\subsection{Standard output} \index{output!main output file}

\siesta\ writes a log of its workings to standard output (unit 6),
which is usually redirected to an ``output file''.

A brief description follows. See the example cases in the
siesta/Tests directory for illustration.

The program starts writing the version of the code which is
used. Then, the input \fdflib\ file is dumped into the output file as is
(except for empty lines). The program does part of the reading and
digesting of the data at the beginning within the \program{redata}
subroutine. It prints some of the information it digests. It is
important to note that it is only part of it, some other information
being accessed by the different subroutines when they need it during
the run (in the spirit of \fdflib\ input).  A complete list of the input
used by the code can be found at the end in the file \file{fdf.log},
including defaults used by the code in the run.

After that, the program reads the pseudopotentials, factorizes them
into Kleinman-Bylander form, and generates (or reads) the atomic basis
set to be used in the simulation. These stages are documented in the
output file.

The simulation begins after that, the output showing information of
the MD (or CG) steps and the SCF cycles within.  Basic descriptions of
the process and results are presented. The user has the option to
customize it, however,\index{output!customization} by defining
different options that control the printing of informations like
coordinates, forces, $\vec k$ points, etc.  The options are discussed
in the appropriate sections, but take into account the behavior of the
legacy \fdf{LongOutput} option, as in the current implementation might
silently activate output to the main .out file at the expense of
auxiliary files.

\begin{fdflogicalF}{LongOutput}
  \index{output!long}

  \siesta\ can write to standard output different data sets depending
  on the values for output options described below.  By default
  \siesta\ will not write most of them. They can be large for large
  systems (coordinates, eigenvalues, forces, etc.)  and, if written to
  standard output, they accumulate for all the steps of the
  dynamics. \siesta\ writes the information in other files (see Output
  Files) in addition to the standard output, and these can be
  cumulative or not.

  Setting \fdf{LongOutput} to \fdftrue\ changes the default of some
  options, obtaining more information in the output (verbose).  In
  particular, it redefines the defaults for the following:

  \begin{itemize}

    \item \fdf{WriteKpoints}%
    \index{output!grid $\vec k$ points}

    \item \fdf{WriteKbands}%
    \index{output!band $\vec k$ points}

    \item \fdf{WriteCoorStep}%
    \index{output!atomic coordinates!in a dynamics step}

    \item \fdf{WriteForces}%
    \index{output!forces}

    \item \fdf{WriteEigenvalues}%
    \index{output!eigenvalues}

    \item \fdf{WriteWaveFunctions}%
    \index{output!wave functions}

    \item \fdf{WriteMullikenPop}%
    \index{output!Mulliken analysis}%
    \index{Mulliken population analysis}%
    (it sets it to 1)

  \end{itemize}

  The specific changing of any of these options has precedence.

\end{fdflogicalF}


\subsection{Output to dedicated files}%
\index{output!dedicated files}

\siesta\ can produce a wealth of information in dedicated files,
with specific formats, that can be used for further analysis. See the
appropriate sections, and the appendix on file formats.
Please take into account the behavior of
\fdf{LongOutput}, as in the current implementation might silently
activate output to the main .out file at the expense of auxiliary
files.

