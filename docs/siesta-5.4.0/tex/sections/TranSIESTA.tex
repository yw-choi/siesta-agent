\newcommand\Nelec{N_{\mathfrak{E}}}


\siesta\ includes the possibility of performing calculations of
electronic transport properties using the \tsiesta\ method. This
Section describes how to use these
capabilities, and a reference guide to the relevant \fdflib\
options. We describe here only the additional options available for
\tsiesta\ calculations, while the rest of the \siesta\ functionalities
and variables are described in the previous sections of this User's
Guide.

An accompanying Python toolbox is available which will assist with
\tsiesta\ calculations. Please use (and cite) \sisl\cite{sisl}.


\subsection{Source code structure}

In this implementation, the \tsiesta\ routines have been grouped in a
set of modules whose file names begin with \texttt{m\_ts} or
\texttt{ts}.


\subsection{Compilation}

Prior to \siesta\ 4.1 \tsiesta\ was a separate executable. Now
\tsiesta\ is fully incorporated into \siesta. \emph{Only} compile
\siesta\ and the full functionality is present.
Sec.~\ref{sec:compilation} for details on compiling \siesta.


\subsection{Brief description}
\label{sec:transiesta:description}

The \tsiesta\ method is a procedure to solve the electronic
structure of an open system formed by a finite structure sandwiched
between semi-infinite metallic leads. A finite bias can be applied
between leads, to drive a finite current. The method is described
in detail in \citet{Brandbyge2002,Papior2017}. In practical terms,
calculations using \tsiesta\ involve the solution of the
electronic density from the DFT Hamiltonian using Greens functions
techniques, instead of the usual diagonalization procedure. Therefore,
\tsiesta\ calculations involve a \siesta\ run, in which a
set of routines are invoked to solve the Greens functions and the
charge density for the open system. These routines are packed in a set
of modules, and we will refer to it as the '\tsiesta\ module'
in what follows.

\tsiesta\ was originally developed by Mads Brandbyge, Jos\'e-Luis
Mozos, Pablo Ordej\'on, Jeremy Taylor and Kurt
Stokbro\cite{Brandbyge2002}. It consisted, mainly, in setting up an
interface between \siesta\ and the (tight-binding) transport codes
developed by M. Brandbyge and K. Stokbro. Initially everything was
written in Fortran-77. As \siesta\ started to be translated to
Fortran-90, so were the \tsiesta\ parts of the code. This was
accomplished by Jos\'e-Luis Mozos, who also worked on the
parallelization of \tsiesta.
%
Subsequently Frederico D. Novaes extended \tsiesta\ to allow $k$-point
sampling for transverse directions. Additional extensions was
added by Nick R. Papior during 2012.

The current \tsiesta\ module has been completely rewritten by Nick
R. Papior and encompass highly advanced inversion algorithms as well
as allowing $N\geq1$ electrode setups among many new
features. Furthermore, the utility \tbtrans\ has also been fully
re-coded (by Nick R. Papior) to be a generic tight-binding code
capable of analyzing physics from the Greens function perspective in
$N\ge1$ setups\cite{Papior2017}.


\begin{itemize}
  \item%
  Transport calculations involve \emph{electrode} (EL) calculations,
  and subsequently the Scattering Region (SR) calculation. The
  \emph{electrode} calculations are usual \siesta\ calculations, but
  where files \sysfile{HSX}/\sysfile{TSHS} (the \sysfile{TSHS} is deprecated),
  and optionally \sysfile{TSDE}, are
  generated. These files contain the information necessary for
  calculation of the self-energies. If any electrodes have identical
  structures (see below) the same files can and should be used to
  describe those. In general, however, electrodes can be different and
  therefore two different \sysfile{HSX} files must be generated. The
  location of these electrode files must be specified in the \fdflib\
  input file of the SR calculation, see \fdf{TS!Elec.<>!HS}.

  \item %
  For the SR, \tsiesta\ starts with the usual \siesta\ procedure,
  converging a Density Matrix (DM) with the usual Kohn-Sham scheme for
  periodic systems. It uses this solution as an initial input for the
  Greens function self consistent cycle. Effectively you will start a
  \tsiesta\ calculation from a fully periodic calculation. This is why
  the $0\, V$ calculation should be the only calculation where you start
  from \siesta.

  \tsiesta\ stores the SCF DM in a file named \sysfile{TSDE}. In a rerun of
  the same system (meaning the same \fdf{SystemLabel}), if the
  code finds a \sysfile{TSDE} file in the directory, it will take this
  DM as the initial input and this is then considered a continuation
  run. In this case it does not perform an initial \siesta\ run. It
  must be clear that when starting a calculation from scratch, in the
  end one will find both files, \sysfile{DM} and \sysfile{TSDE}.
  %
  The first one stores the \siesta\ density matrix (periodic boundary
  conditions in all directions and no voltage), and the latter the
  \tsiesta\ solution.

  \item %
  When performing several bias calculations, it is heavily advised to
  run different bias' in different directories. To drastically improve
  convergence (and throughput) one should copy the \sysfile{TSDE} from
  the closest, previously, calculated bias to the current bias.

  \item %
  The \sysfile{TSDE} may be read equivalently as the
  \sysfile{DM}. Thus, it may be used by fx. \program{denchar} to
  analyze the non-equilibrium charge density. Alternatively one can
  use \sisl\cite{sisl} to interpolate the DM and EDM to speed up
  convergence.

  \item %
  As in the case of \siesta\ calculations, what \tsiesta\ does is to
  obtain a converged DM, but for open boundary conditions and possibly
  a finite bias applied between electrodes. The corresponding
  Hamiltonian matrix (once self consistency is achieved) of the SR is
  also stored in a \sysfile{HSX} file. Subsequently, transport
  properties are obtained in a post-processing procedure using the
  \tbtrans\ code (located in the \program{Util/TS/TBtrans}
  directory). We note that the \sysfile{HSX} files contain all the
  needed structural information (atomic positions, matrix elements,
  \ldots), and so the input (fdf) flags for the geometry and basis
  have no influence of the subsequent \tbtrans\ calculations.

  \item %
  When the non-equilibrium calculation uses different electrodes one
  should use so-called \emph{buffer} atoms behind the electrodes to act
  as additional screening regions when calculating the initial guess
  (using \siesta) for \tsiesta. Essentially they may be used to
  achieve a better ``bulk-like'' environment at the electrodes in the
  SR calculation.


  \item%
  An important parameter is the lower bound of the energy contours. It
  is a good practice, to start with a \siesta\ calculation for the SR
  and look at the eigenvalues of the system. The lower bound of the
  contours must be \emph{well} below the lowest eigenvalue.

  \item%
  Periodic boundary conditions are assumed in 2 cases.

  \begin{enumerate}
    \item For $\Nelec\neq 2$ all lattice vectors are periodic, users
    \emph{must} manually define \fdf{TS!kgrid!MonkhorstPack}

    \item For $\Nelec=2$ \tsiesta\ will auto-detect if both electrodes
    are semi-infinite along the same lattice vector. If so, only 1 $k$
    point will be used along that lattice vector.
  \end{enumerate}

  \item%
  The default algorithm for matrix inversion is the BTD method, before
  starting a \tsiesta\ calculation please run with the analyzation
  step \fdf{TS!Analyze} (note this is very fast and can be done on any
  desktop computer, regardless of system size).

  \item%
  Importantly(!) the $k$-point sampling need typically be much higher
  in a \tbtrans\ calculation to achieve a converged transmission
  function.

  \item%
  Energies from \tsiesta\ are \emph{not} to be trusted since the open
  boundaries complicates the energy calculation. Therefore care needs
  to be taken when comparing energies between different calculations
  and/or different bias'.

  \item%
  Always ensure that charges are preserved in the scattering region
  calculation. Doing the SCF an output like the following will be shown:
  \begin{output}[fontsize=\footnotesize]
ts-q:         D        E1        C1        E2        C2        dQ
ts-q:   436.147   392.146     3.871   392.146     3.871  7.996E-3
  \end{output}
  Always ensure the last column (\code{dQ}) is a very small fraction of
  the total number of electrons. Ideally this should be $0$.
  %
  For $0$ bias calculations this should be very small, typically less
  than $0.1\,\%$ of the total charge in the system. If this is not the
  case, it probably means that there is not enough screening towards the
  electrodes which can be solved by adding more electrode layers between
  the electrode and the scattering region. This layer thickness is
  \emph{very} important to obtain a correct open boundary calculation.

  \item%
  Do \emph{not} perform \tsiesta\ calculations using semi-conducting
  electrodes. The basic premise of \tsiesta\ calculations is that the
  electrodes \emph{behave like bulk} in the electrode regions of the
  SR. This means that the distance between the electrode and the
  perturbed must equal the screening length of the electrode.

  This is problematic for semi-conducting systems since they
  intrinsically have a very long screening length.

  In addition, the Fermi-level of semi-conductors are not well-defined
  since it may be placed anywhere in the band gap.

\end{itemize}


\subsection{Electrodes}

To calculate the electronic structure of a system under external bias,
\tsiesta\ attaches the system to semi-infinite electrodes which extend
to their respective semi-infinite directions. Examples of electrodes
would include surfaces, nanowires, nanotubes or fully infinite
regions. The electrode must be large enough (in the semi-infinite
direction) so that orbitals within the unit cell only interact with a
single nearest neighbor cell in the semi-infinite direction (the size
of the unit cell can thus be derived from the range of support for the
orbital basis functions). \tsiesta\ will stop if this is not
enforced. The electrodes are generated by a separate \tsiesta\ run on
a bulk system. This implies that the proper bulk properties are
obtained by a sufficiently high $k$-point sampling. If in doubt, use
100 $k$-points along the semi-infinite direction. The results are
saved in a file with extension \sysfile{HSX} which contains a
description of the electrode unit cell, the position of the atoms
within the unit cell, as well as the Hamiltonian and overlap matrices
that describe the electronic structure of the lead. One can generate a
variety of electrodes and the typical use of \tsiesta\ would involve
reusing the same electrode for several setups. At runtime, the
\tsiesta\ coordinates are checked against the electrode coordinates
and the program stops if there is a mismatch to a certain precision
($10^{-4}\,\mathrm{Bohr}$). Note that the atomic coordinates are
compared relatively. Hence the \emph{input} atomic coordinates of the
electrode and the device need not be the same (see e.g. the tests in
the \shell{Tests}\index{Tests} directory.

To run an electrode calculation one should do:
\begin{shellexample}
  siesta --electrode RUN.fdf
\end{shellexample}
or define these options in the electrode fdf files:
\fdf{Save!HS} and \fdf{TS!DE.Save} to \fdf*{true} (the above
\code{--electrode} is a shorthand to forcefully define the two options).


\subsubsection{Matching coordinates}

Here are some rules required to successfully construct the appropriate
coordinates of the scattering region. Contrary to versions prior to
4.1, the order of atoms is largely irrelevant. One may define all
electrodes, then subsequently the device, or vice versa. Similarly,
buffer atoms are not restricted to be the first/last atoms.

However, atoms in any given electrode \emph{must} be consecutive in
the device file. I.e. if an electrode input option is given by:
\begin{fdfexample}
  %block TS.Elec.<>
    HS ../elec-<>/siesta.HSX
    bloch 1 3 1
    used-atoms 4
    electrode-position 10
    ...
  %endblock
\end{fdfexample}
then the atoms from $10$ to $10+4*3-1$ must coincide with the atoms of
the calculation performed in the \program{../elec-<>/}
subdirectory. The above options will be discussed in the following
section.

When using the Bloch expansion (highly recommended if your system
allows it) it is advised to follow the \emph{tiling} method. However
both of the below sequences are allowed.

\paragraph{Tile} \fdfindex{TS!Elec.<>!Bloch}%
Here the atoms are copied and displaced by the full
electrode. Generally this expansion should be preferred over the
\emph{repeat} expansion due to much faster execution.
\begin{fdfexample}
  iaD = 10 ! as per the above input option
  do iC = 0 , nC - 1
  do iB = 0 , nB - 1
  do iA = 0 , nA - 1
    do iaE = 1 , na_u
      xyz_device(:, iaD) = xyz_elec(:, iaE) + &
          cell_elec(:, 1) * iA + &
          cell_elec(:, 2) * iB + &
          cell_elec(:, 3) * iC
      iaD = iaD + 1
    end do
  end do
  end do
  end do
\end{fdfexample}

By using \sisl\cite{sisl} one can achieve the tiling scheme
by using the following command-line utility on an input
\program{ELEC.fdf} structure with the minimal electrode:
\begin{codeexample}
  sgeom -tx 1 -ty 3 -tz 1 ELEC.fdf DEVICE_ELEC.fdf
\end{codeexample}

\paragraph{Repeat} \fdfindex{TS!Elec.<>!Bloch}%
Here the atoms are copied individually. Generally
this expansion should \emph{not} be used since it is much slower than
tiling.
\begin{fdfexample}
  iaD = 10 ! as per the above input option
  do iaE = 1 , na_u
    do iC = 0 , nC - 1
    do iB = 0 , nB - 1
    do iA = 0 , nA - 1
      xyz_device(:, iaD) = xyz_elec(:, iaE) + &
          cell_elec(:, 1) * iA + &
          cell_elec(:, 2) * iB + &
          cell_elec(:, 3) * iC
      iaD = iaD + 1
    end do
    end do
    end do
  end do
\end{fdfexample}

By using \sisl\cite{sisl} one can achieve the repeating scheme by
using the following command-line utility on an input
\program{ELEC.fdf} structure with the minimal electrode:
\begin{codeexample}
  sgeom -rz 1 -ry 3 -rx 1 ELEC.fdf DEVICE_ELEC.fdf
\end{codeexample}



\subsubsection{Principal layer interactions} %
\index{transiesta!electrode!principal layer}%

It is \emph{extremely} important that the electrodes only interact
with one neighboring supercell due to the self-energy
calculation\cite{Sancho1985}. \tsiesta\ will print out a block as this
(\shell{<>} is the electrode name):
\begin{verbatim}
 <> principal cell is perfect!
\end{verbatim}
if the electrode is correctly setup and it only interacts with its
neighboring supercell.
%
In case the electrode is erroneously setup, something similar to the
following will be shown in the output file.
\begin{verbatim}
 <> principal cell is extending out with 96 elements:
    Atom 1 connects with atom 3
    Orbital 8 connects with orbital 26
    Hamiltonian value: |H(8,6587)|@R=-2 =  0.651E-13 eV
    Overlap          :  S(8,6587)|@R=-2 =   0.00
\end{verbatim}
It is imperative that you have a \emph{perfect} electrode as otherwise
nonphysical results will occur. This means that you need to add more
layers in your electrode calculation (and hence also in your
scattering region). An example is an ABC stacking electrode. If the
above error is shown one \emph{has} to create an electrode with ABCABC
stacking in order to retain periodicity.

By default \tsiesta\ will die if there are connections beyond the
principal cell. One may control whether this is allowed or not by
using \fdf{TS!Elecs!Neglect.Principal}.



\subsection{Convergence of electrodes and scattering regions}

For successful \tsiesta\ calculations it is imperative that the
electrodes and scattering regions are well-converged.
%
The basic principle is equivalent to the \siesta\ convergence, see
Sec.~\ref{sec:scf}.

The steps should be something along the line of (only done at
$0\, V$).
\begin{enumerate}

  \item%
  Converge electrodes and find optimal \fdf{Mesh!Cutoff},
  \fdf{kgrid!MonkhorstPack} etc.

  Electrode $k$ points should be very high along the semi-infinite
  direction. The default is $100$, but at least $>50$ should easily be
  reachable.


  \item%
  Use the parameters from the electrodes and also converge the
  same parameters for the scattering region SCF.

  This is an iterative process since the scattering region forces the
  electrodes to use equivalent $k$ points (see
  \fdf{TS!Elec.<>!check-kgrid}).

  Note that $k$ points should be limited in the \tsiesta\ run, see
  \fdf{TS!kgrid!MonkhorstPack}.

  One should always use the same parameters in both the electrode and
  scattering region calculations, except the number of $k$ points for
  the electrode calculations along their respective semi-infinite
  directions.


  \item%
  Once \tsiesta\ is completed one should also converge the
  number of $k$ points for \tbtrans. Note that $k$ point sampling in
  \tbtrans\ should generally be much denser but \emph{always} fulfill
  $N_k^{\tsiesta}\geq N_k^\tbtrans$

\end{enumerate}

The converged parameters obtained at $0\,\mathrm V$ should be used for
all subsequent bias calculations. Remember to copy the \sysfile{TSDE}
from the closest, previously, calculated bias for restart and much
faster convergence.


\tsiesta\ is also more difficult to converge during the SCF
steps. This may be due to several interrelated problems:
%
\begin{itemize}

  \item%
  A too short screening distance between the scattering atoms
  and the electrode layers.


  \item%
  In case buffer atoms (\fdf{TS!Atoms.Buffer}) are used with
  vacuum on the backside it may be that there are too few buffer atoms
  to accurately screen off the vacuum region for a sufficiently good
  initial guess. This effect is only true for $0\,\mathrm V$
  calculations.


  \item%
  The mixing parameters may need to be smaller than for \siesta,
  see Sec.~\ref{sec:scf:mix} and it is never guaranteed that it will
  converge. It is \emph{always} a trial and error method, there are
  \emph{no} omnipotent mixing parameters.


  \item%
  Very high bias' may be extremely difficult to
  converge. Generally one can force bias convergence by doing smaller
  steps of bias. E.g. if problems arise at $0.5\,\mathrm V$ with an
  initial DM from a $0.25\,\mathrm V$ calculation, one could try and
  $0.3\,\mathrm V$ first.


  \item%
  If a particular bias point is hard to converge, even by doing
  the previous step, it may be related to an eigenstate close to the
  chemical potentials of either electrode (e.g. a molecular eigenstate
  in the junction). In such cases one could try an even higher bias
  and see if this converges more smoothly.

\end{itemize}



\subsection{NEGF equations}
\label{sec:negf-equations}

The options available for \tsiesta\ will impact how the calculation
is performed. It is vital that the users carefully read this section
and the options that refer to these.

The NEGF equation are primarily concerning the Green function:
\begin{equation}
  \label{eq:negf:green}
  \G(E) = \big[ (E+i\eta)\SO - \Ham - \sum_\elec \SE_\elec(E).
\end{equation}
The electrode self-energy is calculated from the bulk electrode
calculation
\begin{equation}
  \SE_\elec(E) \leftarrow \big\{\Ham_\elec, \SO_\elec\big\}.
\end{equation}

\tsiesta\ has options to discern which Hamiltonian elements can be
used in which parts of the calculation.
Default is that the electrode matrices ($\Ham_\elec, \SO_\elec$) are
used whenever the electrode enters a matrix. Lets show a partitioning
of the Green function for a particular electrode ($z=E+i\eta$)
\begin{equation}
  \label{eq:negf:green-elec}
  \G(z) =
  \begin{bmatrix}
    \mathbf M_{\elec,\elec} & \mathbf M_{\elec, D} & \dots
    \\
    \mathbf M_{D,\elec} & \mathbf M_{D, D} &
    \\
    \vdots && \ddots
    \\
  \end{bmatrix}^{-1}=
  \begin{bmatrix}
    (z+\mu_elec)\SO_\elec - \Ham_\elec - \SE_\elec(E) & z\SO_{\elec,D} - \Ham_{\elec,D} & \dots
    \\
    z\SO_{D,\elec} - \Ham_{D,\elec} & z\SO_D - \Ham_D &
    \\
    \vdots & & \ddots
    \\
  \end{bmatrix}^{-1}.
\end{equation}
The following options alter the above equation slightly:
\begin{itemize}
  \item \fdf{TS!Elec.<>!Bulk}
  \item \fdf{TS!Elec.<>!Eta}
  \item \fdf{TS!Elec.<>!chemical-potential}
  \item \fdf{TS!Elec.<>!V-fraction} (experts only!)
  \item \fdf{TS!Elec.<>!delta-Ef} (experts only!)
\end{itemize}


\subsection{\texorpdfstring{\tsiesta\ }{TranSIESTA} Options}

The fdf options shown here are only to be used at the input file for
the scattering region. When using \tsiesta\ for electrode
calculations, only the usual \siesta\ options are relevant.
%
Note that since \tsiesta\ is a generic $\Nelec$ electrode NEGF code the
input options are heavily changed compared to versions prior to 4.1.

\subsubsection{Quick and dirty}

Since 4.1, \tsiesta\ has been fully re-implemented. And so have
\emph{every} input fdf-flag. To accommodate an easy transition between
previous input files and the new version format a small utility called
\program{ts2ts}. It may be compiled in \program{Util/TS/ts2ts}. It is
recommended that you use this tool if you are familiar with previous
\tsiesta\ versions.

%
One may input options as in the old \tsiesta\ version and then run
\begin{fdfexample}
  ts2ts OLD.fdf > NEW.fdf
\end{fdfexample}
which translates all keys to the new, equivalent, input format. If you
are familiar with the old-style flags this is highly recommendable
while becoming comfortable with the new input format. Please note that
some defaults have changed to more conservative values in the newer
release.

If one does not know the old flags and wish to get a basic example of
an input file, a script \program{Util/TS/tselecs.sh} exists that can
create the basic input for $\Nelec$ electrodes. One may call it like:
\begin{shellexample}
  tselecs.sh -2 > TWO_ELECTRODE.fdf
  tselecs.sh -3 > THREE_ELECTRODE.fdf
  tselecs.sh -4 > FOUR_ELECTRODE.fdf
  ...
\end{shellexample}
where the first call creates an input fdf for 2 electrode setups, the
second for a 3 electrode setup, and so on. See the help (\program{-h})
for the program for additional options.

Before endeavoring on large scale calculations you are advised to run
an analyzation of the system at hand, you may run your system as
\begin{shellexample}
  siesta -fdf TS.Analyze RUN.fdf > analyze.out
\end{shellexample}
which will analyze the sparsity pattern and print out several
different pivoting schemes. Please see \fdf{TS!Analyze} for more
information.


\subsubsection{General options}

One have to set \fdf{SolutionMethod} to \fdf*{transiesta} to enable
\tsiesta.

\begin{fdfentry}{TS!SolutionMethod}[string]<btd|mumps|full>

  Control the algorithm used for calculating the Green
  function. Generally the BTD method is the fastest and this option
  need not be changed.

  \begin{fdfoptions}
    \option[BTD]%
    \fdfindex*{TS!SolutionMethod:BTD}%
    Use the block-tri-diagonal algorithm for matrix inversion.

    This is generally the recommended method.

    \option[MUMPS]%
    \fdfindex*{TS!SolutionMethod:MUMPS}%
    Use sparse matrix inversion algorithm (MUMPS). This requires
    \tsiesta\ to be compiled with MUMPS.
    \index{MUMPS}%
    \index{External library!MUMPS}%

    \option[full]%
    \fdfindex*{TS!SolutionMethod:full}%
    Use full matrix inversion algorithm (LAPACK). Generally only
    usable for debugging purposes.

  \end{fdfoptions}

\end{fdfentry}

\begin{fdfentry}{TS!Voltage}[energy]<$0\,\mathrm{eV}$>

  Define the reference applied bias. For $\Nelec=2$ electrode calculations
  this refers to the actual potential drop between the electrodes,
  while for $\Nelec\neq2$ this is a reference bias. In the latter case it
  \emph{must} be equivalent to the maximum difference between the
  chemical potential of any two electrodes.

  \note Specifying \shell{-V}\fdfindex{Command line options:-V} on the
  command-line overwrites the value in the fdf file.

\end{fdfentry}

\begin{fdfentry}{TS!kgrid!MonkhorstPack}[block]<\fdfvalue{kgrid!MonkhorstPack}>

  $k$ points used for the \tsiesta\ calculation.

  For $\Nelec\neq2$ this should always be defined. Always take care to
  use only 1 $k$ point along non-periodic lattice vectors. An
  electrode semi-infinite region is considered non-periodic since it
  is integrated out through the self-energies.

  This defaults to \fdf{kgrid!MonkhorstPack}.

\end{fdfentry}

\begin{fdfentry}{TS!Atoms.Buffer}[block/list]
  \fdfindex{TS.BufferAtomsLeft|see TS!Atoms.Buffer}%
  \fdfindex{TS.BufferAtomsRight|see TS!Atoms.Buffer}%

  Specify atoms that will be removed in the \tsiesta\ SCF. They are
  not considered in the calculation and may be used to improve the
  initial guess for the Hamiltonian.


  An intended use for buffer atoms is to ensure a bulk behavior in the
  electrode regions when electrodes are different. As an example: a 2
  electrode calculation with left consisting of Au atoms and the right
  consisting of Pt atoms. In such calculations one cannot create a
  periodic geometry along the transport direction. One needs to add
  vacuum between the Au and Pt atoms that comprise the
  electrodes. However, this creates an artificial edge of the
  electrostatic environment for the electrodes since in \siesta\ there
  is vacuum, whereas in \tsiesta\ the effective Hamiltonian sees a
  bulk environment. To ensure that \siesta\ also exhibits a bulk
  environment on the electrodes we add \emph{buffer} atoms towards the
  vacuum region to screen off the electrode region. These
  \emph{buffer} atoms is thus a technicality that has no influence on
  the \tsiesta\ calculation but they are necessary to ensure the
  electrode bulk properties.

  The above discussion is even more important when doing $\Nelec$-electrode
  calculations.

  \note all lines are additive for the buffer atoms and the input
  method is similar to that of \fdf{Geometry!Constraints} for the
  \fdf*{atom} line(s).

  \begin{fdfexample}
    %block TS.Atoms.Buffer
       atom [ 1 -- 5 ]
    %endblock
    # Or equivalently as a list
    TS.Atoms.Buffer [1 -- 5]
  \end{fdfexample}
  will remove atoms [1--5] from the calculation.

\end{fdfentry}

\begin{fdfentry}{TS!ElectronicTemperature}[energy]<\fdfvalue{ElectronicTemperature}>

  Define the temperature used for the Fermi distributions for the
  chemical potentials.
  %
  See \fdf{TS!ChemPot.<>!ElectronicTemperature}.

\end{fdfentry}

\begin{fdfentry}{TS!SCF!DM.Tolerance}[real]<\fdfvalue{SCF.DM!Tolerance}>%
  \fdfdepend{SCF.DM!Tolerance,SCF.DM!Converge}

  The density matrix tolerance for the \tsiesta\ SCF cycle.

\end{fdfentry}

\begin{fdfentry}{TS!SCF!H.Tolerance}[energy]<\fdfvalue{SCF.H!Tolerance}>%
  \fdfdepend{SCF.H!Tolerance,SCF.H!Converge}

  The Hamiltonian tolerance for the \tsiesta\ SCF cycle.

\end{fdfentry}

\begin{fdflogicalT}{TS!SCF!dQ.Converge}

  Whether \tsiesta\ should check whether the total charge is within a
  provided tolerance, see \fdf{TS!SCF!dQ.Tolerance}.

\end{fdflogicalT}

\begin{fdfentry}{TS!SCF!dQ.Tolerance}[real]<$\mathrm{Q(device)}\cdot 10^{-3}$>%
  \fdfdepend{TS!SCF!dQ.Converge}

  The charge tolerance during the SCF.

  The charge is not stable in \tsiesta\ calculations and this flag
  ensures that one does not, by accident, do post-processing of files
  where the charge distribution is completely wrong.

  A too high tolerance may heavily influence the electrostatics of the
  simulation.

  \note Please see \fdf{TS!dQ} for ways to reduce charge loss in
  equilibrium calculations.

\end{fdfentry}

\begin{fdfentry}{TS!SCF.Initialize}[string]<diagon|transiesta>%

  Control which initial guess should be used for \tsiesta. The general
  way is the \fdf*{diagon} solution method (which is preferred),
  however, one can start a \tsiesta\ run immediately. If you start
  directly with \tsiesta\ please refer to these flags:
  \fdf{TS!Elecs!DM.Init} and \fdf{TS!Fermi.Initial}.

  \note Setting this to \fdf*{transiesta} is highly experimental and
  convergence may be extremely poor.

\end{fdfentry}

\begin{fdfentry}{TS!Fermi.Initial}[energy]<$\sum^{N_E}_iE_F^i/N_E$>

  Manually set the initial Fermi level to a predefined value.

  \note this may also be used to change the Fermi level for
  calculations where you restart calculations. Using this feature is
  highly experimental.

\end{fdfentry}

\begin{fdfentry}{TS!Weight.Method}[string]<orb-orb|[[un]correlated+][sum|tr]-atom-[atom|orb]|mean>

  Control how the NEGF weighting scheme is conducted. Generally one
  should only use the \fdf*{orb-orb} while the others are present for
  more advanced usage. They refer to how the weighting coefficients of
  the different non-equilibrium contours are performed. In the
  following the weight are denoted in a two-electrode setup while they
  are generalized for multiple electrodes.

  \def\mypropto{\,\oppropto^{||}\,} %
  \def\mn{{\mu\nu}} %
  Define the normalised geometric mean as $\mypropto$ via
  \begin{equation}
    w\mypropto \langle\cdot^L\rangle\equiv
    \frac{\langle\cdot^L\rangle}{\langle\cdot^L\rangle+\langle\cdot^R\rangle}.
  \end{equation}

  When applying a bias, \tsiesta\ will printout the following during
  the SCF cycle:
\begin{output}[fontsize=\footnotesize]
ts-err-D: ij(  447,   447), M =  1.8275, ew = -.257E-2, em = 0.258E-2. avg_em = 0.542E-06
ts-err-E: ij(  447,   447), M = -6.7845, ew = 0.438E-3, em = -.439E-3. avg_em = -.981E-07
ts-w-q:               qP1       qP2
ts-w-q:           219.150   216.997
ts-q:         D        E1        C1        E2        C2        dQ
ts-q:   436.147   392.146     3.871   392.146     3.871  7.996E-3
  \end{output}
  %
  The extra output corresponds to fine details in the integration
  scheme.
  \begin{description}[labelindent=3em, leftmargin=4.5em]
    \itemsep 10pt
    \parsep 0pt

    \item[\texttt{ts-err-*}] are estimated error outputs from the
    different integrals, for the density matrix (\texttt{D}) and the
    energy density matrix (\texttt{E}), see Eq.~(12) in
    \cite{Papior2017}. All values (except \texttt{avg\_em}) are for
    the given orbital site

    \begin{description}
      \itemsep 4pt
      \parsep 0pt

      \item[\texttt{ij(A,B)}] refers to the matrix element between orbital
      \texttt{A} and \texttt{B}

      \item[\texttt{M}] is the weighted matrix element value,
      $\sum_{\elec}w_\elec\DM^\elec$

      \item[\texttt{ew}] is the maximum difference between
      $\sum_{\elec}w_\elec\DM^\elec-\DM^\elec$ for all $\elec$.

      \item[\texttt{em}] is the maximum difference between
      $\DM^{\elec'}-\DM^\elec$ for all combinations of $\elec$ and
      $\elec'$.

      \item[\texttt{avg\_em}] is the averaged difference of \texttt{em} for all
      orbital sites.

    \end{description}

    \item[\texttt{ts-w-q}] is the Mulliken charge from the different
    integrals: $\Tr[w_\elec\DM^\elec\SO]$

  \end{description}

  \begin{fdfoptions}

    \option[orb-orb]%
    \fdfindex*{TS!Weight.Method:orb-orb}%
    Weight each orbital-density matrix element individually.

    \option[tr-atom-atom]%
    \fdfindex*{TS!Weight.Method:tr-atom-atom}%
    Weight according to the trace of the atomic density matrix sub-blocks
    \begin{equation}
      w_{ij}^{\Tr} \mypropto
      \sqrt{
          % First the i'th atom
          \sum_{\in\{i\}}(\Delta\rho_{\mu\mu}^L)^2
          \; % ensure a little space between them
          % second the j'th atom
          \sum_{\in\{j\}}(\Delta\rho_{\mu\mu}^L)^2
      }
    \end{equation}

    \option[tr-atom-orb]%
    \fdfindex*{TS!Weight.Method:tr-atom-orb}%

    Weight according to the trace of the atomic density matrix
    sub-block times the weight of the orbital weight
    \begin{equation}
      w_{ij,\mn}^{\Tr} \mypropto
      \sqrt{
          w_{ij}^{\Tr}
          w_{ij,\mn}
      }
    \end{equation}

    \option[sum-atom-atom]%
    \fdfindex*{TS!Weight.Method:sum-atom-atom}%

    Weight according to the total sum of the atomic density matrix
    sub-blocks
    \begin{equation}
      w_{ij,\mn}^{\Sigma} \mypropto
      \sqrt{
          % First the i'th atom
          \sum_{\in\{i\}}(\Delta\rho_{\mn}^L)^2
          \; % ensure a little space between them
          % second the j'th atom
          \sum_{\in\{j\}}(\Delta\rho_{\mn}^L)^2
      }
    \end{equation}

    \option[sum-atom-orb]%
    \fdfindex*{TS!Weight.Method:sum-atom-orb}%

    Weight according to the total sum of the atomic density matrix
    sub-block times the weight of the orbital weight
    \begin{equation}
      w_{ij,\mn}^{\Sigma} \mypropto
      \sqrt{
          w_{ij}^{\Sigma}
          w_{ij,\mn}
      }
    \end{equation}

    \option[mean]%
    \fdfindex*{TS!Weight.Method:mean}%

    A standard average.

  \end{fdfoptions}


  Each of the methods (except \fdf*{mean}) comes in a correlated and
  uncorrelated variant where $\sum$ is either outside or inside the
  square, respectively.

\end{fdfentry}

\begin{fdfentry}{TS!Weight.k.Method}[string]<correlated|uncorrelated>

  Control weighting \emph{per} $k$-point or the full sum. I.e. if
  \fdf*{uncorrelated} is used it will weight $n_k$ times if there are
  $n_k$ $k$-points in the Brillouin zone.

\end{fdfentry}

\begin{fdflogicalT}{TS!Forces}

  Control whether the forces are calculated. If \emph{not} \tsiesta\
  will use slightly less memory and the performance slightly
  increased, however the final forces shown are incorrect.

  If this is \fdftrue\ the file \sysfile{TSFA} (and possibly the
  \sysfile{TSFAC}) will be created. They contain forces for the atoms
  that are having updated density-matrix elements
  (\fdf{TS!Elec.<>!DM-update:all}).

  Generally one should not expect good forces close to the
  electrode/device interface since this typically has some
  electrostatic effects that are inherent to the \tsiesta\ method.
  Forces on atoms \emph{far} from the electrode can safely be
  analyzed.

\end{fdflogicalT}

\begin{fdfentry}{TS!dQ}[string]<none|buffer|fermi>
  \fdfindex*{TS!dQ:fermi}

  Any excess/deficiency of charge can be re-adjusted after each
  \tsiesta\ cycle to reduce charge fluctuations in the cell.

  \note recommended to \emph{only} use charge corrections for
  $0\,\mathrm{V}$ calculations.

  The non-neutral charge in \tsiesta\ cycles is an expression of one
  of the following things:
  \begin{enumerate}
    \item An incorrect screening towards the electrodes. To check
    this, simply add more electrode layers towards the device at each
    electrode and see how the charge evolves. It should tend to zero.

    The best way to check this is to follow these steps:
    \begin{enumerate}
      \item%
      Perform a \siesta-only calculation (the resulting DM
      should be used as the starting point for both following
      calculations)

      \item%
      Perform a \tsiesta\ calculation with the option
      \fdf{TS!Elecs!DM.Init:diagon} (please note that the electrode
      option has precedence, so remove any entry from the
      \fdf{TS!Elec.<>} block)

      \item%
      Perform a \tsiesta\ calculation with the option
      \fdf{TS!Elec.<>!DM-init:bulk} (please note that the electrode
      option has precedence, so remove any entry from the
      \fdf{TS!Elec.<>} block)

    \end{enumerate}

    Now compare the final output and the initial charge distribution,
    e.g.:
    \begin{output}
>>> TS.Elecs.DM.Init diagon
transiesta: Charge distribution, target =    396.00000
Total charge                  [Q]  :   396.00000

>>> TS.Elecs.DM.Init bulk
transiesta: Charge distribution, target =    396.00000
Total charge                  [Q]  :   395.9995
\end{output}

    The above shows that there is very little charge difference
    between the bulk electrode DM and the scattering region. This
    ensures that the charge distribution are similar and that your
    electrode is sufficiently screened.

    Additionally one may compare the final output such as total
    energies, calculated DOS and ADOS (see \tbtrans). If the two
    calculations show different properties, one should carefully
    examine the system setup.

    \item An incorrect reference energy level. In \tsiesta\ the Fermi
    level is calculated from the \siesta\ SCF. However, the \siesta\
    Fermi level corresponds to a periodic calculation and \emph{not}
    an open system calculation such as NEGF.

    If the first step shows a good screening towards the electrode it
    is usually the reference energy level, then use \fdf{TS!dQ:fermi}.

    \item A combination of the above, this is the typical case.
  \end{enumerate}

  \begin{fdfoptions}

    \option[none]%
    No charge corrections are introduced.

    \option[buffer]%
    Excess/missing electrons are placed in the buffer regions (buffer
    atoms are required to exist)

    \option[fermi] %
    Correct the charge filling by calculating a new reference energy
    level (referred to as the Fermi level). \\
    We approximate the contribution to be constant around the Fermi
    level and find
    \begin{equation}
      \label{eq:fermi-shift}
      \mathrm{d}E_F = \frac{Q'-Q}{Q|_{E_F}},
    \end{equation}
    where $Q'$ is the charge from a \tsiesta\ SCF step
    and $Q|_{E_F}$ is the equilibrium charge at the current Fermi
    level, $Q$ is the supposed charge to reside in the
    calculation. Fermi correction utilizes Eq.~\eqref{eq:fermi-shift} for
    the first correction and all subsequent corrections are based on a
    cubic spline interpolation to faster converge the
    ``correct'' Fermi level.

    This method will create a file called \file{TS\_FERMI}.

    \note correcting the reference energy level is a costly
    operation since the SCF cycle typically gets
    \emph{corrupted} resulting in many more SCF cycles.

  \end{fdfoptions}

\end{fdfentry}

\begin{fdfentry}{TS!dQ!Factor}[real]<0.8>

  Any positive value close to $1$. $0$ means no charge correction. $1$
  means total charge correction. This will reduce the fluctuations in
  the SCF and setting this to $1$ may result in difficulties in
  converging.

\end{fdfentry}

\begin{fdfentry}{TS!dQ!Fermi.Tolerance}[real]<0.01>

  The tolerance at which the charge correction will converge. Any
  excess/missing charge ($|Q'-Q|>\mathrm{Tol}$) will result in a
  correction for the Fermi level.

\end{fdfentry}

\begin{fdfentry}{TS!dQ!Fermi.Max}[energy]<$1.5\,\mathrm{eV}$>%

  The maximally allowed value that the Fermi level will change from a
  charge correction using the Fermi correction method. In case the
  Fermi level lies in between two bands a DOS of $0$ at the Fermi
  level will make the Fermi change equal to $\infty$. This is not
  physical and the user can thus truncate the correction.

  \note If you know the band-gab, setting this to $1/4$ (or smaller)
  of the band gab seems like a better value than the rather
  arbitrarily default one.

\end{fdfentry}

\begin{fdfentry}{TS!dQ!Fermi.Eta}[energy]<$1\,\mathrm{meV}$>%

  The $\eta$ value that we extrapolate the charge at the poles to.
  Usually a smaller $\eta$ value will mean larger changes in the
  Fermi level. If the charge convergence w.r.t. the Fermi level is
  fluctuating a lot one should increase this $\eta$ value.

\end{fdfentry}

\begin{fdflogicalT}{TS!HS.Save}
  \fdfindex*{TS!HS.Save:true}

  Must be \fdftrue\ for saving the Hamiltonian (\sysfile{TSHS}). Can only be set if
  \fdf{SolutionMethod} is not \fdf*{transiesta}.

  The default is \fdffalse\ for \fdf{SolutionMethod} different from
  \fdf*{transiesta} and if \code{--electrode} has not been passed as a
  command line argument.

  \note The \sysfile{TSHS} file format is deprecated and may be removed in
  future \siesta\ versions. The \sysfile{TS.HSX} file format (\fdf{Save!HS}) is
  feature complete, and should be used.

\end{fdflogicalT}

\begin{fdflogicalT}{TS!DE.Save}
  \fdfindex*{TS!DE.Save:true}

  Must be \fdftrue\ for saving the density and energy density matrix
  for continuation runs (\sysfile{TSDE}). Can only be set if
  \fdf{SolutionMethod} is not \fdf*{transiesta}.

  The default is \fdffalse\ for \fdf{SolutionMethod} different from
  \fdf*{transiesta} and if \code{--electrode} has not been passed as a
  command line argument.

\end{fdflogicalT}

\begin{fdflogicalF}{TS!S.Save}

  This is a flag mainly used for the Inelastica code to produce
  overlap matrices for Pulay corrections. This should only be used by
  advanced users.

\end{fdflogicalF}


\begin{fdflogicalF}{TS!SIESTA.Only}

  Stop \tsiesta\ right after the initial diagonalization run in
  \siesta. Upon exit it will also create the \sysfile{TSDE} file which
  may be used for initialization runs later.

  This may be used to start several calculations from the same initial
  density matrix, and it may also be used to rescale the Fermi level
  of electrodes. The rescaling is primarily used for semi-conductors
  where the Fermi levels of the device and electrodes may be
  misaligned.

\end{fdflogicalF}


\begin{fdflogicalF}{TS!Analyze}

  When using the BTD solution method (\fdf{TS!SolutionMethod}) this
  will analyze the Hamiltonian and printout an analysis of the
  sparsity pattern for optimal choice of the BTD partitioning
  algorithm.

  This yields information regarding the \fdf{TS!BTD!Pivot} flag.

  \note we advice users to \emph{always} run an analyzation step prior
  to actual calculation and select the \emph{best} BTD format. This
  analyzing step is very fast and may be performed on small
  work-station computers, even on systems of $\gg10,000$ orbitals.

  To run the analyzing step you may do:
  \begin{shellexample}
    siesta -fdf TS.Analyze RUN.fdf > analyze.out
  \end{shellexample}
  note that there is little gain on using MPI and it should complete
  within a few minutes, no matter the number of orbitals.

  Choosing the best one may be difficult. Generally one should choose
  the pivoting scheme that uses the least amount of memory. However,
  one should also choose the method with the largest block-size being
  as small as possible. As an example:
  \begin{output}[fontsize=\footnotesize]
TS.BTD.Pivot atom+GPS
...
    BTD partitions (7):
     [ 2984, 2776, 192, 192, 1639, 4050, 105 ]
    BTD matrix block size [max] / [average]: 4050 /   1705.429
    BTD matrix elements in % of full matrix:   47.88707 %

TS.BTD.Pivot atom+GGPS
...
    BTD partitions (6):
     [ 2880, 2916, 174, 174, 2884, 2910 ]
    BTD matrix block size [max] / [average]: 2916 /   1989.667
    BTD matrix elements in % of full matrix:   48.62867 %

  \end{output}
  Although the GPS method uses the least amount of memory, the GGPS
  will likely perform better as the largest block in GPS is $4050$
  vs. $2916$ for the GGPS method.

\end{fdflogicalF}

\begin{fdflogicalF}{TS!Analyze.Graphviz}
  \fdfdepend{TS!Analyze}

  If performing the analysis, also create the connectivity graph and
  store it as \file{GRAPHVIZ\_atom.gv} or \file{GRAPHVIZ\_orbital.gv}
  to be post-processed in Graphviz\footnote{\url{www.graphviz.org}}.

\end{fdflogicalF}


\subsection{\texorpdfstring{$k$}{k}-point sampling}


The options for $k$-point sampling are identical to the \siesta\
options, \fdf{kgrid!MonkhorstPack}, \fdf{kgrid!Cutoff} or
\fdf{kgrid!File}.

One may however use specific \tsiesta\ $k$-points by using these
options:

\begin{fdfentry}{TS.kgrid!MonkhorstPack}[block]<\fdfvalue{kgrid!MonkhorstPack}>%

  See \fdf{kgrid!MonkhorstPack} for details.

\end{fdfentry}

\begin{fdfentry}{TS.kgrid!Cutoff}[length]<$0.\,\mathrm{Bohr}$>

  See \fdf{kgrid!Cutoff} for details.

\end{fdfentry}

\begin{fdfentry}{TS.kgrid!File}[string]<none>

  See \fdf{kgrid!File} for details.

\end{fdfentry}


\subsubsection{Algorithm specific options}

These options adhere to the specific solution methods available for
\tsiesta. For instance the \fdf*{TS.BTD.*} options adhere only when
using \fdf{TS!SolutionMethod:BTD}, similarly for options with
\fdf*{MUMPS}.

\begin{fdfentry}{TS!BTD!Pivot}[string]<\nonvalue{first electrode}>

  Decide on the partitioning for the BTD matrix. One may denote either
  \fdf*{atom+} or \fdf*{orb+} as a prefix which does the analysis on
  the atomic sparsity pattern or the full orbital sparsity pattern,
  respectively. If neither are used it will default to \fdf*{atom+}.

  Please see \fdf{TS!Analyze}.

  \begin{fdfoptions}

    \option[<elec-name>|CG-<elec-name>]%
    The partitioning will be a connectivity graph starting from the
    electrode denoted by the name. This name \emph{must} be found in
    the \fdf{TS!Elecs} block. One can append more than one electrode
    to simultaneously start from more than 1 electrode. This may be
    necessary for multi-terminal calculations.

    \option[rev-CM] %
    Use the reverse Cuthill-McKee for pivoting the matrix elements to
    reduce bandwidth. One may omit \fdf*{rev-} to use the standard
    Cuthill-McKee algorithm (not recommended).

    This pivoting scheme depends on the initial starting
    electrodes, append \fdf*{+<elec-name>} to start the Cuthill-McKee
    algorithm from the specified electrode(s).

    \option[GPS] %
    Use the Gibbs-Poole-Stockmeyer algorithm for reducing the
    bandwidth.

    \option[GGPS] %
    Use the generalized Gibbs-Poole-Stockmeyer algorithm for reducing
    the bandwidth.

    \note this algorithm does not work on dis-connected graphs.

    \option[PCG] %
    Use the perphiral connectivity graph algorithm for reducing the
    bandwidth.

    This pivoting scheme \emph{may} depend on the initial starting
    electrode(s), append \fdf*{+<elec-name>} to initialize the PCG
    algorithm from the specified electrode(s).

  \end{fdfoptions}

  Examples are
  \begin{fdfexample}
    TS.BTD.Pivot atom+GGPS
    TS.BTD.Pivot GGPS
    TS.BTD.Pivot orb+GGPS
    TS.BTD.Pivot orb+PCG+Left
  \end{fdfexample}
  where the first two are equivalent. The 3rd and 4th are more heavy
  on analysis and will typically not improve the bandwidth reduction.

\end{fdfentry}

\begin{fdfentry}{TS!BTD!Optimize}[string]<speed|memory>

  When selecting the smallest blocks for the BTD matrix there are
  certain criteria that may change the size of each block. For very
  memory consuming jobs one may choose the \fdf*{memory}.

  \note often both methods provide \emph{exactly} the same BTD matrix
  due to constraints on the matrix.

\end{fdfentry}

\begin{fdfentry}{TS!BTD!Guess1.Min}[int]<\nonvalue{empirically determined}>
  \fdfdepend{TS!BTD!Guess1.Max}

  Constructing the blocks for the BTD starts by \emph{guessing} the
  first block size. One could guess on all different block sizes, but
  to speed up the process one can define a smaller range of guesses by
  defining \fdf{TS!BTD!Guess1.Min} and \fdf{TS!BTD!Guess1.Max}.

  The initial guessed block size will be between the two values.

  By default this is $1/4$ of the minimum bandwidth for a selected
  first set of orbitals.

  \note setting this to 1 may sometimes improve the final BTD matrix
  blocks.

\end{fdfentry}

\begin{fdfentry}{TS!BTD!Guess1.Max}[int]<\nonvalue{empirically determined}>
  \fdfdepend{TS!BTD!Guess1.Min}

  See \fdf{TS!BTD!Guess1.Min}.

  \note for improved initialization performance setting Min/Max flags
  to the first block size for a given pivoting scheme will drastically
  reduce the search space and make initialization much
  faster.

\end{fdfentry}

\begin{fdfentry}{TS!BTD!Spectral}[string]<propagation|column>

  How to compute the spectral function ($G\Gamma G^\dagger$).

  For $\Nelec<4$ this defaults to \fdf*{propagation} which should be the
  fastest.

  For $\Nelec\ge4$ this defaults to \fdf*{column}.

  Check which has the best performance for your system if you endeavor
  on huge amounts of calculations for the same system.

\end{fdfentry}


\begin{fdfentry}{TS!MUMPS!Ordering}[string]<\nonvalue{read MUMPS
      manual}>

  One may select from a number of different matrix orderings which are
  all described in the MUMPS manual.

  The following list of orderings are available (without detailing
  their differences): %
  \fdf*{auto}, \fdf*{AMD}, \fdf*{AMF}, \fdf*{SCOTCH}, \fdf*{PORD},
  \fdf*{METIS}, \fdf*{QAMD}.

\end{fdfentry}

\begin{fdfentry}{TS!MUMPS!Memory}[integer]<20>

  Specify a factor for the memory consumption in MUMPS. See the
  \fdf*{INFOG(9)} entry in the MUMPS manual. Generally if \tsiesta\
  dies and \fdf*{INFOG(9)=-9} one should increase this number.

\end{fdfentry}

\begin{fdfentry}{TS!MUMPS!BlockingFactor}[integer]<112>

  Specify the number of internal block sizes. Larger numbers increases
  performance at the cost of memory.

  \note this option may heavily influence performance.

\end{fdfentry}

\subsubsection{Poisson solution for fixed boundary conditions}

\tsiesta\ requires fixed boundary conditions and forcing this is an
intricate and important detail.

It is important that these options are exactly the same if one reuses
the \sysfile{TSDE} files.

\begin{fdfentry}{TS!Poisson}[string]<ramp|elec-box|\nonvalue{file}>

  Define how the correction of the Poisson equation is
  superimposed. The default is to apply the linear correction across
  the entire cell (if there are two semi-infinite aligned
  electrodes). Otherwise this defaults to the \emph{box} solution
  which will introduce spurious effects at the electrode
  boundaries. In this case you are encouraged to supply a \fdf*{file}.

  If the input is a \fdf*{file}, it should be a NetCDF file containing
  the grid information which acts as the boundary conditions for the
  SCF cycle.
  The grid information should conform to the grid size of the
  unit-cell in the simulation.
  %
  \note the file option is only applicable if compiled with CDF4
  compliance.

  \begin{fdfoptions}
    \option[ramp]%
    \fdfindex*{TS!Poisson:ramp}%

    Apply the ramp for the full cell. This is the default for 2
    electrodes.

    \option[<file>]%
    \fdfindex*{TS!Poisson:<file>}%

    Specify an external file used as the boundary conditions for the
    applied bias. This is encouraged to use for $\Nelec>2$ electrode
    calculations but may also be used when an \emph{a priori}
    potential profile is know.

    The file should contain something similar to this output
    (\code{ncdump -h}):
    \begin{output}[fontsize=\footnotesize]
netcdf <file> {
dimensions:
	one = 1 ;
	a = 43 ;
	b = 451 ;
	c = 350 ;
variables:
	double Vmin(one) ;
		Vmin:unit = "Ry" ;
	double Vmax(one) ;
		Vmax:unit = "Ry" ;
	double V(c, b, a) ;
		V:unit = "Ry" ;
}
    \end{output}
    Note that the units should be in Ry. \code{Vmax}/\code{Vmin}
    should contain the maximum/minimum fixed boundary conditions in
    the Poisson solution. This is used internally by \tsiesta\ to
    scale the potential to arbitrary $V$. This enables the Poisson
    solution to only be solved \emph{once} independent on subsequent
    calculations. For chemical potential configurations where the
    Poisson solution is not linearly dependent one have to create
    separate files for each applied bias.


    \option[elec-box]%
    \fdfindex*{TS!Poisson:elec-box}%

    The default potential profile for $\Nelec>2$, or when the electrodes
    does are not aligned (in terms of their transport direction).

    \note usage of this Poisson solution is \emph{highly}
    discouraged. Please see \fdf{TS!Poisson:<file>}.

  \end{fdfoptions}

\end{fdfentry}

\begin{fdfentry}{TS!Hartree.Fix}[string]<[-+][ABC]>

  Specify which plane to fix the Hartree potential at. For regular (2
  electrode calculations with a single transport direction) this
  should not be set.
  %
  For $\Nelec\neq2$ electrode systems one \emph{have} to specify a
  plane to fix. One can specify one or several planes to fix. Users
  are encouraged to fix the plane where the entire plane has the
  highest/lowest potential.

\end{fdfentry}

\begin{fdfentry}{TS!Hartree.Fix!Frac}[real]<$1.$>

  Fraction of the correction that is applied.

  \note this is an experimental feature!

\end{fdfentry}

\begin{fdfentry}{TS!Hartree.Offset}[energy]<$0\,\mathrm{eV}$>

  An offset in the Hartree potential to match the electrode potential.

  This value may be useful in certain cases where the Hartree
  potentials are very different between the electrode and device
  region calculations.

  This should not be changed between different bias calculations. It
  directly relates to the reference energy level ($E_F$).

\end{fdfentry}

\subsubsection{Electrode description options}

As \tsiesta\ supports $\Nelec$ electrodes one needs to specify all
electrodes in a generic input format.

Note that electrodes should be \emph{metallic} so that the Fermi-level
is well defined. Please see Sec.~\ref{sec:transiesta:description} for
more details.

\begin{fdfentry}{TS!Elecs}[block]

  Each line denote an electrode which is queried in \fdf{TS!Elec.<>}
  for its setup.

\end{fdfentry}

\begin{fdfentry}{TS!Elec.<>}[block]

  Each line represents a setting for electrode \fdf*{<>}.
  There are a few lines that \emph{must} be present, \fdf*{HS},
  \fdf*{semi-inf-dir}, \fdf*{electrode-pos}, \fdf*{chem-pot}. The
  remaining options are optional.

  \note Options prefixed with \fdf*{tbt} are neglected in \tsiesta\
  calculations. In \tbtrans\ calculations these flags has precedence
  over the other options and \emph{must} be placed at the end of the
  block.

  \begin{fdfoptions}

    \option[HS]%
    \fdfindex*{TS!Elec.<>!HS}%
    The Hamiltonian information from the initial electrode
    calculation. This file retains the geometrical information as well
    as the Hamiltonian, overlap matrix and the Fermi-level of the
    electrode.
    %
    This is a file-path and the Hamiltonian file need not be in the same
    directory, i.e. the path can be relative.

    \note \sysfile{HSX}, \sysfile{TSHS} and \sysfile{nc} files are supported.
    See \fdf{Save!HS}, \fdf{TS.HS!Save} or \fdf{CDF!Save} are the relevant flags
    for this.

    \note Please note that \tsiesta\ expects a metallic electrode.
    Results can not be trusted for semi-conductors.

    \option[semi-inf-direction|semi-inf-dir|semi-inf]%
    \fdfindex*{TS!Elec.<>!semi-inf-direction}%
    The semi-infinite direction of the electrode with respect to the
    electrode unit-cell.

    It may be one of \fdf*{[-+][abc]}, \fdf*{[-+]A[123]}, \fdf*{ab},
    \fdf*{ac}, \fdf*{bc} or \fdf*{abc}. The latter four all refer to a
    real-space self-energy as described in \cite{Papior2019}.

    \note this direction is \emph{not} with respect to the scattering
    region unit cell. It is with respect to the electrode unit
    cell. \tsiesta\ will figure out the alignment of the electrode
    unit cell and the scattering region unit-cell.

    \option[chemical-potential|chem-pot|mu]%
    \fdfindex*{TS!Elec.<>!chemical-potential}%
    The chemical potential that is associated with this
    electrode. This is a string that should be present in the
    \fdf{TS!ChemPots} block.

    \option[electrode-position|elec-pos]%
    \fdfindex*{TS!Elec.<>!electrode-position}%
    The index of the electrode in the scattering region.
    This may be given by either \fdf*{elec-pos <idx>}, which refers to
    the first atomic index of the electrode residing at index
    \fdf*{<idx>}. Else the electrode position may be given via
    \fdf*{elec-pos end <idx>} where the last index of the electrode
    will be located at \fdf*{<idx>}.

    \option[used-atoms]%
    \fdfindex*{TS!Elec.<>!used-atoms}%
    \fdfdepend{TS!Elec.<>!semi-inf-direction}%
    Number of atoms from the electrode calculation that is used in the
    scattering region as electrode. This may be useful when the
    periodicity of the electrodes forces extensive electrodes in the
    semi-infinite direction.

    If the semi-infinite direction is \emph{positive}, the first atoms will
    be retained.
    Contrary, if the semi-infinite direction is \emph{negative}, the last
    atoms will be retained.

    \note do not set this if you use all atoms in the electrode.

    \option[Bulk]%
    \fdfindex*{TS!Elec.<>!Bulk}%
    \fdfindex*{TS!Elec.<>!Bulk:true}%
    \fdfindex*{TS!Elec.<>!Bulk:false}%
    Control whether the Hamiltonian of the electrode region in the
    scattering region is enforced \emph{bulk} or whether the
    Hamiltonian is taken from the scattering region elements.

    This defaults to \fdftrue. If there are buffer atoms \emph{behind}
    the electrode it may be advantageous to set this to false to
    extend the electrode region, otherwise it is recommended to keep
    the default.

    This option changes how $\mathbf M_{\elec,\elec}$, see Eq.~\eqref{eq:negf:green-elec}, is setup.

    For \fdftrue\ $\big\{\Ham_\elec, \SO_\elec\big\}$ are taken from the
    electrode file (\fdf{TS!Elec.<>!HS}).

    For \fdffalse\ $\big\{\Ham_\elec, \SO_\elec\big\}$ are substituted by
    the device calculations electrode region.
    I.e. it is the self-consistent Hamiltonian.

    \option[DM-update]%
    \fdfindex*{TS!Elec.<>!DM-update}%
    \fdfindex*{TS!Elec.<>!DM-update:none}%
    \fdfindex*{TS!Elec.<>!DM-update:all}%
    \fdfdepend{TS!Elec.<>!Bulk}%
    String of values \fdf*{none}, \fdf*{cross-terms} or \fdf*{all}
    which controls which part of the electrode density matrix elements
    that are updated.
    %
    The density matrices that comprises an electrode and
    device-electrode region can be written as (omitting the central
    device region)
    \begin{equation}
      \label{eq:ts-dm-update}
      \DM =
      \begin{bmatrix}
        \DM_{\mathfrak e} & \DM_{\mathfrak eD} & 0
        \\
        \DM_{D\mathfrak e} & \ddots & \ddots
        \\
        0  & \ddots
      \end{bmatrix}
    \end{equation}
    This flag determines whether $\DM_{\mathfrak e}$ (\fdf*{all}) or
    $\DM_{\mathfrak eD}$ (\fdf*{cross-terms} and \fdf*{all}) or
    neither (\fdf*{none}) are updated in the SCF. The density matrices
    contains the charges and thus affects the Hamiltonian and Poisson
    solutions. Generally the default value will suffice and is
    recommended.

    If \fdf{TS!Elec.<>!Bulk:false} this is forced to \fdf*{all} and
    cannot be changed.

    If \fdf{TS!Elec.<>!Bulk:true} this defaults to \fdf*{cross-terms},
    but may be changed.

    \note if this is \fdf*{none} the forces on the atoms coupled to
    the electrode regions are \emph{not} to be trusted. The value
    \fdf*{none} should be avoided, if possible.

    \option[DM-init]%
    \fdfindex*{TS!Elec.<>!DM-init}%
    \fdfindex*{TS!Elec.<>!DM-init:diagon}%
    \fdfindex*{TS!Elec.<>!DM-init:bulk}%
    \fdfdepend{TS!Elecs!DM.Init,TS!Elec.<>!Bulk,TS!Voltage}%
    String of values \fdf*{bulk}, \fdf*{diagon} (default) or
    \fdf*{force-bulk} which controls whether the DM is initially
    overwritten by the DM from the bulk electrode calculation. This
    requires the DM file for the electrode to be present. Only
    \fdf*{force-bulk} will have effect if $V\neq0$. Otherwise this
    option only affects $V=0$ calculations.

    The density matrix elements in the electrodes of the scattering
    region may be forcefully set to the bulk values by reading in the
    DM of the corresponding electrode. If one uses
    \fdf{TS!Elec.<>!Bulk:false} it may be dis-advantageous to set this
    to \fdf*{bulk}.
    If the system is well setup (good screening towards electrodes),
    setting this to \fdf*{bulk} may be advantageous.

    This option may be used to check how good the electrodes are
    screened, see \fdf{TS!dQ:fermi}.

    \option[out-of-core]%
    \fdfindex*{TS!Elec.<>!Out-of-core}%
    \fdfdepend{TS!Elec.<>!Gf}%
    If \fdftrue\ (default) the GF files are created which contain
    the surface Green function.
    If \fdffalse\ the surface Green function will be calculated when
    needed.
    Setting this to \fdffalse\ will heavily degrade performance and
    it is highly discouraged!

    \option[Gf]%
    \fdfindex*{TS!Elec.<>!Gf}%
    String with filename of the surface Green function data
    (\sysfile{TSGF*}). This may be used to place a common surface
    Green function file in a top directory which may then be used in
    all calculations using the same electrode and the same contour.
    %
    If doing many calculations with the same electrode and $\mathbf k$, $E$ grids,
    then this can greatly improve throughput. It has a minor cost of disk-space.
    Note that the energy-grids are dependent on the applied bias.

    \option[Gf-Reuse]%
    \fdfindex*{TS!Elec.<>!Gf-Reuse}%
    \fdfdepend{TS!Elec.<>!Out-of-core,TS!Elec.<>!Gf}%
    Logical deciding whether the surface Green function file should be
    re-used or deleted.
    %
    If this is \fdffalse\ the surface Green function file is deleted
    and re-created upon start.

    \option[pre-expand]%
    \fdfindex*{TS!Elec.<>!pre-expand}%
    \fdfdepend{TS!Elec.<>!out-of-core}%
    String denoting how the expansion of the surface Green function
    file will be performed. This only affects the Green function file
    if \fdf*{Bloch} is larger than 1. By default the Green function
    file will contain the fully expanded surface Green function, but
    not Hamiltonian and overlap matrices (\fdf*{Green}). One may
    reduce the file size by setting this to \fdf*{Green} which only
    expands the surface Green function. Finally \fdf*{none} may be
    passed to reduce the file size to the bare minimum.
    %
    For performance reasons \fdf*{all} is preferred.

    If disk-space is a limited resource and the \sysfile{TSGF*} files
    are really big, try \fdf*{none}.

    \option[Eta]%
    \fdfindex*{TS!Elec.<>!Eta}%
    \fdfdepend{TS!Elecs!Eta}%
    Control the imaginary energy ($\eta$) of the surface Green
    function for this electrode.

    The imaginary part is \emph{only} used in the non-equilibrium
    contours since the equilibrium are already lifted into the complex
    plane. Thus this $\eta$ reflects the imaginary part in the
    $G\Gamma G^\dagger$ calculations. Ensure that all imaginary values
    are larger than $0$ as otherwise \tsiesta\ may seg-fault.

    \note if this energy is negative the complex value associated with
    the non-equilibrium contour is used. This is particularly useful
    when providing a user-defined contour along the real axis.

    See Sec.~\ref{sec:negf-equations} for details.
    This options changes the $\eta$ value in the calculated self-energy
    ($\SE(E+i\eta)$), while it does not change the $\eta$ value used
    in the device region.

    \option[DE]%
    \fdfindex*{TS!Elec.<>!DE}%
    \fdfdepend{TS!Elec.<>!DM-init}%
    Density and energy density matrix file for the electrode. This may
    be used to initialize the density matrix elements in the electrode
    region by the bulk values. See \fdf{TS!Elec.<>!DM-init:bulk}.

    \note this should only be performed on one \tsiesta\ calculation
    as then the scattering region \sysfile{TSDE} contains the
    electrode density matrix.

    \option[Bloch]%
    \fdfindex*{TS!Elec.<>!Bloch}%
    $3$ integers should be present on this line which each denote the
    number of times bigger the scattering region electrode is compared
    to the electrode, in each lattice direction. Remark that these
    expansion coefficients are with regard to the electrode unit-cell.
    This is denoted ``Bloch'' because it is an expansion based on
    Bloch waves.

    \note Using symmetries such as periodicity will greatly increase
    performance.

    \option[Bloch-A/a1|B/a2|C/a3]%
    \fdfindex{TS!Elec.<>!Bloch}%
    Specific Bloch expansions in each of the electrode unit-cell
    direction. See \fdf*{Bloch} for details.

    \option[Accuracy]%
    \fdfindex*{TS!Elec.<>!Accuracy}%
    \fdfdepend{TS!Elecs!Accuracy}%
    Control the convergence accuracy required for the self-energy
    calculation when using the Lopez-Sancho, Lopez-Sancho iterative
    scheme.

    \note advanced use \emph{only}.

    \option[delta-Ef]%
    \fdfindex*{TS!Elec.<>!delta-Ef}%
    Specify an offset for the Fermi-level of the electrode. This will
    directly be added to the Fermi-level found in the electrode file.

    Effectively this will transform the used chemical potential to
    \begin{equation}
      \mu'_{\mathrm{used}} = \mu_{\mathrm{used}} + \delta E_F.
    \end{equation}

    \note this option only makes sense for semi-conducting electrodes
    since it shifts the entire electronic structure. This is because
    the Fermi-level may be arbitrarily placed anywhere in the band
    gap. It is the users responsibility to define a value which does
    not introduce a potential drop between the electrode and device
    region. Please do not use unless you really know what you are
    doing.

    \option[V-fraction]%
    \fdfindex*{TS!Elec.<>!V-fraction}%

    Specify the fraction of the chemical potential shift in the
    electrode-device coupling region. This corresponds to altering Eq.~\eqref{eq:negf:green-elec} by:
    \begin{equation}
      \Ham_{\elec,D} \leftarrow \Ham_{\elec,D} +
      \mu_{\elec} \mathrm{V-fraction} \SO_{\elec,D}
    \end{equation}
    in the coupling region. Consequently the value \emph{must} be
    between $0$ and $1$.

    \note this option \emph{only} makes sense for
    \fdf{TS!Elec.<>!DM-update:none} since otherwise the electrostatic
    potential will be incorporated in the Hamiltonian.

    Only expert users should play with this number.

    \option[check-kgrid]%
    \fdfindex*{TS!Elec.<>!check-kgrid}%
    For $\Nelec$ electrode calculations the $\mathbf k$ mesh will sometimes
    not be equivalent for the electrodes and the device region
    calculations. However, \tsiesta\ requires that the device and
    electrode $\mathbf k$ samplings are commensurate. This flag
    controls whether this check is enforced for a given electrode.

    \note only use if fully aware of the implications!

  \end{fdfoptions}

\end{fdfentry}

There are several flags which are globally controlling the variables
for the electrodes (with \fdf{TS!Elec.<>} taking precedence).

\begin{fdflogicalT}{TS!Elecs!Bulk}

  This globally controls how the Hamiltonian is treated in all
  electrodes.
  %
  See \fdf{TS!Elec.<>!Bulk}.

\end{fdflogicalT}

\begin{fdfentry}{TS!Elecs!Eta}[energy]<$1\,\mathrm{meV}$>

  Globally control the imaginary energy ($\eta$) used for the surface
  Green function calculation on the non-equilibrium contour.
  %
  See \fdf{TS!Elec.<>!Eta} for extended details on the usage of this
  flag.

\end{fdfentry}

\begin{fdfentry}{TS!Elecs!Accuracy}[energy]<$10^{-13}\,\mathrm{eV}$>

  Globally control the accuracy required for convergence of the self-energy.
  %
  See \fdf{TS!Elec.<>!Accuracy}.

\end{fdfentry}

\begin{fdflogicalF}{TS!Elecs!Neglect.Principal}

  If this is \fdffalse\ \tsiesta\ dies if there are connections beyond
  the principal cell.

  \note set this to \fdftrue\ with care, non-physical results may
  arise. Use at your own risk!

\end{fdflogicalF}

\begin{fdflogicalT}{TS!Elecs!Gf.Reuse}

  Globally control whether the surface Green function files should
  be re-used (\fdftrue) or re-created (\fdffalse).

  See \fdf{TS!Elec.<>!Gf-Reuse}.

\end{fdflogicalT}

\begin{fdflogicalT}{TS!Elecs!Out-of-core}

  Whether the electrodes will calculate the self energy at each SCF
  step. Using this will not require the surface Green function files
  but at the cost of heavily degraded performance.

  See \fdf{TS!Elec.<>!Out-of-core}.

\end{fdflogicalT}

\begin{fdfentry}{TS!Elecs!DM.Update}[string]<cross-terms|all|none>

  Globally controls which parts of the electrode density matrix
  gets updated.

  See \fdf{TS!Elec.<>!DM-update}.

\end{fdfentry}

\begin{fdfentry}{TS!Elecs!DM.Init}[string]<diagon|bulk|force-bulk>
  \fdfindex*{TS!Elecs!DM.Init:bulk}%
  \fdfindex*{TS!Elecs!DM.Init:diagon}%

  Specify how the density matrix elements in the electrode regions of
  the scattering region will be initialized when starting \tsiesta.

  See \fdf{TS!Elec.<>!DM-init}.

\end{fdfentry}

\begin{fdfentry}{TS!Elecs!Coord.EPS}[length]<$0.001\,\mathrm{Ang}$>

  When using Bloch expansion of the self-energies one may experience
  difficulties in obtaining perfectly aligned electrode coordinates.

  This parameter controls how strict the criteria for equivalent
  atomic coordinates is. If \tsiesta\ crashes due to mismatch between
  the electrode atomic coordinates and the scattering region
  calculation, one may increase this criteria. This should only be
  done if one is sure that the atomic coordinates are almost similar
  and that the difference in electronic structures of the two may be
  negligible.

\end{fdfentry}


\subsubsection{Chemical potentials}
\label{sec:ts:chem-pot}

For $\Nelec$ electrodes there will also be $N_\mu$ chemical
potentials. They are defined via blocks similar to \fdf{TS!Elecs}.

\begin{fdfentry}{TS!ChemPots}[block]

  Each line denotes a new chemical potential which is defined in the
  \fdf{TS!ChemPot.<>} block.

\end{fdfentry}

\begin{fdfentry}{TS!ChemPot.<>}[block]

  Each line defines a setting for the chemical potential named
  \fdf*{<>}.

  \begin{fdfoptions}

    \option[chemical-shift|mu]%
    \fdfindex*{TS!ChemPot.<>!chemical-shift}%
    \fdfindex*{TS!ChemPot.<>!mu}%

    Define the chemical shift (an energy) for this chemical
    potential. One may specify the shift in terms of the applied bias
    using \fdf*{V/<integer>} instead of explicitly typing the energy.

    \option[contour.eq]%
    \fdfindex*{TS!ChemPot.<>!contour.eq}%
    A subblock which defines the integration curves for the
    equilibrium contour for this equilibrium chemical potential. One
    may supply as many different contours to create whatever shape of
    the contour

    Its format is
    \begin{fdfexample}
      contour.eq
       begin
        <contour-name-1>
        <contour-name-2>
        ...
       end
    \end{fdfexample}

    \note If you do \emph{not} specify \fdf*{contour.eq} in the block
    one will automatically use the continued fraction method and you
    are encouraged to use $50$ or more poles\cite{Ozaki2010}.

    \option[ElectronicTemperature|Temp|kT]%
    \fdfindex*{TS!ChemPot.<>!ElectronicTemperature}%
    \fdfindex*{TS!ChemPot.<>!Temp}%
    \fdfindex*{TS!ChemPot.<>!kT}%

    Specify the electronic temperature (as an energy or in
    Kelvin). This defaults to \fdf{TS!ElectronicTemperature}.

    One may specify this in units of \fdf{TS!ElectronicTemperature} by
    using the unit \fdf*{kT}.

    \option[contour.eq.pole]%
    \fdfindex*{TS!ChemPot.<>!contour.eq.pole}%

    Define the number of poles used via an energy
    specification. \tsiesta\ will automatically convert the energy to
    the closest number of poles (rounding up).

    \note this has precedence over
    \fdf{TS!ChemPot.<>!contour.eq.pole.N} if it is specified
    \emph{and} a positive energy. Set this to a negative energy to
    directly control the number of poles.

    \option[contour.eq.pole.N]%
    \fdfindex*{TS!ChemPot.<>!contour.eq.pole.N}%

    Define the number of poles via an integer.

    \note this will only take effect if
    \fdf{TS!ChemPot.<>!contour.eq.pole} is a negative energy.

  \end{fdfoptions}

  \note It is important to realize that the parametrization in 4.1 of
  the voltage into the chemical potentials enables one to have a
  \emph{single} input file which is never required to be changed, even
  when changing the applied bias (if using the command line options
  for specifying the applied bias).
  %
  This is different from 4.0 and prior versions since one had to
  manually change the \fdf*{TS.biasContour.NumPoints} for each applied
  bias.

\end{fdfentry}

These options complicate the input sequence for regular $2$ electrode
which is unfortunate.

Using \program{tselecs.sh -only-mu} yields this output:
\begin{fdfexample}
  %block TS.ChemPots
    Left
    Right
  %endblock
  %block TS.ChemPot.Left
    mu V/2
    contour.eq
      begin
        C-Left
        T-Left
      end
  %endblock
  %block TS.ChemPot.Right
    mu -V/2
    contour.eq
      begin
        C-Right
        T-Right
      end
  %endblock
\end{fdfexample}

Note that the default is a $2$ electrode setup with chemical
potentials associated directly with the electrode names
``Left''/``Right''. Each chemical potential has two parts of the
equilibrium contour named according to their name.



\subsubsection{Complex contour integration options}

Specifying the contour for $\Nelec$ electrode systems is a bit
extensive due to the possibility of more than 2 chemical
potentials. Please use the \program{Util/TS/tselecs.sh} as a means to
create default input blocks.

The contours are split in two segments. One, being the equilibrium
contour of each of the different chemical potentials. The second for
the non-equilibrium contour. The equilibrium contours are shifted
according to their chemical potentials with respect to a reference
energy. Note that for \tsiesta\ the reference energy is named the
Fermi-level, which is rather unfortunate (for non-equilibrium but not
equilibrium). Fortunately the non-equilibrium contours are defined
from different chemical potentials Fermi functions, and as such this
contour is defined in the window of the minimum and maximum chemical
potentials. Because the reference energy is the periodic Fermi level
it is advised to retain the average chemical potentials equal to
$0$. Otherwise applying different bias will shift transmission curves
calculated via \tbtrans\ relative to the average chemical potential.

In this section the equilibrium contours are defined, and in the next
section the non-equilibrium contours are defined.

\begin{fdfentry}{TS!Contours!Eq.Pole}[energy]<$1.5\,\mathrm{eV}$>

  The imaginary part of the line integral crossing the chemical
  potential. Note that the actual number of poles may differ between
  different calculations where the electronic temperatures are
  different.

  \note if the energy specified is negative,
  \fdf{TS!Contours!Eq.Pole.N} takes effect.

\end{fdfentry}

\begin{fdfentry}{TS!Contours!Eq.Pole.N}[integer]<8>

  Manually select the number poles for the equilibrium contour.

  \note this flag will only take effect if \fdf{TS!Contours!Eq.Pole}
  is a negative energy.

\end{fdfentry}

\begin{fdfentry}{TS!Contour.<>}[block]

  Specify a contour named \fdf*{<>} with options within the block.

  The names \fdf*{<>} are taken from the
  \fdf{TS!ChemPot.<>!contour.eq} block in the chemical potentials.

  The format of this block is made up of at least $4$ lines, in the
  following order of appearance.

  \begin{fdfoptions}

    \option[part]%
    \fdfindex*{TS!Contour.<>!part}%

    Specify which part of the equilibrium contour this is:
    \begin{fdfoptions}

      \option[circle]%
      The initial circular part of the contour

      \option[square]%
      The initial square part of the contour

      \option[line]%
      The straight line of the contour

      \option[tail]%
      The final part of the contour \emph{must} be a tail which
      denotes the Fermi function tail.

    \end{fdfoptions}

    \option[from \emph{a} to \emph{b}]%
    \fdfindex*{TS!Contour.<>!from}%

    Define the integration range on the energy axis.
    Thus \emph{a} and \emph{b} are energies.

    The parameters may also be given values \fdf*{prev}/\fdf*{next}
    which is the equivalent of specifying the same energy as the
    previous contour it is connected to.

    \note that \emph{b} may be supplied as \fdf*{inf} for \fdf*{tail}
    parts.

    \option[points/delta]%
    \fdfindex*{TS!Contour.<>!points}%
    \fdfindex*{TS!Contour.<>!delta}%

    Define the number of integration points/energy separation.
    If specifying the number of points an integer should be supplied.

    If specifying the separation between consecutive points an energy
    should be supplied.

    \option[method]%
    \fdfindex*{TS!Contour.<>!method}%

    Specify the numerical method used to conduct the integration. Here
    a number of different numerical integration schemes are accessible

    \begin{fdfoptions}
      \option[mid|mid-rule]%
      Use the mid-rule for integration.

      \option[simpson|simpson-mix]%
      Use the composite Simpson $3/8$ rule (three point Newton-Cotes).

      \option[boole|boole-mix]%
      Use the composite Booles rule (five point Newton-Cotes).

      \option[G-legendre]%
      Gauss-Legendre quadrature.

      \note has \fdf*{opt left}

      \note has \fdf*{opt right}

      \option[tanh-sinh]%
      Tanh-Sinh quadrature.

      \note has \fdf*{opt precision <>}

      \note has \fdf*{opt left}

      \note has \fdf*{opt right}

      \option[G-Fermi]%
      Gauss-Fermi quadrature (only on tails).

    \end{fdfoptions}

    \option[opt]%
    \fdfindex*{TS!Contour.<>!opt}%

    Specify additional options for the \fdf*{method}. Only a selected
    subset of the methods have additional options.

  \end{fdfoptions}

\end{fdfentry}

These options complicate the input sequence for regular $2$ electrode
which is unfortunate. However, it allows highly customizable contours.

Using \program{tselecs.sh -only-c} yields this output:
\begin{fdfexample}
  TS.Contours.Eq.Pole 2.5 eV
  %block TS.Contour.C-Left
    part circle
     from -40. eV + V/2 to -10 kT + V/2
       points 25
        method g-legendre
         opt right
  %endblock
  %block TS.Contour.T-Left
    part tail
     from prev to inf
       points 10
        method g-fermi
  %endblock
  %block TS.Contour.C-Right
    part circle
     from -40. eV -V/2 to -10 kT -V/2
       points 25
        method g-legendre
         opt right
  %endblock
  %block TS.Contour.T-Right
    part tail
     from prev to inf
       points 10
        method g-fermi
  %endblock
\end{fdfexample}
These contour options refer to input options for the chemical
potentials as shown in Sec.~\ref{sec:ts:chem-pot}
(p.~\pageref{sec:ts:chem-pot}). Importantly one should note the shift
of the contours corresponding to the chemical potential (the shift
corresponds to difference from the reference energy used in \tsiesta).


\subsubsection{Bias contour integration options}

The bias contour is similarly defined as the equilibrium
contours. Please use the \program{Util/TS/tselecs.sh} as a means to
create default input blocks.

\begin{fdfentry}{TS!Contours.nEq!Eta}[energy]<$\operatorname{min}[\eta_{\mathfrak e}]/10$>
  \fdfdepend{TS!Elecs!Eta}%

  The imaginary part ($\eta$) of the device states. While this may be
  set to $0$ for most systems it defaults to the minimum $\eta$ value
  for the electrodes ($\operatorname{min}[\eta_{\mathfrak
      e}]/10$). This ensures that the device broadening is always
  smaller than the electrodes while allowing broadening of localized
  states.

\end{fdfentry}

\begin{fdfentry}{TS!Contours.nEq!Fermi.Cutoff}[energy]<$5\,k_BT$>

  The bias contour is limited by the Fermi function tails. Numerically
  it does not make sense to integrate to infinity.
  %
  This energy defines where the bias integration window is turned into
  zero. Thus above $-|V|/2-E$ or below $|V|/2+E$ the DOS is defined as
  exactly zero.

\end{fdfentry}

\begin{fdfentry}{TS!Contours.nEq}[block]

  Each line defines a new contour on the non-equilibrium bias
  window. The contours defined \emph{must} be defined in
  \fdf{TS!Contour.nEq.<>}.

  These contours must all be \fdf*{part line} or \fdf*{part tail}.

\end{fdfentry}

\begin{fdfentry}{TS!Contour.nEq.<>}[block]

  This block is \emph{exactly} equivalently defined as the
  \fdf{TS!Contour.<>}. See page \pageref{TS!Contour.<>}.

\end{fdfentry}

The default options related to the non-equilibrium bias contour are
defined as this:
\begin{fdfexample}
  %block TS.Contours.nEq
    neq
  %endblock TS.Contours.nEq
  %block TS.Contour.nEq.neq
    part line
     from -|V|/2 - 5 kT to |V|/2 + 5 kT
       delta 0.01 eV
        method mid-rule
  %endblock TS.Contour.nEq.neq
\end{fdfexample}
If one chooses a different reference energy than $0$, then the limits
should change accordingly. Note that here \fdf*{kT} refers to
\fdf{TS!ElectronicTemperature}.


\subsection{Output}

\subsection{Standard output} \index{output!main output file}

\siesta\ writes a log of its workings to standard output (unit 6),
which is usually redirected to an ``output file''.

A brief description follows. See the example cases in the
siesta/Tests directory for illustration.

The program starts writing the version of the code which is
used. Then, the input \fdflib\ file is dumped into the output file as is
(except for empty lines). The program does part of the reading and
digesting of the data at the beginning within the \program{redata}
subroutine. It prints some of the information it digests. It is
important to note that it is only part of it, some other information
being accessed by the different subroutines when they need it during
the run (in the spirit of \fdflib\ input).  A complete list of the input
used by the code can be found at the end in the file \file{fdf.log},
including defaults used by the code in the run.

After that, the program reads the pseudopotentials, factorizes them
into Kleinman-Bylander form, and generates (or reads) the atomic basis
set to be used in the simulation. These stages are documented in the
output file.

The simulation begins after that, the output showing information of
the MD (or CG) steps and the SCF cycles within.  Basic descriptions of
the process and results are presented. The user has the option to
customize it, however,\index{output!customization} by defining
different options that control the printing of informations like
coordinates, forces, $\vec k$ points, etc.  The options are discussed
in the appropriate sections, but take into account the behavior of the
legacy \fdf{LongOutput} option, as in the current implementation might
silently activate output to the main .out file at the expense of
auxiliary files.

\begin{fdflogicalF}{LongOutput}
  \index{output!long}

  \siesta\ can write to standard output different data sets depending
  on the values for output options described below.  By default
  \siesta\ will not write most of them. They can be large for large
  systems (coordinates, eigenvalues, forces, etc.)  and, if written to
  standard output, they accumulate for all the steps of the
  dynamics. \siesta\ writes the information in other files (see Output
  Files) in addition to the standard output, and these can be
  cumulative or not.

  Setting \fdf{LongOutput} to \fdftrue\ changes the default of some
  options, obtaining more information in the output (verbose).  In
  particular, it redefines the defaults for the following:

  \begin{itemize}

    \item \fdf{WriteKpoints}%
    \index{output!grid $\vec k$ points}

    \item \fdf{WriteKbands}%
    \index{output!band $\vec k$ points}

    \item \fdf{WriteCoorStep}%
    \index{output!atomic coordinates!in a dynamics step}

    \item \fdf{WriteForces}%
    \index{output!forces}

    \item \fdf{WriteEigenvalues}%
    \index{output!eigenvalues}

    \item \fdf{WriteWaveFunctions}%
    \index{output!wave functions}

    \item \fdf{WriteMullikenPop}%
    \index{output!Mulliken analysis}%
    \index{Mulliken population analysis}%
    (it sets it to 1)

  \end{itemize}

  The specific changing of any of these options has precedence.

\end{fdflogicalF}


\subsection{Output to dedicated files}%
\index{output!dedicated files}

\siesta\ can produce a wealth of information in dedicated files,
with specific formats, that can be used for further analysis. See the
appropriate sections, and the appendix on file formats.
Please take into account the behavior of
\fdf{LongOutput}, as in the current implementation might silently
activate output to the main .out file at the expense of auxiliary
files.


