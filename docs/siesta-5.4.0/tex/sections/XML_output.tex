From version 2.0, \siesta\ includes an option to write its output to an
XML file. The XML it produces is in accordance with the CMLComp subset of
version 2.2 of the Chemical Markup Language. Further information
and resources can be found at \url{http://cmlcomp.org/} and tools for working
with the XML file can be found in the \texttt{Util/CMLComp} directory.

The main motivation for standarised XML (CML) output is as a step
towards standarising formats for uses like the following.

\begin{itemize}

\item To have \siesta\ communicating with other software, either
for postprocessing or as part of a larger workflow scheme. In such a
scenario, the XML output of one \siesta\ simulation may be easily parsed
in order to direct further simulations. Detailed discussion of this is
out of the scope of this manual.

\item To generate webpages showing \siesta\ output in a more accessible,
graphically rich, fashion. This section will explain how to do this.

\end{itemize}

\subsection{Controlling XML output}

\begin{fdflogicalF}{XML!Write}

  Determine if the main XML file should be created for this run.

\end{fdflogicalF}

\subsection{Converting XML to XHTML}

The translation of the \siesta\ XML output to a HTML-based webpage is
done using XSLT technology. The stylesheets conform to XSLT-1.0 plus
EXSLT extensions; an xslt processor capable of dealing with this is
necessary. However, in order to make the system easy to use, a script
called ccViz is provided in \texttt{Util/CMLComp} that works on most Unix or
Mac OS X systems. It is run like so:

\texttt{./ccViz SystemLabel.xml}

A new file will be produced. Point your web-browser at \texttt{SystemLabel.xhtml}
to view the output.

The generated webpages include support for viewing three-dimensional
interactive images of the system. If you want to do this, you will
either need jMol (\url{http://jmol.sourceforge.net}) installed or access
to the internet. As this
is a Java applet, you will also need a working Java Runtime
Environment and browser plugin - installation instructions for these
are outside the scope of this manual, though. However, the webpages
are still useful and may be viewed without this plugin.

An online version of this tool is avalable from
\url{http://cmlcomp.org/ccViz/}, as are updated versions of
the ccViz script.