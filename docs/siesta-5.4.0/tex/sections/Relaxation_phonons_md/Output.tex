Every time the atoms move, either during coordinate relaxation or
molecular dynamics, their positions \textbf{predicted for next step}
and \textbf{current} velocities are stored in file \sysfile{XV}. The
shape of the unit cell and its associated 'velocity' (in
Parrinello-Rahman dynamics) are also stored in this file.

\begin{fdflogicalT}{WriteCoorInitial}
  \index{output!atomic coordinates!initial}

  It determines whether the initial atomic coordinates of the
  simulation are dumped into the main output file. These coordinates
  correspond to the ones actually used in the first step (see the
  section on precedence issues in structural input) and are output in
  Cartesian coordinates in Bohr units.

  It is not affected by the setting of \fdf{LongOutput}.

\end{fdflogicalT}


\begin{fdflogicalF}{WriteCoorStep}
  \index{output!atomic coordinates!in a dynamics step}

  If \fdftrue, it writes the atomic coordinates to standard
  output at every MD time step or relaxation step. The coordinates are
  always written in the \sysfile{XV} file, but overriden at
  every step. They can be also accumulated in the \sysfile*{MD}
  or \sysfile{MDX} files depending on \fdf{WriteMDHistory}.

\end{fdflogicalF}

\begin{fdflogicalF}{WriteForces}
  \index{output!forces}

  If \fdftrue, it writes the atomic forces to the output file at every
  MD time step or relaxation step.  Note that the forces of the last
  step can be found in the file \sysfile{FA}. If constraints are used,
  the file \sysfile{FAC} is also written.

\end{fdflogicalF}

\begin{fdflogicalF}{WriteMDHistory}
  \index{output!molecular dynamics!history}

  If \fdftrue, \siesta\ accumulates the molecular dynamics trajectory
  in the following files:
  \begin{itemize}
    \item%
    \sysfile{MD} : atomic coordinates and velocities (and lattice
    vectors and their time derivatives, if the dynamics implies
    variable cell). The information is stored unformatted for
    postprocessing with utility programs to analyze the MD trajectory.

    \item%
    \sysfile{MDE} : shorter description of the run, with energy,
    temperature, etc., per time step.

  \end{itemize}
  These files are accumulative even for different runs.

  \index{output!molecular dynamics!history}

  The trajectory of a molecular dynamics run (or a conjugate gradient
  minimization) can be accumulated in different files: SystemLabel.MD,
  SystemLabel.MDE, and SystemLabel.ANI. The first file keeps the whole
  trajectory information, meaning positions and velocities at every time
  step, including lattice vectors if the cell varies. NOTE that the
  positions (and maybe the cell vectors) stored at each time step are
  the \textbf{predicted} values for the next step. Care should be taken if
  joint position-velocity correlations need to be computed from this
  file.  The second gives global information (energy, temperature, etc),
  and the third has the coordinates in a form suited for XMol animation.
  See the \fdf{WriteMDHistory} and \fdf{WriteMDXmol} data descriptors
  above for information. \siesta\ always appends new information on
  these files, making them accumulative even for different runs.

  The \program{iomd} subroutine can generate both an unformatted file
  \sysfile*{MD} (default) or ASCII formatted files \sysfile*{MDX} and
  \sysfile*{MDC} containing the atomic and lattice trajectories,
  respectively. Edit the file to change the settings if desired.

\end{fdflogicalF}


\begin{fdflogicalT}{Write.OrbitalIndex}

  If \fdftrue\ it causes the writing of an extra file
  named \sysfile{ORB\_INDX} containing all orbitals used in the
  calculation.

  Its formatting is clearly specified at the end of the file.

\end{fdflogicalT}
