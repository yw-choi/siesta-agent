Every time the atoms move, either during coordinate relaxation or
molecular dynamics, their \textbf{positions predicted for next step} and
\textbf{current velocities} are stored in file SystemLabel.XV, where
SystemLabel is the value of that \fdflib\ descriptor (or 'siesta' by
default).  The shape of the unit cell and its associated 'velocity'
(in Parrinello-Rahman dynamics) are also stored in this file. For MD
runs of type Verlet, Parrinello-Rahman, Nose,
Nose-Parrinello-Rahman, or Anneal, a file named SystemLabel.VERLET\_RESTART,
SystemLabel.PR\_RESTART, SystemLabel.NOSE\_RESTART,
SystemLabel.NPR\_RESTART, or SystemLabel.ANNEAL\_RESTART,
respectively, is created to hold the values
of auxiliary variables needed for a completely seamless
continuation.

If the restart file is not available, a simulation can still make use
of the XV information, and ``restart'' by basically repeating the
last-computed step (the positions are shifted backwards by using a
single Euler-like step with the current velocities as derivatives).
While this feature does not result in seamless continuations, it
allows cross-restarts (those in which a simulation of one kind (e.g.,
Anneal) is followed by another (e.g., Nose)), and permits
to re-use dynamical information from old runs.

This restart fix is not satisfactory from a fundamental point of view,
so the MD subsystem in \siesta\ will have to be redesigned
eventually. In the meantime, users are reminded that the scripting
hooks being steadily introduced (see \shell{Util/Scripting}) might be used to
create custom-made MD scripts.
