Useful for structural optimizations and constant-pressure molecular
dynamics.

\begin{fdfentry}{Target!Pressure}[pressure]<$0\,\mathrm{GPa}$>
  \fdfindex*{MD.TargetPressure}
  \fdfdeprecates{MD.TargetPressure}

  Target pressure for Parrinello-Rahman method, variable cell
  optimizations, and annealing options.

  \note this is only compatible with
  \fdf{MD.TypeOfRun} \fdf*{ParrinelloRahman}, \fdf*{NoseParrinelloRahman},
  \fdf*{CG}, \fdf*{Broyden} or \fdf*{FIRE} (variable cell), or \fdf*{Anneal}
  (if \fdf{MD.AnnealOption} \fdf*{Pressure} or \fdf*{TemperatureandPressure}).

\end{fdfentry}


\begin{fdfentry}{Target!Stress.Voigt}[block]<$-1$ $-1$ $-1$ $0$ $0$ $0$>
  \fdfdeprecates{MD.TargetStress}

  External or target stress tensor for variable cell optimizations.
  Stress components are given in a line, in the Voigt order \texttt{xx, yy,
      zz, yz, xz, xy}. In units of \fdf{Target!Pressure}, but
  with the opposite sign. For example, a uniaxial compressive stress
  of 2 GPa along the 100 direction would be given by
  \begin{fdfexample}
     Target.Pressure  2. GPa
     %block Target.Stress.Voigt
         -1.0  0.0  0.0  0.0  0.0  0.0
     %endblock
  \end{fdfexample}

  Only used if \fdf{MD.TypeOfRun} is \fdf*{CG}, \fdf*{Broyden} or
  \fdf*{FIRE} and \fdf{MD.VariableCell} is \fdftrue.

\end{fdfentry}

\begin{fdfentry}{MD.TargetStress}[block]<$-1$ $-1$ $-1$ $0$ $0$ $0$>
  \fdfdeprecatedby{Target!Stress.Voigt}

  Same as \fdf{Target!Stress.Voigt} but the order is same as older
  \siesta\ version (prior to 4.1). Order is \texttt{xx, yy, zz, xy,
      xz, yz}.

\end{fdfentry}


\begin{fdflogicalF}{MD.RemoveIntramolecularPressure}
  \index{removal of intramolecular pressure}

  If \fdftrue, the contribution to the stress coming from the internal
  degrees of freedom of the molecules will be subtracted from the
  stress tensor used in variable-cell optimization or variable-cell
  molecular-dynamics.  This is done in an approximate manner, using
  the virial form of the stress, and assumming that the ``mean force''
  over the coordinates of the molecule represents the
  ``inter-molecular'' stress. The correction term was already computed
  in earlier versions of \siesta\ and used to report the ``molecule
  pressure''. The correction is now computed molecule-by-molecule if
  the Zmatrix format is used.

  If the intra-molecular stress is removed, the corrected static and
  total stresses are printed in addition to the uncorrected items.
  The corrected Voigt form is also printed.

  \note versions prior to 4.1 (also 4.1-beta releases) printed the
  Voigt stress-tensor in this format: \shell{[x, y, z, xy, yz,
      xz]}. In 4.1 and later \siesta\ \emph{only} show the correct
  Voigt representation: \shell{[x, y, z, yz, xz, xy]}.

\end{fdflogicalF}
