If \fdf{MD.TypeOfRun} is \fdf*{FC}, \siesta\ sets up a special outer
geometry loop that displaces individual atoms along the coordinate
directions to build the force-constant matrix.
\index{output!molecular dynamics!Force Constants Matrix}

The output (see below) can be analyzed to extract phonon frequencies
and vectors with the VIBRA\index{VIBRA} package in the \program{Util/Vibra}
directory. For computing the Born effective charges together with the
force constants, see \fdf{BornCharge}.

\begin{fdfentry}{FC!Displacement}[length]<$0.04\,\mathrm{Bohr}$>
  \fdfdeprecates{MD!FCDispl}

  Displacement to use for the computation of the force constant
  matrix\index{Force Constants Matrix} for phonon calculations.

\end{fdfentry}

\begin{fdfentry}{FC!First}[integer]<$1$>
  \fdfdeprecates{MD!FCFirst}

  Index of first atom to displace for the computation of the force
  constant matrix\index{Force Constants Matrix} for phonon
  calculations.

\end{fdfentry}

\begin{fdfentry}{FC!Last}[integer]<\fdfvalue{FC!First}>
  \fdfdeprecates{MD!FCLast}

  Index of last atom to displace for the computation of the force
  constant matrix\index{Force Constants Matrix} for phonon
  calculations.

\end{fdfentry}

The force-constants matrix is written in file \sysfile{FC}.  The
format is the following: for the displacement of each atom in each
direction, the forces on each of the other atoms is writen (divided by
the value of the displacement), in units of eV/\AA$^2$. Each line has
the forces in the $x$, $y$ and $z$ direction for one of the atoms.

If constraints are used, the file \sysfile{FCC} is also written.

\begin{fdflogicalF}{FC!Save.dHS}

    For \fdf{MD.TypeOfRun:FC}, if \fdftrue, SIESTA produces a single netCDF
    file \sysfile*{dHSdR.nc} with the derivatives of the Hamiltonian and
    overlap matrix for each displaced atom along each Cartesian direction. 
    \textit{The derivatives are only calculated in the unit cell, not the 
    auxiliary supercell}.

\end{fdflogicalF}

\begin{fdfentry}{FC!dHdR.Tolerance}[force]<$-1\,\mathrm{Ry/Bohr}$>

  Threshold controlling which elements of the Hamiltonian derivative should be
  stored in \sysfile*{dHSdR.nc}. All matrix elements smaller than the threshold
  are discarded. If threshold is negative no elements are discarded.

\end{fdfentry}


\begin{fdfentry}{FC!dSdR.Tolerance}[inverse length]<$-1\,\mathrm{1/Bohr}$>

    Threshold controlling which elements of the Hamiltonian derivative should be
    stored in \sysfile*{dHSdR.nc}. All matrix elements smaller than the threshold
    are discarded. If threshold is negative no elements are discarded.
  
\end{fdfentry}
