

This functionality is not \siesta-specific, but is implemented to
provide a more complete simulation package. The program has an outer
geometry loop: it computes the electronic structure (and
thus the forces and stresses) for a given geometry, updates the
atomic positions (and maybe the cell vectors) accordingly and moves on
to the next cycle.
%
If there are molecular dynamics options missing you are highly
recommend to look into \fdf{MD.TypeOfRun:Lua} or
\fdf{MD.TypeOfRun:Master}.


Several options for MD and structural optimizations are
implemented, selected by
\begin{fdfentry}{MD.TypeOfRun}[string]<CG>

  \begin{fdfoptions}

    \option[CG]%
    \fdfindex*{MD.TypeOfRun:CG}%
    Performs an atomic coordinates optimization by using the conjugate gradients
    method. If \fdf{MD.VariableCell} is enabled (see below), the optimization
    includes the cell vectors.

    \option[Broyden]%
    \fdfindex*{MD.TypeOfRun:Broyden}%
    Performs an atomic coordinates optimization by using a modified Broyden
    method, which falls within the Quasi-Newton family of algorithms. If
    \fdf{MD.VariableCell} is enabled (see below), the optimization includes
    the cell vectors.

    \option[FIRE]%
    \fdfindex*{MD.TypeOfRun:FIRE}%
    Performs an atomic coordinates optimization by using the Fast Inertial
    Relaxation Engine (E. Bitzek et al, PRL 97, 170201, (2006)). If
    \fdf{MD.VariableCell} is enabled (see below), the optimization includes
    the cell vectors.
    FIRE avoids the need for linear search, thus making each individual iteration
    faster when compared to Quasi-Newton methods. However, it also needs more
    iterations to converge, so its efficiency is system-dependent.

    \option[Verlet]%
    \fdfindex*{MD.TypeOfRun:Verlet}%
    Standard Velocity-Verlet algorithm for NVE molecular dynamics.

    \option[Nose]%
    \fdfindex*{MD.TypeOfRun:Nose}%
    Constant temperature (NVT) MD with using a Nos\'e thermostat.

    \option[ParrinelloRahman]%
    \fdfindex*{MD.TypeOfRun:ParrinelloRahman}%
    Constant pressure (NPE) MD, controlled by the Parrinello-Rahman method.

    \option[NoseParrinelloRahman]%
    \fdfindex*{MD.TypeOfRun:NoseParrinelloRahman}%
    Constant temperature and pressure (NPT) MD using both methods above, the
    Nos\'e thermostat and the Parrinello-Rahman method.

    \option[Anneal]%
    \fdfindex*{MD.TypeOfRun:Anneal}%
    Constant temperature and/or pressure MD (see the variable
    \fdf{MD.AnnealOption} below), using a very simple velocity rescaling method
    (i.e. Berendsen thermostat and/or barostat \cite{Berendsen84}). It can be
    used to quickly equilibrate a system to a desired temperature and pressure.
    However, atomic velocities resulting from this option are non-canonical and
    thus tend to produce physically-inaccurate results. Therefore, one must
    rely on the Nos\'e and/or ParrinelloRahman options for production MD runs
    after the equilibration is done.

    \option[FC]%
    \fdfindex*{MD.TypeOfRun:FC}%
    Compute force constants matrix\index{Force Constants Matrix} for
    phonon calculations.

    \option[Master|Forces]%
    \fdfindex*{MD.TypeOfRun:Master}%
    \fdfindex*{MD.TypeOfRun:Forces}%
    Receive coordinates from, and return forces to, an external
    driver program, using MPI, Unix pipes, or Inet sockets for
    communication. The routines in module \program{fsiesta} allow the
    user's program to perform this communication transparently, as if
    \siesta\ were a conventional force-field subroutine. See
    \shell{Util/SiestaSubroutine/README} for details. WARNING: if this
    option is specified without a driver program sending data, siesta
    may hang without any notice.

    See directory \shell{Util/Scripting} \index{Scripting} for other driving
    options.

    \option[Lua]%
    \fdfindex*{MD.TypeOfRun:Lua}%
    Fully control the MD cycle and convergence path using an external
    Lua script.

    With an external Lua script one may control nearly everything from
    a script. One can query \emph{any} internal data-structures in
    \siesta\ and, similarly, return \emph{any} data thus overwriting
    the internals. A list of ideas which may be implemented in such a
    Lua script are:
    \begin{itemize}
      \item New geometry relaxation algorithms

      \item NEB calculations

      \item New MD routines

      \item Convergence tests of \fdf{Mesh!Cutoff} and
      \fdf{kgrid.MonkhorstPack}, or other parameters (currently basis
      set optimizations cannot be performed in the Lua script).

    \end{itemize}
    Sec.~\ref{sec:lua} for additional details (and a description of
    \program{flos} which implements some of the above mentioned items).

    Using this option requires the compilation of \siesta\ with the
    \program{flook} library.%
    \index{flook}\index{External library!flook}%
    If \siesta\ is not compiled as prescribed in Sec.~\ref{sec:libs}
    this option will make \siesta\ die.

    \option[TDED]%
    \fdfindex*{MD.TypeOfRun:TDED}%

    New option to perform time-dependent electron dynamics simulations
    (TDED) within RT-TDDFT. For more details see
    Sec.~\ref{sec:tddft}.

    The second run of \siesta\ uses this option with the files
    \sysfile{TDWF} and \sysfile{TDXV} present in the working
    directory.  In this option ions and electrons are assumed to move
    simultaneously. The occupied electronic states are time-evolved
    instead of the usual SCF calculations in each step.  Choose this
    option even if you intend to do only-electron dynamics. If you
    want to do an electron dynamics-only calculation set
    \fdf{MD.FinalTimeStep} equal to $1$. For optical response
    calculations switch off the external field during the second
    run. The \fdf{MD.LengthTimeStep}, unlike in the standard MD
    simulation, is defined by mulitpilication of \fdf{TDED!TimeStep}
    and \fdf{TDED!Nsteps}. In TDDFT calculations, the user defined
    \fdf{MD.LengthTimeStep} is ignored.


  \end{fdfoptions}

  \note if \fdf{Compat!Pre-v4-Dynamics} is \fdftrue\ this will default
  to \fdf*{Verlet}.

  Note that some options specified in later variables (like quenching)
  modify the behavior of these MD options.

  Appart from being able to act as a force subroutine for a driver
  program that uses module fsiesta, \siesta\ is also prepared to
  communicate with the i-PI code (see
  \url{https://github.com/i-pi/i-pi}).
  To do this, \siesta\ must be started after i-PI (it acts as a client
  of i-PI, communicating with it through Inet or Unix sockets), and
  the following lines must be present in the .fdf data file:
  \begin{fdfexample}
     MD.TypeOfRun      Master     # equivalent to 'Forces'
     Master.code       i-pi       # ( fsiesta | i-pi )
     Master.interface  socket     # ( pipes | socket | mpi )
     Master.address    localhost  # or driver's IP, e.g. 150.242.7.140
     Master.port       10001      # 10000+siesta_process_order
     Master.socketType inet       # ( inet | unix )
  \end{fdfexample}

\end{fdfentry}



\subsection{Compatibility with pre-v4 versions}
\index{Backward compatibility}

Starting in the summer of 2015, some changes were made to the behavior
of the program regarding default dynamics options and choice of
coordinates to work with during post-processing of the electronic
structure. The changes are:

\begin{itemize}
  \item %
  The default dynamics option is ``CG'' instead of ``Verlet''.

  \item%
  The coordinates, if moved by the dynamics routines, are reset to
  their values at the previous step for the analysis of the electronic
  structure (band structure calculations, DOS, LDOS, etc).

  \item%
  Some output files reflect the values of the ``un-moved''
  coordinates.

  \item%
  The default convergence criteria is now \emph{both} density and
  Hamiltonian convergence, see \fdf{SCF.DM!Converge} and
  \fdf{SCF.H!Converge}.

\end{itemize}

To recover the previous behavior, the user can turn on the
compatibility switch \fdf*{Compat!Pre-v4-Dynamics}, which is off by
default.

Note that complete compatibility cannot be perfectly guaranteed.


\subsection{Structural relaxation}

In this mode of operation, the program moves the atoms (and optionally
the cell vectors) trying to minimize the forces (and stresses) on
them.

These are the options common to all relaxation methods. If the Zmatrix
input option is in effect (see Sec.~\ref{sec:Zmatrix}) the
Zmatrix-specific options take precedence.  The 'MD' prefix is
misleading but kept for historical reasons.

\begin{fdflogicalF}{MD.VariableCell}
  \index{cell relaxation}

  If \fdftrue, the lattice is relaxed together with the atomic
  coordinates. It allows to target hydrostatic pressures or arbitrary
  stress tensors. See \fdf{MD.MaxStressTol},
  \fdf{Target!Pressure}, \fdf{Target!Stress.Voigt},
  \fdf{Constant!Volume}, and
  \fdf{MD.PreconditionVariableCell}.

  \note only compatible with \fdf{MD.TypeOfRun:CG},
  \fdf*{Broyden} or \fdf*{fire}.

\end{fdflogicalF}


\begin{fdflogicalF}{Constant!Volume}
  \fdfindex*{MD.ConstantVolume}
  \fdfdeprecates{MD.ConstantVolume}%
  \index{constant-volume cell relaxation}

  If \fdftrue, the cell volume is kept constant in a variable-cell
  relaxation: only the cell shape and the atomic coordinates are
  allowed to change.  Note that it does not make much sense to specify
  a target stress or pressure in this case, except for anisotropic
  (traceless) stresses.  See \fdf{MD.VariableCell},
  \fdf{Target.Stress.Voigt}.

  \note only compatible with \fdf{MD.TypeOfRun:CG},
  \fdf*{Broyden} or \fdf*{fire}.

\end{fdflogicalF}

\begin{fdflogicalF}{MD.RelaxCellOnly}
  \index{relaxation of cell parameters only}

  If \fdftrue, only the cell parameters are relaxed (by the Broyden or
  FIRE method, not CG). The atomic coordinates are re-scaled to the
  new cell, keeping the fractional coordinates constant. For
  \fdf{Zmatrix} calculations, the fractional position of the first
  atom in each molecule is kept fixed, and no attempt is made to
  rescale the bond distances or angles.

  \note only compatible with \fdf{MD.TypeOfRun:Broyden} or \fdf*{fire}.

\end{fdflogicalF}

\begin{fdfentry}{MD.MaxForceTol}[force]<$0.04\,\mathrm{eV/Ang}$>

  Force tolerance in coordinate optimization.
  Run stops if the maximum atomic force is
  smaller than \fdf{MD.MaxForceTol} (see \fdf{MD.MaxStressTol}
  for variable cell).

\end{fdfentry}

\begin{fdfentry}{MD.MaxStressTol}[pressure]<$1\,\mathrm{GPa}$>

  Stress tolerance in variable-cell CG optimization. Run stops if the
  maximum atomic force is smaller than \fdf{MD.MaxForceTol} and the
  maximum stress component is smaller than \fdf{MD.MaxStressTol}.

  Special consideration is needed if used with Sankey-type basis sets,
  since the combination of orbital kinks at the cutoff radii and the
  finite-grid integration originate discontinuities in the stress
  components, whose magnitude depends on the cutoff radii (or energy
  shift) and the mesh cutoff. The tolerance has to be larger than the
  discontinuities to avoid endless optimizations if the target stress
  happens to be in a discontinuity.

\end{fdfentry}

\begin{fdfentry}{MD.Steps}[integer]<0>
  \fdfindex*{MD.NumCGsteps}
  \fdfdeprecates{MD.NumCGsteps}

  Maximum number of steps in a minimization routine
  (the minimization will stop if tolerance is reached before; see
  \fdf{MD.MaxForceTol} below).

  \note The old flag \fdf{MD.NumCGsteps} will remain for historical
  reasons.

\end{fdfentry}

\begin{fdfentry}{MD.MaxDispl}[length]<$0.2\,\mathrm{Bohr}$>
  \fdfindex*{MD.MaxCGDispl}
  \fdfdeprecates{MD.MaxCGDispl}

  Maximum atomic displacements in an optimization move.

  In the Broyden optimization method, it is also possible to limit
  indirectly the \textit{initial\/} atomic displacements using
  \fdf{MD.Broyden.Initial.Inverse.Jacobian}. For the \fdf*{FIRE} method, the
  same result can be obtained by choosing a small time step.

  Note that there are Zmatrix-specific options that override this option.

  \note The old flag \fdf{MD.MaxCGDispl} will remain for historical
  reasons.

\end{fdfentry}

\begin{fdfentry}{MD.PreconditionVariableCell}[length]<$5\,\mathrm{Ang}$>

  A length to multiply to the strain components in a variable-cell
  optimization. The strain components enter the minimization on the
  same footing as the coordinates. For good efficiency, this length
  should make the scale of energy variation with strain similar to the
  one due to atomic displacements. It is also used for the application
  of the \fdf{MD.MaxDispl} value to the strain components.

\end{fdfentry}


\begin{fdfentry}{ZM.ForceTolLength}[force]<$0.00155574\,\mathrm{Ry/Bohr}$>

  Parameter that controls the convergence with respect to forces on
  Z-matrix lengths

\end{fdfentry}


\begin{fdfentry}{ZM.ForceTolAngle}[torque]<$0.00356549\,\mathrm{Ry/rad}$>

  Parameter that controls the convergence with respect to forces on
  Z-matrix angles

\end{fdfentry}

\begin{fdfentry}{ZM.MaxDisplLength}[length]<$0.2\,\mathrm{Bohr}$>

  Parameter that controls the maximum change in a Z-matrix length
  during an optimisation step.

\end{fdfentry}

\begin{fdfentry}{ZM.MaxDisplAngle}[angle]<$0.003\,\mathrm{rad}$>

  Parameter that controls the maximum change in a Z-matrix angle
  during an optimisation step.

\end{fdfentry}



\subsubsection{Conjugate-gradients optimization}

This was historically the default geometry-optimization method, and
all the above options were introduced specifically for it, hence their
names. The following pertains only to this method:

\index{Conjugate-gradient history information}
\begin{fdflogicalF}{MD.UseSaveCG}
  \index{reading saved data!CG}

  Instructs to read the conjugate-gradient hystory information stored
  in file \sysfile{CG} by a previous run.

  \note to get actual continuation of iterrupted CG runs, use
  together with \fdf{MD.UseSaveXV} \fdftrue\ with the \sysfile*{XV}
  file generated in the same run as the CG file.  If the required file
  does not exist, a warning is printed but the program does not
  stop. Overrides \fdf{UseSaveData}.

  \note no such feature exists yet for a Broyden-based relaxation.

\end{fdflogicalF}

\subsubsection{Broyden optimization}

It uses the modified Broyden algorithm to
build up the Jacobian matrix. (See D.D. Johnson, PRB 38, 12807
(1988)). (Note: This is not BFGS.)

\begin{fdfentry}{MD.Broyden!History.Steps}[integer]<$5$>
  \index{Broyden optimization}

  Number of relaxation steps during which the modified Broyden
  algorithm builds up the Jacobian matrix.

\end{fdfentry}

\begin{fdflogicalT}{MD.Broyden!Cycle.On.Maxit}

  Upon reaching the maximum number of history data sets which are kept
  for Jacobian estimation, throw away the oldest and shift the rest to
  make room for a new data set. The alternative is to re-start the
  Broyden minimization algorithm from a first step of a diagonal
  inverse Jacobian (which might be useful when the minimization is
  stuck).

\end{fdflogicalT}

\begin{fdfentry}{MD.Broyden!Initial.Inverse.Jacobian}[real]<$1$>

  Initial inverse Jacobian for the optimization procedure. (The units
  are those implied by the internal \siesta\ usage. The default value
  seems to work well for most systems.

\end{fdfentry}



\subsubsection{FIRE relaxation}

Implementation of the Fast Inertial Relaxation Engine (FIRE) method
(E. Bitzek et al, PRL 97, 170201, (2006) in a manner compatible with
the CG and Broyden modes of relaxation. (An older implementation
activated by the \fdf*{MD.FireQuench} variable is still available).

\begin{fdfentry}{MD.FIRE.TimeStep}[time]<\fdfvalue{MD.LengthTimeStep}>

  The (fictitious) time-step for FIRE relaxation.  This is the main
  user-variable when the option \fdf*{FIRE} for
  \fdf{MD.TypeOfRun} is active.

  \note the default value is encouraged to be changed as the link to
  \fdf{MD.LengthTimeStep} is misleading.

  There are other low-level options tunable by the user (see the
  routines \texttt{fire\_optim} and \texttt{cell\_fire\_optim} for
  more details.

\end{fdfentry}


\ifdeprecated
% The below options are deprecated in favor of:
% MD.TypeOfRun fire

\subsubsection{Quenched MD}

These methods are really based on molecular dynamics, but are used for
structural relaxation.

Note that the Zmatrix input option (see Sec.~\ref{sec:Zmatrix}) is not
compatible with molecular dynamics. The initial geometry can be
specified using the Zmatrix format, but the Zmatrix generalized
coordinates will not be updated.

Note also that the force and stress tolerances have no effect on
the termination conditions of these methods. They run for the number
of MD steps requested (this is arguably a bug).

\begin{fdflogicalF}{MD.Quench}

  Logical option to perform a power quench during the molecular
  dynamics.  In the power quench, each velocity component is set to
  zero if it is opposite to the corresponding force of that
  component. This affects atomic velocities, or unit-cell velocities
  (for cell shape optimizations).

  \note only applicable for \fdf{MD.TypeOfRun:Verlet} or
  \fdf*{ParrinelloRahman}.
  %
  It is incompatible with Nose thermostat options.

  \note \fdf{MD.Quench} is superseded by \fdf{MD.FireQuench} (see
  below).

\end{fdflogicalF}


\begin{fdflogicalF}{MD.FireQuench}

  See the new option \fdf*{FIRE} for \fdf{MD.TypeOfRun}.

  Logical option to perform a FIRE quench during a Verlet molecular
  dynamics run, as described by Bitzek \textit{et al.} in
  Phys. Rev. Lett. \textbf{97}, 170201 (2006). It is a relaxation
  algorithm, and thus the dynamics are of no interest per se: the
  initial time-step can be played with (it uses
  \fdf{MD.LengthTimeStep} as initial $\Delta t$), as well as the
  initial temperature (recommended 0) and the atomic masses
  (recommended equal). Preliminary tests seem to indicate that the
  combination of $\Delta t = 5$ fs and a value of 20 for the atomic
  masses works reasonably. The dynamics stops when the force tolerance
  is reached (\fdf{MD.MaxForceTol}). The other parameters
  controlling the algorithm (initial damping, increase and decrease
  thereof etc.) are hardwired in the code, at the recommended values
  in the cited paper, including $\Delta t_{max} = 10$ fs.

  Only available for \fdf{MD.TypeOfRun:Verlet}
  It is incompatible with Nose thermostat options. No variable
  cell option implemented for this at this stage.
  \fdf{MD.FireQuench} supersedes \fdf{MD.Quench}. This option is
  deprecated. The new option \fdf*{FIRE} for \fdf{MD.TypeOfRun} should be
  used instead.

\end{fdflogicalF}


\fi

\subsection{Target stress options}

Useful for structural optimizations and constant-pressure molecular
dynamics.

\begin{fdfentry}{Target!Pressure}[pressure]<$0\,\mathrm{GPa}$>
  \fdfindex*{MD.TargetPressure}
  \fdfdeprecates{MD.TargetPressure}

  Target pressure for Parrinello-Rahman method, variable cell
  optimizations, and annealing options.

  \note this is only compatible with
  \fdf{MD.TypeOfRun} \fdf*{ParrinelloRahman}, \fdf*{NoseParrinelloRahman},
  \fdf*{CG}, \fdf*{Broyden} or \fdf*{FIRE} (variable cell), or \fdf*{Anneal}
  (if \fdf{MD.AnnealOption} \fdf*{Pressure} or \fdf*{TemperatureandPressure}).

\end{fdfentry}


\begin{fdfentry}{Target!Stress.Voigt}[block]<$-1$ $-1$ $-1$ $0$ $0$ $0$>
  \fdfdeprecates{MD.TargetStress}

  External or target stress tensor for variable cell optimizations.
  Stress components are given in a line, in the Voigt order \texttt{xx, yy,
      zz, yz, xz, xy}. In units of \fdf{Target!Pressure}, but
  with the opposite sign. For example, a uniaxial compressive stress
  of 2 GPa along the 100 direction would be given by
  \begin{fdfexample}
     Target.Pressure  2. GPa
     %block Target.Stress.Voigt
         -1.0  0.0  0.0  0.0  0.0  0.0
     %endblock
  \end{fdfexample}

  Only used if \fdf{MD.TypeOfRun} is \fdf*{CG}, \fdf*{Broyden} or
  \fdf*{FIRE} and \fdf{MD.VariableCell} is \fdftrue.

\end{fdfentry}

\begin{fdfentry}{MD.TargetStress}[block]<$-1$ $-1$ $-1$ $0$ $0$ $0$>
  \fdfdeprecatedby{Target!Stress.Voigt}

  Same as \fdf{Target!Stress.Voigt} but the order is same as older
  \siesta\ version (prior to 4.1). Order is \texttt{xx, yy, zz, xy,
      xz, yz}.

\end{fdfentry}


\begin{fdflogicalF}{MD.RemoveIntramolecularPressure}
  \index{removal of intramolecular pressure}

  If \fdftrue, the contribution to the stress coming from the internal
  degrees of freedom of the molecules will be subtracted from the
  stress tensor used in variable-cell optimization or variable-cell
  molecular-dynamics.  This is done in an approximate manner, using
  the virial form of the stress, and assumming that the ``mean force''
  over the coordinates of the molecule represents the
  ``inter-molecular'' stress. The correction term was already computed
  in earlier versions of \siesta\ and used to report the ``molecule
  pressure''. The correction is now computed molecule-by-molecule if
  the Zmatrix format is used.

  If the intra-molecular stress is removed, the corrected static and
  total stresses are printed in addition to the uncorrected items.
  The corrected Voigt form is also printed.

  \note versions prior to 4.1 (also 4.1-beta releases) printed the
  Voigt stress-tensor in this format: \shell{[x, y, z, xy, yz,
      xz]}. In 4.1 and later \siesta\ \emph{only} show the correct
  Voigt representation: \shell{[x, y, z, yz, xz, xy]}.

\end{fdflogicalF}


\subsection{Molecular dynamics}

In this mode of operation, the program moves the atoms (and optionally
the cell vectors) in response to the forces (and stresses), using the
classical equations of motion.

Note that the \fdf{Zmatrix} input option (see Sec.~\ref{sec:Zmatrix}) is not
compatible with molecular dynamics. The initial geometry can be
specified using the Zmatrix format, but the Zmatrix generalized
coordinates will not be updated.


\begin{fdfentry}{MD.InitialTimeStep}[integer]<$1$>

  Initial time step of the MD simulation.  In the current version of
  \siesta\ it must be 1.

  Used only if \fdf{MD.TypeOfRun} is not \fdf*{CG} or \fdf*{Broyden}.

\end{fdfentry}

\begin{fdfentry}{MD.FinalTimeStep}[integer]<\fdfvalue{MD.Steps}>

  Final time step of the MD simulation.

\end{fdfentry}


\begin{fdfentry}{MD.LengthTimeStep}[time]<$1\,\mathrm{fs}$>

  Length of the time step of the MD simulation.

\end{fdfentry}

\begin{fdfentry}{MD.InitialTemperature}[temperature/energy]<$0\,\mathrm K$>

  Initial temperature for the MD run. The atoms are assigned random
  velocities drawn from the Maxwell-Bolzmann distribution with the
  corresponding temperature. The constraint of zero center of mass
  velocity is imposed.

  \note only used if \fdf{MD.TypeOfRun} \fdf*{Verlet}, \fdf*{Nose},
  \fdf*{ParrinelloRahman}, \fdf*{NoseParrinelloRahman} or
  \fdf*{Anneal}.

\end{fdfentry}

\begin{fdfentry}{MD.TargetTemperature}[temperature/energy]<$0\,\mathrm K$>

  Target temperature for Nose thermostat and annealing options.

  \note only used if \fdf{MD.TypeOfRun} \fdf*{Nose},
  \fdf*{NoseParrinelloRahman} or
  \fdf*{Anneal} if \fdf{MD.AnnealOption} is \fdf*{Temperature} or
  \fdf*{TemperatureandPressure}.

\end{fdfentry}

\begin{fdfentry}{MD.NoseMass}[moment of inertia]<$100\,\mathrm{Ry\,fs^2}$>

  Generalized mass of Nose variable.  This determines the time scale
  of the Nose variable dynamics, and the coupling of the thermal bath
  to the physical system.

  Only used for Nose MD runs.

\end{fdfentry}

\begin{fdfentry}{MD.ParrinelloRahmanMass}[moment of inertia]<$100\,\mathrm{Ry\,fs^2}$>

  Generalized mass of Parrinello-Rahman variable.  This determines the
  time scale of the Parrinello-Rahman variable dynamics, and its
  coupling to the physical system.

  Only used for Parrinello-Rahman MD runs.

\end{fdfentry}

\begin{fdfentry}{MD.AnnealOption}[string]<TemperatureAndPressure>

  Type of annealing MD to perform. The target temperature or pressure
  are achieved by velocity and unit cell rescaling, in a given time
  determined by the variable \fdf{MD.TauRelax} below. These are based on
  the Berendsen thermostat and barostats, respectively \cite{Berendsen84}.
  \begin{fdfoptions}
    \option[Temperature]%
    Reach a target temperature by velocity rescaling

    \option[Pressure]%
    Reach a target pressure by scaling of the unit cell size and shape

    \option[TemperatureandPressure]%
    Reach a target temperature and pressure by velocity rescaling and
    by scaling of the unit cell size and shape
  \end{fdfoptions}

  Only applicable for \fdf{MD.TypeOfRun:Anneal}.

\end{fdfentry}

\begin{fdfentry}{MD.TauRelax}[time]<$100\,\mathrm{fs}$>

  Relaxation time to reach target temperature and/or pressure in
  annealing MD. Note that this is a ``relaxation time'', and as such
  it gives a rough estimate of the time needed to achieve the given
  targets. As a normal simulation also exhibits oscillations, the
  actual time needed to reach the \emph{averaged} targets will be
  significantly longer.

  When using the barostat, the actual time required to reach the target
  pressure will depend on the ratio between \fdf{MD.TauRelax} and
  \fdf{MD.BulkModulus}.

  Only applicable for \fdf{MD.TypeOfRun:Anneal}.

\end{fdfentry}

\begin{fdfentry}{MD.BulkModulus}[pressure]<$100\,\mathrm{Ry/Bohr^3}$>

  Estimate (may be rough) of the bulk modulus of the system.  This is
  needed to set the rate of change of cell shape to reach target
  pressure in annealing MD.

  Only applicable for \fdf{MD.TypeOfRun} \fdf*{Anneal}, when
  \fdf{MD.AnnealOption} is \fdf*{Pressure} or \fdf*{TemperatureAndPressure}

\end{fdfentry}


\subsection{Output options for dynamics}

\tsiesta\ generates several output files.
\begin{description}
  \itemsep 10pt
  \parsep 0pt

  \item[\sysfile{DM}]: The \siesta\ density matrix. \siesta\ initially
  performs a calculation at zero bias assuming periodic boundary conditions in all
  directions, and no voltage, which is used as a starting point for the \tsiesta\
  calculation.

  \item[\sysfile{TSDE}]: The \tsiesta\ density matrix and energy
  density matrix. During a \tsiesta\ run, the \sysfile{DM} values are
  used for the density matrix in the buffer (if used) and electrode
  regions. The coupling terms may or may not be updated in a \tsiesta\
  run, see \fdf{TS!Elec.<>!DM-update}.

  \item[\sysfile{TS.HSX}]: The Hamiltonian corresponding to
  \sysfile{TSDE}. This file also contains geometry information
  etc. needed by \tsiesta\ and \tbtrans.

  \item[\sysfile{TSHS}]: The Hamiltonian corresponding to
  \sysfile{TSDE}. This file also contains geometry information
    etc. needed by \tsiesta\ and \tbtrans. Deprecated since the \sysfile{TS.HSX} file
    contains the same information.

  \item[\sysfile{TS.KP}]: The $k$-points used in the \tsiesta\ calculation. See
  \siesta\ \sysfile{KP} file for formatting information.

  \item[\sysfile{TSFA}]: Forces only on atoms in the device
  region. See \fdf{TS!Forces} for details.

  \item[\sysfile{TSCCEQ*}]: The equilibrium complex contour integration paths.

  \item[\sysfile{TSCCNEQ*}]: The non-equilibrium complex contour
  integration paths for \emph{correcting} the equilibrium contours.

  \item[\sysfile{TSGF*}]: Self-energy files containing the used
  self-energies from the leads. These are very large files used in the
  SCF loop. Once completed one can safely delete these files.
  %
  For heavily increased throughput these files may be re-used for the
  same electrode settings in various calculations.

\end{description}

\subsection{Utilities for analysis:
    \texorpdfstring{\tbtrans}{TBtrans}}
\index{tbtrans@\tbtrans}

Please see the separate \tbtrans\ manual
(\href{run:tbtrans.pdf}{tbtrans.pdf}).



\subsection{Restarting geometry optimizations and MD runs}

Every time the atoms move, either during coordinate relaxation or
molecular dynamics, their \textbf{positions predicted for next step} and
\textbf{current velocities} are stored in file SystemLabel.XV, where
SystemLabel is the value of that \fdflib\ descriptor (or 'siesta' by
default).  The shape of the unit cell and its associated 'velocity'
(in Parrinello-Rahman dynamics) are also stored in this file. For MD
runs of type Verlet, Parrinello-Rahman, Nose,
Nose-Parrinello-Rahman, or Anneal, a file named SystemLabel.VERLET\_RESTART,
SystemLabel.PR\_RESTART, SystemLabel.NOSE\_RESTART,
SystemLabel.NPR\_RESTART, or SystemLabel.ANNEAL\_RESTART,
respectively, is created to hold the values
of auxiliary variables needed for a completely seamless
continuation.

If the restart file is not available, a simulation can still make use
of the XV information, and ``restart'' by basically repeating the
last-computed step (the positions are shifted backwards by using a
single Euler-like step with the current velocities as derivatives).
While this feature does not result in seamless continuations, it
allows cross-restarts (those in which a simulation of one kind (e.g.,
Anneal) is followed by another (e.g., Nose)), and permits
to re-use dynamical information from old runs.

This restart fix is not satisfactory from a fundamental point of view,
so the MD subsystem in \siesta\ will have to be redesigned
eventually. In the meantime, users are reminded that the scripting
hooks being steadily introduced (see \shell{Util/Scripting}) might be used to
create custom-made MD scripts.


\subsection{Use of general constraints}

\textbf{Note:} The Zmatrix format (see Sec.~\ref{sec:Zmatrix}) provides
an alternative constraint formulation which can be useful for system
involving molecules.

\begin{fdfentry}{Geometry!Constraints}[block]
  \index{constraints in relaxations}

  Constrains certain atomic coordinates or cell parameters in a
  consistent method.

  There are a high number of configurable parameters that may be used
  to control the relaxation of the coordinates.

  \note \siesta\ prints out a small section of how the constraints are
  recognized.

  \def\directionalconstraint{\note these specifications are apt for \emph{directional}
    constraints.}


  \begin{fdfoptions}
    \option[atom|position]%
    Fix certain atomic coordinates.

    This option takes a variable number of integers which each
    correspond to the atomic index (or input sequence) in
    \fdf{AtomicCoordinatesAndAtomicSpecies}.

    \fdf*{atom} is now the preferred input option while
    \fdf*{position} still works for backwards compatibility.

    One may also specify ranges of atoms according to:

    \begin{fdfoptions}
      \option[{atom \emph{A} [\emph{B} [\emph{C} [\dots]]]}]%
      A sequence of atomic indices which are constrained.

      % Generic input (compatible with the <= 4.0)
      \option[{atom from \emph{A} to \emph{B} [step \emph{s}]}]%
      Here atoms \emph{A} up to and including \emph{B} are
      constrained.
      %
      If \fdf*{step <s>} is given, the range
      \emph{A}:\emph{B} will be taken in steps of \emph{s}.

      \begin{fdfexample}
        atom from 3 to 10 step 2
      \end{fdfexample}
      will constrain atoms 3, 5, 7 and 9.

      \option[{atom from \emph{A} plus/minus \emph{B} [step
          \emph{s}]}]%
      Here atoms \emph{A} up to and including $\emph{A}+\emph{B}-1$
      are constrained.
      %
      If \fdf*{step <s>} is given, the range
      \emph{A}:$\emph{A}+\emph{B}-1$ will be taken in steps of
      \emph{s}.

      % Generic input (compatible with the <= 4.0)
      \option[atom {[\emph{A}, \emph{B} -\mbox{}- \emph{C} [step \emph{s}], \emph{D}]}]%
      Equivalent to \fdf*{from \dots to} specification, however in a
      shorter variant. Note that the list may contain arbitrary number
      of ranges and/or individual indices.

      \begin{fdfexample}
        atom [2, 3 -- 10 step 2, 6]
      \end{fdfexample}
      will constrain atoms 2, 3, 5, 7, 9 and 6.

      \begin{fdfexample}
        atom [2, 3 -- 6, 8]
      \end{fdfexample}
      will constrain atoms 2, 3, 4, 5, 6 and 8.

      \option[atom all]%
      Constrain all atoms.

    \end{fdfoptions}

    \directionalconstraint


    \option[Z]%
    Equivalent to \fdf*{atom} with all indices of the atoms that
    have atomic number equal to the specified number.

    \directionalconstraint


    \option[species-i]%
    Equivalent to \fdf*{atom} with all indices of the atoms that
    have species according to the \fdf{ChemicalSpeciesLabel} and
    \fdf{AtomicCoordinatesAndAtomicSpecies}.

    \directionalconstraint


    \option[center]%
    One may retain the coordinate center of a
    range of atoms (say molecules or other groups of atoms).

    Atomic indices may be specified according to \fdf*{atom}.

    \directionalconstraint


    \option[rigid|molecule]%
    Move a selection of atoms together as though they where one atom.

    The forces are summed and averaged to get a net-force on the
    entire molecule.

    Atomic indices may be specified according to \fdf*{atom}.

    \directionalconstraint


    \option[rigid-max|molecule-max]%
    Move a selection of atoms together as though they where one atom.

    The maximum force acting on one of the atoms in the selection will
    be expanded to act on all atoms specified.

    Atomic indices may be specified according to \fdf*{atom}.

    \directionalconstraint


    \option[cell-angle]%
    Control whether the cell angles ($\alpha$, $\beta$, $\gamma$) may
    be altered.

    This takes either one or more of
    \fdf*{alpha}/\fdf*{beta}/\fdf*{gamma} as argument.

    \fdf*{alpha} is the angle between the 2nd and 3rd cell vector.

    \fdf*{beta} is the angle between the 1st and 3rd cell vector.

    \fdf*{gamma} is the angle between the 1st and 2nd cell vector.

    \note currently only one angle can be constrained at a time and it
    forces only the spanning vectors to be relaxed.


    \option[cell-vector]%
    Control whether the cell vectors ($A$, $B$, $C$) may be altered.

    This takes either one or more of \fdf*{A}/\fdf*{B}/\fdf*{C} as
    argument.

    Constraining the cell-vectors are only allowed if they only have a
    component along their respective Cartesian
    direction. I.e. \fdf*{B} must only have a $y$-component.


    \option[stress]%
    Control which of the 6 stress components are constrained.

    Numbers $1\le i\le6$ where $1$ corresponds
    to the \emph{XX} stress-component, $2$ is \emph{YY}, $3$ is
    \emph{ZZ}, $4$ is \emph{YZ}/\emph{ZY}, $5$ is \emph{XZ}/\emph{ZX}
    and $6$ is \emph{XY}/\emph{YX}.

    The text specifications are also allowed.


    \option[routine]%
    This calls the \program{constr} routine specified in the file:
    \file{constr.f}. Without having changed the corresponding source
    file, this does nothing.
    See details and comments in the source-file.


    \option[clear]%
    Remove constraints on selected atoms from all previously specified
    constraints.

    This may be handy when specifying constraints via \fdf*{Z} or
    \fdf*{species-i}.

    Atomic indices may be specified according to \fdf*{atom}.


    \option[clear-prev]
    Remove constraints on selected atoms from the \emph{previous} specified
    constraint.

    This may be handy when specifying constraints via \fdf*{Z} or
    \fdf*{species-i}.

    Atomic indices may be specified according to \fdf*{atom}.

    \note two consecutive \fdf*{clear-prev} may be used in conjunction
    as though the atoms where specified on the same line.

  \end{fdfoptions}

  It is instructive to give an example of the input options presented.

  Consider a benzene molecule ($\mathrm{C}_6\mathrm{H}_6$) and we wish
  to relax all Hydrogen atoms (and no stress in $x$ and $y$
  directions). This may be accomplished with this
  \begin{fdfexample}
    %block Geometry.Constraints
      Z 6
      stress 1 2
    %endblock
  \end{fdfexample}
  Or as in this example
  \begin{fdfexample}
    %block AtomicCoordinatesAndAtomicSpecies
      ... ... ... 1   # C 1
      ... ... ... 2   # H 2
      ... ... ... 1   # C 3
      ... ... ... 2   # H 4
      ... ... ... 1   # C 5
      ... ... ... 2   # H 6
      ... ... ... 1   # C 7
      ... ... ... 2   # H 8
      ... ... ... 1   # C 9
      ... ... ... 2   # H 10
      ... ... ... 1   # C 11
      ... ... ... 2   # H 12
      stress XX YY
    %endblock
    %block Geometry.Constraints
      atom from 1 to 12 step 2
      stress XX YY
    %endblock
    %block Geometry.Constraints
      atom [1 -- 12 step 2]
      stress XX 2
    %endblock
    %block Geometry.Constraints
      atom all
      clear-prev [2 -- 12 step 2]
      stress 1 YY
    %endblock
  \end{fdfexample}
  where the 3 last blocks all create the same result.

  Finally, the \emph{directional} constraint is an important and often
  useful feature.
  The directional constraints will subtract the force projected onto
  the direction specified. Hence an $x$ directional constraint will
  remove the force component along the $x$ direction $f_x\to 0$.

  When relaxing complex structures it may be advantageous to first
  relax along a given direction (where you expect the stress to be the
  largest) and subsequently let it fully relax. Another example would
  be to relax the binding distance between a molecule and a surface,
  before relaxing the entire system by forcing the molecule and
  adsorption site to relax together.
  %
  To use directional constraints one may provide an additional 3
  \emph{reals} after the \fdf*{atom}/\fdf*{rigid}.
  For instance in the previous example (benzene) one may first relax
  all Hydrogen atoms along the $y$ and $z$ Cartesian vector by
  constraining the $x$ Cartesian vector
  \begin{fdfexample}
    %block Geometry.Constraints
      Z 6 # constrain Carbon
      Z 1 1. 0. 0. # constrain Hydrogen along x Cartesian vector
    %endblock
  \end{fdfexample}
  Note that you \emph{must} append a ``.'' to denote it a real. The
  vector specified need not be normalized. Also, if you want it to
  be constrained along the $x$-$y$ vector you may do
  \begin{fdfexample}
    %block Geometry.Constraints
      Z 6
      Z 1 1. 1. 0.
    %endblock
  \end{fdfexample}

  Therefore the directional constraint will remove the force
  components that projects onto the direction specified.


\end{fdfentry}


\subsection{Phonon calculations}

If \fdf{MD.TypeOfRun} is \fdf*{FC}, \siesta\ sets up a special outer
geometry loop that displaces individual atoms along the coordinate
directions to build the force-constant matrix.
\index{output!molecular dynamics!Force Constants Matrix}

The output (see below) can be analyzed to extract phonon frequencies
and vectors with the VIBRA\index{VIBRA} package in the \program{Util/Vibra}
directory. For computing the Born effective charges together with the
force constants, see \fdf{BornCharge}.

\begin{fdfentry}{FC!Displacement}[length]<$0.04\,\mathrm{Bohr}$>
  \fdfdeprecates{MD!FCDispl}

  Displacement to use for the computation of the force constant
  matrix\index{Force Constants Matrix} for phonon calculations.

\end{fdfentry}

\begin{fdfentry}{FC!First}[integer]<$1$>
  \fdfdeprecates{MD!FCFirst}

  Index of first atom to displace for the computation of the force
  constant matrix\index{Force Constants Matrix} for phonon
  calculations.

\end{fdfentry}

\begin{fdfentry}{FC!Last}[integer]<\fdfvalue{FC!First}>
  \fdfdeprecates{MD!FCLast}

  Index of last atom to displace for the computation of the force
  constant matrix\index{Force Constants Matrix} for phonon
  calculations.

\end{fdfentry}

The force-constants matrix is written in file \sysfile{FC}.  The
format is the following: for the displacement of each atom in each
direction, the forces on each of the other atoms is writen (divided by
the value of the displacement), in units of eV/\AA$^2$. Each line has
the forces in the $x$, $y$ and $z$ direction for one of the atoms.

If constraints are used, the file \sysfile{FCC} is also written.

\begin{fdflogicalF}{FC!Save.dHS}

    For \fdf{MD.TypeOfRun:FC}, if \fdftrue, SIESTA produces a single netCDF
    file \sysfile*{dHSdR.nc} with the derivatives of the Hamiltonian and
    overlap matrix for each displaced atom along each Cartesian direction. 
    \textit{The derivatives are only calculated in the unit cell, not the 
    auxiliary supercell}.

\end{fdflogicalF}

\begin{fdfentry}{FC!dHdR.Tolerance}[force]<$-1\,\mathrm{Ry/Bohr}$>

  Threshold controlling which elements of the Hamiltonian derivative should be
  stored in \sysfile*{dHSdR.nc}. All matrix elements smaller than the threshold
  are discarded. If threshold is negative no elements are discarded.

\end{fdfentry}


\begin{fdfentry}{FC!dSdR.Tolerance}[inverse length]<$-1\,\mathrm{1/Bohr}$>

    Threshold controlling which elements of the Hamiltonian derivative should be
    stored in \sysfile*{dHSdR.nc}. All matrix elements smaller than the threshold
    are discarded. If threshold is negative no elements are discarded.
  
\end{fdfentry}

