Some of the real arrays used in \siesta\ are by default
single-precision, to save memory. This applies to the array that holds
the values of the basis orbitals on the real-space grid, to the
historical data sets in Broyden mixing, and to the arrays used in the
O(N) routines. Note that the grid functions (charge densities,
potentials, etc) are (since mid January 2010) in double
precision by default.

The following options and pre-processing symbols control the
precision selection.

\begin{itemize}

  \item Add \texttt{-DWITH\_GRID\_SP} to the CMake invocation to use
    single-precision for all the grid magnitudes, including the
    orbitals array and charge densities and potentials.  This will
    cause some numerical differences and will have a negligible effect
    on memory consumption, since the orbitals array is the main user
    of memory on the grid, and it is single-precision by default. This
    setting will recover the default behavior of versions prior to
    4.0.\index{Grid precision}


  \item Use \texttt{-DFortran\_FLAGS="-DGRID\_DP"} to use
    double-precision for all the grid magnitudes, including the
    orbitals array. This will significantly increase the memory used
    for large problems, with negligible differences in accuracy.


  \item Use \texttt{-DFortran\_FLAGS="-DBROYDEN\_DP"} to use
    double-precision arrays for the data sets in the Broyden mixing
    for SCF convergence acceleration.\index{Broyden mixing}

  \item Use \texttt{-DFortran\_FLAGS="-DON\_DP"} to use
    double-precision for all the arrays in the O(N) routines.

\end{itemize}