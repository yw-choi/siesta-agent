The Lindhard Function calculator computes the 2D Lindhard response
~\cite{doi:10.1088/1361-648X/ab8522} from \siesta\ outputs. The outputs
required are the EIG file (with the eigenvalues for each k-point) and the KP
file (with the coordinates for each k-point). The original \siesta\ calculation
must be done with the \fdf{TimeReversalSymmetryForKpoints} option set to
\fdffalse. Other than that, the Lindhard-function specific options should be
added to the same fdf file as the \siesta\ calculation. Then, the utility can be
used as:

\begin{verbatim}
  lindhard < <fdf file> > <output file>
\end{verbatim}

The lindhard utility accepts the following options:

\begin{fdfentry}{Lindhard!Temperature}[temperature]<$300\,\mathrm{K}$>

  The temperature for the Fermi distribution function used to calculate the
  Lindhard response.
\end{fdfentry}

\begin{fdfentry}{Lindhard!FirstBand}[integer]<$0$>
    Index of the first band to be included in the calculation; i.e., the lowest
    energy eigenvalue.
\end{fdfentry}

\begin{fdfentry}{Lindhard!LastBand}[integer]<$0$>
    Index of the last band to be included in the calculation; i.e., the highest
    energy eigenvalue.
\end{fdfentry}

\begin{fdfentry}{Lindhard!ngridx}[integer]<$0$>
    Number of k-grid points in the direction of the first reciprocal lattice
    vector. Must be an even number.
\end{fdfentry}

\begin{fdfentry}{Lindhard!ngridy}[integer]<$0$>
    Number of k-grid points in the direction of the second reciprocal lattice
    vector. Must be an even number.
\end{fdfentry}

\begin{fdfentry}{Lindhard!ngridz}[integer]<$0$>
    Number of k-grid points in the direction of the third reciprocal lattice
    vector. Must be an even number.
\end{fdfentry}

\begin{fdfentry}{Lindhard!nq1}[integer]<$1$>
    Step size in the first direction. In the direction of the first reciprocal
    lattice vector, calculate only every nq1 points. This only affects the
    granularity of the output, not the calculation itself.
\end{fdfentry}

\begin{fdfentry}{Lindhard!nq2}[integer]<$1$>
    Step size in the second direction. In the direction of the second reciprocal
    lattice vector, calculate only every nq2 points. This only affects the
    granularity of the output, not the calculation itself.
\end{fdfentry}