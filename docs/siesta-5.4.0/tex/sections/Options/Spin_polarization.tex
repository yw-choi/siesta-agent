

\begin{fdfentry}{Spin}[string]<non-polarized>
    \fdfdeprecates{SpinPolarized,NonCollinearSpin,SpinOrbit}
  
    Choose the spin-components in the simulation.
  
    \note This flag has precedence over \fdf*{SpinOrbit}, \fdf*{NonCollinearSpin} and
    \fdf*{SpinPolarized} while these deprecated flags may still be used.
    \begin{fdfoptions}
  
      \option[non-polarized]%
      \fdfindex*{Spin:non-polarized}%
      Perform a calculation with spin-degeneracy (only one component).
  
      \option[polarized]%
      \fdfindex*{SpinPolarized}%
      \fdfindex*{Spin:polarized}%
      Perform a calculation with colinear spin (two spin components).
  
      \option[non-colinear]%
      \fdfindex*{NonCollinearSpin}%
      \fdfindex*{Spin:non-colinear}%
      Perform a calculation with non-colinear spin (4 spin components),
      up-down and angles.
  
      Refs: T. Oda et al, PRL, \textbf{80}, 3622 (1998);
      V. M. Garc\'{\i}a-Su\'arez et al, Eur. Phys. Jour. B \textbf{40}, 371 (2004);
      V. M. Garc\'{\i}a-Su\'arez et al, Journal of
      Phys: Cond. Matt \textbf{16}, 5453 (2004).
  
      \option[spin-orbit]%
      \fdfindex*{SpinOrbit}%
      \fdfindex*{Spin:spin-orbit}%
      Performs calculations including the spin-orbit coupling. By default the
      full SO option is set. To perform an on-site SO calculation (see~\ref{sec:onsite-SOC})
      this option has to be \fdf*{spin-orbit+onsite}. This requires the
      pseudopotentials to be relativistic.
  
      See Sect.~\ref{sec:spin-orbit} for further specific spin-orbit options.
  
    \end{fdfoptions}
  
    \siesta\ can read a \sysfile*{DM} with different spin structure by
    adapting the information to the currently selected spin
    multiplicity, averaging or splitting the spin components equally, as
    needed. This may be used to greatly increase convergence.
  
    Certain options may not be used together with specific
    parallelization routines.
  
  \end{fdfentry}
  
  \begin{fdflogicalF}{Spin!Fix}
    \index{fixed spin state}\index{LSD}
    \fdfindex*{FixSpin}[|Spin!Fix]%
  
    If \fdftrue, the calculation is done with a fixed value of the spin
    of the system, defined by variable \fdf{Spin!Total}. This option can
    only be used for colinear spin polarized calculations.
  
  \end{fdflogicalF}
  
  \begin{fdfentry}{Spin!Total}[real]<$0$>
    \index{fixed spin state}\index{LSD}
    \index{spin}
    \fdfindex*{TotalSpin}[|Spin!Total]
  
    Value of the imposed total spin polarization of the system (in units
    of the electron spin, 1/2). It is only used if \fdf{Spin!Fix} \fdftrue.
  
  \end{fdfentry}
  
  \begin{fdfentry}{Spin!Spiral}[block]
    \fdfdepend{Spin}
  
    Specify the spiral $q$ vector for the non-collinear spin.
  
    \begin{fdfexample}
      Spin.Spiral.Scale ReciprocalLatticeVectors
      %block Spin.Spiral
        0. 0. 0.5
      %endblock
    \end{fdfexample}
  
    \note this option only applies for non-collinear spin (not for
    spin-orbit).
  
    \note this part of the code has not been tested, we would welcome
    any person who could assert its correctness and provide tests. Use
    with \emph{extreme} care.
  
  \end{fdfentry}
  
  \begin{fdfentry}{Spin!Spiral.Scale}[string]
    \fdfdepend{Spin!Spiral}
  
    Specifies the scale of the spiral vector $q$ vectors given in \fdf{Spin!Spiral}.
    The options are:
    \begin{fdfoptions}
      \option[pi/a]%
      vector is given in Cartesian coordinates, in units
      of $\pi/a$, where $a$ is the lattice constant (\fdf{LatticeConstant})
  
      \option[ReciprocalLatticeVectors]%
      vector is given in reciprocal-lattice-vector coordinates
  
    \end{fdfoptions}
  
  \end{fdfentry}
  
  \begin{fdflogicalF}{SingleExcitation}
  
    If \fdftrue, \siesta\ calculates a very rough approximation to the
    lowest excited state by swapping the populations of the HOMO and the
    LUMO. If there is no spin polarisation, it is half swap only.  It is
    done for the first spin component (up) and first $k$ vector.
  
  \end{fdflogicalF}
  