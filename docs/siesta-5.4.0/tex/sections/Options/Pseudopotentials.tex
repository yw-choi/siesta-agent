\siesta\ uses pseudopotentials to represent the electron-ion
interaction (as do most plane-wave codes and in contrast to so-called
``all-electron'' programs). In particular, the pseudopotentials are of
the ``norm-conserving'' kind.

The pseudopotentials will be read by \siesta\ from different files, according
to the species information in the block
\fdf{ChemicalSpeciesLabel}).\index{pseudopotential!files}
Recall that an optional \textit{ps-file-spec} can be present for each
species. If absent, \textit{ps-file-spec} defaults to the species'
label (\textit{Chemical\_label}).

The name of the files can be:
\begin{itemize}

\item \textit{ps-file-spec}\texttt{.vps} (unformatted) or
\item \textit{ps-file-spec}\texttt{.psf} (ASCII) or
\item \textit{ps-file-spec}\texttt{.psml} (PSML format)

\end{itemize}

\noindent

Files are searched by default in the current directory. In addition,
the environment variable \shell{SIESTA\_PS\_PATH} can be used to
provide a set of alternate paths in which to search for files.

The rules for pseudopotential file discovery, given
\textit{ps-file-spec}, are:

\begin{itemize}
  \item If \textit{ps-file-spec} does not have an extension, the
        following rules are applied in turn adding each of \shell{.vps,.psf,.psml}
        to \textit{ps-file-spec}, in that order of preference. The
        search ends with the first finding.

  \item If \textit{ps-file-spec} is not an absolute path,
    (e.g. \shell{Si} or \shell{C.psf}, or \shell{pbe/C.psml}), the
    search is done on the implied path (i.e. with reference to the
    current directory), and in each of the sections in
    \shell{SIESTA\_PS\_PATH} with \textit{ps-file-spec} (and possibly
    an extension) appended.

  \item If \textit{ps-file-spec} is an absolute path,
    (e.g. \shell{/home/user/Si} or \shell{/data/ps/C.psml}, the
    search is done \emph{only} in the path, possibly with
    an extension  appended.
\end{itemize}



\index{pseudopotential!ATOM code}
Pseudopotential files in the \shell{.psf} format  can be generated by the \program{ATOM} program,
(see \shell{Pseudo/README.ATOM}) and by a number of other codes such as
\program{APE}. The \shell{.vps} format is a binary version of the
\shell{.psf} format, and is deprecated.

\index{PSML format}\index{pseudopotential!PSML format}
Pseudopotential files in the PSML format (see~\citet{Garcia2018}) can
be produced by the combination of \program{ATOM} and \program{psop} (see directory
\shell{Pseudo/vnl-operator})\index{PSML!from \siesta's vnl-operator}
in a form fully compatible with the \siesta\ procedures to
generate the non-local pseudopotential operator.  Notably, they can
also be produced  by suitably patched versions of D.R. Hamann's \program{oncvpsp}
program (see directory
\shell{Pseudo/Third-Party-Tools/ONCVPSP}).\index{pseudopotential!oncvpsp
  code}\index{PSML!from oncvpsp code}. The \program{oncvpsp} code can
generate several projectors per $l$ channel, leading to
pseudopotentials that are more transferable.

For more information on the format itself and the PSML
ecosystem of generators and client ab-initio codes, please see
\url{http://esl.cecam.org/PSML}.

Note that curated databases of high-quality PSML files are available.
In particular, the Pseudo-Dojo project \url{https://www.pseudo-dojo.org}
offers PSML files for almost the whole periodic table, together with a
report of the tests carried out during the generation procedure.

In this connection, it should be stressed that \textbf{all
  pseudopotentials should be thoroughly tested} before using them. We
refer you to the standard literature on pseudopotentials, to the
\program{ATOM} manual, and to the Pseudo-Dojo site for more
information.

Please take into account the following when using PSML files:

\begin{itemize}

  \item If present in the execution directory, \shell{.psf} files take precedence
  over \shell{.psml} files.  That is, if both
  \textit{Chemical\_label}\texttt{.psf} and \textit{Chemical\_label}\texttt{.psml}
  are present, \siesta\ will process the former.

  \item PSML files typically contain semilocal potentials, a local
  potential, and non-local projectors. By default, \siesta\ will use the
  local potential and non-local projectors from the PSML file, unless the
  respective options \fdf{PSML!Vlocal} and \fdf{PSML!KB.projectors}
  are set to \fdffalse. These options are \fdftrue\ by default.
  Several combinations are  possible with these options:
  \begin{itemize}

    \item The recommended (and default) is to use the local potential and projectors
    from the PSML file.

    \item One could use only the semilocal potentials from the PSML file,
  and proceed to generate a local potential and KB projectors with the
  traditional \siesta\ algorithm.

  \item One could use the semilocal potentials and the local potential
    from the PSML file, and generate a set of KB projectors from them.

  \end{itemize}


  \item In order to generate its basis set of pseudo-atomic orbitals (PAOs),
  \siesta\ still needs the semilocal parts of the pseudopotential.
  Currently all available PSML files (generated by \program{ATOM+psop} or \program{ONCVPSP})
  contain semilocal potentials, but this might change in the future
  (for example, when a PSML file is obtained from a projectors-only
  UPF file). This restriction will be lifted in a later version:
  \siesta\ will then be able to use the full pseudopotential operator to
  generate the PAOs.

  \item For the \emph{full} (default) version of spin-orbit-coupling
    (SOC), SIESTA uses fully relativistic ($lj$) projectors. These are
    available in PSML files generated by \program{ONCVPSP} in
    fully-relativistic mode, if the \shell{psfile} option \shell{upf}
    or \shell{both} is used in the appropriate place in the input
    file. To obtain appropriate PSML files with the
    \program{ATOM+psop} chain (see the directory
    \shell{Pseudo/vnl-operator}), the projector generation with \program{psop}
    must use the \shell{-r} option.  Note that $lj$ projectors can
    still be directly generated by \siesta\ from relativistic
    semilocal potentials.

  \item Fully-relativistic PSML files with only $lj$ non-local projectors
  cannot be used directly in calculations not involving ``full''
  SOC. For this, \siesta\ needs the ``scalar-relativistic'' projectors. An
  algorithm for direct generation of SR projectors from an $lj$ set
  already exists as part of the \program{oncvpsp} code, and it will be
  integrated in a forthcoming version. In the meantime, while in
  principle it is possible to read only the semilocal potentials from
  the file and proceed to generate the appropriate projectors, it is
  better to use PSML files which contain both (actually three) sets of
  non-local projectors: ``sr'', ``so'', and $lj$. These can be obtained
  with \program{ONCVPSP} with the \shell{both} option. (For the \program{ATOM+psop} chain, it is
  currently necessary to run \program{psop} twice (once with the -r option) and
  generate two different PSML files, and then ``graft'' the ``sr'' set
  into the file containing the $lj$ set.)

  \item  A large number of PSML files obtained from the Pseudo-Dojo database
  are generated with (several) semicore shells. Dealing with them has
  uncovered a few weaknesses in the standard heuristics used
  traditionally in \siesta\ to generate basis sets:

  \begin{itemize}

     \item Sometimes it was not possible to execute successfully the
       legacy split-norm algorithm. Now, the default is to use
       \fdf{PAO!SplitTailNorm:true}, with a simpler, more robust algorithm.
       See the section on split-norm for full details.

     \item The default perturbative scheme for polarization orbitals
       can fail in very specific cases. When the polarization orbital
       has to have a node due to the presence of a lower-lying orbital
       with the same $l$, the program can (if enabled by the
       \fdf{PAO!Polarization!NonPerturbative.Fallback} option, which is
       \fdftrue by default)  automatically switch to
       using a non-perturbative scheme. In other cases, include the
       \textit{Chemical\_label} in the \fdf{PAO!Polarization!Scheme} block to request a
       non-perturbative scheme:

         \begin{fdfexample}
    %block PAO.Polarization.Scheme
      Mg non-perturbative
    %endblock PAO.Polarization.Scheme
         \end{fdfexample}

    Please see the relevant section for a fuller explanation.

    \item A number of improvements to the PAO generation code have
      been made while implementing support for PSML
      pseudopotentials. In particular, \siesta\ can now automatically
      detect and generate basis sets for atoms with semicore shells
      without the explicit use of a \fdf{PAO!Basis} block.

    \end{itemize}

\end{itemize}
