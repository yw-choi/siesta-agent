\begin{fdflogicalF}{OpticalCalculation}
    \index{Dielectric function,optical absorption}
  
    If specified, the imaginary part of the dielectric function will be
    calculated and stored in a file called \sysfile{EPSIMG}. The
    calculation is performed using the simplest approach based on the
    dipolar transition matrix elements between different eigenfunctions
    of the self-consistent Hamiltonian. For molecules the calculation is
    performed using the position operator matrix elements, while for
    solids the calculation is carried out in the momentum space
    formulation. Corrections due to the non-locality of the
    pseudopotentials are introduced in the usual way.
  
  \end{fdflogicalF}
  
  \begin{fdfentry}{Optical.Energy.Minimum}[energy]<$0\,\mathrm{Ry}$>
  
    This specifies the minimum of the energy range in which the
    frequency spectrum will be calculated.
  
  \end{fdfentry}
  
  \begin{fdfentry}{Optical.Energy.Maximum}[energy]<$10\,\mathrm{Ry}$>
  
    This specifies the maximum of the energy range in which the
    frequency spectrum will be calculated.
  
  \end{fdfentry}
  
  \begin{fdfentry}{Optical.Broaden}[energy]<$0\,\mathrm{Ry}$>
  
    If this is value is set then a Gaussian broadening will be applied
    to the frequency values.
  
  \end{fdfentry}
  
  \begin{fdfentry}{Optical.Scissor}[energy]<$0\,\mathrm{Ry}$>
  
    Because of the tendency of DFT calculations to under estimate the
    band gap, a rigid shift of the unoccupied states, known as the
    scissor operator, can be added to correct the gap and thereby
    improve the calculated results. This shift is only applied to the
    optical calculation and no where else within the calculation.
  
  \end{fdfentry}
  
  \begin{fdfentry}{Optical.NumberOfBands}[integer]<all bands>
  
    This option controls the number of bands that are included in the
    optical property calculation. Clearly this number must be larger
    than the number of occupied bands and less than or equal to the
    number of basis functions (which determines the number of unoccupied
    bands available). Note, while including all the bands may be the
    most accurate choice this will also be the most expensive!
  
  \end{fdfentry}
  
  \begin{fdfentry}{Optical.Mesh}[block]
  
    This block contains 3 numbers that determine the mesh size used for
    the integration across the Brillouin zone. For example:
    \begin{fdfexample}
        %block  Optical.Mesh
          5 5 5
        %endblock  Optical.Mesh
    \end{fdfexample}
    The three values represent the number of mesh points in the
    direction of each reciprocal lattice vector.
  
  \end{fdfentry}
  
  
  \begin{fdflogicalF}{Optical.OffsetMesh}
  
    If set to true, then the mesh is offset away from the gamma point
    for odd numbers of points.
  
  \end{fdflogicalF}
  
  \begin{fdfentry}{Optical.PolarizationType}[string]<polycrystal>
  
    This option has three possible values that represent the type of
    polarization to be used in the calculation. The options are
    \begin{fdfoptions}
      \option[polarized]%
      implies the application of an electric field in a given direction
  
      \option[unpolarized]%
      implies the propagation of light in a given direction
  
      \option[polycrystal]%
      In the case of the first two options a direction in space must be
      specified for the electric field or propagation using the
      \fdf{Optical.Vector} data block.
  
    \end{fdfoptions}
  
  \end{fdfentry}
  
  \begin{fdfentry}{Optical.Vector}[block]
  
    This block contains 3 numbers that specify the vector direction for
    either the electric field or light propagation, for a polarized or
    unpolarized calculation, respectively. A typical block might look
    like:
    \begin{fdfexample}
        %block  Optical.Vector
          1.0 0.0 0.5
        %endblock  Optical.Vector
    \end{fdfexample}
  
  \end{fdfentry}
  
  
  