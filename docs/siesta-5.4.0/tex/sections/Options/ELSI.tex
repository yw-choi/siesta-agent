The ELSI library provides a unified interface to a growing collection of solvers,
including ELPA (diagonalization), PEXSI, OMM, density-matrix
purification (using NTPoly), SIPS (based on spectrum
slicing), EigenExa, and MAGMA, and ChaSE. It can be downloaded and installed freely from
\url{http://elsi-interchange.org}.

The current interface in \siesta\ is based on the ``DM'' mode in ELSI:
sparse H and S matrices are passed to the library, and sparse DM and
EDM matrices are returned. Internally, the library performs any extra
operations needed.  K-points and/or spin-polarization (in collinear
form) are also supported, in fully parallel mode. This means that
k-points/spin-polarized calculations will be carried out splitting the
communicator for orbital distribution into nk$\times$nspin copies,
where nk is the number of k-points and nspin the number of spin
components. The total number of MPI processors must be a multiple of
nk$\times$nspin.

See Sec.~\ref{sec:libs} for details on installing \siesta\ with ELSI support.

\subsubsection{Input parameters}

The most important parameter is the solver to be used. Most other
options are given default values by the program which are nearly
always appropriate.

\begin{fdfentry}{ELSI!Solver}[string]<ELPA>
  A specification of the solver wanted. Possible values are
  \begin{itemize}
  \item elpa
  \item omm
  \item pexsi
  \item ntpoly
  \item sips
  \item eigenexa
  \item magma
  \end{itemize}
\end{fdfentry}

The EigenExa and MAGMA solvers are available after version
2.4.1 of the ELSI library, but they are optional, as are the
PEXSI and SIPS solvers. Only the solvers actually supported by
the ELSI library linked with \siesta\ can be used.

Note that the native interface in \siesta\ for an older version of the
PEXSI solver is still available (\texttt{-DWITH\_PEXSI=ON}).

There is also native support for the ELPA library (\texttt{-DWITH\_ELPA=ON}), which
can coexist with the ELSI interface if ELSI has been compiled with (the same)
external ELPA library.  The native OMM interface in
\siesta\ (enabled by default) has no coexistence problems with ELSI.

The compatibility options for libraries are handled by CMake, but some
corner cases might remain.

\begin{fdfentry}{ELSI!Broadening!Method}[string]<''fermi''>
The scheme for level broadening. Valid options are ``fermi'',
``gauss'', ``mp'' (Methfessel-Paxton), ``cold''
(Marzari-Vanderbilt cold smearing).
\end{fdfentry}

\begin{fdfentry}{ELSI!BuildDM!Method}[string]<''unset''>
The scheme for building the DM/EDM matrices. It is best to leave it unspecified, except when
running benchmarks or tests. Valid options are ``unset'' (no preference),
``elpa-multiply-at-a'' (not always available), ``elpa-multiply'' ``rank-update'' , ``pdgemm''
(not recommended, except for tests).
\end{fdfentry}

\begin{fdfentry}{ELSI!Output!Level}[integer]<0>
 The level of detail provided in the output. Possible values are 0, 1,
 2, and 3.
\end{fdfentry}

\begin{fdfentry}{ELSI!Output!Json}[integer]<1>
  If not 0, a separate log file in JSON format will be written out.
\end{fdfentry}

Other fdf options are mapped in a direct way to the options described in the
ELSI manual:

\begin{fdfentry}{ELSI!Broadening!MPOrder}[integer]<1>
\end{fdfentry}

\begin{fdfentry}{ELSI!Ill-Condition!Check}[integer]<0>
\end{fdfentry}

\begin{fdfentry}{ELSI!Ill-Condition!Tolerance}[real]<$10^{-5}$>
\end{fdfentry}

\begin{fdfentry}{ELSI!ELPA!Flavor}[integer]<2>
  Use the two-stage (conversion first to banded, then to tridiagonal form) of the ELPA
  solver by default.
\end{fdfentry}

\begin{fdfentry}{ELSI!ELPA!NSinglePrecision}[integer]<0>
  Number of initial scf steps carried out in single precision (if
  supported by the externally compiled version of the ELPA
  library). Typically, most scf steps can be computed in single
  precision, with only a final one or two in double precision, to
  achieve the standard results. The use of this option increases
  performance significantly.

\end{fdfentry}

\begin{fdfentry}{ELSI!ELPA!Autotune}[integer]<0>
  Use the auto-tune feature (if supported by the externally
  compiled version of the ELPA library). See the ELPA documentation
  for more details.
\end{fdfentry}

\begin{fdfentry}{ELSI!ELPA!GPU}[integer]<0>
  Request use of GPU acceleration (if supported by the compiled
  version of the ELPA library).
\end{fdfentry}

\begin{fdfentry}{ELSI!OMM!Flavor}[integer]<0>
\end{fdfentry}

\begin{fdfentry}{ELSI!OMM!ELPA!Steps}[integer]<3>
\end{fdfentry}

\begin{fdfentry}{ELSI!OMM!Tolerance}[real]<$10^{-9}$>
\end{fdfentry}

\begin{fdfentry}{ELSI!PEXSI!Method}[integer]< {\bf 3}>
  Method for computing the poles: 1: Contour integral,
  2: Minimax rational approximation, 3 (default): Adaptive Antoulas-Anderson
  (AAA). Please see the manual for a fuller discussion.

  Support for setting this parameter was introduced in ELSI in v2.4.1. If
  your library does not have it, please define the preprocessor flag
  \texttt{-DSIESTA\_\_ELSI\_DOES\_NOT\_HAVE\_PEXSI\_METHOD} at compile time.

\end{fdfentry}

\begin{fdfentry}{ELSI!PEXSI!TasksPerPole}[integer]< {\bf no default}>
  See the discussion on the three-level PEXSI parallelism in the ELSI manual.
\end{fdfentry}

\begin{fdfentry}{ELSI!PEXSI!TasksSymbolic}[integer]<1>
\end{fdfentry}

\begin{fdfentry}{ELSI!PEXSI!Number-of-Poles}[integer]<20>
\end{fdfentry}

\begin{fdfentry}{ELSI!PEXSI!Number-of-Mu-Points}[integer]<2>
\end{fdfentry}

\begin{fdfentry}{ELSI!PEXSI!Inertia-Tolerance}[real]<$0.05$>
\end{fdfentry}

\begin{fdfentry}{ELSI!PEXSI!Initial-Mu-Min}[real]<$-1.0\,\mathrm{Ry}$>
Initial lower bound for $\mu$ at the beginning of the scf loop.
\end{fdfentry}

\begin{fdfentry}{ELSI!PEXSI!Initial-Mu-Max}[real]<$0.0\,\mathrm{Ry}$>
Initial upper bound for $\mu$ at the beginning of the scf loop.
\end{fdfentry}

\begin{fdfentry}{ELSI!NTPoly!Method}[integer]<2>
\end{fdfentry}

\begin{fdfentry}{ELSI!NTPoly!Filter}[real]<$10^{-9}$>
\end{fdfentry}

\begin{fdfentry}{ELSI!NTPoly!Tolerance}[real]<$10^{-6}$>
\end{fdfentry}

\begin{fdfentry}{ELSI!SIPS!Slices}[integer]<{\bf no default}>
  See the discussion in the ELSI manual.
\end{fdfentry}

\begin{fdfentry}{ELSI!SIPS!ELPA!Steps}[integer]<2>
\end{fdfentry}

\begin{fdfentry}{ELSI!EigenExa!Method}[integer]<2>
Selects between the tridiagonalization (1) and pentadiagonalization
(2) methods. The latter is usually faster and more scalable.
\end{fdfentry}

\begin{fdfentry}{ELSI!MAGMA!Solver-Method}[integer]<1>
Selects between the 1-stage (1) and 2-stage (2) solvers in the MAGMA
library. Please see the documentation of the MAGMA library for the
details about compilation and support for accelerator devices.
\end{fdfentry}
