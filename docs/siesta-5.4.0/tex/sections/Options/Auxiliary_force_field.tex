It is possible to supplement the DFT interactions with a limited
set of force-field options, typically useful to simulate dispersion
interactions. It is not yet possible to turn off DFT and base the
dynamics only on the force field. The \program{GULP} program should be
used for that.

\begin{fdfentry}{MM!Potentials}[block]

  This block allows the input
  of molecular mechanics potentials between species. The following
  potentials are currently implemented:
  \begin{itemize}
    \item%
    C6, C8, C10 powers of the Tang-Toennes damped dispersion
    potential.

    \item%
    A harmonic interaction.

    \item%
    A dispersion potential of the Grimme type (similar to the C6
    type but with a different damping function). (See S. Grimme,
    J. Comput. Chem. Vol 27, 1787-1799 (2006)). See also
    \fdf{MM!Grimme.D} and \fdf{MM!Grimme.S6} below.

  \end{itemize}

  The format of the input is the two species numbers that are to
  interact, the potential name (C6, C8, C10, harm, or Grimme), followed
  by the potential parameters. For the damped dispersion potentials the
  first number is the coefficient and the second is the exponent of the
  damping term (i.e., a reciprocal length). A value of zero for the
  latter term implies no damping. For the harmonic potential the force
  constant is given first, followed by r0. For the Grimme potential C6
  is given first, followed by the (corrected) sum of the van der Waals
  radii for the interacting species (a real length). Positive values of
  the C6, C8, and C10 coefficients imply attractive potentials.

  \begin{fdfexample}
    %block MM.Potentials
      1 1 C6 32.0 2.0
      1 2 harm 3.0 1.4
      2 3 Grimme 6.0 3.2
    %endblock MM.Potentials
  \end{fdfexample}

  To automatically create input for Grimme's method, please see the
  utility: \program{Util/Grimme} which can read an fdf file and create
  the correct input for Grimme's method.

\end{fdfentry}

\begin{fdfentry}{MM!Cutoff}[length]<$30\,\mathrm{Bohr}$>

  Specifies the distance out to which molecular mechanics
  potential will act before being treated as going to zero.

\end{fdfentry}

\begin{fdfentry}{MM!UnitsEnergy}[unit]<eV>

  Specifies the units to be used for energy in the
  molecular mechanics potentials.

\end{fdfentry}

\begin{fdfentry}{MM!UnitsDistance}[unit]<Ang>

  Specifies the units to be used for distance in the
  molecular mechanics potentials.

\end{fdfentry}

\begin{fdfentry}{MM!Grimme.D}[real]<$20.0$>

  Specifies the scale factor $d$ for the scaling function
  in the Grimme dispersion potential (see above).

\end{fdfentry}

\begin{fdfentry}{MM!Grimme.S6}[real]<$1.66$>

  Specifies the overall fitting factor $s_6$ for the
  Grimme dispersion potential (see above). This number depends on the
  quality of the basis set, the exchange-correlation functional, and the
  fitting set.

\end{fdfentry}
