This calculation of the band structure is performed optionally after
the geometry loop finishes, and the output information written
to the \sysfile{bands} file (see below for the format).

\begin{fdfentry}{BandLinesScale}[string]<pi/a>

  Specifies the scale of the $k$ vectors given in \fdf{BandLines}
  and \fdf{BandPoints} below.  The options are:
  \begin{fdfoptions}
    \option[pi/a]%
    k-vector coordinates are given in Cartesian coordinates, in units
    of $\pi/a$, where $a$ is the lattice constant. NOTE: this results
    in a k-cell that is orthorhombic, so it might not be useful for systems
    in which the unit cell itself is non-orthorhombic.

    \option[ReciprocalLatticeVectors]%
    $k$ vectors are given in reciprocal-lattice-vector coordinates
  \end{fdfoptions}

  \note you might need to define explicitly a LatticeConstant tag in
  your fdf file if you do not already have one, and make it consistent
  with the scale of the k-points and any unit-cell vectors you might
  have already defined.

\end{fdfentry}

\begin{fdfentry}{BandLines}[block]

  Specifies the lines along which band energies are calculated
  (usually along high-symmetry directions).  An example for an FCC
  lattice is:
  \begin{fdfexample}
     %block BandLines
       1  1.000  1.000  1.000  L        # Begin at L
      20  0.000  0.000  0.000  \Gamma   # 20 points from L to gamma
      25  2.000  0.000  0.000  X        # 25 points from gamma to X
      30  2.000  2.000  2.000  \Gamma   # 30 points from X to gamma
     %endblock BandLines
  \end{fdfexample}
  where the last column is an optional \LaTeX\ label for use in the
  band plot. If only given points (not lines) are required, simply
  specify 1 in the first column of each line. The first column of the
  first line must be always 1.

  \note this block is not used if \fdf{BandPoints} is present.

\end{fdfentry}

\begin{fdfentry}{BandPoints}[block]

  Band energies are calculated for the list of arbitrary $k$ points
  given in the block. Units defined by \fdf{BandLinesScale} as for
  \fdf{BandLines}. The generated \sysfile{bands} file will contain the
  $k$ point coordinates (in a.u.) and the corresponding band energies
  (in eV). Example:
  \begin{fdfexample}
     %block BandPoints
        0.000  0.000  0.000   # This is a comment. eg this is gamma
        1.000  0.000  0.000
        0.500  0.500  0.500
     %endblock BandPoints
  \end{fdfexample}

  See also \fdf{BandLines}.
\end{fdfentry}


\begin{fdflogicalF}{WriteKbands}
  \index{output!band $\vec k$ points}

  If \fdftrue, it writes the coordinates of the $\vec k$ vectors
  defined for band plotting, to the main output file.

\end{fdflogicalF}

\begin{fdflogicalF}{WriteBands}
  \index{output!band structure}%
  \index{band structure}

  If \fdftrue, it writes the Hamiltonian eigenvalues corresponding to
  the $\vec k$ vectors defined for band plotting, in the main output
  file.

\end{fdflogicalF}


\subsubsection{Format of the .bands file}

\begin{verbatim}

FermiEnergy (all energies in eV) \\
kmin, kmax (along the k-lines path, i.e. range of k in the band plot) \\
Emin, Emax (range of all eigenvalues) \\
NumberOfBands, NumberOfSpins (1 or 2), NumberOfkPoints \\
k1, ((ek(iband,ispin,1),iband=1,NumberOfBands),ispin=1,NumberOfSpins) \\
k2, ek \\
 . \\
 . \\
 . \\
klast, ek \\
NumberOfkLines \\
kAtBegOfLine1, kPointLabel \\
kAtEndOfLine1, kPointLabel \\
  . \\
  . \\
  . \\
kAtEndOfLastLine, kPointLabel \\
\end{verbatim}

\noindent
The \program{gnubands}\index{gnubands@\texttt{gnubands}} postprocessing
utility program (found in the \shell{Util/Bands} directory) reads the
\sysfile{bands} for plotting.  See the \fdf{BandLines} data
descriptor above for more information.


\subsubsection{Output of wavefunctions associated to bands}
\label{sec:wf-bands}

The user can optionally request that the wavefunctions corresponding
to the computed bands be written to file.  They are written to the
\sysfile{bands.WFSX} file.
The relevant options are:

\begin{fdflogicalF}{WFS.Write.For.Bands}
  \index{output of wave functions for bands}

  Instructs the program to compute and write the wave functions
  associated to the bands specified (by a \fdf{BandLines} or a
  \fdf{BandPoints} block) to the file \sysfile{WFSX}.

  The information in this file might be useful, among other things, to
  generate ``fatbands'' plots, in which both band eigenvalues and
  information about orbital projections is presented.
  \index{fatbands} See the \program{fat} program in the
  \texttt{Util/COOP} directory for details.

\end{fdflogicalF}

\begin{fdfentry}{WFS.Band.Min}[integer]<1>
  \index{output of wave functions for bands}

  Specifies the lowest band index of the wave-functions to be written
  to the file \sysfile{WFSX} for each $k$-point (all $k$-points in the
  band set are affected).

\end{fdfentry}

\begin{fdfentry}{WFS.Band.Max}[integer]<number of orbitals>
  \index{output of wave functions for bands}

  Specifies the highest band index of the wave-functions to be written
  to the file \sysfile{WFSX} for each $k$-point (all $k$-points in the
  band set are affected).

\end{fdfentry}
