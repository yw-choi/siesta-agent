
  \program{wannier90} (\url{http://www.wannier.org}) is a code to generate
  maximally localized wannier functions according to the original
  Marzari and Vanderbilt recipe.

  A wrapper interface between \program{siesta} and \program{wannier90}
  (version 3.1.0) has been implemented, so that \program{wannier90} can
  be called from \program{siesta} on-the-fly, as well as used as a
  post-processing tool.

  It is strongly recommended to read the original papers on which this
  method is based and the  documentation of \program{wannier90} code.
  Here we shall focus only on those internal \siesta\ variables
  required to produce the files that will be processed
  by \program{wannier90}.

  \subsubsection{\program{wannier90}\, as a postprocessing tool}

  This interface is analogous to that found in other programs. The user
  first runs \program{wannier90} in pre-processing mode to get a \program{.nnkp} file.
  Then \siesta\ is run with the appropriate options to generate the files needed
  by a wannierization run with \program{wannier90}.

  A complete list of examples and tests (including molecules, metals,
  semiconductors, insulators, magnetic systems, plotting of Fermi surfaces
  or interpolation of bands), can be downloaded from

   \url{http://personales.unican.es/junqueraj/Wannier-examples.tar.gz}

  \textbf{NOTE}: The Bloch functions produced by a first-principles code
        have arbitrary phases that depend on the number of processors
        used and other possibly non-reproducible details of the
        calculation. In what follows it is essential to maintain
        consistency in the handling of the overlap and Bloch-function
        files produced and fed to \program{wannier90}.


  \begin{fdflogicalF}{Siesta2Wannier90.WriteMmn}

    This flag determines whether the overlaps between the periodic part
    of the Bloch states at neighbour k-points are computed and dumped
    into a file in the format required by \program{wannier90}.  These
    overlaps are defined in Eq. (27) in the paper by N. Marzari
    \textit{et al.}, Review of Modern Physics \textbf{84}, 1419 (2012),
    or Eq. (1.7) of the Wannier90 User Guide, Version 2.0.1.

    The k-points for which the overlaps will be computed are read from a
    \sysfile*{nnkp} file produced by \program{wannier90}. It is strongly
    recommended for the user to read the corresponding user guide.

    The overlap matrices are written in a file with extension
    \sysfile*{mmn}.

  \end{fdflogicalF}

  \begin{fdflogicalF}{Siesta2Wannier90.WriteAmn}

    This flag determines whether the overlaps between Bloch states and
    trial localized orbitals are computed and dumped into a file in the
    format required by \program{wannier90}.  These projections are
    defined in Eq. (16) in the paper by N. Marzari \textit{et al.},
    Review of Modern Physics \textbf{84}, 1419 (2012), or Eq. (1.8) of
    the Wannier90 User Guide, Version 2.0.1.

    The localized trial functions to use are taken from the
    \sysfile*{nnkp} file produced by \program{wannier90}. It is strongly
    recommended for the user to read the corresponding user guide.

    The overlap matrices are written in a file with extension
    \sysfile*{amn}.

  \end{fdflogicalF}

  \begin{fdflogicalF}{Siesta2Wannier90.WriteEig}

    Flag that determines whether the Kohn-Sham eigenvalues (in eV) at
    each point in the Monkhorst-Pack mesh required by
    \program{wannier90} are written to file.  This file is mandatory in
    \program{wannier90} if any of disentanglement, plot\_bands,
    plot\_fermi\_surface or hr\_plot options are set to true in the
    \program{wannier90} input file.

    The eigenvalues are written in a file with extension \sysfile*{eigW}.
    This extension is chosen to avoid name clashes with \siesta's
    standard eigenvalue file in case-insensitive filesystems.

  \end{fdflogicalF}

  \begin{fdflogicalF}{Siesta2Wannier90.WriteUnk}

    Produces \file{UNKXXXXX.Y} files which contain the periodic part
    of a Bloch function in the unit cell on a grid given by global
    unk\_nx, unk\_ny, unk\_nz variables.  The name of the output files
    is assumed to have the previous form, where the \texttt{XXXXXX}
    refer to the k-point index (from 00001 to the total number of
    k-points considered), and the \texttt{Y} refers to the spin
    component (1 or 2)

    The periodic part of the Bloch functions is defined by
    \begin{equation}
      u_{n \vec{k}} (\vec{r}) =
      \sum_{\vec{R} \mu} c_{n \mu}(\vec{k})
      e^{i \vec{k} \cdot ( \vec{r}_{\mu} + \vec{R} - \vec{r} )}
      \phi_{\mu} (\vec{r} - \vec{r}_{\mu} - \vec{R} ) ,
    \end{equation}
    where $\phi_{\mu} (\vec{r} - \vec{r}_{\mu} - \vec{R} )$ is a basis
    set atomic orbital centered on atom $\mu$ in the unit cell
    $\vec{R}$, and $c_{n \mu}(\vec{k})$ are the coefficients of the wave
    function. The latter must be identical to the ones used for
    wannierization in $M_{mn}$. (See the above comment about arbitrary
    phases.)

  \end{fdflogicalF}

  \begin{fdfentry}{Siesta2Wannier90.UnkGrid1}[integer]<\nonvalue{mesh
        points along $A$}>

    Number of points along the first lattice vector in the grid where
    the periodic part of the wave functions will be plotted.

  \end{fdfentry}

  \begin{fdfentry}{Siesta2Wannier90.UnkGrid2}[integer]<\nonvalue{mesh
        points along $B$}>

    Number of points along the second lattice vector in the grid where
    the periodic part of the wave functions will be plotted.

  \end{fdfentry}

  \begin{fdfentry}{Siesta2Wannier90.UnkGrid3}[integer]<\nonvalue{mesh
        points along $C$}>

    Number of points along the third lattice vector in the grid where
    the periodic part of the wave functions will be plotted.

  \end{fdfentry}


  \begin{fdflogicalT}{Siesta2Wannier90.UnkGridBinary}

    Flag that determines whether the periodic part of the wave function
    in the real space grid is written in binary format (default) or in
    ASCII format.

  \end{fdflogicalT}

  \begin{fdfentry}{Siesta2Wannier90.NumberOfBands}[integer]<occupied bands>

    In spin unpolarized calculations, number of bands that will be
    initially considered by \siesta\ to generate the information
    required by \program{wannier90}. Note that it should be at least as
    large as the index of the highest-lying band in the
    \program{wannier90} post-processing. For example, if the
    wannierization is going to involve bands 3 to 5, the \siesta\ number
    of bands should be at least 5. Bands 1 and 2 should appear in a
    ``excluded'' list.

    \note you are highly encouraged to explicitly specify the number of
    bands.

  \end{fdfentry}

  \begin{fdfentry}{Siesta2Wannier90.NumberOfBandsUp}[integer]<\fdfvalue{Siesta2Wannier90.NumberOfBands}>

    In spin-polarized calculations, number of bands with spin up that
    will be initially considered by \siesta\ to generate the information
    required by \program{wannier90}.

  \end{fdfentry}

  \begin{fdfentry}{Siesta2Wannier90.NumberOfBandsDown}[integer]<\fdfvalue{Siesta2Wannier90.NumberOfBands}>

    In spin-polarized calculations, number of bands with spin down that
    will be initially considered by \siesta\ to generate the information
    required by \program{wannier90}.

  \end{fdfentry}


  \subsubsection{\program{wannier90}\, called on-the-fly within
      \program{siesta}}
  \label{sec:w90:library}

   A wrapper interface to \program{wannier90} can be compiled and called directly
   from \program{siesta}. This presents several advantages:

   \begin{itemize}
     \item No need to prepare two different input files.
     \item No need to run \program{wannier90} in pre-processing mode.
     \item We can use the basis set of \program{siesta} (numerical
           atomic orbitals) as the initial guess for the projections.
     \item Wannierization of different manifolds can be done in the
           same run of \program{siesta}.
     \item The unitary matrices connecting the Bloch and the Wannier
           representations are available within \program{siesta}.
     \item The coefficients of the Wannier functions in the
           basis of the atomic orbitals of the supercell in \program{siesta}
           are written in a file with \sysfile{WANNX} extension. Then,
           the Wannier functions can be plotted using
           \program{denchar}, following the same method as for the wave functions.
   \end{itemize}

   Further details of the compilation of \siesta\ with this
   functionality can be found in the file
   \file{External/Wannier/README.md}. Note also the (slightly outdated) presentation
   \url{https://personales.unican.es/junqueraj/JavierJunquera_files/Metodos/Wannier/Exercise-Wannier90-within-siesta.pdf}

  \begin{fdfentry}{Wannier!Manifolds}[block]

    Each line denotes the name of a manifold to be processed by Wannier90.

    Options for each manifold is specified in the \fdf{Wannier!Manifold.<>}

  \end{fdfentry}


  \begin{fdfentry}{Wannier!Manifold.<>}[block]

    Each line represents a setting for the Wannier manifold to be
    processed.

    \begin{fdfoptions}

      \option[bands]%
      \fdfindex*{Wannier!Manifold.<>!bands}%

      Two integers specifying the initial and final band of the manifold
      to be wannierized.

      \note required input


      \option[trial-orbitals]%
      \fdfindex*{Wannier!Manifold.<>!trial-orbitals}%

      Indices of the orbitals that will be used
      as localized trial orbitals in the first step of the minimization
      of the spreading.
      The user has to specify the same number of atomic orbitals as the number
      of Wannier functions required.
      For the sake of readiness, the number of trial orbitals can be split in
      several lines, all of them starting with \fdf*{trial-orbitals}.
      These indices can be found by inspection of the
      \file{SystemLabel.ORB\_INDX} file.
      If there are negative integers in this line, then the projectors will
      be generated \`a-la-Wannier90, with the instructions given in the
      WannierProjectors block.

      If there are negative numbers, it is not strictly required that
      they must appear after the list of positive indices.

      \note required input

      \option[spreading.nitt]%
      \fdfindex*{Wannier!Manifold.<>!spreading.nitt}%

      Number of iterations that \program{wannier90}
      will carry out to minimize the spreading.
      If zero, then the procedure is the same as a L\"owdin orthonormalization.
      In such a case, the resulting Wannier function will keep the symmetry
      of the trial projection function, but it will not be maximally localized.

      \option[wannier-plot]%
      \fdfindex*{Wannier!Manifold.<>!wannier-plot}%

      Instructs \program{wannier90} to produce the files
      required to plot the Wannier functions
      (if \fdf{w90.in.siesta.compute.unk} is set to true).
      The integer refers to the size of the supercell for plotting the
      Wannier functions (see the variable
      \fdf{wannier.plot.supercell} in the \program{wannier90} User's Guide).
      This will produce files with the .xsf extension, that can be directly plotted
      with \program{xcrysden}.

      \option[fermi-surface-plot]%
      \fdfindex*{Wannier!Manifold.<>!fermi-surface-plot}%

      Is the file required to plot the Fermi surface computed?
      If true, this will produce files with the .bxsf extension, that can be directly plotted
      with \program{xcrysden}.

      \option[write-hr]%
      \fdfindex*{Wannier!Manifold.<>!write-hr}%

      Is the file with the Hamiltonian in real space in a basis of Wannier functions written?
      If true, this will produce files with the \_hr.dat extension.


      \option[write-tb]%
      \fdfindex*{Wannier!Manifold.<>!write-tb}%

      Is the file with the tight-binding parameters in a basis of Wannier functions written?
      (this includes the lattice vectors, Hamiltonian
      in real space, and position operator in a basis of Wannier functions.
      If true, this will produce files with the \_tb.dat extension.

      \option[write-unk]%
      \fdfindex*{Wannier!Manifold.<>!write-unk}%

      Are the files that contain the periodic part
      of a Bloch function in the unit cell on a grid computed?
      If true, files like those described in \fdf{Siesta2Wannier90.WriteUnk}
      are written for this manifold, and then
      the corresponding .xsf files directly readable by \program{xcrysden}
      will be produced.
      The computation of the UNK files might be rather expensive.
      To plot the shape of the Wannier functions, the expansion of the Wannier functions
      in the basis of Numerical Atomic Orbitals to produce .WFSX files, and the
      subsequent use of \program{denchar} is recommended.

      If a disentanglement procedure is required two extra lines are mandatory:

      \option[window]%
      \fdfindex*{Wannier!Manifold.<>!window}%

      It refers to the bottom and top of the outer
      energy window for band disentanglement. The units for the energy are
      introduced as the last character string of the line.

      \option[window.frozen]%
      \fdfindex*{Wannier!Manifold.<>!window.frozen}%

      It refers to the bottom and top of the inner
      energy window for band disentanglement. The units for the energy are
      introduced as the last character string of the line.
      This is the energy window where some Bloch states are forced to be
      preserved identically in the projected manifold.

      \option[threshold]%
      \fdfindex*{Wannier!Manifold.<>!window.threshold}%

      Specification of the threshold for the real part of the
      coefficients of a Wannier in a basis of NAO that will be written in
      a \sysfile{WANNX} extension file.
      This file can be used to plot the Wannier functions using
      \program{denchar}, following the same method as for the wave functions.
      This threshold is particularized for a particular manifold.

    \end{fdfoptions}

    An example of a manifold:

    \begin{fdfexample}
  %block Wannier.Manifold.second
    bands 21 23
    trial-orbitals [24 25 27]
    spreading.nitt 0
    wannier_plot 3
    fermi_surface_plot true
    write_hr true
    write_tb true
  %endblock
    \end{fdfexample}

  \end{fdfentry}



  \iffalse
  % These lines should be deleted once documentation is complete
  \begin{fdfentry}{NumberOfBandManifoldsForWannier}[integer]<\fdfvalue{Number of band manifolds to be wannierized}>

    Number of band manifolds that will be considered for Wannierization.

    Multiple wannierizations on different band manifolds can be performed in the same
    \program{siesta} run.

    \note The default value will be 0.

  \end{fdfentry}

  \begin{fdfentry}{WannierManifolds}[block]
  <\fdfvalue{Information on the band manifolds to be wannierized}>

    Specifies all the information required to Wannierized a given
    manifold of bands.
    An example for an isolated manifold (top of the valence band of
    bulk SrTiO$_{3}$) is:

    \begin{fdfexample}
       %block WannierManifolds
         1                  # Sequential index of the manifold
         12  20             # Indices of the initial and final band of the manifold
         9                  # Number of bands for Wannier transformation
         36 37 38 49 50 51 62 63 64  # Indices of the orbitals to be used as localized trial orbitals
         num_iter 0         # Number of iterations for the minimization of \Omega
         wannier_plot  3    # Plot the Wannier function
         fermi_surface_plot # Plot the Fermi surface
         write_hr           # Write the Hamiltonian in the WF basis
         write_tb           # Write lattice vectors, Hamiltonian, and position operator in WF basis
       %endblock WannierManifolds
    \end{fdfexample}

    The first line is the sequential index of the manifold,
    from 1 to \fdf{NumberOfBandManifoldsForWannier}.
    The lines of this block must be repeated as many times as the
    number of band manifolds that will be wannierized.

    The second line refers to the initial and final band of the manifold
    to be wannierized.

    The third line is the number of bands that will be retained for
    wannierization.
    The number of bands retained for Wannierization must be the same
    as the number of localized orbitals that will be introduced
    as a first guess in the following lines.
    This number is equal to the number of bands in the manifold if we are
    dealing with isolated group of bands.
    But if the desired bands lie within a limited energy range but
    overlap and hybridize with other bands which extend further in energy
    (as is the case in metals),
    then the number of Wannier functions might be different from the
    number of bands in the manifolds.
    A disentanglement procedure is required, that will be carried out
    following the recipe given by
    I. Souza, N. Marzari and D. Vanderbilt,
    Phys. Rev. B \textbf{65}, 035109 (2002)
    "Maximally localized Wannier functions for entangled energy bands".
    If there are more than two integers in this third line,
    the total number of bands for Wannierization will be the product
    of the two, and we will need as many lines to introduce the
    initial localized functions as the number indicated by the second integer.

    The fourth line (and subsequents if there are more than two integers
    in the third line) refers to the indices of the orbitals that will be used
    as localized trial orbitals in the first step of the minimization
    of the spreading.
    The user has to specify the same number of atomic orbitals as the number
    of Wannier functions required.
    These indices can be found by inspection of the
    \file{SystemLabel.ORB\_INDX} file.
    If there are negative integers in this line, then the projectors will
    be generated \`a-la-Wannier90, with the instructions given in the
    WannierProjectors block.
    If there are negative numbers, then they must appear always after
    the list of positive indices.

    The fifth line is the number of iterations that \program{wannier90}
    will carry out to minimize the spreading.
    If zero, then the procedure is the same as a L\"owdin orthonormalization.
    In such a case, the resulting Wannier function will keep the symmetry
    of the trial projection function, but it will not be maximally localized.

    The sixth function instructs \program{wannier90} to produce the files
    required to plot the Wannier functions
    (if \fdf{w90.in.siesta.compute.unk} is set to true).
    The integer refers to the size of the supercell for plotting the
    Wannier functions (see the variable
    \fdf{wannier.plot.supercell} in the \program{wannier90} User's Guide).
    This will produce files with the .xsf extension, that can be directly plotted
    with \program{xcrysden}.

    The seventh line instructs \program{wannier90} to produce the files
    required to plot the Fermi surface.
    This will produce files with the .bxsf extension, that can be directly plotted
    with \program{xcrysden}.

    The eigth line is a instruction to write the Hamiltonian in real space
    in a basis of Wannier functions.
    This will produce files with the \_hr.dat extension.

    The nineth line is a instuction to write the lattice vectors, Hamiltonian
    in real space, and position operator in a basis of Wannier functions.
    This will produce files with the \_tb.dat extension.

    If a disentanglement procedure is required two extra lines are mandatory:

    The first extra line refers to the bottom and top of the outer
    energy window for band disentanglement (in eV).

    The second extra line refers to the bottom and top of the inner
    energy window for band disentanglement (in eV).
    This is the energy window where some Bloch states are forced to be
    preserved identically in the projected manifold.

  \end{fdfentry}

  \fi

  \begin{fdfentry}{Wannier!Projectors}[block]
  <\fdfvalue{projection functions as in wannier90}>

  Information on the projection functions \`a-la-\program{wannier90},
  used to construct the
  initial guesses for the unitary transformations.

  These are used when some of the atomic orbitals in the
  \fdf{trial-orbitals} lines of the block \fdf{Wannier!Manifold.<>} are negative.

   For instance, to specify the projectors for the bottom of the
   conduction band of bulk SrTiO$_{3}$, we can write a
   block \fdf{Wannier!Manifold.<>} as

   \begin{fdfexample}
  %block Wannier.Manifold.example
     # Indices of the initial and final band of the manifold
     bands  21  23
     # Number of bands for Wannier transformation
     trial-orbitals  -1 -2 -3
     spreading.nitt 0         # Number of iterations for the minimization of \Omega
     wannier-plot  3    # Plot the Wannier function
     fermi-surface-plot # Plot the Fermi surface
     write-hr           # Write the Hamiltonian in the WF basis
     write-tb           # Write lattice vectors, Hamiltonian, and position operator in WF basis
  %endblock
    \end{fdfexample}

   Then, the three projector functions that will be generated
   following the recipe of \program{wannier90} will be

    \begin{fdfexample}
  %block Wannier.Projectors.example
     0.5 0.5 0.5  2  2  1 0.00  0.00  1.00   1.00  0.00  0.00    1.00
     0.5 0.5 0.5  2  3  1 0.00  0.00  1.00   1.00  0.00  0.00    1.00
     0.5 0.5 0.5  2  5  1 0.00  0.00  1.00   1.00  0.00  0.00    1.00
  %endblock
    \end{fdfexample}

   The different lines in this block are written following the \program{wannier90} format
   provided in the .nnkp file.

   centre: three real numbers; projection function centre in crystallographic
   co-ordinates relative to the direct lattice vectors.

   $l$ $m_{r}$ $r$: three integers; $l$ and $m_{r}$ specify the angular part
   $\Theta_{l m_{r}}(\theta,\varphi)$, and $r$ specifies the radial part
   $R_{r} (r)$ of the projection function (see Tables 3.1, 3.2 and 3.3
   of the \program{wannier90} User's Guide).

   z-axis: three real numbers; default is 0.0 0.0 1.0;
   defines the axis from which the polar angle $\theta$
   in spherical polar coordinates is measured.

   x-axis: three real numbers; must be orthogonal to z-axis\;
   default is 1.0 0.0 0.0 or a vector perpendicular to z-axis
   if z-axis is given; defines the axis from with the azimuthal
   angle $\varphi$ in spherical polar coordinates is measured.

   zona: real number; the value of $\frac{Z}{a}$ associated with
   the radial part of the atomic orbital. Units are in reciprocal Angstrom.

  \end{fdfentry}


  \begin{fdfentry}{Wannier!Manifolds!Threshold}[real]<$10^{-6}$>

    Global specification of the threshold for the real part of the
    coefficients of a Wannier in a basis of NAO that will be written in
    a \sysfile{WANNX} extension file.
    This file can be used to plot the Wannier functions using
    \program{denchar}, following the same method as for the wave functions.

    Individual manifolds can be controlled via
    \fdf{Wannier!Manifold.<>!threshold}

  \end{fdfentry}

  \begin{fdflogicalF}{Wannier!Manifolds!Unk}

    Global flag that determines whether the periodic part of the wave function
    in the real space grid will be computed (as using Siesta2Wannier90.WriteUnk),
    and whether the files \file{xsf} directly readable by \program{xcrysden}
    will be produced.

    Individual manifolds can be controlled via
    \fdf{Wannier!Manifold.<>!write-unk}.

    The computation of the UNK files might be rather expensive.
    To plot the shape of the Wannier functions, the expansion of the Wannier functions
    in the basis of Numerical Atomic Orbitals to produce \sysfile{WFSX} files, and the
    subsequent use of \program{denchar} is recommended.

  \end{fdflogicalF}

  \begin{fdfentry}{Wannier!k}[list/block]<$\Gamma$-point>
    \index{output!kpoint mesh for Wannierization}

    Dimension of the Monkhorst-Pack grid of k-points that will be used
    during the wannierization.
    The overlap matrices between periodic parts of the wavefunctions
    at neighbour k-points in this grid will be computed.

    \begin{fdfexample}
  Wannier.k [4 4 4]
  # Or equivalently
  %block Wannier.k
     4  4  4
  %endblock
    \end{fdfexample}
  \end{fdfentry}
