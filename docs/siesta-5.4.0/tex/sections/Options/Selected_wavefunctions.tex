The user can optionally request that specific wavefunctions are
written to file. These wavefunctions are re-computed after the
geometry loop (if any) finishes, using the last (presumably converged)
density matrix produced during the last self-consistent field loop
(after a final mixing). They are written to the
\sysfile{selected.WFSX} file.

Note that the complete set of wavefunctions obtained during the last
iteration of the SCF loop will be written to SystemLabel.fullBZ.WFSX
if the \fdf{COOP.Write} option is in effect.

Note that the complete set of wavefunctions obtained during the last
iteration of the SCF loop will be written to a NetCDF file
\file{WFS.nc} if the \fdf{Diag!UseNewDiagk} option is in effect.

\begin{fdfentry}{WaveFuncKPointsScale}[string]<pi/a>

    Specifies the scale of the $k$ vectors given in
    \fdf{WaveFuncKPoints} below.  The options are:
    \begin{fdfoptions}
      \option[pi/a]%
      k-vector coordinates are given in Cartesian coordinates, in units
      of $\pi/a$, where $a$ is the lattice constant
  
      \option[ReciprocalLatticeVectors]%
      $k$ vectors are given in reciprocal-lattice-vector coordinates
  
    \end{fdfoptions}
  
  \end{fdfentry}
  
  \begin{fdfentry}{WaveFuncKPoints}[block]
  
    Specifies the $k$-points at which the electronic wavefunction
    coefficients are written.  An example for an FCC lattice is:
    \begin{fdfexample}
       %block WaveFuncKPoints
       0.000  0.000  0.000  from 1 to 10   # Gamma wavefuncs 1 to 10
       2.000  0.000  0.000  1 3 5          # X wavefuncs 1,3 and 5
       1.500  1.500  1.500                 # K wavefuncs, all
       %endblock WaveFuncKPoints
    \end{fdfexample}
    The index of a wavefunction is defined by its energy, so that the
    first one has lowest energy.
  
    The user can also narrow the energy-range used with the
    \fdf{WFS.Energy.Min} and \fdf{WFS.Energy.Max} options (both take
    an energy (with units) as extra argument -- see
    section~\ref{sec:coop}). Care should be taken to make sure that the
    actual values of the options make sense.
  
    The output of the wavefunctions in described in Section
    \ref{sec:wf-output-user}.
  
  \end{fdfentry}
  
  
  \begin{fdflogicalF}{WriteWaveFunctions}
    \index{output!wave functions}
  
    If \fdftrue, it writes to the output file a list of the
    wavefunctions actually written to the \sysfile{selected.WFSX} file,
    which is always produced.
  
  \end{fdflogicalF}
  
  The unformatted WFSX file contains the information of the
  k-points for which wavefunctions coefficients are written, and the
  energies and coefficients of each wavefunction which was specified in
  the input file (see \fdf{WaveFuncKPoints} descriptor above). It also contains information
  on the atomic species and the orbitals for postprocessing purposes.
  
  \textbf{NOTE:} The \sysfile{WFSX} file is in a more compact
  form than the old WFS, and the wavefunctions are output in single
  precision. The \program{Util/WFS/wfsx2wfs} program can be used to
  convert to the old format.
  
  \noindent
  The \program{readwf}\index{readwf} and
  \program{readwfsx}\index{readwfsx} postprocessing utilities programs
  (found in the \shell{Util/WFS} directory) read the \sysfile{WFS} or
  \sysfile{WFSX} files, respectively, and generate a readable file.