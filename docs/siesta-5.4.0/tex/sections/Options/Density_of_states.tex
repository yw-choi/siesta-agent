\subsubsection{Total density of states}
There are several options to obtain the
total density of states:
\begin{itemize}
  \index{output!eigenvalues}

  \item The Hamiltonian eigenvalues for the SCF sampling $\vec k$ points can be
  dumped into \sysfile{EIG} in a format analogous to SystemLabel.bands,
  but without the kmin, kmax, emin, emax information, and without
  the abscissa. The \program{Eig2DOS}\index{Eig2DOS@\textsc{Eig2DOS}}
  postprocessing utility can be then used to obtain the density of
  states.\index{density of states}
  See the \fdf{WriteEigenvalues} descriptor.

  \item As a side-product of a partial-density-of-states calculation
  (see below)

  \item As one of the files produced by the \program{Util/COOP/mprop} during
  the off-line analysis of the electronic structure. This method
  allows the flexibility of specifying energy ranges and resolutions
  at will, without re-running \siesta\ See Sec.~\ref{sec:coop}.

  \item Using the inertia-counting routines in the PEXSI solver (see
  Sec.~\ref{pexsi-dos}).

\end{itemize}

The k-point specification for the partial and local density of states
calculations described in the following two sections may optionally be
given by

\begin{fdfentry}{DOS.kgrid.?}<kgrid.?>

  The generic DOS k-grid specification.

  See Sec.~\ref{ssec:k-points} for details. If \emph{any} of
  \fdf*{DOS.kgrid.MonkhorstPack}, \fdf*{DOS.kgrid.Cutoff} or
  \fdf*{DOS.kgrid.File} is present, they will be used, otherwise fall
  back to the SCF k-point sampling (\fdf*{kgrid.?}).

  \note \fdf{DOS.kgrid.?} options are the default values for
  \fdf{ProjectedDensityOfStates} and \fdf{LocalDensityOfStates}, but
  they do not affect the sampling used to generate the \sysfile{EIG}
  file. This feature might be implemented in a later version.

\end{fdfentry}


\subsubsection{Partial (projected) density of states}

There are two options to obtain the partial density of states
\begin{itemize}

  \item Using the options below

  \item Using the \program{Util/COOP/mprop} program for the off-line analysis of
  the electronic structure in PDOS mode. This method allows the
  flexibility of specifying energy ranges, orbitals, and resolutions
  at will, without re-running \siesta. See Sec.~\ref{sec:coop}.

\end{itemize}

\begin{fdfentry}{ProjectedDensityOfStates}[block]
  \index{output!projected density of states}

  Instructs to write the Total Density Of States (Total DOS) and the
  Projected Density Of States (PDOS) on the basis orbitals, between
  two given energies, in files \sysfile{DOS} and \sysfile{PDOS},
  respectively.  The block must be a single line with the energies of
  the range for PDOS projection, (relative to the program's zero,
  i.e. the same as the eigenvalues printed by the program), the peak
  width (an energy) for broadening the eigenvalues, the number of
  points in the energy window, and the energy units.  An example is:
  \begin{fdfexample}
     %block ProjectedDensityOfStates
        -20.00  10.00  0.200  500  eV
     %endblock ProjectedDensityOfStates
  \end{fdfexample}
  Optionally one may start the line with \shell{EF} as this:
  \begin{fdfexample}
     %block ProjectedDensityOfStates
        EF -20.00  10.00  0.200  500  eV
     %endblock ProjectedDensityOfStates
  \end{fdfexample}
  This specifies the energies with respect to the Fermi-level.

  The broadening of the states is the Gaussian distribution with the peak
  width being $w$:
  \begin{equation}
    f(E) = \frac{1}{w\sqrt{\pi}}\exp\left[-\left(\frac{E-\epsilon}{w}\right)^2\right],
  \end{equation}
  where $\epsilon$ is the eigenvalue of the state. Note that the peak width
  is equivalent to $\sigma\sqrt2=w$, with $\sigma$ being the standard deviation.

  By default the projected density of states is generated for the same
  grid of points in reciprocal space as used for the SCF calculation.
  However, a separate set of K-points, usually on a finer grid, can be
  generated by using \fdf{PDOS.kgrid.?} Note that if a gamma point
  calculation is being used in the SCF part, especially as part of a
  geometry optimisation, and this is then to be run with a grid of
  K-points for the PDOS calculation it is more efficient to run the
  SCF phase first and then restart to perform the PDOS evaluation
  using the density matrix saved from the SCF phase.

  \note the two energies of the range must be ordered, with lowest
  first.

  The total DOS is stored in a file called \sysfile{DOS}.  The format
  of this file is:
  \begin{shellexample}
   Energy value, Total DOS (spin up), Total DOS (spin down)
  \end{shellexample}

  The Projected Density Of States for all the orbitals in the unit
  cell is dumped sequentially into a file called
  \sysfile{PDOS}. This file is structured using spacing and
  xml tags. A machine-readable (but not very human readable) xml file
  \sysfile{PDOS.xml} is also produced. Both can be processed by the
  program in \program{Util/pdosxml}. The \sysfile{PDOS} file can be
  processed by utilites in \program{Util/Contrib/APostnikov}.

  In all cases, the units for the DOS are (number of states/eV), and the
  Total DOS, $g(\epsilon)$, is normalized as follows:
  \begin{equation}
    \int_{-\infty}^\infty g (\epsilon) d\epsilon =
    \text{number of basis orbitals in unit cell}
  \end{equation}

\end{fdfentry}

\begin{fdfentry}{PDOS.kgrid.?}<\fdfvalue{DOS.kgrid.?}>

  This is PDOS only specification for the k-points. I.e. if one wishes
  to use a specific k-point sampling. These options are equivalent to
  the \fdf{kgrid!Cutoff}, \fdf{kgrid!MonkhorstPack} and
  \fdf{kgrid!File} options. Refer to them for additional details.

  If \fdf{PDOS.kgrid.?} does not exist, then \fdf{DOS.kgrid.?} is
  checked, and if that does not exist then \fdf*{kgrid.?} options are
  used.

\end{fdfentry}


\subsubsection{Local density of states}

The LDOS is formally the DOS weighted by the amplitude of the
corresponding wavefunctions at different points in space, and is then
a function of energy and position. \siesta\ can output the LDOS
integrated over a range of energies. This information can be used to
obtain simple STM images in the Tersoff-Hamann approximation (See
\program{Util/STM/simple-stm}).

\begin{fdfentry}{LocalDensityOfStates}[block]
  \index{output!local density of states}

  Instructs to write the LDOS, integrated between two given energies,
  at the mesh used by DHSCF, in file \sysfile{LDOS}. This file can be
  read by routine IORHO, which may be used by an application program
  in later versions.  The block must be a single line with the
  energies of the range for LDOS integration (relative to the
  program's zero, i.e. the same as the eigenvalues printed by the
  program) and their units.  An example is:
  \begin{fdfexample}
     %block LocalDensityOfStates
        -3.50  0.00   eV
     %endblock LocalDensityOfStates
  \end{fdfexample}

  One may optionally write \shell{EF} as the first word to specify that
  the energies are with respect to the Fermi level
  \begin{fdfexample}
     %block LocalDensityOfStates
       EF -3.50  0.00   eV
     %endblock LocalDensityOfStates
  \end{fdfexample}
  would calculate the LDOS from $-3.5\,\mathrm{eV}$ below the
  Fermi-level up to the Fermi-level.

  One may use \fdf{LDOS.kgrid.?} to fine-tune the k-point sampling in
  the LDOS calculation.

  \note the two energies of the range must be ordered, with lowest
  first.

  \note this flag is not compatible with \fdf{PEXSI!LDOS}.

  If netCDF support is compiled in, the file \file{LDOS.grid.nc} is
  produced.

\end{fdfentry}

\begin{fdfentry}{LDOS.kgrid.?}<\fdfvalue{DOS.kgrid.?}>

  This is LDOS only specification for the k-points. I.e. if one wishes
  to use a specific k-point sampling. These options are equivalent to
  the \fdf{kgrid!Cutoff}, \fdf{kgrid!MonkhorstPack} and
  \fdf{kgrid!File} options. Refer to them for additional details.

  If \fdf{LDOS.kgrid.?} does not exist, then \fdf{DOS.kgrid.?} is
  checked, if that does not exist then \fdf*{kgrid.?} are used.

\end{fdfentry}

