The current implementation has the possibility of adding D3 corrections to DFT
calculations (See Grimme, J. Chem. Phys. 132 (2010), 154104. DOI:
10.1063/1.3382344). The following options provide a great deal of fine-tuning
within this model; see in the above reference for
insight on the parameters Sn, rSn and alpha, which correspond to the following
equations:

\begin{centering}

$ E_{D3} = E_{2body} + E_{3body} $

$ E_{2body} = \sum_{A,B} \frac{ s_{6} C_{6}^{AB}}{ (r_{AB})^6 } f_{6}(r_{AB})
            + \sum_{A,B} \frac{ s_{8} C_{8}^{AB}}{ (r_{AB})^8 } f_{8}(r_{AB})$ ;
$ f_{n}(r_{AB}) = \frac{ 1}{
    1 + 6 \left (\frac{r_{AB}}{Sr_{n} R_{0}^{AB}} \right )^{-\alpha_{n}} } $

\end{centering}

The 3-body interaction is also calculated but there are no input parameters
involved except for enabling or disabling it entirely. In this case, the value
of $\alpha$ is always 16 and the value of $S_{r}$ is 4/3.

\begin{centering}

$ E_{3body} = \sum_{A,B,C} f_{3}(r_{ABC}) E_{ABC} $

$ E_{ABC} = \frac{ 1 + 3 cos(\theta_{ABC}) cos(\theta_{BCA}) cos(\theta_{ACB})
            }{ ( r_{AB} r_{BC} r_{AC} ) ^{3} } C_{9}^{ABC} $ ;
$ C_{9}^{ABC} = - \sqrt{ C_{6}^{AB} C_{6}^{BC} C_{6}^{AC} }$

\end{centering}

\begin{fdflogicalF}{DFTD3}
  If \fdftrue, D3 corrections are enabled for the current calculation.

\end{fdflogicalF}

\begin{fdflogicalT}{DFTD3!UseXCDefaults}
  When doing D3 corrections, SIESTA may use default parameters for the D3
  model which where already available for some functionals. At the moment
  this covers only PBE, PBESol, RevPBE, RPBE, LYP, BLYP, but more of them
  may be added in the future. With LIBXC, HS6 and PBE0 are also available.

\end{fdflogicalT}

\begin{fdfentry}{DFTD3!2BodyCutOff}[length]<$60.0 bohr$>
  Cut-off distance for 2-body dispersion interactions. Interactions corresponding
  to atom pairs farther away than this distance are ignored.

\end{fdfentry}

\begin{fdfentry}{DFTD3!3BodyCutOff}[length]<$40.0 bohr$>
  Cut-off distance for 3-body dispersion interactions. Interactions corresponding
  to atom pairs farther away than this distance are ignored.

\end{fdfentry}

\begin{fdfentry}{DFTD3!CoordinationCutoff}[length]<$10.0 bohr$>
  Cut-off distance for coordination number calculation (i.e. first neighbours
  count). This is relevant for the correct calculation of the C6 and C8 factors.

\end{fdfentry}

\begin{fdflogicalT}{DFTD3!BJdamping}
  If \fdftrue, uses the Becke-Johnson damping for D3 interaction. If not,
  uses the zero-damping variant.

\end{fdflogicalT}

\begin{fdfentry}{DFTD3!s6}[real]<$1.0$>
  Sets the value for the s6 coefficient in the D3 model, with s6 being
  the factor that multiplies the C6 interaction terms.

\end{fdfentry}

\begin{fdfentry}{DFTD3!rs6}[real]<$1.0$>
  Sets the value for the rs6, which is the prefactor present in the
  C6 damping function.

\end{fdfentry}

\begin{fdfentry}{DFTD3!s8}[real]<$1.0$>
  Sets the value for the s8 coefficient in the D3 model, with s8 being
  the factor that multiplies the C8 interaction terms.

\end{fdfentry}

\begin{fdfentry}{DFTD3!rs8}[real]<$1.0$>
  Sets the value for the rs8, which is the prefactor present in the
  C8 damping function. This is usually set to 1.0 and not changed.

\end{fdfentry}

\begin{fdfentry}{DFTD3!alpha}[real]<$14.0$>
  Sets the value for the the exponent in the C6 damping function. The C8
  damping function automatically takes the value of alpha + 2.

\end{fdfentry}

\begin{fdfentry}{DFTD3!a1}[real]<$0.4$>
  Value of the a1 coefficient for Becke-Johnson damping.

\end{fdfentry}

\begin{fdfentry}{DFTD3!a2}[real]<$5.0$>
  Value of the a2 coefficient for Becke-Johnson damping.

\end{fdfentry}

\subsubsection{ DFTD3 and periodicity }
SIESTA will try to guess system periodicity from the existing coordinates and
lattice vectors; however, detection might fail in non-orthogonal simulation
cells, giving the following output:

\begin{shellexample}
Could not guess the right amount of periodicity directions for D3.
Enabling periodicity on all directions.
\end{shellexample}

You can enforce the desired periodicity using the \fdf{DFTD3!Periodic} list
option:

\begin{fdfentry}{DFTD3!Periodic}[list]<[1 2 3]>
  Indicates which lattice vectors are periodic. A value of [1 3], for example,
  will enforce the periodicity for Simple-D3 in the directions of the first and
  third lattice vectors.
\end{fdfentry}

\subsubsection{ A note on LIBXC functionals }
SIESTA has now LIBXC functionality enabled via GRIDXC. However, not
every single one of the posibilities provided by that library are present
in the standard D3 model. Most of the one that are already present, are
already the standard SIESTA GGA functionals. So in case you want to try
something different, we recommend referring to the following webpage for
already existing D3 parameters:

\href{
  https://www.chemie.uni-bonn.de/pctc/mulliken-center/software/dft-d3/dft-d3
}{
  https://www.chemie.uni-bonn.de/pctc/mulliken-center/software/dft-d3/dft-d3
}

Don't forget to set \fdf{DFTD3!UseXCDefaults} to \shell{F} when adding
external parameters.