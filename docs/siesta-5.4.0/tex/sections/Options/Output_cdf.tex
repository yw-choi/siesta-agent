\note this requires \siesta\ compiled with CDF4 support.

To unify and construct a simple output file for an entire \siesta\
calculation a generic NetCDF file will be created if \siesta\ is
compiled with \program{ncdf} support, see Sec.~\ref{sec:libs} and the
\program{ncdf} section\index{compile!pre-processor!-DNCDF\_4}.

Generally all output to NetCDF flags,
\fdf{SaveElectrostaticPotential}, etc. apply to this file as well.

One may control the output file with compressibility and parallel I/O,
if needed.

\begin{fdflogicalF}{CDF!Save}

  Create the \sysfile{nc} file which is a NetCDF file.

  This file will be created with a large set of \emph{groups} which
  make separating the quantities easily. Also it will inherently
  denote the units for the stored quantities.

  \note this option is not available for MD/relaxations, only for
  force constant runs.

\end{fdflogicalF}


\begin{fdfentry}{CDF!Compress}[integer]<$0$>

  Integer between 0 and 9. The former represents \emph{no} compressing
  and the latter is the highest compressing.

  The higher the number the more computation time is spent on
  compressing the data. A good compromise between speed and
  compression is $3$.

  \note if one requests parallel I/O (\fdf{CDF!MPI}) this will
  automatically be set to $0$. One cannot perform parallel IO and
  compress the data simultaneously.

  \note instead of using \siesta\ for compression you may compress
  after execution by:
  \begin{shellexample}
    nccopy -d 3 -s noncompressed.nc compressed.nc
  \end{shellexample}

\end{fdfentry}


\begin{fdflogicalF}{CDF!MPI}

  Write \sysfile{nc} in parallel using MPI for increased
  performance. This has almost no memory overhead but may for very
  large number of processors saturate the file-system.

  \note this is an experimental flag.

\end{fdflogicalF}

\begin{fdfentry}{CDF!Grid.Precision}[string]<single|double>

  At which precision should the real-space grid quantities be stored,
  such as the density, electrostatic potential etc.

\end{fdfentry}

