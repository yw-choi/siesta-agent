For successful \tsiesta\ calculations it is imperative that the
electrodes and scattering regions are well-converged.
%
The basic principle is equivalent to the \siesta\ convergence, see
Sec.~\ref{sec:scf}.

The steps should be something along the line of (only done at
$0\, V$).
\begin{enumerate}

  \item%
  Converge electrodes and find optimal \fdf{Mesh!Cutoff},
  \fdf{kgrid!MonkhorstPack} etc.

  Electrode $k$ points should be very high along the semi-infinite
  direction. The default is $100$, but at least $>50$ should easily be
  reachable.


  \item%
  Use the parameters from the electrodes and also converge the
  same parameters for the scattering region SCF.

  This is an iterative process since the scattering region forces the
  electrodes to use equivalent $k$ points (see
  \fdf{TS!Elec.<>!check-kgrid}).

  Note that $k$ points should be limited in the \tsiesta\ run, see
  \fdf{TS!kgrid!MonkhorstPack}.

  One should always use the same parameters in both the electrode and
  scattering region calculations, except the number of $k$ points for
  the electrode calculations along their respective semi-infinite
  directions.


  \item%
  Once \tsiesta\ is completed one should also converge the
  number of $k$ points for \tbtrans. Note that $k$ point sampling in
  \tbtrans\ should generally be much denser but \emph{always} fulfill
  $N_k^{\tsiesta}\geq N_k^\tbtrans$

\end{enumerate}

The converged parameters obtained at $0\,\mathrm V$ should be used for
all subsequent bias calculations. Remember to copy the \sysfile{TSDE}
from the closest, previously, calculated bias for restart and much
faster convergence.


\tsiesta\ is also more difficult to converge during the SCF
steps. This may be due to several interrelated problems:
%
\begin{itemize}

  \item%
  A too short screening distance between the scattering atoms
  and the electrode layers.


  \item%
  In case buffer atoms (\fdf{TS!Atoms.Buffer}) are used with
  vacuum on the backside it may be that there are too few buffer atoms
  to accurately screen off the vacuum region for a sufficiently good
  initial guess. This effect is only true for $0\,\mathrm V$
  calculations.


  \item%
  The mixing parameters may need to be smaller than for \siesta,
  see Sec.~\ref{sec:scf:mix} and it is never guaranteed that it will
  converge. It is \emph{always} a trial and error method, there are
  \emph{no} omnipotent mixing parameters.


  \item%
  Very high bias' may be extremely difficult to
  converge. Generally one can force bias convergence by doing smaller
  steps of bias. E.g. if problems arise at $0.5\,\mathrm V$ with an
  initial DM from a $0.25\,\mathrm V$ calculation, one could try and
  $0.3\,\mathrm V$ first.


  \item%
  If a particular bias point is hard to converge, even by doing
  the previous step, it may be related to an eigenstate close to the
  chemical potentials of either electrode (e.g. a molecular eigenstate
  in the junction). In such cases one could try an even higher bias
  and see if this converges more smoothly.

\end{itemize}

