% Manual for the SIESTA program
%
% To generate the printed version:
%
% pdflatex siesta
% splitidx siesta.idx (optional if you want a split index)
% makeindex siesta    (Optional if you have a current siesta.ind)
% pdflatex siesta
%
%

% NOTE:
% If you want to reference a fdf flag, use:
%  \fdf{#1}
% This command will automatically create the necessary link
% to the created entry of the flag. Thus every flag will
% eventually become a link to the correct.
% If you want to create an fdf flag which is not referenced
% For instance a sub-block segment, use \fdf*{#1}.
% The \fdf command will automatically convert ! to . but retains
% the ! for indexing of the flag. Thus \fdf{FDF!Flag} will print
% as FDF.Flag, but indexed as \index{FDF!Flag}. If you reference
% a flag you must use the exclamation mark as used in the fdfentry
% environment.
% If you want to reference a fdf flag with an argument do it like
% this:
%   \fdf{Label:Argument}
% in case the index exists for this.
%
% Two very used fdf-flags are \fdftrue and \fdffalse which
% are short-hands for \fdf*{true} and \fdf*{false}.
%
% To document a new flag, do this:
%
%  \begin{fdfentry}{FDF.Flag}
%
%    Description
%
%  \end{fdfentry}
%
% Remark that it is not necessary to create any index commands
% as that is handled by the fdfentry environment.
% Optionally one may specify the type of variable it corresponds
% to:
%
%  \begin{fdfentry}{FDF.Flag}[integer]<0>
%
%    Description
%
%  \end{fdfentry}
%
% where [#1] may be any of:
%    block, integer, real, energy, length, logical
% and <#1> is the default value of the flag.
%
% There are two shorthands for the logical input with T/F default:
%
%  \begin{fdflogicalT}{FlagDefaultTrue}
%
%    This flag is default \fdftrue.
%
%  \end{fdflogicalT}
%  \begin{fdflogicalF}{FlagDefaultFalse}
%
%    This flag is default \fdffalse.
%
%  \end{fdflogicalF}
%
% Sometimes it is useful to create an index value without
% having to create the entry (say if a flag has been
% superseeded by a new flag), in this case you should do:
%
%  \begin{fdfentry}{FDF.Flag}[integer]<0>
%    \fdfindex*{Old.Flag}
%
%    Description
%
%  \end{fdfentry}
%
% Additionally one may create dependency flags to make it
% easy to see dependencies between flags
%  \begin{fdfentry}{FDF.Flag}[integer]<0>
%    \fdfdepend{FDF.First, FDF.Second}
%
%    Description
%
%  \end{fdfentry}
% will create additional information in the PDF with
% proper dependencies and hyperlinks to the fdf-flags.
% To document a flag has replaced another flag, use
%  \fdfdeprecates{...} in the same manner as \fdfdepend{...}.
%
%
% !!!IMPORTANT!!!
% Do NOT use \bf, \it, \tt, \rm, \em etc!
%
% If you want to emphasize a word in a sentence, prefer to
% use \emph{#1}.
% If you want bold, use \textbf{#1}
% If you want italics, use \textit{#1}.

% In addition to the above specifications one may add developer
% notes by encapsulating sections in:
%  \ifdeveloper
%    content only shown when compiling the developer version
%  \fi
% This may be handy for doing documentation in-line together
% with the regular documentation.

% Include settings appropriate for siesta
%
% This TeX file includes the default environments for creating fdf blocks
% in the documentation of SIESTA.
%
% These utility functions have been implemented by:
%  Nick R. Papior, 2016 (nickpapior <at> gmail.com)
%
\input{tex/helpers/setup.tex}

\makeatletter

% Include standard packages
\input{tex/helpers/packages.tex}

% Include physics quantities
%
% This TeX file includes the default commands used in the SIESTA manual. 
%
%  Nick R. Papior, 2016 (nickpapior <at> gmail.com)

% Define some often used physical quantities

\newcommand{\Ham}{\mathbf{H}}
\newcommand{\SO}{\mathbf{S}}
\newcommand{\DM}{\boldsymbol{\rho}}
\newcommand{\EDM}{\boldsymbol{\mathcal{E}}}
\newcommand{\elec}{\mathfrak{e}}
\newcommand{\G}{\mathbf{G}}
\newcommand{\SE}{\boldsymbol{\Sigma}}

\DeclareMathOperator{\Var}{Var}
\DeclareMathOperator{\Tr}{Tr}
\DeclareMathOperator*{\oppropto}{\propto}
\DeclareMathOperator\Res{R}
\DeclareMathOperator\RRes{\Delta R}

% Declare mathematical operators
\newcommand{\bra}[1]{\langle#1|}
\newcommand{\ket}[1]{|#1\rangle}
\newcommand{\kb}[2]{\ket{#1}\bra{#2}}
\newcommand{\bk}[2]{\langle#1|#2\rangle}


% Include default commands/macros for the siesta manual
%
% This TeX file includes the default commands used in the SIESTA manual. 
%
%  Nick R. Papior, 2016 (nickpapior <at> gmail.com)
%

\newcommand{\program}[1]{\texttt{#1}}
\newcommand{\shell}[1]{\texttt{#1}}
\newcommand{\code}[1]{\texttt{#1}}

\newcommand{\method}[1]{\textsc{#1}}

\NewDocumentCommand\siesta{}{\method{SIESTA}}
\NewDocumentCommand\tsiesta{}{\method{TranSIESTA}}
\NewDocumentCommand\tbtrans{}{\method{TBtrans}}
\NewDocumentCommand\phtrans{}{\method{PHtrans}}
\NewDocumentCommand\sisl{}{\method{sisl}}
\NewDocumentCommand\fdflib{}{\method{fdf}}

\newcommand{\note}{\textbf{NOTE:} }

% % Enable a non-active | char in the output
\newcommand{\pipe}{\siesta@bar}

% ------- ___PANDOC_IGNORE___ ------------
% The following latex code is too complex for PANDOC,
% so it must be cut before converting with pandoc.

% Enable macros as shorthands for specific 
% characters
\begingroup
  \catcode`\|=12
  \gdef\siesta@bar{|}
  \catcode`\|=13
  \gdef\siesta@active@bar{|}
  \catcode`\_=12
  \gdef\siesta@underscore{_}
  \catcode`\ =13
  \gdef\siesta@active@space{ }
  \catcode`\^^I=13
  \gdef\siesta@active@tab{^^I}
  \catcode`\<=13
  \gdef\siesta@active@less{<}
  \catcode`\>=13
  \gdef\siesta@active@greater{>}
\endgroup


% Ensure that we may use texttt to print underscore and braces
\let\origtexttt=\texttt
\def\texttt#1{%
    \begingroup%
    %\catcode`\|=13%
    %\expandafter\let\siesta@active@bar\siesta@bar%
    \def\textunderscore{\char`\_}%
    \def\textunderscore{\char`\_}%
    \def\textbraceleft{\char`\{}%
    \def\textbraceright{\char`\}}%
    \def\textless{\textlangle}%
    \def\textgreater{\textrangle}%
    \origtexttt{#1}%
    \endgroup%
}


% Specific routine for collecting input from the | | segment
\def\siesta@verb{%
    \ifmmode%
       \expandafter%
       \siesta@bar%
     \else%
      \begingroup
        \siesta@verb@preparecatcodes@
        \toksdef\t@siesta@verb=0
        \t@siesta@verb={}%
        \expandafter\siesta@verb@collect
    \fi%
}

% Start the collecting routine
\def\siesta@verb@collect#1{%
    \def\siesta@temp{#1}%
    \ifx\siesta@temp\siesta@bar
      % ok, finish:
      \edef\siesta@verb@collect@next{%
          % this command will also handle control sequences.
          \noexpand\endgroup%
          \noexpand\fdf{\the\t@siesta@verb}%
      }%
    \else%
      \ifx\siesta@temp\siesta@active@bar
        \edef\siesta@verb@collect@next{%
            \noexpand\endgroup%
            \noexpand\fdf{\the\t@siesta@verb}%
        }%
      \else
        \t@siesta@verb=\expandafter{\the\t@siesta@verb #1}%
        \let\siesta@verb@collect@next=\siesta@verb@collect%
      \fi%
    \fi%
    \siesta@verb@collect@next%
}%

\def\siesta@verb@preparecatcodes@{%
    \let\do\@makeother%
    \dospecials%
    \catcode`\%=12 % THATS IMPORTANT! Do *not* handle comments!
    % these catcodes are expected by the pretty printer...
    \catcode`\ =13
    \catcode`\^^I=13
    \expandafter\def\siesta@active@space{\space}%
    \expandafter\def\siesta@active@tab{\space\space\space\space}%
}%


% Create shorthand for | | as fdf keys
\AtBeginDocument{%
%    \catcode`\|=12
%    \expandafter\let\siesta@active@bar=\siesta@verb
}



%%% Local Variables:
%%% mode: latex
%%% TeX-master: "../siesta"
%%% End:


% This includes how the indexes of the manual should be handled
\input{tex/helpers/index.tex}

% Include default commands for creating output
\input{tex/helpers/output.tex}

% Enable the specific environments for fdf-keys used throughout the documentation
\input{tex/helpers/fdf.tex}

\makeatother


%%% Local Variables:
%%% mode: latex
%%% TeX-master: "../siesta"
%%% End:


% Local command for the software version for printing
% \unskip helps when space/newline chars are present in the
% SIESTA.version file.
\providecommand\softwareversion{\input{../SIESTA.version}\unskip}

% Title (note that \input above will fail in the title, we however
% don't care since it is only meaningful for final publications where
% direct text is used.)
\title{SIESTA manual \softwareversion}

% Set the date
\date{\today}

% Specify Authors
\author{%
    Emilio Artacho, %
    Jose Maria Cela, %
    Julian D. Gale, %
    Alberto Garcia, %
    Javier Junquera, %
    Richard M. Martin, %
    Pablo Ordejon, %
    Daniel Sanchez-Portal, %
    Jose M. Soler, %
    Nick R. Papior%
}

% Ensure the information is written in the PDF (see tex/setup.tex)
\setpdfmetadata

\begin{document}

% TITLE PAGE --------------------------------------------------------------

\begin{titlepage}

\begin{center}

\vspace{1cm}
\ifdeveloper
 {\Huge \textsc{D e v e l o p e r' s \, \, G u i d e}}
\else
 {\Huge \textsc{U s e r' s \, \, G u i d e}}
\fi

\vspace{1cm}
\hrulefill
\vspace{1cm}

{\Huge \textbf{S I E S T A \, \, \softwareversion}\par}

\vspace{1cm}
\hrulefill
\vspace{0.5cm}

{\Large \today}

\vspace{1.5cm}
{\Large \url{https://siesta-project.org}}

\vspace{2.5cm}
\siesta\ Steering Committee:
\vspace{1.0cm}

\begin{tabular}{ll}

  Emilio Artacho &
  \textit{CIC-Nanogune and University of Cambridge} \\

  Jos\'e Mar\'{\i}a Cela &
  \textit{Barcelona Supercomputing Center} \\

  Julian D. Gale &
  \textit{Curtin University of Technology, Perth} \\

  Alberto Garc\'{\i}a &
  \textit{Institut de Ci\`encia de Materials, CSIC, Barcelona} \\

  Javier Junquera &
  \textit{Universidad de Cantabria, Santander} \\

  Richard M. Martin &
  \textit{University of Illinois at Urbana-Champaign} \\

  Pablo Ordej\'on &
  \textit{Centre de Investigaci\'o en Nanoci\`encia} \\
  &
  \textit{  i Nanotecnologia, (CSIC-ICN), Barcelona} \\

  Nick R\"ubner Papior &
  \textit{Technical University of Denmark} \\

  Daniel S\'anchez-Portal &
  \textit{Unidad de F\'{\i}sica de Materiales,} \\
  &
  \textit{Centro Mixto CSIC-UPV/EHU, San Sebasti\'an} \\

  Jos\'e M. Soler &
  \textit{Universidad Aut\'onoma de Madrid} \\

\end{tabular}

\vspace{0.5cm}

\siesta\ is Copyright \copyright\ 1996-\the\year{} by The Siesta Group

\end{center}

\end{titlepage}

% END TITLE PAGE --------------------------------------------------------------

\newpage

\section*{Contributors to \siesta}
\addcontentsline{toc}{section}{Contributors to \texorpdfstring{\siesta}{Siesta}}

\input{tex/sections/Contributors.tex}

\newpage
\tableofcontents
\newpage

\section{INTRODUCTION}

\textit{This Reference Manual contains descriptions of all the input,
  output and execution features of \siesta, but is not really a
  tutorial introduction to the program. Interested users can find
  tutorial material prepared for \siesta\ schools and workshops at
  the web page} \url{https://docs.siesta-project.org}


\siesta\index{Siesta@\siesta} (Spanish Initiative for
Electronic Simulations with
Thousands of Atoms) is both a method and its computer program implementation,
to perform electronic structure calculations and \textit{ab initio} molecular
dynamics simulations of molecules and solids. Its main characteristics are:
\begin{itemize}
\item
It uses the standard Kohn-Sham selfconsistent density functional
method in the local density (LDA-LSD) and generalized gradient (GGA)
approximations, as well as in a non local functional that includes
van der Waals interactions (VDW-DF).
\item
It uses norm-conserving pseudopotentials in their fully nonlocal
(Kleinman-Bylander) form.
\item
It uses atomic orbitals as a basis set, allowing unlimited multiple-zeta
and angular momenta, polarization and off-site orbitals. The radial
shape of every orbital is numerical and any shape can be used and provided
by the user, with the only condition that it has to be of finite support,
i.e., it has to be strictly zero beyond a user-provided distance from the
corresponding nucleus.
Finite-support basis sets are the key for calculating the Hamiltonian
and overlap matrices in $O(N)$ operations.
\item
Projects the electron wavefunctions and density onto a real-space
grid in order to calculate the Hartree and exchange-correlation
potentials and their matrix elements.
\item
Besides the standard Rayleigh-Ritz eigenstate method, it allows
the use of localized linear combinations of the occupied orbitals
(valence-bond or Wannier-like functions), making the computer
time and memory scale linearly with the number of atoms.
Simulations with several hundred atoms are feasible with
modest workstations.
\item
It is written in Fortran 2003 and memory is allocated dynamically.
\item
It may be compiled for serial or parallel execution (under MPI).

\end{itemize}

It routinely provides:
\begin{itemize}
  \item Total and partial energies.
  \item Atomic forces.
  \item Stress tensor.
  \item Electric dipole moment.
  \item Atomic, orbital and bond populations (Mulliken).
  \item Electron density.
\end{itemize}

And also (though not all options are compatible):
\begin{itemize}
  \item Geometry relaxation, fixed or variable cell.
  \item Constant-temperature molecular dynamics (Nose thermostat).
  \item Variable cell dynamics (Parrinello-Rahman).
  \item Spin polarized calculations (colinear or not).
  \item k-sampling of the Brillouin zone.
  \item Local and orbital-projected density of states.
  \item COOP and COHP curves for chemical bonding analysis.
  \item Dielectric polarization.
  \item Vibrations (phonons).
  \item Band structure.
  \item Ballistic electron transport under non-equilibrium (through \tsiesta)
\end{itemize}


Starting from version 3.0, \siesta\ includes the \tsiesta\index{TranSIESTA@\tsiesta}
module. \tsiesta\ provides the ability to model open-boundary systems where ballistic
electron transport is taking place.  Using \tsiesta\ one can compute electronic
transport properties, such as the zero bias conductance and the I-V characteristic, of a
nanoscale system in contact with two electrodes at different electrochemical potentials.
The method is based on using non equilibrium Greens functions (NEGF), that are
constructed using the density functional theory Hamiltonian obtained from a given electron
density. A new density is computed using the NEGF formalism, which closes the DFT-NEGF
self consistent cycle.

Starting from version 4.1, \tsiesta\ is an intrinsic part of the
\siesta\ code. I.e. a separate executable is not necessary
anymore. See Sec.~\ref{sec:transiesta} for details.

For more details on the formalism, see the main \tsiesta\
reference cited below. A section has been added to this User's Guide,
that describes the necessary steps involved in doing transport
calculations, together with the currently implemented input options.

\vspace{0.5cm}
{\large \textbf{References:} }

\begin{itemize}

\item
``Unconstrained minimization approach for electronic computations
that scales linearly with system size''
P. Ordej\'on, D. A. Drabold, M. P. Grumbach and R. M. Martin,
Phys. Rev. B \textbf{48}, 14646 (1993);
``Linear system-size methods for electronic-structure calculations''
Phys. Rev. B \textbf{51} 1456 (1995), and references therein.

Description of the order-\textit{N} eigensolvers
implemented in this code.

\item
``Self-consistent order-$N$ density-functional
calculations for very large systems''
P. Ordej\'on, E. Artacho and J. M. Soler,
Phys. Rev. B \textbf{53}, 10441, (1996).

Description of a previous version of this methodology.

\item
``Density functional method for very large systems with LCAO basis sets''
D. S\'anchez-Portal, P. Ordej\'on, E. Artacho and J. M. Soler,
Int. J. Quantum Chem., \textbf{65}, 453 (1997).

Description of the present method and code.

\item
``Linear-scaling ab-initio calculations for large and complex systems''
E. Artacho, D. S\'anchez-Portal, P. Ordej\'on, A. Garc\'{\i}a and
J. M. Soler, Phys. Stat. Sol. (b) \textbf{215}, 809 (1999).

Description of the numerical atomic orbitals (NAOs) most commonly
used in the code, and brief review of applications as of March 1999.

\item
``Numerical atomic orbitals for linear-scaling calculations''
J. Junquera, O. Paz, D. S\'anchez-Portal, and E. Artacho, Phys. Rev. B
 \textbf{64}, 235111, (2001).

Improved, soft-confined NAOs.

\item
``The \siesta\ method for ab initio order-$N$ materials simulation''
J. M. Soler, E. Artacho, J.D. Gale, A. Garc\'{\i}a, J. Junquera, P. Ordej\'on,
and D. S\'anchez-Portal, J. Phys.: Condens. Matter \textbf{14}, 2745-2779 (2002)

Extensive description of the \siesta\ method.

\item
``Computing the properties of materials from first principles
with  \siesta'', D. S\'anchez-Portal, P. Ordej\'on,
and E. Canadell, Structure and Bonding \textbf{113},
103-170 (2004).

Extensive review of applications as of summer 2003.

\item
 ``Improvements on non-equilibrium and transport Green function techniques: The next-generation TranSIESTA'',
 Nick Papior, Nicolas Lorente, Thomas Frederiksen, Alberto Garc\'{\i}a and
 Mads Brandbyge, Computer Physics Communications, \textbf{212}, 8--24 (2017).

 Description of the \tsiesta\ method.

\item
 ``Density-functional method for nonequilibrium electron transport'',
 Mads Brandbyge, Jose-Luis Mozos, Pablo Ordej\'on, Jeremy Taylor,
 and Kurt Stokbro, Phys. Rev. B \textbf{65}, 165401 (2002).

 Description of the original \tsiesta\ method (prior to 4.1).

 \item
 ``Siesta: Recent developments and applications'',
 Alberto Garc\'{\i}a, \emph{et al.}, J. Chem. Phys. \textbf{152}, 204108
 (2020).

 Extensive review of applications and developments as of 2020.

\end{itemize}

For more information you can visit the web page
\url{https://siesta-project.org}.

\section{COMPILATION}
\label{sec:compilation}

Prior to \siesta\ 4.1 \tsiesta\ was a separate executable. Now
\tsiesta\ is fully incorporated into \siesta. \emph{Only} compile
\siesta\ and the full functionality is present.
Sec.~\ref{sec:compilation} for details on compiling \siesta.


\section{EXECUTION OF THE PROGRAM}


A fast way to test your installation of \siesta\ and get a feeling for
the workings of the program is implemented in directory
\shell{Tests}\index{Tests}.
Assuming that you have built \siesta\ in \program{\_build}, you can
do

\begin{shellexample}
  cd _build
  ctest -L simple
\end{shellexample}
%

to test the execution of an assortment of tests. Executing
\program{ctest} with no other options will run all possible tests.
Output verification is also available via the
\program{VERIFY\_TESTS} environment variable:

\begin{shellexample}
  VERIFY_TESTS=1 ctest -L simple
\end{shellexample}

Other examples are provided in the \shell{Examples} directory.

Further information about the running of \siesta\, with tutorials and
how-to's on various topics, including the generation of
pseudopotentials with the \program{ATOM} code, can be found in the
documentation site \url{https://docs.siesta-project.org}.

\subsection{Specific execution options}

\siesta\ may be executed in different forms. The basic execution form
is
\begin{shellexample}
  siesta < RUN.fdf > RUN.out
\end{shellexample}
which uses a \emph{pipe} statement.
%
\siesta\ 4.1 and later does not require one to pipe in the input file
and the input file may instead be specified on the command line:
\begin{shellexample}
  siesta RUN.fdf > RUN.out
\end{shellexample}
\siesta\ 4.1 and later also accepts special flags and options
described in what follows:
\begin{itemize}
\item
All flags must start by one or more dashes (\shell{-}).
The number of leading dashes is irrelevant,
as long as there is at least one of them.
\item
Some flags (e.g., \shell[-L])
must be followed by a properly formed option string.
Other flags (e.g., \shell[-elec])
are logical toggles and they are not followed by option strings.
\item
Flags and option strings must all be separated by spaces
(and only spaces are valid separators for this).
\item
Option strings may be quoted.
Option strings that contain spaces need to either be quoted
or have the spaces replaced by a colon (\shell{:}) or by
an equal sign (\shell{=}).
\item
If the input file is not piped in, it can be given as an argument:
\begin{shellexample}
  siesta -L Hello -V 0.25:eV RUN.fdf > RUN.out
  siesta -L Hello RUN.fdf -V 0.25:eV > RUN.out
\end{shellexample}
\end{itemize}

The list of available flags and options is:
\begin{fdfoptions}

  \option[-help|-h]%
  \fdfindex*{Command line options:-h}%
  Print a help instruction and quit.

  \option[-version|-v]%
  \fdfindex*{Command line options:-v}%
  Print version information and quit.

  \option[-out|-o]%
  \fdfindex*{Command line options:-out}%
  \fdfindex*{Command line options:-o}%
  Specify the output file (instead of printing to the terminal).
  Example:

  \begin{shellexample}
  siesta --out RUN.out
  \end{shellexample}

  \option[-L]%
  \fdfindex*{Command line options:-L}%
  Override, temporarily, the \fdf{SystemLabel} flag.
  Example:

  \begin{shellexample}
  siesta -L Hello
  \end{shellexample}

  \option[-electrode|-elec]%
  \fdfindex*{Command line options:-electrode}%
  \fdfindex*{Command line options:-elec}%
  \fdfoverwrite{Save!HS,TS!HS.Save,TS!DE.Save}
  Denote this as an electrode calculation which forces the
  \sysfile{TS.HSX}, \sysfile{TSHS} and \sysfile{TSDE} files to be saved.

  \note This is equivalent to specifying \fdf{Save!HS:true}, \fdf{TS!HS.Save:true} and
  \fdf{TS!DE.Save:true} in the input file.

  \option[-V]%
  \fdfindex*{Command line options:-V}%
  \fdfoverwrite{TS!Voltage}
  Specify the bias for the current \tsiesta\ run.
  If no units are specified, $\mathrm{eV}$ are assumed.
  Example: any of the following three commands set the applied bias to
  $0.25\,\mathrm{eV}$:

  \begin{shellexample}
  siesta -V 0.25:eV
  siesta -V "0.25 eV"
  siesta -V 0.25
  \end{shellexample}

  \note This is equivalent to specifying \fdf{TS!Voltage} in the input
  file.

  \option[-fdf]%
  \fdfindex*{Command line options:-fdf}%
  Specify any FDF option string.  For example, another way to specify the bias
  of the example of the previous option would be:

  \begin{shellexample}
  siesta --fdf TS.Voltage=0.25:eV
  \end{shellexample}

\end{fdfoptions}


\section{THE FLEXIBLE DATA FORMAT (FDF)}
\index{FDF}

The main input file, typically with extension \program{.fdf},\index{input file}
contains the physical data of the system and the parameters of
the simulation to be performed.
This file is written in a special format called FDF, developed by
Alberto Garc\'{\i}a and Jos\'e M. Soler. This format allows data to be
given in any order, or to be omitted in favor of default values.
Refer to documentation of \program{libfdf} for details.
Here we offer a glimpse of it through the following rules:

\begin{itemize}

  \item[$\bullet$] The \fdflib\ syntax is a ``data label'' followed by
  its value.  Values that are not specified in the datafile are
  assigned a default value.

  \item[$\bullet$] \fdflib\ labels are case insensitive, and
  characters - \_ .  in a data label are ignored. Thus,
  \fdf*{LatticeConstant} and \fdf*{lattice\_constant} represent the
  same label.

\item[$\bullet$] All text following the \# character is taken as comment.

\item[$\bullet$] Logical values can be specified as T, true, .true.,
yes, F, false, .false., no. Blank is also equivalent to true.

\item[$\bullet$] Character strings should \textbf{not} be in apostrophes.

\item[$\bullet$] Real values which represent a physical magnitude must be
followed by its units. Look at function fdf\_convfac in
file $\sim$/siesta/Src/fdf/fdf.f for the units that are currently supported.
It is important to include a decimal point in a real number to distinguish
it from an integer, in order to prevent ambiguities when mixing the types
on the same input line.

\item[$\bullet$] Complex data structures are called blocks and are
placed between ``\%block label''\index{block@\%block} and a
``\%endblock label'' (without the quotes).

\item[$\bullet$] You may ``include'' other \fdflib\ files and redirect
the search for a particular data label to another file.  If a data
label appears more than once, its first appearance is used.

\item[$\bullet$] If the same label is specified twice, the first one
takes precedence.

\item[$\bullet$] If a label is misspelled it will not be recognized
(there is no internal list of ``accepted'' tags in the program). You
can check the actual value used by \siesta\ by looking for the label
in the output \file{fdf.log} file.

\end{itemize}

\noindent
These are some examples:

\begin{verbatim}
           SystemName      Water molecule  # This is a comment
           SystemLabel     h2o
           Spin polarized
           SaveRho
           NumberOfAtoms         64
           LatticeConstant       5.42 Ang
           %block LatticeVectors
                    1.000  0.000  0.000
                    0.000  1.000  0.000
                    0.000  0.000  1.000
           %endblock LatticeVectors
           KgridCutoff < BZ_sampling.fdf

           # Reading the coordinates from a file
           %block AtomicCoordinatesAndAtomicSpecies < coordinates.data

           # Even reading more FDF information from somewhere else
           %include mydefaults.fdf
\end{verbatim}

The file \file{fdf-XXXXX.log} contains all the parameters used by
\siesta\ in a given run, both those specified in the input fdf file
and those taken by default. They are written in fdf format, so that
you may reuse them as input directly. Input data blocks are copied to
the \file{fdf.log} file only if you specify the \textit{dump} option
for them.  In practice, the name of a FDF log file contains a sequence
of digits (e.g., \shell{fdf-12345.log}) chosen on-the-fly in
order to have a reduced chance of overwriting other FDF log files that
may be present in the same directory.

\section{PROGRAM OUTPUT}

\subsection{Standard output} \index{output!main output file}

\siesta\ writes a log of its workings to standard output (unit 6),
which is usually redirected to an ``output file''.

A brief description follows. See the example cases in the
siesta/Tests directory for illustration.

The program starts writing the version of the code which is
used. Then, the input \fdflib\ file is dumped into the output file as is
(except for empty lines). The program does part of the reading and
digesting of the data at the beginning within the \program{redata}
subroutine. It prints some of the information it digests. It is
important to note that it is only part of it, some other information
being accessed by the different subroutines when they need it during
the run (in the spirit of \fdflib\ input).  A complete list of the input
used by the code can be found at the end in the file \file{fdf.log},
including defaults used by the code in the run.

After that, the program reads the pseudopotentials, factorizes them
into Kleinman-Bylander form, and generates (or reads) the atomic basis
set to be used in the simulation. These stages are documented in the
output file.

The simulation begins after that, the output showing information of
the MD (or CG) steps and the SCF cycles within.  Basic descriptions of
the process and results are presented. The user has the option to
customize it, however,\index{output!customization} by defining
different options that control the printing of informations like
coordinates, forces, $\vec k$ points, etc.  The options are discussed
in the appropriate sections, but take into account the behavior of the
legacy \fdf{LongOutput} option, as in the current implementation might
silently activate output to the main .out file at the expense of
auxiliary files.

\begin{fdflogicalF}{LongOutput}
  \index{output!long}

  \siesta\ can write to standard output different data sets depending
  on the values for output options described below.  By default
  \siesta\ will not write most of them. They can be large for large
  systems (coordinates, eigenvalues, forces, etc.)  and, if written to
  standard output, they accumulate for all the steps of the
  dynamics. \siesta\ writes the information in other files (see Output
  Files) in addition to the standard output, and these can be
  cumulative or not.

  Setting \fdf{LongOutput} to \fdftrue\ changes the default of some
  options, obtaining more information in the output (verbose).  In
  particular, it redefines the defaults for the following:

  \begin{itemize}

    \item \fdf{WriteKpoints}%
    \index{output!grid $\vec k$ points}

    \item \fdf{WriteKbands}%
    \index{output!band $\vec k$ points}

    \item \fdf{WriteCoorStep}%
    \index{output!atomic coordinates!in a dynamics step}

    \item \fdf{WriteForces}%
    \index{output!forces}

    \item \fdf{WriteEigenvalues}%
    \index{output!eigenvalues}

    \item \fdf{WriteWaveFunctions}%
    \index{output!wave functions}

    \item \fdf{WriteMullikenPop}%
    \index{output!Mulliken analysis}%
    \index{Mulliken population analysis}%
    (it sets it to 1)

  \end{itemize}

  The specific changing of any of these options has precedence.

\end{fdflogicalF}


\subsection{Output to dedicated files}%
\index{output!dedicated files}

\siesta\ can produce a wealth of information in dedicated files,
with specific formats, that can be used for further analysis. See the
appropriate sections, and the appendix on file formats.
Please take into account the behavior of
\fdf{LongOutput}, as in the current implementation might silently
activate output to the main .out file at the expense of auxiliary
files.



\section{DETAILED DESCRIPTION OF PROGRAM OPTIONS}

The fdf options shown here are only to be used at the input file for
the scattering region. When using \tsiesta\ for electrode
calculations, only the usual \siesta\ options are relevant.
%
Note that since \tsiesta\ is a generic $\Nelec$ electrode NEGF code the
input options are heavily changed compared to versions prior to 4.1.

\subsubsection{Quick and dirty}

Since 4.1, \tsiesta\ has been fully re-implemented. And so have
\emph{every} input fdf-flag. To accommodate an easy transition between
previous input files and the new version format a small utility called
\program{ts2ts}. It may be compiled in \program{Util/TS/ts2ts}. It is
recommended that you use this tool if you are familiar with previous
\tsiesta\ versions.

%
One may input options as in the old \tsiesta\ version and then run
\begin{fdfexample}
  ts2ts OLD.fdf > NEW.fdf
\end{fdfexample}
which translates all keys to the new, equivalent, input format. If you
are familiar with the old-style flags this is highly recommendable
while becoming comfortable with the new input format. Please note that
some defaults have changed to more conservative values in the newer
release.

If one does not know the old flags and wish to get a basic example of
an input file, a script \program{Util/TS/tselecs.sh} exists that can
create the basic input for $\Nelec$ electrodes. One may call it like:
\begin{shellexample}
  tselecs.sh -2 > TWO_ELECTRODE.fdf
  tselecs.sh -3 > THREE_ELECTRODE.fdf
  tselecs.sh -4 > FOUR_ELECTRODE.fdf
  ...
\end{shellexample}
where the first call creates an input fdf for 2 electrode setups, the
second for a 3 electrode setup, and so on. See the help (\program{-h})
for the program for additional options.

Before endeavoring on large scale calculations you are advised to run
an analyzation of the system at hand, you may run your system as
\begin{shellexample}
  siesta -fdf TS.Analyze RUN.fdf > analyze.out
\end{shellexample}
which will analyze the sparsity pattern and print out several
different pivoting schemes. Please see \fdf{TS!Analyze} for more
information.


\subsubsection{General options}

One have to set \fdf{SolutionMethod} to \fdf*{transiesta} to enable
\tsiesta.

\begin{fdfentry}{TS!SolutionMethod}[string]<btd|mumps|full>

  Control the algorithm used for calculating the Green
  function. Generally the BTD method is the fastest and this option
  need not be changed.

  \begin{fdfoptions}
    \option[BTD]%
    \fdfindex*{TS!SolutionMethod:BTD}%
    Use the block-tri-diagonal algorithm for matrix inversion.

    This is generally the recommended method.

    \option[MUMPS]%
    \fdfindex*{TS!SolutionMethod:MUMPS}%
    Use sparse matrix inversion algorithm (MUMPS). This requires
    \tsiesta\ to be compiled with MUMPS.
    \index{MUMPS}%
    \index{External library!MUMPS}%

    \option[full]%
    \fdfindex*{TS!SolutionMethod:full}%
    Use full matrix inversion algorithm (LAPACK). Generally only
    usable for debugging purposes.

  \end{fdfoptions}

\end{fdfentry}

\begin{fdfentry}{TS!Voltage}[energy]<$0\,\mathrm{eV}$>

  Define the reference applied bias. For $\Nelec=2$ electrode calculations
  this refers to the actual potential drop between the electrodes,
  while for $\Nelec\neq2$ this is a reference bias. In the latter case it
  \emph{must} be equivalent to the maximum difference between the
  chemical potential of any two electrodes.

  \note Specifying \shell{-V}\fdfindex{Command line options:-V} on the
  command-line overwrites the value in the fdf file.

\end{fdfentry}

\begin{fdfentry}{TS!kgrid!MonkhorstPack}[block]<\fdfvalue{kgrid!MonkhorstPack}>

  $k$ points used for the \tsiesta\ calculation.

  For $\Nelec\neq2$ this should always be defined. Always take care to
  use only 1 $k$ point along non-periodic lattice vectors. An
  electrode semi-infinite region is considered non-periodic since it
  is integrated out through the self-energies.

  This defaults to \fdf{kgrid!MonkhorstPack}.

\end{fdfentry}

\begin{fdfentry}{TS!Atoms.Buffer}[block/list]
  \fdfindex{TS.BufferAtomsLeft|see TS!Atoms.Buffer}%
  \fdfindex{TS.BufferAtomsRight|see TS!Atoms.Buffer}%

  Specify atoms that will be removed in the \tsiesta\ SCF. They are
  not considered in the calculation and may be used to improve the
  initial guess for the Hamiltonian.


  An intended use for buffer atoms is to ensure a bulk behavior in the
  electrode regions when electrodes are different. As an example: a 2
  electrode calculation with left consisting of Au atoms and the right
  consisting of Pt atoms. In such calculations one cannot create a
  periodic geometry along the transport direction. One needs to add
  vacuum between the Au and Pt atoms that comprise the
  electrodes. However, this creates an artificial edge of the
  electrostatic environment for the electrodes since in \siesta\ there
  is vacuum, whereas in \tsiesta\ the effective Hamiltonian sees a
  bulk environment. To ensure that \siesta\ also exhibits a bulk
  environment on the electrodes we add \emph{buffer} atoms towards the
  vacuum region to screen off the electrode region. These
  \emph{buffer} atoms is thus a technicality that has no influence on
  the \tsiesta\ calculation but they are necessary to ensure the
  electrode bulk properties.

  The above discussion is even more important when doing $\Nelec$-electrode
  calculations.

  \note all lines are additive for the buffer atoms and the input
  method is similar to that of \fdf{Geometry!Constraints} for the
  \fdf*{atom} line(s).

  \begin{fdfexample}
    %block TS.Atoms.Buffer
       atom [ 1 -- 5 ]
    %endblock
    # Or equivalently as a list
    TS.Atoms.Buffer [1 -- 5]
  \end{fdfexample}
  will remove atoms [1--5] from the calculation.

\end{fdfentry}

\begin{fdfentry}{TS!ElectronicTemperature}[energy]<\fdfvalue{ElectronicTemperature}>

  Define the temperature used for the Fermi distributions for the
  chemical potentials.
  %
  See \fdf{TS!ChemPot.<>!ElectronicTemperature}.

\end{fdfentry}

\begin{fdfentry}{TS!SCF!DM.Tolerance}[real]<\fdfvalue{SCF.DM!Tolerance}>%
  \fdfdepend{SCF.DM!Tolerance,SCF.DM!Converge}

  The density matrix tolerance for the \tsiesta\ SCF cycle.

\end{fdfentry}

\begin{fdfentry}{TS!SCF!H.Tolerance}[energy]<\fdfvalue{SCF.H!Tolerance}>%
  \fdfdepend{SCF.H!Tolerance,SCF.H!Converge}

  The Hamiltonian tolerance for the \tsiesta\ SCF cycle.

\end{fdfentry}

\begin{fdflogicalT}{TS!SCF!dQ.Converge}

  Whether \tsiesta\ should check whether the total charge is within a
  provided tolerance, see \fdf{TS!SCF!dQ.Tolerance}.

\end{fdflogicalT}

\begin{fdfentry}{TS!SCF!dQ.Tolerance}[real]<$\mathrm{Q(device)}\cdot 10^{-3}$>%
  \fdfdepend{TS!SCF!dQ.Converge}

  The charge tolerance during the SCF.

  The charge is not stable in \tsiesta\ calculations and this flag
  ensures that one does not, by accident, do post-processing of files
  where the charge distribution is completely wrong.

  A too high tolerance may heavily influence the electrostatics of the
  simulation.

  \note Please see \fdf{TS!dQ} for ways to reduce charge loss in
  equilibrium calculations.

\end{fdfentry}

\begin{fdfentry}{TS!SCF.Initialize}[string]<diagon|transiesta>%

  Control which initial guess should be used for \tsiesta. The general
  way is the \fdf*{diagon} solution method (which is preferred),
  however, one can start a \tsiesta\ run immediately. If you start
  directly with \tsiesta\ please refer to these flags:
  \fdf{TS!Elecs!DM.Init} and \fdf{TS!Fermi.Initial}.

  \note Setting this to \fdf*{transiesta} is highly experimental and
  convergence may be extremely poor.

\end{fdfentry}

\begin{fdfentry}{TS!Fermi.Initial}[energy]<$\sum^{N_E}_iE_F^i/N_E$>

  Manually set the initial Fermi level to a predefined value.

  \note this may also be used to change the Fermi level for
  calculations where you restart calculations. Using this feature is
  highly experimental.

\end{fdfentry}

\begin{fdfentry}{TS!Weight.Method}[string]<orb-orb|[[un]correlated+][sum|tr]-atom-[atom|orb]|mean>

  Control how the NEGF weighting scheme is conducted. Generally one
  should only use the \fdf*{orb-orb} while the others are present for
  more advanced usage. They refer to how the weighting coefficients of
  the different non-equilibrium contours are performed. In the
  following the weight are denoted in a two-electrode setup while they
  are generalized for multiple electrodes.

  \def\mypropto{\,\oppropto^{||}\,} %
  \def\mn{{\mu\nu}} %
  Define the normalised geometric mean as $\mypropto$ via
  \begin{equation}
    w\mypropto \langle\cdot^L\rangle\equiv
    \frac{\langle\cdot^L\rangle}{\langle\cdot^L\rangle+\langle\cdot^R\rangle}.
  \end{equation}

  When applying a bias, \tsiesta\ will printout the following during
  the SCF cycle:
\begin{output}[fontsize=\footnotesize]
ts-err-D: ij(  447,   447), M =  1.8275, ew = -.257E-2, em = 0.258E-2. avg_em = 0.542E-06
ts-err-E: ij(  447,   447), M = -6.7845, ew = 0.438E-3, em = -.439E-3. avg_em = -.981E-07
ts-w-q:               qP1       qP2
ts-w-q:           219.150   216.997
ts-q:         D        E1        C1        E2        C2        dQ
ts-q:   436.147   392.146     3.871   392.146     3.871  7.996E-3
  \end{output}
  %
  The extra output corresponds to fine details in the integration
  scheme.
  \begin{description}[labelindent=3em, leftmargin=4.5em]
    \itemsep 10pt
    \parsep 0pt

    \item[\texttt{ts-err-*}] are estimated error outputs from the
    different integrals, for the density matrix (\texttt{D}) and the
    energy density matrix (\texttt{E}), see Eq.~(12) in
    \cite{Papior2017}. All values (except \texttt{avg\_em}) are for
    the given orbital site

    \begin{description}
      \itemsep 4pt
      \parsep 0pt

      \item[\texttt{ij(A,B)}] refers to the matrix element between orbital
      \texttt{A} and \texttt{B}

      \item[\texttt{M}] is the weighted matrix element value,
      $\sum_{\elec}w_\elec\DM^\elec$

      \item[\texttt{ew}] is the maximum difference between
      $\sum_{\elec}w_\elec\DM^\elec-\DM^\elec$ for all $\elec$.

      \item[\texttt{em}] is the maximum difference between
      $\DM^{\elec'}-\DM^\elec$ for all combinations of $\elec$ and
      $\elec'$.

      \item[\texttt{avg\_em}] is the averaged difference of \texttt{em} for all
      orbital sites.

    \end{description}

    \item[\texttt{ts-w-q}] is the Mulliken charge from the different
    integrals: $\Tr[w_\elec\DM^\elec\SO]$

  \end{description}

  \begin{fdfoptions}

    \option[orb-orb]%
    \fdfindex*{TS!Weight.Method:orb-orb}%
    Weight each orbital-density matrix element individually.

    \option[tr-atom-atom]%
    \fdfindex*{TS!Weight.Method:tr-atom-atom}%
    Weight according to the trace of the atomic density matrix sub-blocks
    \begin{equation}
      w_{ij}^{\Tr} \mypropto
      \sqrt{
          % First the i'th atom
          \sum_{\in\{i\}}(\Delta\rho_{\mu\mu}^L)^2
          \; % ensure a little space between them
          % second the j'th atom
          \sum_{\in\{j\}}(\Delta\rho_{\mu\mu}^L)^2
      }
    \end{equation}

    \option[tr-atom-orb]%
    \fdfindex*{TS!Weight.Method:tr-atom-orb}%

    Weight according to the trace of the atomic density matrix
    sub-block times the weight of the orbital weight
    \begin{equation}
      w_{ij,\mn}^{\Tr} \mypropto
      \sqrt{
          w_{ij}^{\Tr}
          w_{ij,\mn}
      }
    \end{equation}

    \option[sum-atom-atom]%
    \fdfindex*{TS!Weight.Method:sum-atom-atom}%

    Weight according to the total sum of the atomic density matrix
    sub-blocks
    \begin{equation}
      w_{ij,\mn}^{\Sigma} \mypropto
      \sqrt{
          % First the i'th atom
          \sum_{\in\{i\}}(\Delta\rho_{\mn}^L)^2
          \; % ensure a little space between them
          % second the j'th atom
          \sum_{\in\{j\}}(\Delta\rho_{\mn}^L)^2
      }
    \end{equation}

    \option[sum-atom-orb]%
    \fdfindex*{TS!Weight.Method:sum-atom-orb}%

    Weight according to the total sum of the atomic density matrix
    sub-block times the weight of the orbital weight
    \begin{equation}
      w_{ij,\mn}^{\Sigma} \mypropto
      \sqrt{
          w_{ij}^{\Sigma}
          w_{ij,\mn}
      }
    \end{equation}

    \option[mean]%
    \fdfindex*{TS!Weight.Method:mean}%

    A standard average.

  \end{fdfoptions}


  Each of the methods (except \fdf*{mean}) comes in a correlated and
  uncorrelated variant where $\sum$ is either outside or inside the
  square, respectively.

\end{fdfentry}

\begin{fdfentry}{TS!Weight.k.Method}[string]<correlated|uncorrelated>

  Control weighting \emph{per} $k$-point or the full sum. I.e. if
  \fdf*{uncorrelated} is used it will weight $n_k$ times if there are
  $n_k$ $k$-points in the Brillouin zone.

\end{fdfentry}

\begin{fdflogicalT}{TS!Forces}

  Control whether the forces are calculated. If \emph{not} \tsiesta\
  will use slightly less memory and the performance slightly
  increased, however the final forces shown are incorrect.

  If this is \fdftrue\ the file \sysfile{TSFA} (and possibly the
  \sysfile{TSFAC}) will be created. They contain forces for the atoms
  that are having updated density-matrix elements
  (\fdf{TS!Elec.<>!DM-update:all}).

  Generally one should not expect good forces close to the
  electrode/device interface since this typically has some
  electrostatic effects that are inherent to the \tsiesta\ method.
  Forces on atoms \emph{far} from the electrode can safely be
  analyzed.

\end{fdflogicalT}

\begin{fdfentry}{TS!dQ}[string]<none|buffer|fermi>
  \fdfindex*{TS!dQ:fermi}

  Any excess/deficiency of charge can be re-adjusted after each
  \tsiesta\ cycle to reduce charge fluctuations in the cell.

  \note recommended to \emph{only} use charge corrections for
  $0\,\mathrm{V}$ calculations.

  The non-neutral charge in \tsiesta\ cycles is an expression of one
  of the following things:
  \begin{enumerate}
    \item An incorrect screening towards the electrodes. To check
    this, simply add more electrode layers towards the device at each
    electrode and see how the charge evolves. It should tend to zero.

    The best way to check this is to follow these steps:
    \begin{enumerate}
      \item%
      Perform a \siesta-only calculation (the resulting DM
      should be used as the starting point for both following
      calculations)

      \item%
      Perform a \tsiesta\ calculation with the option
      \fdf{TS!Elecs!DM.Init:diagon} (please note that the electrode
      option has precedence, so remove any entry from the
      \fdf{TS!Elec.<>} block)

      \item%
      Perform a \tsiesta\ calculation with the option
      \fdf{TS!Elec.<>!DM-init:bulk} (please note that the electrode
      option has precedence, so remove any entry from the
      \fdf{TS!Elec.<>} block)

    \end{enumerate}

    Now compare the final output and the initial charge distribution,
    e.g.:
    \begin{output}
>>> TS.Elecs.DM.Init diagon
transiesta: Charge distribution, target =    396.00000
Total charge                  [Q]  :   396.00000

>>> TS.Elecs.DM.Init bulk
transiesta: Charge distribution, target =    396.00000
Total charge                  [Q]  :   395.9995
\end{output}

    The above shows that there is very little charge difference
    between the bulk electrode DM and the scattering region. This
    ensures that the charge distribution are similar and that your
    electrode is sufficiently screened.

    Additionally one may compare the final output such as total
    energies, calculated DOS and ADOS (see \tbtrans). If the two
    calculations show different properties, one should carefully
    examine the system setup.

    \item An incorrect reference energy level. In \tsiesta\ the Fermi
    level is calculated from the \siesta\ SCF. However, the \siesta\
    Fermi level corresponds to a periodic calculation and \emph{not}
    an open system calculation such as NEGF.

    If the first step shows a good screening towards the electrode it
    is usually the reference energy level, then use \fdf{TS!dQ:fermi}.

    \item A combination of the above, this is the typical case.
  \end{enumerate}

  \begin{fdfoptions}

    \option[none]%
    No charge corrections are introduced.

    \option[buffer]%
    Excess/missing electrons are placed in the buffer regions (buffer
    atoms are required to exist)

    \option[fermi] %
    Correct the charge filling by calculating a new reference energy
    level (referred to as the Fermi level). \\
    We approximate the contribution to be constant around the Fermi
    level and find
    \begin{equation}
      \label{eq:fermi-shift}
      \mathrm{d}E_F = \frac{Q'-Q}{Q|_{E_F}},
    \end{equation}
    where $Q'$ is the charge from a \tsiesta\ SCF step
    and $Q|_{E_F}$ is the equilibrium charge at the current Fermi
    level, $Q$ is the supposed charge to reside in the
    calculation. Fermi correction utilizes Eq.~\eqref{eq:fermi-shift} for
    the first correction and all subsequent corrections are based on a
    cubic spline interpolation to faster converge the
    ``correct'' Fermi level.

    This method will create a file called \file{TS\_FERMI}.

    \note correcting the reference energy level is a costly
    operation since the SCF cycle typically gets
    \emph{corrupted} resulting in many more SCF cycles.

  \end{fdfoptions}

\end{fdfentry}

\begin{fdfentry}{TS!dQ!Factor}[real]<0.8>

  Any positive value close to $1$. $0$ means no charge correction. $1$
  means total charge correction. This will reduce the fluctuations in
  the SCF and setting this to $1$ may result in difficulties in
  converging.

\end{fdfentry}

\begin{fdfentry}{TS!dQ!Fermi.Tolerance}[real]<0.01>

  The tolerance at which the charge correction will converge. Any
  excess/missing charge ($|Q'-Q|>\mathrm{Tol}$) will result in a
  correction for the Fermi level.

\end{fdfentry}

\begin{fdfentry}{TS!dQ!Fermi.Max}[energy]<$1.5\,\mathrm{eV}$>%

  The maximally allowed value that the Fermi level will change from a
  charge correction using the Fermi correction method. In case the
  Fermi level lies in between two bands a DOS of $0$ at the Fermi
  level will make the Fermi change equal to $\infty$. This is not
  physical and the user can thus truncate the correction.

  \note If you know the band-gab, setting this to $1/4$ (or smaller)
  of the band gab seems like a better value than the rather
  arbitrarily default one.

\end{fdfentry}

\begin{fdfentry}{TS!dQ!Fermi.Eta}[energy]<$1\,\mathrm{meV}$>%

  The $\eta$ value that we extrapolate the charge at the poles to.
  Usually a smaller $\eta$ value will mean larger changes in the
  Fermi level. If the charge convergence w.r.t. the Fermi level is
  fluctuating a lot one should increase this $\eta$ value.

\end{fdfentry}

\begin{fdflogicalT}{TS!HS.Save}
  \fdfindex*{TS!HS.Save:true}

  Must be \fdftrue\ for saving the Hamiltonian (\sysfile{TSHS}). Can only be set if
  \fdf{SolutionMethod} is not \fdf*{transiesta}.

  The default is \fdffalse\ for \fdf{SolutionMethod} different from
  \fdf*{transiesta} and if \code{--electrode} has not been passed as a
  command line argument.

  \note The \sysfile{TSHS} file format is deprecated and may be removed in
  future \siesta\ versions. The \sysfile{TS.HSX} file format (\fdf{Save!HS}) is
  feature complete, and should be used.

\end{fdflogicalT}

\begin{fdflogicalT}{TS!DE.Save}
  \fdfindex*{TS!DE.Save:true}

  Must be \fdftrue\ for saving the density and energy density matrix
  for continuation runs (\sysfile{TSDE}). Can only be set if
  \fdf{SolutionMethod} is not \fdf*{transiesta}.

  The default is \fdffalse\ for \fdf{SolutionMethod} different from
  \fdf*{transiesta} and if \code{--electrode} has not been passed as a
  command line argument.

\end{fdflogicalT}

\begin{fdflogicalF}{TS!S.Save}

  This is a flag mainly used for the Inelastica code to produce
  overlap matrices for Pulay corrections. This should only be used by
  advanced users.

\end{fdflogicalF}


\begin{fdflogicalF}{TS!SIESTA.Only}

  Stop \tsiesta\ right after the initial diagonalization run in
  \siesta. Upon exit it will also create the \sysfile{TSDE} file which
  may be used for initialization runs later.

  This may be used to start several calculations from the same initial
  density matrix, and it may also be used to rescale the Fermi level
  of electrodes. The rescaling is primarily used for semi-conductors
  where the Fermi levels of the device and electrodes may be
  misaligned.

\end{fdflogicalF}


\begin{fdflogicalF}{TS!Analyze}

  When using the BTD solution method (\fdf{TS!SolutionMethod}) this
  will analyze the Hamiltonian and printout an analysis of the
  sparsity pattern for optimal choice of the BTD partitioning
  algorithm.

  This yields information regarding the \fdf{TS!BTD!Pivot} flag.

  \note we advice users to \emph{always} run an analyzation step prior
  to actual calculation and select the \emph{best} BTD format. This
  analyzing step is very fast and may be performed on small
  work-station computers, even on systems of $\gg10,000$ orbitals.

  To run the analyzing step you may do:
  \begin{shellexample}
    siesta -fdf TS.Analyze RUN.fdf > analyze.out
  \end{shellexample}
  note that there is little gain on using MPI and it should complete
  within a few minutes, no matter the number of orbitals.

  Choosing the best one may be difficult. Generally one should choose
  the pivoting scheme that uses the least amount of memory. However,
  one should also choose the method with the largest block-size being
  as small as possible. As an example:
  \begin{output}[fontsize=\footnotesize]
TS.BTD.Pivot atom+GPS
...
    BTD partitions (7):
     [ 2984, 2776, 192, 192, 1639, 4050, 105 ]
    BTD matrix block size [max] / [average]: 4050 /   1705.429
    BTD matrix elements in % of full matrix:   47.88707 %

TS.BTD.Pivot atom+GGPS
...
    BTD partitions (6):
     [ 2880, 2916, 174, 174, 2884, 2910 ]
    BTD matrix block size [max] / [average]: 2916 /   1989.667
    BTD matrix elements in % of full matrix:   48.62867 %

  \end{output}
  Although the GPS method uses the least amount of memory, the GGPS
  will likely perform better as the largest block in GPS is $4050$
  vs. $2916$ for the GGPS method.

\end{fdflogicalF}

\begin{fdflogicalF}{TS!Analyze.Graphviz}
  \fdfdepend{TS!Analyze}

  If performing the analysis, also create the connectivity graph and
  store it as \file{GRAPHVIZ\_atom.gv} or \file{GRAPHVIZ\_orbital.gv}
  to be post-processed in Graphviz\footnote{\url{www.graphviz.org}}.

\end{fdflogicalF}


\vspace{5pt}
\section{STRUCTURAL RELAXATION, PHONONS, AND MOLECULAR DYNAMICS}



This functionality is not \siesta-specific, but is implemented to
provide a more complete simulation package. The program has an outer
geometry loop: it computes the electronic structure (and
thus the forces and stresses) for a given geometry, updates the
atomic positions (and maybe the cell vectors) accordingly and moves on
to the next cycle.
%
If there are molecular dynamics options missing you are highly
recommend to look into \fdf{MD.TypeOfRun:Lua} or
\fdf{MD.TypeOfRun:Master}.


Several options for MD and structural optimizations are
implemented, selected by
\begin{fdfentry}{MD.TypeOfRun}[string]<CG>

  \begin{fdfoptions}

    \option[CG]%
    \fdfindex*{MD.TypeOfRun:CG}%
    Performs an atomic coordinates optimization by using the conjugate gradients
    method. If \fdf{MD.VariableCell} is enabled (see below), the optimization
    includes the cell vectors.

    \option[Broyden]%
    \fdfindex*{MD.TypeOfRun:Broyden}%
    Performs an atomic coordinates optimization by using a modified Broyden
    method, which falls within the Quasi-Newton family of algorithms. If
    \fdf{MD.VariableCell} is enabled (see below), the optimization includes
    the cell vectors.

    \option[FIRE]%
    \fdfindex*{MD.TypeOfRun:FIRE}%
    Performs an atomic coordinates optimization by using the Fast Inertial
    Relaxation Engine (E. Bitzek et al, PRL 97, 170201, (2006)). If
    \fdf{MD.VariableCell} is enabled (see below), the optimization includes
    the cell vectors.
    FIRE avoids the need for linear search, thus making each individual iteration
    faster when compared to Quasi-Newton methods. However, it also needs more
    iterations to converge, so its efficiency is system-dependent.

    \option[Verlet]%
    \fdfindex*{MD.TypeOfRun:Verlet}%
    Standard Velocity-Verlet algorithm for NVE molecular dynamics.

    \option[Nose]%
    \fdfindex*{MD.TypeOfRun:Nose}%
    Constant temperature (NVT) MD with using a Nos\'e thermostat.

    \option[ParrinelloRahman]%
    \fdfindex*{MD.TypeOfRun:ParrinelloRahman}%
    Constant pressure (NPE) MD, controlled by the Parrinello-Rahman method.

    \option[NoseParrinelloRahman]%
    \fdfindex*{MD.TypeOfRun:NoseParrinelloRahman}%
    Constant temperature and pressure (NPT) MD using both methods above, the
    Nos\'e thermostat and the Parrinello-Rahman method.

    \option[Anneal]%
    \fdfindex*{MD.TypeOfRun:Anneal}%
    Constant temperature and/or pressure MD (see the variable
    \fdf{MD.AnnealOption} below), using a very simple velocity rescaling method
    (i.e. Berendsen thermostat and/or barostat \cite{Berendsen84}). It can be
    used to quickly equilibrate a system to a desired temperature and pressure.
    However, atomic velocities resulting from this option are non-canonical and
    thus tend to produce physically-inaccurate results. Therefore, one must
    rely on the Nos\'e and/or ParrinelloRahman options for production MD runs
    after the equilibration is done.

    \option[FC]%
    \fdfindex*{MD.TypeOfRun:FC}%
    Compute force constants matrix\index{Force Constants Matrix} for
    phonon calculations.

    \option[Master|Forces]%
    \fdfindex*{MD.TypeOfRun:Master}%
    \fdfindex*{MD.TypeOfRun:Forces}%
    Receive coordinates from, and return forces to, an external
    driver program, using MPI, Unix pipes, or Inet sockets for
    communication. The routines in module \program{fsiesta} allow the
    user's program to perform this communication transparently, as if
    \siesta\ were a conventional force-field subroutine. See
    \shell{Util/SiestaSubroutine/README} for details. WARNING: if this
    option is specified without a driver program sending data, siesta
    may hang without any notice.

    See directory \shell{Util/Scripting} \index{Scripting} for other driving
    options.

    \option[Lua]%
    \fdfindex*{MD.TypeOfRun:Lua}%
    Fully control the MD cycle and convergence path using an external
    Lua script.

    With an external Lua script one may control nearly everything from
    a script. One can query \emph{any} internal data-structures in
    \siesta\ and, similarly, return \emph{any} data thus overwriting
    the internals. A list of ideas which may be implemented in such a
    Lua script are:
    \begin{itemize}
      \item New geometry relaxation algorithms

      \item NEB calculations

      \item New MD routines

      \item Convergence tests of \fdf{Mesh!Cutoff} and
      \fdf{kgrid.MonkhorstPack}, or other parameters (currently basis
      set optimizations cannot be performed in the Lua script).

    \end{itemize}
    Sec.~\ref{sec:lua} for additional details (and a description of
    \program{flos} which implements some of the above mentioned items).

    Using this option requires the compilation of \siesta\ with the
    \program{flook} library.%
    \index{flook}\index{External library!flook}%
    If \siesta\ is not compiled as prescribed in Sec.~\ref{sec:libs}
    this option will make \siesta\ die.

    \option[TDED]%
    \fdfindex*{MD.TypeOfRun:TDED}%

    New option to perform time-dependent electron dynamics simulations
    (TDED) within RT-TDDFT. For more details see
    Sec.~\ref{sec:tddft}.

    The second run of \siesta\ uses this option with the files
    \sysfile{TDWF} and \sysfile{TDXV} present in the working
    directory.  In this option ions and electrons are assumed to move
    simultaneously. The occupied electronic states are time-evolved
    instead of the usual SCF calculations in each step.  Choose this
    option even if you intend to do only-electron dynamics. If you
    want to do an electron dynamics-only calculation set
    \fdf{MD.FinalTimeStep} equal to $1$. For optical response
    calculations switch off the external field during the second
    run. The \fdf{MD.LengthTimeStep}, unlike in the standard MD
    simulation, is defined by mulitpilication of \fdf{TDED!TimeStep}
    and \fdf{TDED!Nsteps}. In TDDFT calculations, the user defined
    \fdf{MD.LengthTimeStep} is ignored.


  \end{fdfoptions}

  \note if \fdf{Compat!Pre-v4-Dynamics} is \fdftrue\ this will default
  to \fdf*{Verlet}.

  Note that some options specified in later variables (like quenching)
  modify the behavior of these MD options.

  Appart from being able to act as a force subroutine for a driver
  program that uses module fsiesta, \siesta\ is also prepared to
  communicate with the i-PI code (see
  \url{https://github.com/i-pi/i-pi}).
  To do this, \siesta\ must be started after i-PI (it acts as a client
  of i-PI, communicating with it through Inet or Unix sockets), and
  the following lines must be present in the .fdf data file:
  \begin{fdfexample}
     MD.TypeOfRun      Master     # equivalent to 'Forces'
     Master.code       i-pi       # ( fsiesta | i-pi )
     Master.interface  socket     # ( pipes | socket | mpi )
     Master.address    localhost  # or driver's IP, e.g. 150.242.7.140
     Master.port       10001      # 10000+siesta_process_order
     Master.socketType inet       # ( inet | unix )
  \end{fdfexample}

\end{fdfentry}



\subsection{Compatibility with pre-v4 versions}
\index{Backward compatibility}

Starting in the summer of 2015, some changes were made to the behavior
of the program regarding default dynamics options and choice of
coordinates to work with during post-processing of the electronic
structure. The changes are:

\begin{itemize}
  \item %
  The default dynamics option is ``CG'' instead of ``Verlet''.

  \item%
  The coordinates, if moved by the dynamics routines, are reset to
  their values at the previous step for the analysis of the electronic
  structure (band structure calculations, DOS, LDOS, etc).

  \item%
  Some output files reflect the values of the ``un-moved''
  coordinates.

  \item%
  The default convergence criteria is now \emph{both} density and
  Hamiltonian convergence, see \fdf{SCF.DM!Converge} and
  \fdf{SCF.H!Converge}.

\end{itemize}

To recover the previous behavior, the user can turn on the
compatibility switch \fdf*{Compat!Pre-v4-Dynamics}, which is off by
default.

Note that complete compatibility cannot be perfectly guaranteed.


\subsection{Structural relaxation}

In this mode of operation, the program moves the atoms (and optionally
the cell vectors) trying to minimize the forces (and stresses) on
them.

These are the options common to all relaxation methods. If the Zmatrix
input option is in effect (see Sec.~\ref{sec:Zmatrix}) the
Zmatrix-specific options take precedence.  The 'MD' prefix is
misleading but kept for historical reasons.

\begin{fdflogicalF}{MD.VariableCell}
  \index{cell relaxation}

  If \fdftrue, the lattice is relaxed together with the atomic
  coordinates. It allows to target hydrostatic pressures or arbitrary
  stress tensors. See \fdf{MD.MaxStressTol},
  \fdf{Target!Pressure}, \fdf{Target!Stress.Voigt},
  \fdf{Constant!Volume}, and
  \fdf{MD.PreconditionVariableCell}.

  \note only compatible with \fdf{MD.TypeOfRun:CG},
  \fdf*{Broyden} or \fdf*{fire}.

\end{fdflogicalF}


\begin{fdflogicalF}{Constant!Volume}
  \fdfindex*{MD.ConstantVolume}
  \fdfdeprecates{MD.ConstantVolume}%
  \index{constant-volume cell relaxation}

  If \fdftrue, the cell volume is kept constant in a variable-cell
  relaxation: only the cell shape and the atomic coordinates are
  allowed to change.  Note that it does not make much sense to specify
  a target stress or pressure in this case, except for anisotropic
  (traceless) stresses.  See \fdf{MD.VariableCell},
  \fdf{Target.Stress.Voigt}.

  \note only compatible with \fdf{MD.TypeOfRun:CG},
  \fdf*{Broyden} or \fdf*{fire}.

\end{fdflogicalF}

\begin{fdflogicalF}{MD.RelaxCellOnly}
  \index{relaxation of cell parameters only}

  If \fdftrue, only the cell parameters are relaxed (by the Broyden or
  FIRE method, not CG). The atomic coordinates are re-scaled to the
  new cell, keeping the fractional coordinates constant. For
  \fdf{Zmatrix} calculations, the fractional position of the first
  atom in each molecule is kept fixed, and no attempt is made to
  rescale the bond distances or angles.

  \note only compatible with \fdf{MD.TypeOfRun:Broyden} or \fdf*{fire}.

\end{fdflogicalF}

\begin{fdfentry}{MD.MaxForceTol}[force]<$0.04\,\mathrm{eV/Ang}$>

  Force tolerance in coordinate optimization.
  Run stops if the maximum atomic force is
  smaller than \fdf{MD.MaxForceTol} (see \fdf{MD.MaxStressTol}
  for variable cell).

\end{fdfentry}

\begin{fdfentry}{MD.MaxStressTol}[pressure]<$1\,\mathrm{GPa}$>

  Stress tolerance in variable-cell CG optimization. Run stops if the
  maximum atomic force is smaller than \fdf{MD.MaxForceTol} and the
  maximum stress component is smaller than \fdf{MD.MaxStressTol}.

  Special consideration is needed if used with Sankey-type basis sets,
  since the combination of orbital kinks at the cutoff radii and the
  finite-grid integration originate discontinuities in the stress
  components, whose magnitude depends on the cutoff radii (or energy
  shift) and the mesh cutoff. The tolerance has to be larger than the
  discontinuities to avoid endless optimizations if the target stress
  happens to be in a discontinuity.

\end{fdfentry}

\begin{fdfentry}{MD.Steps}[integer]<0>
  \fdfindex*{MD.NumCGsteps}
  \fdfdeprecates{MD.NumCGsteps}

  Maximum number of steps in a minimization routine
  (the minimization will stop if tolerance is reached before; see
  \fdf{MD.MaxForceTol} below).

  \note The old flag \fdf{MD.NumCGsteps} will remain for historical
  reasons.

\end{fdfentry}

\begin{fdfentry}{MD.MaxDispl}[length]<$0.2\,\mathrm{Bohr}$>
  \fdfindex*{MD.MaxCGDispl}
  \fdfdeprecates{MD.MaxCGDispl}

  Maximum atomic displacements in an optimization move.

  In the Broyden optimization method, it is also possible to limit
  indirectly the \textit{initial\/} atomic displacements using
  \fdf{MD.Broyden.Initial.Inverse.Jacobian}. For the \fdf*{FIRE} method, the
  same result can be obtained by choosing a small time step.

  Note that there are Zmatrix-specific options that override this option.

  \note The old flag \fdf{MD.MaxCGDispl} will remain for historical
  reasons.

\end{fdfentry}

\begin{fdfentry}{MD.PreconditionVariableCell}[length]<$5\,\mathrm{Ang}$>

  A length to multiply to the strain components in a variable-cell
  optimization. The strain components enter the minimization on the
  same footing as the coordinates. For good efficiency, this length
  should make the scale of energy variation with strain similar to the
  one due to atomic displacements. It is also used for the application
  of the \fdf{MD.MaxDispl} value to the strain components.

\end{fdfentry}


\begin{fdfentry}{ZM.ForceTolLength}[force]<$0.00155574\,\mathrm{Ry/Bohr}$>

  Parameter that controls the convergence with respect to forces on
  Z-matrix lengths

\end{fdfentry}


\begin{fdfentry}{ZM.ForceTolAngle}[torque]<$0.00356549\,\mathrm{Ry/rad}$>

  Parameter that controls the convergence with respect to forces on
  Z-matrix angles

\end{fdfentry}

\begin{fdfentry}{ZM.MaxDisplLength}[length]<$0.2\,\mathrm{Bohr}$>

  Parameter that controls the maximum change in a Z-matrix length
  during an optimisation step.

\end{fdfentry}

\begin{fdfentry}{ZM.MaxDisplAngle}[angle]<$0.003\,\mathrm{rad}$>

  Parameter that controls the maximum change in a Z-matrix angle
  during an optimisation step.

\end{fdfentry}



\subsubsection{Conjugate-gradients optimization}

This was historically the default geometry-optimization method, and
all the above options were introduced specifically for it, hence their
names. The following pertains only to this method:

\index{Conjugate-gradient history information}
\begin{fdflogicalF}{MD.UseSaveCG}
  \index{reading saved data!CG}

  Instructs to read the conjugate-gradient hystory information stored
  in file \sysfile{CG} by a previous run.

  \note to get actual continuation of iterrupted CG runs, use
  together with \fdf{MD.UseSaveXV} \fdftrue\ with the \sysfile*{XV}
  file generated in the same run as the CG file.  If the required file
  does not exist, a warning is printed but the program does not
  stop. Overrides \fdf{UseSaveData}.

  \note no such feature exists yet for a Broyden-based relaxation.

\end{fdflogicalF}

\subsubsection{Broyden optimization}

It uses the modified Broyden algorithm to
build up the Jacobian matrix. (See D.D. Johnson, PRB 38, 12807
(1988)). (Note: This is not BFGS.)

\begin{fdfentry}{MD.Broyden!History.Steps}[integer]<$5$>
  \index{Broyden optimization}

  Number of relaxation steps during which the modified Broyden
  algorithm builds up the Jacobian matrix.

\end{fdfentry}

\begin{fdflogicalT}{MD.Broyden!Cycle.On.Maxit}

  Upon reaching the maximum number of history data sets which are kept
  for Jacobian estimation, throw away the oldest and shift the rest to
  make room for a new data set. The alternative is to re-start the
  Broyden minimization algorithm from a first step of a diagonal
  inverse Jacobian (which might be useful when the minimization is
  stuck).

\end{fdflogicalT}

\begin{fdfentry}{MD.Broyden!Initial.Inverse.Jacobian}[real]<$1$>

  Initial inverse Jacobian for the optimization procedure. (The units
  are those implied by the internal \siesta\ usage. The default value
  seems to work well for most systems.

\end{fdfentry}



\subsubsection{FIRE relaxation}

Implementation of the Fast Inertial Relaxation Engine (FIRE) method
(E. Bitzek et al, PRL 97, 170201, (2006) in a manner compatible with
the CG and Broyden modes of relaxation. (An older implementation
activated by the \fdf*{MD.FireQuench} variable is still available).

\begin{fdfentry}{MD.FIRE.TimeStep}[time]<\fdfvalue{MD.LengthTimeStep}>

  The (fictitious) time-step for FIRE relaxation.  This is the main
  user-variable when the option \fdf*{FIRE} for
  \fdf{MD.TypeOfRun} is active.

  \note the default value is encouraged to be changed as the link to
  \fdf{MD.LengthTimeStep} is misleading.

  There are other low-level options tunable by the user (see the
  routines \texttt{fire\_optim} and \texttt{cell\_fire\_optim} for
  more details.

\end{fdfentry}


\ifdeprecated
% The below options are deprecated in favor of:
% MD.TypeOfRun fire

\subsubsection{Quenched MD}

These methods are really based on molecular dynamics, but are used for
structural relaxation.

Note that the Zmatrix input option (see Sec.~\ref{sec:Zmatrix}) is not
compatible with molecular dynamics. The initial geometry can be
specified using the Zmatrix format, but the Zmatrix generalized
coordinates will not be updated.

Note also that the force and stress tolerances have no effect on
the termination conditions of these methods. They run for the number
of MD steps requested (this is arguably a bug).

\begin{fdflogicalF}{MD.Quench}

  Logical option to perform a power quench during the molecular
  dynamics.  In the power quench, each velocity component is set to
  zero if it is opposite to the corresponding force of that
  component. This affects atomic velocities, or unit-cell velocities
  (for cell shape optimizations).

  \note only applicable for \fdf{MD.TypeOfRun:Verlet} or
  \fdf*{ParrinelloRahman}.
  %
  It is incompatible with Nose thermostat options.

  \note \fdf{MD.Quench} is superseded by \fdf{MD.FireQuench} (see
  below).

\end{fdflogicalF}


\begin{fdflogicalF}{MD.FireQuench}

  See the new option \fdf*{FIRE} for \fdf{MD.TypeOfRun}.

  Logical option to perform a FIRE quench during a Verlet molecular
  dynamics run, as described by Bitzek \textit{et al.} in
  Phys. Rev. Lett. \textbf{97}, 170201 (2006). It is a relaxation
  algorithm, and thus the dynamics are of no interest per se: the
  initial time-step can be played with (it uses
  \fdf{MD.LengthTimeStep} as initial $\Delta t$), as well as the
  initial temperature (recommended 0) and the atomic masses
  (recommended equal). Preliminary tests seem to indicate that the
  combination of $\Delta t = 5$ fs and a value of 20 for the atomic
  masses works reasonably. The dynamics stops when the force tolerance
  is reached (\fdf{MD.MaxForceTol}). The other parameters
  controlling the algorithm (initial damping, increase and decrease
  thereof etc.) are hardwired in the code, at the recommended values
  in the cited paper, including $\Delta t_{max} = 10$ fs.

  Only available for \fdf{MD.TypeOfRun:Verlet}
  It is incompatible with Nose thermostat options. No variable
  cell option implemented for this at this stage.
  \fdf{MD.FireQuench} supersedes \fdf{MD.Quench}. This option is
  deprecated. The new option \fdf*{FIRE} for \fdf{MD.TypeOfRun} should be
  used instead.

\end{fdflogicalF}


\fi

\subsection{Target stress options}

Useful for structural optimizations and constant-pressure molecular
dynamics.

\begin{fdfentry}{Target!Pressure}[pressure]<$0\,\mathrm{GPa}$>
  \fdfindex*{MD.TargetPressure}
  \fdfdeprecates{MD.TargetPressure}

  Target pressure for Parrinello-Rahman method, variable cell
  optimizations, and annealing options.

  \note this is only compatible with
  \fdf{MD.TypeOfRun} \fdf*{ParrinelloRahman}, \fdf*{NoseParrinelloRahman},
  \fdf*{CG}, \fdf*{Broyden} or \fdf*{FIRE} (variable cell), or \fdf*{Anneal}
  (if \fdf{MD.AnnealOption} \fdf*{Pressure} or \fdf*{TemperatureandPressure}).

\end{fdfentry}


\begin{fdfentry}{Target!Stress.Voigt}[block]<$-1$ $-1$ $-1$ $0$ $0$ $0$>
  \fdfdeprecates{MD.TargetStress}

  External or target stress tensor for variable cell optimizations.
  Stress components are given in a line, in the Voigt order \texttt{xx, yy,
      zz, yz, xz, xy}. In units of \fdf{Target!Pressure}, but
  with the opposite sign. For example, a uniaxial compressive stress
  of 2 GPa along the 100 direction would be given by
  \begin{fdfexample}
     Target.Pressure  2. GPa
     %block Target.Stress.Voigt
         -1.0  0.0  0.0  0.0  0.0  0.0
     %endblock
  \end{fdfexample}

  Only used if \fdf{MD.TypeOfRun} is \fdf*{CG}, \fdf*{Broyden} or
  \fdf*{FIRE} and \fdf{MD.VariableCell} is \fdftrue.

\end{fdfentry}

\begin{fdfentry}{MD.TargetStress}[block]<$-1$ $-1$ $-1$ $0$ $0$ $0$>
  \fdfdeprecatedby{Target!Stress.Voigt}

  Same as \fdf{Target!Stress.Voigt} but the order is same as older
  \siesta\ version (prior to 4.1). Order is \texttt{xx, yy, zz, xy,
      xz, yz}.

\end{fdfentry}


\begin{fdflogicalF}{MD.RemoveIntramolecularPressure}
  \index{removal of intramolecular pressure}

  If \fdftrue, the contribution to the stress coming from the internal
  degrees of freedom of the molecules will be subtracted from the
  stress tensor used in variable-cell optimization or variable-cell
  molecular-dynamics.  This is done in an approximate manner, using
  the virial form of the stress, and assumming that the ``mean force''
  over the coordinates of the molecule represents the
  ``inter-molecular'' stress. The correction term was already computed
  in earlier versions of \siesta\ and used to report the ``molecule
  pressure''. The correction is now computed molecule-by-molecule if
  the Zmatrix format is used.

  If the intra-molecular stress is removed, the corrected static and
  total stresses are printed in addition to the uncorrected items.
  The corrected Voigt form is also printed.

  \note versions prior to 4.1 (also 4.1-beta releases) printed the
  Voigt stress-tensor in this format: \shell{[x, y, z, xy, yz,
      xz]}. In 4.1 and later \siesta\ \emph{only} show the correct
  Voigt representation: \shell{[x, y, z, yz, xz, xy]}.

\end{fdflogicalF}


\subsection{Molecular dynamics}

In this mode of operation, the program moves the atoms (and optionally
the cell vectors) in response to the forces (and stresses), using the
classical equations of motion.

Note that the \fdf{Zmatrix} input option (see Sec.~\ref{sec:Zmatrix}) is not
compatible with molecular dynamics. The initial geometry can be
specified using the Zmatrix format, but the Zmatrix generalized
coordinates will not be updated.


\begin{fdfentry}{MD.InitialTimeStep}[integer]<$1$>

  Initial time step of the MD simulation.  In the current version of
  \siesta\ it must be 1.

  Used only if \fdf{MD.TypeOfRun} is not \fdf*{CG} or \fdf*{Broyden}.

\end{fdfentry}

\begin{fdfentry}{MD.FinalTimeStep}[integer]<\fdfvalue{MD.Steps}>

  Final time step of the MD simulation.

\end{fdfentry}


\begin{fdfentry}{MD.LengthTimeStep}[time]<$1\,\mathrm{fs}$>

  Length of the time step of the MD simulation.

\end{fdfentry}

\begin{fdfentry}{MD.InitialTemperature}[temperature/energy]<$0\,\mathrm K$>

  Initial temperature for the MD run. The atoms are assigned random
  velocities drawn from the Maxwell-Bolzmann distribution with the
  corresponding temperature. The constraint of zero center of mass
  velocity is imposed.

  \note only used if \fdf{MD.TypeOfRun} \fdf*{Verlet}, \fdf*{Nose},
  \fdf*{ParrinelloRahman}, \fdf*{NoseParrinelloRahman} or
  \fdf*{Anneal}.

\end{fdfentry}

\begin{fdfentry}{MD.TargetTemperature}[temperature/energy]<$0\,\mathrm K$>

  Target temperature for Nose thermostat and annealing options.

  \note only used if \fdf{MD.TypeOfRun} \fdf*{Nose},
  \fdf*{NoseParrinelloRahman} or
  \fdf*{Anneal} if \fdf{MD.AnnealOption} is \fdf*{Temperature} or
  \fdf*{TemperatureandPressure}.

\end{fdfentry}

\begin{fdfentry}{MD.NoseMass}[moment of inertia]<$100\,\mathrm{Ry\,fs^2}$>

  Generalized mass of Nose variable.  This determines the time scale
  of the Nose variable dynamics, and the coupling of the thermal bath
  to the physical system.

  Only used for Nose MD runs.

\end{fdfentry}

\begin{fdfentry}{MD.ParrinelloRahmanMass}[moment of inertia]<$100\,\mathrm{Ry\,fs^2}$>

  Generalized mass of Parrinello-Rahman variable.  This determines the
  time scale of the Parrinello-Rahman variable dynamics, and its
  coupling to the physical system.

  Only used for Parrinello-Rahman MD runs.

\end{fdfentry}

\begin{fdfentry}{MD.AnnealOption}[string]<TemperatureAndPressure>

  Type of annealing MD to perform. The target temperature or pressure
  are achieved by velocity and unit cell rescaling, in a given time
  determined by the variable \fdf{MD.TauRelax} below. These are based on
  the Berendsen thermostat and barostats, respectively \cite{Berendsen84}.
  \begin{fdfoptions}
    \option[Temperature]%
    Reach a target temperature by velocity rescaling

    \option[Pressure]%
    Reach a target pressure by scaling of the unit cell size and shape

    \option[TemperatureandPressure]%
    Reach a target temperature and pressure by velocity rescaling and
    by scaling of the unit cell size and shape
  \end{fdfoptions}

  Only applicable for \fdf{MD.TypeOfRun:Anneal}.

\end{fdfentry}

\begin{fdfentry}{MD.TauRelax}[time]<$100\,\mathrm{fs}$>

  Relaxation time to reach target temperature and/or pressure in
  annealing MD. Note that this is a ``relaxation time'', and as such
  it gives a rough estimate of the time needed to achieve the given
  targets. As a normal simulation also exhibits oscillations, the
  actual time needed to reach the \emph{averaged} targets will be
  significantly longer.

  When using the barostat, the actual time required to reach the target
  pressure will depend on the ratio between \fdf{MD.TauRelax} and
  \fdf{MD.BulkModulus}.

  Only applicable for \fdf{MD.TypeOfRun:Anneal}.

\end{fdfentry}

\begin{fdfentry}{MD.BulkModulus}[pressure]<$100\,\mathrm{Ry/Bohr^3}$>

  Estimate (may be rough) of the bulk modulus of the system.  This is
  needed to set the rate of change of cell shape to reach target
  pressure in annealing MD.

  Only applicable for \fdf{MD.TypeOfRun} \fdf*{Anneal}, when
  \fdf{MD.AnnealOption} is \fdf*{Pressure} or \fdf*{TemperatureAndPressure}

\end{fdfentry}


\subsection{Output options for dynamics}

\subsection{Standard output} \index{output!main output file}

\siesta\ writes a log of its workings to standard output (unit 6),
which is usually redirected to an ``output file''.

A brief description follows. See the example cases in the
siesta/Tests directory for illustration.

The program starts writing the version of the code which is
used. Then, the input \fdflib\ file is dumped into the output file as is
(except for empty lines). The program does part of the reading and
digesting of the data at the beginning within the \program{redata}
subroutine. It prints some of the information it digests. It is
important to note that it is only part of it, some other information
being accessed by the different subroutines when they need it during
the run (in the spirit of \fdflib\ input).  A complete list of the input
used by the code can be found at the end in the file \file{fdf.log},
including defaults used by the code in the run.

After that, the program reads the pseudopotentials, factorizes them
into Kleinman-Bylander form, and generates (or reads) the atomic basis
set to be used in the simulation. These stages are documented in the
output file.

The simulation begins after that, the output showing information of
the MD (or CG) steps and the SCF cycles within.  Basic descriptions of
the process and results are presented. The user has the option to
customize it, however,\index{output!customization} by defining
different options that control the printing of informations like
coordinates, forces, $\vec k$ points, etc.  The options are discussed
in the appropriate sections, but take into account the behavior of the
legacy \fdf{LongOutput} option, as in the current implementation might
silently activate output to the main .out file at the expense of
auxiliary files.

\begin{fdflogicalF}{LongOutput}
  \index{output!long}

  \siesta\ can write to standard output different data sets depending
  on the values for output options described below.  By default
  \siesta\ will not write most of them. They can be large for large
  systems (coordinates, eigenvalues, forces, etc.)  and, if written to
  standard output, they accumulate for all the steps of the
  dynamics. \siesta\ writes the information in other files (see Output
  Files) in addition to the standard output, and these can be
  cumulative or not.

  Setting \fdf{LongOutput} to \fdftrue\ changes the default of some
  options, obtaining more information in the output (verbose).  In
  particular, it redefines the defaults for the following:

  \begin{itemize}

    \item \fdf{WriteKpoints}%
    \index{output!grid $\vec k$ points}

    \item \fdf{WriteKbands}%
    \index{output!band $\vec k$ points}

    \item \fdf{WriteCoorStep}%
    \index{output!atomic coordinates!in a dynamics step}

    \item \fdf{WriteForces}%
    \index{output!forces}

    \item \fdf{WriteEigenvalues}%
    \index{output!eigenvalues}

    \item \fdf{WriteWaveFunctions}%
    \index{output!wave functions}

    \item \fdf{WriteMullikenPop}%
    \index{output!Mulliken analysis}%
    \index{Mulliken population analysis}%
    (it sets it to 1)

  \end{itemize}

  The specific changing of any of these options has precedence.

\end{fdflogicalF}


\subsection{Output to dedicated files}%
\index{output!dedicated files}

\siesta\ can produce a wealth of information in dedicated files,
with specific formats, that can be used for further analysis. See the
appropriate sections, and the appendix on file formats.
Please take into account the behavior of
\fdf{LongOutput}, as in the current implementation might silently
activate output to the main .out file at the expense of auxiliary
files.



\subsection{Restarting geometry optimizations and MD runs}

Every time the atoms move, either during coordinate relaxation or
molecular dynamics, their \textbf{positions predicted for next step} and
\textbf{current velocities} are stored in file SystemLabel.XV, where
SystemLabel is the value of that \fdflib\ descriptor (or 'siesta' by
default).  The shape of the unit cell and its associated 'velocity'
(in Parrinello-Rahman dynamics) are also stored in this file. For MD
runs of type Verlet, Parrinello-Rahman, Nose,
Nose-Parrinello-Rahman, or Anneal, a file named SystemLabel.VERLET\_RESTART,
SystemLabel.PR\_RESTART, SystemLabel.NOSE\_RESTART,
SystemLabel.NPR\_RESTART, or SystemLabel.ANNEAL\_RESTART,
respectively, is created to hold the values
of auxiliary variables needed for a completely seamless
continuation.

If the restart file is not available, a simulation can still make use
of the XV information, and ``restart'' by basically repeating the
last-computed step (the positions are shifted backwards by using a
single Euler-like step with the current velocities as derivatives).
While this feature does not result in seamless continuations, it
allows cross-restarts (those in which a simulation of one kind (e.g.,
Anneal) is followed by another (e.g., Nose)), and permits
to re-use dynamical information from old runs.

This restart fix is not satisfactory from a fundamental point of view,
so the MD subsystem in \siesta\ will have to be redesigned
eventually. In the meantime, users are reminded that the scripting
hooks being steadily introduced (see \shell{Util/Scripting}) might be used to
create custom-made MD scripts.


\subsection{Use of general constraints}

\textbf{Note:} The Zmatrix format (see Sec.~\ref{sec:Zmatrix}) provides
an alternative constraint formulation which can be useful for system
involving molecules.

\begin{fdfentry}{Geometry!Constraints}[block]
  \index{constraints in relaxations}

  Constrains certain atomic coordinates or cell parameters in a
  consistent method.

  There are a high number of configurable parameters that may be used
  to control the relaxation of the coordinates.

  \note \siesta\ prints out a small section of how the constraints are
  recognized.

  \def\directionalconstraint{\note these specifications are apt for \emph{directional}
    constraints.}


  \begin{fdfoptions}
    \option[atom|position]%
    Fix certain atomic coordinates.

    This option takes a variable number of integers which each
    correspond to the atomic index (or input sequence) in
    \fdf{AtomicCoordinatesAndAtomicSpecies}.

    \fdf*{atom} is now the preferred input option while
    \fdf*{position} still works for backwards compatibility.

    One may also specify ranges of atoms according to:

    \begin{fdfoptions}
      \option[{atom \emph{A} [\emph{B} [\emph{C} [\dots]]]}]%
      A sequence of atomic indices which are constrained.

      % Generic input (compatible with the <= 4.0)
      \option[{atom from \emph{A} to \emph{B} [step \emph{s}]}]%
      Here atoms \emph{A} up to and including \emph{B} are
      constrained.
      %
      If \fdf*{step <s>} is given, the range
      \emph{A}:\emph{B} will be taken in steps of \emph{s}.

      \begin{fdfexample}
        atom from 3 to 10 step 2
      \end{fdfexample}
      will constrain atoms 3, 5, 7 and 9.

      \option[{atom from \emph{A} plus/minus \emph{B} [step
          \emph{s}]}]%
      Here atoms \emph{A} up to and including $\emph{A}+\emph{B}-1$
      are constrained.
      %
      If \fdf*{step <s>} is given, the range
      \emph{A}:$\emph{A}+\emph{B}-1$ will be taken in steps of
      \emph{s}.

      % Generic input (compatible with the <= 4.0)
      \option[atom {[\emph{A}, \emph{B} -\mbox{}- \emph{C} [step \emph{s}], \emph{D}]}]%
      Equivalent to \fdf*{from \dots to} specification, however in a
      shorter variant. Note that the list may contain arbitrary number
      of ranges and/or individual indices.

      \begin{fdfexample}
        atom [2, 3 -- 10 step 2, 6]
      \end{fdfexample}
      will constrain atoms 2, 3, 5, 7, 9 and 6.

      \begin{fdfexample}
        atom [2, 3 -- 6, 8]
      \end{fdfexample}
      will constrain atoms 2, 3, 4, 5, 6 and 8.

      \option[atom all]%
      Constrain all atoms.

    \end{fdfoptions}

    \directionalconstraint


    \option[Z]%
    Equivalent to \fdf*{atom} with all indices of the atoms that
    have atomic number equal to the specified number.

    \directionalconstraint


    \option[species-i]%
    Equivalent to \fdf*{atom} with all indices of the atoms that
    have species according to the \fdf{ChemicalSpeciesLabel} and
    \fdf{AtomicCoordinatesAndAtomicSpecies}.

    \directionalconstraint


    \option[center]%
    One may retain the coordinate center of a
    range of atoms (say molecules or other groups of atoms).

    Atomic indices may be specified according to \fdf*{atom}.

    \directionalconstraint


    \option[rigid|molecule]%
    Move a selection of atoms together as though they where one atom.

    The forces are summed and averaged to get a net-force on the
    entire molecule.

    Atomic indices may be specified according to \fdf*{atom}.

    \directionalconstraint


    \option[rigid-max|molecule-max]%
    Move a selection of atoms together as though they where one atom.

    The maximum force acting on one of the atoms in the selection will
    be expanded to act on all atoms specified.

    Atomic indices may be specified according to \fdf*{atom}.

    \directionalconstraint


    \option[cell-angle]%
    Control whether the cell angles ($\alpha$, $\beta$, $\gamma$) may
    be altered.

    This takes either one or more of
    \fdf*{alpha}/\fdf*{beta}/\fdf*{gamma} as argument.

    \fdf*{alpha} is the angle between the 2nd and 3rd cell vector.

    \fdf*{beta} is the angle between the 1st and 3rd cell vector.

    \fdf*{gamma} is the angle between the 1st and 2nd cell vector.

    \note currently only one angle can be constrained at a time and it
    forces only the spanning vectors to be relaxed.


    \option[cell-vector]%
    Control whether the cell vectors ($A$, $B$, $C$) may be altered.

    This takes either one or more of \fdf*{A}/\fdf*{B}/\fdf*{C} as
    argument.

    Constraining the cell-vectors are only allowed if they only have a
    component along their respective Cartesian
    direction. I.e. \fdf*{B} must only have a $y$-component.


    \option[stress]%
    Control which of the 6 stress components are constrained.

    Numbers $1\le i\le6$ where $1$ corresponds
    to the \emph{XX} stress-component, $2$ is \emph{YY}, $3$ is
    \emph{ZZ}, $4$ is \emph{YZ}/\emph{ZY}, $5$ is \emph{XZ}/\emph{ZX}
    and $6$ is \emph{XY}/\emph{YX}.

    The text specifications are also allowed.


    \option[routine]%
    This calls the \program{constr} routine specified in the file:
    \file{constr.f}. Without having changed the corresponding source
    file, this does nothing.
    See details and comments in the source-file.


    \option[clear]%
    Remove constraints on selected atoms from all previously specified
    constraints.

    This may be handy when specifying constraints via \fdf*{Z} or
    \fdf*{species-i}.

    Atomic indices may be specified according to \fdf*{atom}.


    \option[clear-prev]
    Remove constraints on selected atoms from the \emph{previous} specified
    constraint.

    This may be handy when specifying constraints via \fdf*{Z} or
    \fdf*{species-i}.

    Atomic indices may be specified according to \fdf*{atom}.

    \note two consecutive \fdf*{clear-prev} may be used in conjunction
    as though the atoms where specified on the same line.

  \end{fdfoptions}

  It is instructive to give an example of the input options presented.

  Consider a benzene molecule ($\mathrm{C}_6\mathrm{H}_6$) and we wish
  to relax all Hydrogen atoms (and no stress in $x$ and $y$
  directions). This may be accomplished with this
  \begin{fdfexample}
    %block Geometry.Constraints
      Z 6
      stress 1 2
    %endblock
  \end{fdfexample}
  Or as in this example
  \begin{fdfexample}
    %block AtomicCoordinatesAndAtomicSpecies
      ... ... ... 1   # C 1
      ... ... ... 2   # H 2
      ... ... ... 1   # C 3
      ... ... ... 2   # H 4
      ... ... ... 1   # C 5
      ... ... ... 2   # H 6
      ... ... ... 1   # C 7
      ... ... ... 2   # H 8
      ... ... ... 1   # C 9
      ... ... ... 2   # H 10
      ... ... ... 1   # C 11
      ... ... ... 2   # H 12
      stress XX YY
    %endblock
    %block Geometry.Constraints
      atom from 1 to 12 step 2
      stress XX YY
    %endblock
    %block Geometry.Constraints
      atom [1 -- 12 step 2]
      stress XX 2
    %endblock
    %block Geometry.Constraints
      atom all
      clear-prev [2 -- 12 step 2]
      stress 1 YY
    %endblock
  \end{fdfexample}
  where the 3 last blocks all create the same result.

  Finally, the \emph{directional} constraint is an important and often
  useful feature.
  The directional constraints will subtract the force projected onto
  the direction specified. Hence an $x$ directional constraint will
  remove the force component along the $x$ direction $f_x\to 0$.

  When relaxing complex structures it may be advantageous to first
  relax along a given direction (where you expect the stress to be the
  largest) and subsequently let it fully relax. Another example would
  be to relax the binding distance between a molecule and a surface,
  before relaxing the entire system by forcing the molecule and
  adsorption site to relax together.
  %
  To use directional constraints one may provide an additional 3
  \emph{reals} after the \fdf*{atom}/\fdf*{rigid}.
  For instance in the previous example (benzene) one may first relax
  all Hydrogen atoms along the $y$ and $z$ Cartesian vector by
  constraining the $x$ Cartesian vector
  \begin{fdfexample}
    %block Geometry.Constraints
      Z 6 # constrain Carbon
      Z 1 1. 0. 0. # constrain Hydrogen along x Cartesian vector
    %endblock
  \end{fdfexample}
  Note that you \emph{must} append a ``.'' to denote it a real. The
  vector specified need not be normalized. Also, if you want it to
  be constrained along the $x$-$y$ vector you may do
  \begin{fdfexample}
    %block Geometry.Constraints
      Z 6
      Z 1 1. 1. 0.
    %endblock
  \end{fdfexample}

  Therefore the directional constraint will remove the force
  components that projects onto the direction specified.


\end{fdfentry}


\subsection{Phonon calculations}

If \fdf{MD.TypeOfRun} is \fdf*{FC}, \siesta\ sets up a special outer
geometry loop that displaces individual atoms along the coordinate
directions to build the force-constant matrix.
\index{output!molecular dynamics!Force Constants Matrix}

The output (see below) can be analyzed to extract phonon frequencies
and vectors with the VIBRA\index{VIBRA} package in the \program{Util/Vibra}
directory. For computing the Born effective charges together with the
force constants, see \fdf{BornCharge}.

\begin{fdfentry}{FC!Displacement}[length]<$0.04\,\mathrm{Bohr}$>
  \fdfdeprecates{MD!FCDispl}

  Displacement to use for the computation of the force constant
  matrix\index{Force Constants Matrix} for phonon calculations.

\end{fdfentry}

\begin{fdfentry}{FC!First}[integer]<$1$>
  \fdfdeprecates{MD!FCFirst}

  Index of first atom to displace for the computation of the force
  constant matrix\index{Force Constants Matrix} for phonon
  calculations.

\end{fdfentry}

\begin{fdfentry}{FC!Last}[integer]<\fdfvalue{FC!First}>
  \fdfdeprecates{MD!FCLast}

  Index of last atom to displace for the computation of the force
  constant matrix\index{Force Constants Matrix} for phonon
  calculations.

\end{fdfentry}

The force-constants matrix is written in file \sysfile{FC}.  The
format is the following: for the displacement of each atom in each
direction, the forces on each of the other atoms is writen (divided by
the value of the displacement), in units of eV/\AA$^2$. Each line has
the forces in the $x$, $y$ and $z$ direction for one of the atoms.

If constraints are used, the file \sysfile{FCC} is also written.

\begin{fdflogicalF}{FC!Save.dHS}

    For \fdf{MD.TypeOfRun:FC}, if \fdftrue, SIESTA produces a single netCDF
    file \sysfile*{dHSdR.nc} with the derivatives of the Hamiltonian and
    overlap matrix for each displaced atom along each Cartesian direction. 
    \textit{The derivatives are only calculated in the unit cell, not the 
    auxiliary supercell}.

\end{fdflogicalF}

\begin{fdfentry}{FC!dHdR.Tolerance}[force]<$-1\,\mathrm{Ry/Bohr}$>

  Threshold controlling which elements of the Hamiltonian derivative should be
  stored in \sysfile*{dHSdR.nc}. All matrix elements smaller than the threshold
  are discarded. If threshold is negative no elements are discarded.

\end{fdfentry}


\begin{fdfentry}{FC!dSdR.Tolerance}[inverse length]<$-1\,\mathrm{1/Bohr}$>

    Threshold controlling which elements of the Hamiltonian derivative should be
    stored in \sysfile*{dHSdR.nc}. All matrix elements smaller than the threshold
    are discarded. If threshold is negative no elements are discarded.
  
\end{fdfentry}



\section{DFT+U}


\label{sec:lda+u}

\note This implementation works for both LDA and GGA, hence named
DFT+U in the main text.

\note Current implementation is based on the simplified rotationally
invariant DFT+U formulation of Dudarev and collaborators [see, Dudarev
\textit{et al.}, Phys. Rev. B \textbf{57}, 1505 (1998)].  Although the
input allows to define independent values of the $U$ and $J$
parameters for each atomic shell, in the actual calculation the two
parameters are combined to produce an effective Coulomb repulsion
$U_{\mathrm{eff}}=U-J$. $U_{\mathrm{eff}}$ is the parameter actually
used in the calculations for the time being.


For large or intermediate values of $U_{\mathrm{eff}}$ the convergence
is sometimes difficult. A step-by-step increase of the
value of $U_{\mathrm{eff}}$ can be advisable in such cases.

If DFT+U is used in combination with non-collinear or spin-orbit coupling,
the Liechtenstein approach is implemented, where the $U$ and the exchange $J$ parameters are treated separately [see, A. I. Liechtenstein
\textit{et al.}, Phys. Rev. B \textbf{52}, R5467 (1995)].
The generalization for the spin-orbit or non-collinear cases follows the recipe given by
E. Bousquet and N. Spaldin, Phys. Rev. B \textbf{82}, 220402(R) (2010).
Currently, only the $d$-shell can be considered as the correlated shell where the $U$ and $J$ are applied.
The computation of the occupancies on the orbitals of the correlated shells is done following the same recipe as for the Dudarev approach.
That means that the following entries related with the generation of the DFT+U projectors are still relevant.
However, the input options \fdf{DFTU!FirstIteration}, \fdf{DFTU!ThresholdTol}, \fdf{DFTU!PopTol}, and \fdf{DFTU!PotentialShift}  are irrelevant when DFT+U is used in combination with spin-orbit or non-collinear magnetism.

\begin{fdfentry}{DFTU!ProjectorGenerationMethod}[integer]<2>
  \fdfindex{LDAU!ProjectorGenerationMethod}%

  Generation method of the DFT+U projectors. The DFT+U projectors are
  the localized functions used to calculate the local populations used
  in a Hubbard-like term that modifies the LDA Hamiltonian and
  energy. It is important to recall that DFT+U projectors should be
  quite localized functions. Otherwise the calculated populations
  loose their atomic character and physical meaning. Even more
  importantly, the interaction range can increase so much that
  jeopardizes the efficiency of the calculation.

  Two methods are currently implemented:
  \begin{fdfoptions}
    \option[1]%
    Projectors are slightly-excited numerical atomic orbitals similar
    to those used as an automatic basis set by \siesta.  The radii of
    these orbitals are controlled using the parameter
    \fdf{DFTU!EnergyShift} and/or the data included in the block
    \fdf{DFTU!Proj} (quite similar to the data block \fdf{PAO!Basis}
    used to specify the basis set, see below).

    \option[2]%
    Projectors are exact solutions of the pseudoatomic problem (and,
    in principle, are not strictly localized) which are cut using a
    Fermi function $1/\{1+\exp[(r-r_c)\omega]\}$.  The values of $r_c$
    and $\omega$ are controlled using the parameter \fdf{DFTU!CutoffNorm}
    and/or the data included in the block \fdf{DFTU!Proj}.

  \end{fdfoptions}

\end{fdfentry}

\begin{fdfentry}{DFTU!EnergyShift}[energy]<$0.05\,\mathrm{Ry}$>
  \fdfindex{LDAU!EnergyShift}%

  Energy increase used to define the localization radius of the DFT+U
  projectors (similar to the parameter \fdf{PAO!EnergyShift}).

  \note only used when \fdf{DFTU!ProjectorGenerationMethod} is \fdf*{1}.

\end{fdfentry}

\begin{fdfentry}{DFTU!CutoffNorm}[real]<$0.9$>
  \fdfindex{LDAU!CutoffNorm}%

  Parameter used to define the value of $r_c$ used in the Fermi
  distribution to cut the DFT+U projectors generated according to
  generation method 2 (see above). \fdf{DFTU!CutoffNorm} is the norm of the
  original pseudoatomic orbital contained inside a sphere of radius
  equal to $r_c$.

  \note only used when \fdf{DFTU!ProjectorGenerationMethod} is \fdf*{2}.

\end{fdfentry}


\begin{fdfentry}{DFTU!Proj}[block]
  \fdfindex{LDAU!Proj}%

  Data block used to specify the DFT+U projectors.

  \begin{itemize}
    \item%
    If \fdf{DFTU!ProjectorGenerationMethod} is \fdf*{1}, the
    syntax is as follows:
    \begin{fdfexample}
%block DFTU.Proj      # Define DFT+U projectors
 Fe    2              # Label, l_shells
  n=3 2  E 50.0 2.5   # n (opt if not using semicore levels),l,Softconf(opt)
      5.00  0.35      # U(eV), J(eV) for this shell
      2.30            # rc (Bohr)
      0.95            # scaleFactor (opt)
      0               #    l
      1.00  0.05      # U(eV), J(eV) for this shell
      0.00            # rc(Bohr) (if 0, automatic r_c from DFTU.EnergyShift)
%endblock DFTU.Proj
   \end{fdfexample}

    \item%
    If \fdf{DFTU!ProjectorGenerationMethod} is \fdf*{2}, the
    syntax is as follows:
    \begin{fdfexample}
%block DFTU.Proj      # Define DFTU projectors
 Fe    2              # Label, l_shells
  n=3 2  E 50.0 2.5   # n (opt if not using semicore levels),l,Softconf(opt)
      5.00  0.35      # U(eV), J(eV) for this shell
      2.30  0.15      # rc (Bohr), \omega(Bohr) (Fermi cutoff function)
      0.95            # scaleFactor (opt)
      0               #    l
      1.00  0.05      # U(eV), J(eV) for this shell
      0.00  0.00      # rc(Bohr), \omega(Bohr) (if 0 r_c from DFTU.CutoffNorm
%endblock DFTU.Proj   #                         and \omega from default value)
    \end{fdfexample}
  \end{itemize}

  Certain of the quantites have default values:

  \begin{tabular}{cc}
    $U$ & \fdf*{0.0 eV} \\
    $J$ & \fdf*{0.0 eV} \\
    $\omega$ & \fdf*{0.05 Bohr} \\
    Scale factor & \fdf*{1.0}
  \end{tabular}

  $r_c$ depends on \fdf{DFTU!EnergyShift} or \fdf{DFTU!CutoffNorm}
  depending on the generation method.

\end{fdfentry}

\begin{fdflogicalF}{DFTU!FirstIteration}
  \fdfindex{LDAU!FirstIteration}%

  If \fdftrue, local populations are calculated and Hubbard-like term
  is switch on in the first iteration.  Useful if restarting a
  calculation reading a converged or an almost converged density
  matrix from file.

\end{fdflogicalF}

\begin{fdfentry}{DFTU!ThresholdTol}[real]<$0.01$>
  \fdfindex{LDAU!ThresholdTol}%

  Local populations only calculated and/or updated if the change in the
  density matrix elements (dDmax) is lower than \fdf{DFTU!ThresholdTol}.

\end{fdfentry}

\begin{fdfentry}{DFTU!PopTol}[real]<$0.001$>
  \fdfindex{LDAU!PopTol}%

  Convergence criterium for the DFT+U local populations. In the
  current implementation the Hubbard-like term of the Hamiltonian is
  only updated (except for the last iteration) if the variations of
  the local populations are larger than this value.

\end{fdfentry}

\begin{fdflogicalF}{DFTU!PotentialShift}
  \fdfindex{LDAU!PotentialShift}%

  If set to \fdftrue, the value given to the $U$ parameter in the
  input file is interpreted as a local potential shift. Recording the
  change of the local populations as a function of this potential
  shift, we can calculate the appropriate value of $U$ for the system
  under study following the methology proposed by Cococcioni and
  Gironcoli in Phys. Rev. B \textbf{71}, 035105 (2005).

\end{fdflogicalF}


\section{RT-TDDFT}\index{RT-TDDFT}\index{TDDFT}\label{sec:tddft}


Now it is possible to perform Real-Time Time-Dependent Density
Functional Theory (RT-TDDFT)-based calculations using the
\siesta\ method. This section includes a brief introduction to the
TDDFT method and implementation, shows how to run the TDDFT-based
calculations, and provides a reference guide to the additional input
options.

\subsection{Brief description}
The basic features of the TDDFT have been implemented
within the \siesta\ code. Most of the
details can be found in the paper Phys. Rev. B
\textbf{66} 235416 (2002), by A. Tsolakidis,
D. S\'anchez-Portal and, Richard M. Martin.
However, the practical implementation of the present version
is very different
from the initial version. The present implementation of the TDDFT has been
programmed with the primary aim of calculating the optical
response of clusters and solids, however, it has been successfully used to
calculate the electronic stopping power of solids as well.

For the calculation of the optical response of the electronic
systems a perturbation
to the system is applied at time step 0, and the system is
allowed to reach a self-consistent solution. Then, the perturbation
is switched off for all subsequent time steps, and the electrons
are allowed to evolve according to time-dependent Kohn-Sham
equations. For the case of a cluster the perturbation
is a finite (small) electric field. For the case of bulk
material (not yet fully implemented) the initial perturbation
is different but the main strategy is similar.

The present version of the RT-TDDFT implementation is also capable of
performing a simultaneous  dynamics of electrons and
ions but this is limited to the cases in which forces on the ions are within
ignorable limit.

The general working scheme is as following. First, the system is
allowed to reach a self-consistent solution for some initial
conditions (for example an initial ionic configuration or an applied
external field). The occupied Kohn-Sham orbitals (KSOs) are then selected
and stored in memory.  The occupied KSOs are then made to evolve in
time, and the Hamiltonian is recalculated for each time step.

\subsection{Partial Occupations}

This is a note of caution. This implementation of RT-TDDFT can not propagate partially occupied orbitals. While partial occupation of states is a common occurrence, they must be avoided. The issue of partially occupied states becomes, particularly,  tricky when dealing with metals and k-point sampling at the same time. The code tries to detect partial occupations and stops during the first run but it is not guarantied. Consequently, it can lead to additional or missing charge. Ultimately it is users' responsibility to make sure that the system has no partial occupations and missing or added charge. There are different ways to avoid partial occupations depending on the system and simulation parameters; for example changing spin-polarization and/or adding some k-point shift to k-points.

\subsection{Input options for RT-TDDFT}

A TDDFT calculation requires two runs of \siesta. In the first run
with appropriate flags it calculates the self-consistent initial
state, i.e., only occupied initial KSOs stored in \sysfile{TDWF}
file. The second run uses this file and the structure
file \sysfile{TDXV} as input and evolves the occupied
KSOs.

\begin{fdflogicalF}{TDED!WF.Initialize}

If set to \fdftrue\ in a standard self-consistent \siesta\ calculation, it
makes the program save the KSOs after reaching
self-consistency. This constitutes the first run.

\end{fdflogicalF}

\begin{fdfentry}{TDED!Nsteps}[integer]<1>

Number of electronic time steps between each atomic movement. It can not be less
than $1$.

\end{fdfentry}
\begin{fdfentry}{TDED!TimeStep}[time]<$0.001\,\mathrm{fs}$>
Length of time for each electronic step. The default value is only suggestive. Users must
determine an appropriate value for the electronic time step.

\end{fdfentry}

\begin{fdflogicalF}{TDED!Extrapolate}
  An extrapolated Hamiltonian is applied to evolve KSOs for \fdf{TDED!Extrapolate.Substeps} number of substeps within a sinlge electronic step without re-evaluating the Hamiltonian.
\end{fdflogicalF}

\begin{fdfentry}{TDED!Extrapolate.Substeps}[integer]<3>
  Number of electronic substeps when an extrapolated Hamiltonian is applied
  to propogate the KSOs. Effective only when \fdf{TDED!Extrapolate} set to be true.
\end{fdfentry}


\begin{fdflogicalT}{TDED!Inverse.Linear}

  If \fdftrue\ the inverse of matrix
  \begin{equation}
    \mathbf{S}+\mathrm{i}\mathbf{H}(t)\frac{\mathrm dt}{2}
  \end{equation}
  is calculated by solving a system of linear equations which
  implicitly multiplies the inverted matrix to the right hand side
  matrix. The alternative is explicit inversion and multiplication.
  The two options may differ in performance.

\end{fdflogicalT}

\begin{fdflogicalF}{TDED!WF.Save}

  Option to save wavefunctions at the end of a simulation for a
  possible restart or analysis. Wavefunctions are saved in file
  \sysfile{TDWF}. A TDED restart requires \sysfile{TDWF},
  \sysfile{TDXV}, and \sysfile{VERLET\_RESTART} from the previous
  run.  The first step of the restart is same as the last of the
  previous run.

\end{fdflogicalF}

 \begin{fdflogicalT}{TDED!Write.Etot}

If \fdftrue\ the total energy for every time step is stored in the file
\sysfile{TDETOT}.

\end{fdflogicalT}

 \begin{fdflogicalF}{TDED!Write.Dipole}

If \fdftrue\ a file \sysfile{TDDIPOL} is created that can be
further processed to calculate polarizability.

\end{fdflogicalF}

 \begin{fdflogicalF}{TDED!Write.Eig}

If \fdftrue\ the quantities $\langle \phi(t)|H(t)|\phi(t)\rangle$ in every
time step are calculated and stored in the file
\sysfile{TDEIG}. This is not trivial, hence can increase computational time.

\end{fdflogicalF}

\begin{fdflogicalF}{TDED!Saverho}

  If \fdftrue\ the instantaneous time-dependent density is saved to
  \file{<istep>.TDRho} after every \fdf{TDED!Nsaverho} number of
  steps.

\end{fdflogicalF}

\begin{fdfentry}{TDED!Nsaverho}[integer]<100>

Fixes the number of steps of ion-electron dynamics after which the
instantaneous time-dependent density is saved. May require a lot of
disk space.

\end{fdfentry}



\section{External control of \texorpdfstring{\siesta}{SIESTA}}


\label{sec:lua}

Since \siesta\ 4.1 an additional method of controlling the convergence
and MD of \siesta\ is enabled through external scripting
capability. The external control comes in two variants:
\begin{itemize}
  \item Implicit control of MD through updating/changing parameters
  and optimizing forces. For instance one may use a \fdf*{Verlet} MD
  method but additionally update the forces through some external
  force-field to amend limitations by the \fdf*{Verlet} method for
  your particular case. In the implicit control the molecular dynamics
  is controlled by \siesta.

  \item Explicit control of MD. In this mode the molecular dynamics
  \emph{must} be controlled in the external Lua script and the
  convergence of the geometry should also be controlled via this
  script.

\end{itemize}

The implicit control is in use if \fdf{MD.TypeOfRun} is something
other than \fdf*{lua}, while if the option is \fdf*{lua} the explicit
control is in use.

For examples on the usage of the Lua scripting engine and the power
you may find the library \program{flos}\footnote{This library is
    implemented by Nick R. Papior to further enhance the
    inter-operability with \siesta\ and external contributions.}, see
\url{https://github.com/siesta-project/flos}. At the time of writing
the \program{flos} library already implements new geometry/cell
relaxation schemes and new force-constants algorithms. You are highly
encouraged to use the new relaxation schemes as they may provide
faster convergence of the relaxation.

\begin{fdfentry}{Lua!Script}[file]<\nonvalue{none}>

  Specify a Lua script file which may be used to control the internal
  variables in \siesta. Such a script file must contain at least one
  function named \program{siesta\_comm} with no arguments.

  An example file could be this (note this is Lua code):
  \begin{codeexample}
-- This function (siesta_comm) is REQUIRED
function siesta_comm()

   -- Define which variables we want to retrieve from SIESTA
   get_tbl = {"geom.xa", "E.total"}

   -- Signal to SIESTA which variables we want to explore
   siesta.receive(get_tbl)

   -- Now we have the required variables,
   -- convert to a simpler variable name (not nested tables)
   -- (note the returned quantities are in SIESTA units (Bohr, Ry)
   xa = siesta.geom.xa
   Etot = siesta.E.total

   -- If we know our energy is wrong by 0.001 Ry we may now
   -- change the total energy
   Etot = Etot - 0.001

   -- Return to SIESTA the total energy such that
   -- it internally has the "correct" energy.

   siesta.E.total = Etot
   ret_tbl = {"E.total"}

   siesta.send(ret_tbl)

end
\end{codeexample}

  Within this function there are certain \emph{states} which defines
  different execution points in \siesta:
  \begin{fdfoptions}

    \option[Initialization]%
    This is right after \siesta\ has read the options from the FDF
    file. Here you may query some of the FDF options (and even change
    them) for your particular problem.

    \note \program{siesta.state == siesta.INITIALIZE}.

    \option[Initialize-MD]%
    Right before the SCF step starts. This point is somewhat
    superfluous, but is necessary to communicate the actual meshcutoff
    used\footnote{Remember that the \fdf{Mesh!Cutoff} defined is the
        minimum cutoff used.}.

    \note \program{siesta.state == siesta.INIT\_MD}.

    \option[SCF]%
    Right after \siesta\ has calculated the output density matrix, and
    just after \siesta\ has performed mixing.

    \note \program{siesta.state == siesta.SCF\_LOOP}.

    \option[Forces]%
    This stage is right after \siesta\ has calculated the forces.

    \note \program{siesta.state == siesta.FORCES}.

    \option[Move]%
    This state will \emph{only} be reached if \fdf{MD.TypeOfRun} is
    \fdf*{lua}.

    If one does not return updated atomic coordinates \siesta\ will
    reuse the same geometry as just analyzed.

    \note \program{siesta.state == siesta.MOVE}.

    \option[After-move]%
    Right after determining the atomic coordinates for the next step.
    Therefore, this is the first thing that is done with the new
    atomic coordinates.

    \note \program{siesta.state == siesta.AFTER\_MOVE}.


    \option[Analysis]%
    Just before \siesta\ completes and exits.

    \note \program{siesta.state == siesta.ANALYSIS}.

  \end{fdfoptions}

  Beginning with implementations of Lua scripts may be cumbersome. It
  is recommended to start by using \program{flos}, see
  \url{https://github.com/siesta-project/flos} which contains several
  examples on how to start implementing your own scripts.
  Currently \program{flos} implements a larger variety of relaxation
  schemes, for instance:
  \begin{codeexample}
    local flos = require "flos"
    LBFGS = flos.LBFGS()
    function siesta_comm()
       LBFGS:SIESTA(siesta)
    end
  \end{codeexample}
  which is the most minimal example of using the L-BFGS algorithm for
  geometry relaxation. Note that \program{flos} reads the parameters
  \fdf{MD.MaxDispl} and \fdf{MD.MaxForceTol} through \siesta\
  automatically.

  \note The number of available variables continues to grow and to
  find which quantities are accessible in Lua you may add this small
  code in your Lua script:
  \begin{codeexample}
    siesta.print_allowed()
  \end{codeexample}
  which prints out a list of all accessible variables (note they are
  not sorted).

  If there are any variables you require which are not in the list,
  please contact the developers.

  If you want to stop \siesta\ from Lua you can use the following:
  \begin{codeexample}
    siesta.Stop = true
    siesta.send({"Stop"})
  \end{codeexample}
  which will abort \siesta.

  Remark that since \emph{anything} may be changed via Lua one may
  easily make \siesta\ crash due to inconsistencies in the internal
  logic. This is because \siesta\ does not check what has changed, it
  accepts everything \emph{as is} and continues. Hence, one should be
  careful what is changed.

\end{fdfentry}

\begin{fdflogicalF}{Lua!Debug}

  Debug the Lua script mode by printing out (on stdout) information
  everytime \siesta\ communicates with Lua.

\end{fdflogicalF}

\begin{fdflogicalF}{Lua!Debug.MPI}

  Debug all nodes (if in a parallel run).

\end{fdflogicalF}

\begin{fdflogicalF}{Lua!Interactive}

  Start an interactive Lua session at all the states in the program
  and ask for user-input.
  %
  This is primarily intended for debugging purposes. The interactive
  session is executed just \emph{before} the \code{siesta\_comm}
  function call (if the script is used).

  For serial runs \code{siesta.send} may be used. For parallel runs do
  \emph{not} use \code{siesta.send} as the code is only
  executed on the first MPI node.

  There are various commands that are caught if they are the only
  content on a line:
  \begin{fdfoptions}
    \option[/debug]%
    Turn on/off debugging information.

    \option[/show]%
    Show the currently collected lines of code.

    \option[/clear]%
    Clears the currently collected lines of code.

    \option[;]%
    Run the currently collected lines of code and continue collecting
    lines.

    \option[/run]%
    Same as \code{;}.

    \option[/cont]%
    Run the currently collected lines of code and continue \siesta.

    \option[/stop]%
    Run the currently collected lines of code and stop all future
    interactive Lua sessions.

  \end{fdfoptions}

  Currently this only works if \fdf{Lua!Script} is having a valid Lua
  file (note the file may be empty).

\end{fdflogicalF}



\subsection{Examples of Lua programs}

Please look in the \program{Tests/lua\_*} folders where examples of
basic Lua scripts are found. Below is a description of the \program{*}
examples.


\begin{description}
  \item[h2o] Changes the mixing weight continuously in
  the SCF loop. This will effectively speed up convergence time if one
  can attain the best mixing weight per SCF-step.

  \item[si111] Change the mixing method based on certain convergence
  criteria. I.e. after a certain convergence one can switch to a more
  aggressive mixing method.

\end{description}

A combination of the above two examples may greatly improve
convergence, however, creating a generic method to adaptively change
the mixing parameters may be very difficult to implement. If you do
create such a Lua script, please share it on the mailing list.


\subsection{External MD/relaxation methods}

Using the Lua interface allows a very easy interface for creating
external MD and/or relaxation methods.

A public library (\program{flos},
\url{https://github.com/siesta-project/flos}) already implements a
wider range of relaxation methods than intrinsically enabled in
\siesta. Secondly, by using external scripting mechanisms one can
customize the routines to a much greater extend while simultaneously
create custom constraints.

You are \emph{highly} encouraged to try out the \program{flos} library
(please note that \program{flook} is required, see installation
instructions above).




\section{TRANSIESTA}
\label{sec:transiesta}

\newcommand\Nelec{N_{\mathfrak{E}}}


\siesta\ includes the possibility of performing calculations of
electronic transport properties using the \tsiesta\ method. This
Section describes how to use these
capabilities, and a reference guide to the relevant \fdflib\
options. We describe here only the additional options available for
\tsiesta\ calculations, while the rest of the \siesta\ functionalities
and variables are described in the previous sections of this User's
Guide.

An accompanying Python toolbox is available which will assist with
\tsiesta\ calculations. Please use (and cite) \sisl\cite{sisl}.


\subsection{Source code structure}

In this implementation, the \tsiesta\ routines have been grouped in a
set of modules whose file names begin with \texttt{m\_ts} or
\texttt{ts}.


\subsection{Compilation}

Prior to \siesta\ 4.1 \tsiesta\ was a separate executable. Now
\tsiesta\ is fully incorporated into \siesta. \emph{Only} compile
\siesta\ and the full functionality is present.
Sec.~\ref{sec:compilation} for details on compiling \siesta.


\subsection{Brief description}
\label{sec:transiesta:description}

The \tsiesta\ method is a procedure to solve the electronic
structure of an open system formed by a finite structure sandwiched
between semi-infinite metallic leads. A finite bias can be applied
between leads, to drive a finite current. The method is described
in detail in \citet{Brandbyge2002,Papior2017}. In practical terms,
calculations using \tsiesta\ involve the solution of the
electronic density from the DFT Hamiltonian using Greens functions
techniques, instead of the usual diagonalization procedure. Therefore,
\tsiesta\ calculations involve a \siesta\ run, in which a
set of routines are invoked to solve the Greens functions and the
charge density for the open system. These routines are packed in a set
of modules, and we will refer to it as the '\tsiesta\ module'
in what follows.

\tsiesta\ was originally developed by Mads Brandbyge, Jos\'e-Luis
Mozos, Pablo Ordej\'on, Jeremy Taylor and Kurt
Stokbro\cite{Brandbyge2002}. It consisted, mainly, in setting up an
interface between \siesta\ and the (tight-binding) transport codes
developed by M. Brandbyge and K. Stokbro. Initially everything was
written in Fortran-77. As \siesta\ started to be translated to
Fortran-90, so were the \tsiesta\ parts of the code. This was
accomplished by Jos\'e-Luis Mozos, who also worked on the
parallelization of \tsiesta.
%
Subsequently Frederico D. Novaes extended \tsiesta\ to allow $k$-point
sampling for transverse directions. Additional extensions was
added by Nick R. Papior during 2012.

The current \tsiesta\ module has been completely rewritten by Nick
R. Papior and encompass highly advanced inversion algorithms as well
as allowing $N\geq1$ electrode setups among many new
features. Furthermore, the utility \tbtrans\ has also been fully
re-coded (by Nick R. Papior) to be a generic tight-binding code
capable of analyzing physics from the Greens function perspective in
$N\ge1$ setups\cite{Papior2017}.


\begin{itemize}
  \item%
  Transport calculations involve \emph{electrode} (EL) calculations,
  and subsequently the Scattering Region (SR) calculation. The
  \emph{electrode} calculations are usual \siesta\ calculations, but
  where files \sysfile{HSX}/\sysfile{TSHS} (the \sysfile{TSHS} is deprecated),
  and optionally \sysfile{TSDE}, are
  generated. These files contain the information necessary for
  calculation of the self-energies. If any electrodes have identical
  structures (see below) the same files can and should be used to
  describe those. In general, however, electrodes can be different and
  therefore two different \sysfile{HSX} files must be generated. The
  location of these electrode files must be specified in the \fdflib\
  input file of the SR calculation, see \fdf{TS!Elec.<>!HS}.

  \item %
  For the SR, \tsiesta\ starts with the usual \siesta\ procedure,
  converging a Density Matrix (DM) with the usual Kohn-Sham scheme for
  periodic systems. It uses this solution as an initial input for the
  Greens function self consistent cycle. Effectively you will start a
  \tsiesta\ calculation from a fully periodic calculation. This is why
  the $0\, V$ calculation should be the only calculation where you start
  from \siesta.

  \tsiesta\ stores the SCF DM in a file named \sysfile{TSDE}. In a rerun of
  the same system (meaning the same \fdf{SystemLabel}), if the
  code finds a \sysfile{TSDE} file in the directory, it will take this
  DM as the initial input and this is then considered a continuation
  run. In this case it does not perform an initial \siesta\ run. It
  must be clear that when starting a calculation from scratch, in the
  end one will find both files, \sysfile{DM} and \sysfile{TSDE}.
  %
  The first one stores the \siesta\ density matrix (periodic boundary
  conditions in all directions and no voltage), and the latter the
  \tsiesta\ solution.

  \item %
  When performing several bias calculations, it is heavily advised to
  run different bias' in different directories. To drastically improve
  convergence (and throughput) one should copy the \sysfile{TSDE} from
  the closest, previously, calculated bias to the current bias.

  \item %
  The \sysfile{TSDE} may be read equivalently as the
  \sysfile{DM}. Thus, it may be used by fx. \program{denchar} to
  analyze the non-equilibrium charge density. Alternatively one can
  use \sisl\cite{sisl} to interpolate the DM and EDM to speed up
  convergence.

  \item %
  As in the case of \siesta\ calculations, what \tsiesta\ does is to
  obtain a converged DM, but for open boundary conditions and possibly
  a finite bias applied between electrodes. The corresponding
  Hamiltonian matrix (once self consistency is achieved) of the SR is
  also stored in a \sysfile{HSX} file. Subsequently, transport
  properties are obtained in a post-processing procedure using the
  \tbtrans\ code (located in the \program{Util/TS/TBtrans}
  directory). We note that the \sysfile{HSX} files contain all the
  needed structural information (atomic positions, matrix elements,
  \ldots), and so the input (fdf) flags for the geometry and basis
  have no influence of the subsequent \tbtrans\ calculations.

  \item %
  When the non-equilibrium calculation uses different electrodes one
  should use so-called \emph{buffer} atoms behind the electrodes to act
  as additional screening regions when calculating the initial guess
  (using \siesta) for \tsiesta. Essentially they may be used to
  achieve a better ``bulk-like'' environment at the electrodes in the
  SR calculation.


  \item%
  An important parameter is the lower bound of the energy contours. It
  is a good practice, to start with a \siesta\ calculation for the SR
  and look at the eigenvalues of the system. The lower bound of the
  contours must be \emph{well} below the lowest eigenvalue.

  \item%
  Periodic boundary conditions are assumed in 2 cases.

  \begin{enumerate}
    \item For $\Nelec\neq 2$ all lattice vectors are periodic, users
    \emph{must} manually define \fdf{TS!kgrid!MonkhorstPack}

    \item For $\Nelec=2$ \tsiesta\ will auto-detect if both electrodes
    are semi-infinite along the same lattice vector. If so, only 1 $k$
    point will be used along that lattice vector.
  \end{enumerate}

  \item%
  The default algorithm for matrix inversion is the BTD method, before
  starting a \tsiesta\ calculation please run with the analyzation
  step \fdf{TS!Analyze} (note this is very fast and can be done on any
  desktop computer, regardless of system size).

  \item%
  Importantly(!) the $k$-point sampling need typically be much higher
  in a \tbtrans\ calculation to achieve a converged transmission
  function.

  \item%
  Energies from \tsiesta\ are \emph{not} to be trusted since the open
  boundaries complicates the energy calculation. Therefore care needs
  to be taken when comparing energies between different calculations
  and/or different bias'.

  \item%
  Always ensure that charges are preserved in the scattering region
  calculation. Doing the SCF an output like the following will be shown:
  \begin{output}[fontsize=\footnotesize]
ts-q:         D        E1        C1        E2        C2        dQ
ts-q:   436.147   392.146     3.871   392.146     3.871  7.996E-3
  \end{output}
  Always ensure the last column (\code{dQ}) is a very small fraction of
  the total number of electrons. Ideally this should be $0$.
  %
  For $0$ bias calculations this should be very small, typically less
  than $0.1\,\%$ of the total charge in the system. If this is not the
  case, it probably means that there is not enough screening towards the
  electrodes which can be solved by adding more electrode layers between
  the electrode and the scattering region. This layer thickness is
  \emph{very} important to obtain a correct open boundary calculation.

  \item%
  Do \emph{not} perform \tsiesta\ calculations using semi-conducting
  electrodes. The basic premise of \tsiesta\ calculations is that the
  electrodes \emph{behave like bulk} in the electrode regions of the
  SR. This means that the distance between the electrode and the
  perturbed must equal the screening length of the electrode.

  This is problematic for semi-conducting systems since they
  intrinsically have a very long screening length.

  In addition, the Fermi-level of semi-conductors are not well-defined
  since it may be placed anywhere in the band gap.

\end{itemize}


\subsection{Electrodes}

To calculate the electronic structure of a system under external bias,
\tsiesta\ attaches the system to semi-infinite electrodes which extend
to their respective semi-infinite directions. Examples of electrodes
would include surfaces, nanowires, nanotubes or fully infinite
regions. The electrode must be large enough (in the semi-infinite
direction) so that orbitals within the unit cell only interact with a
single nearest neighbor cell in the semi-infinite direction (the size
of the unit cell can thus be derived from the range of support for the
orbital basis functions). \tsiesta\ will stop if this is not
enforced. The electrodes are generated by a separate \tsiesta\ run on
a bulk system. This implies that the proper bulk properties are
obtained by a sufficiently high $k$-point sampling. If in doubt, use
100 $k$-points along the semi-infinite direction. The results are
saved in a file with extension \sysfile{HSX} which contains a
description of the electrode unit cell, the position of the atoms
within the unit cell, as well as the Hamiltonian and overlap matrices
that describe the electronic structure of the lead. One can generate a
variety of electrodes and the typical use of \tsiesta\ would involve
reusing the same electrode for several setups. At runtime, the
\tsiesta\ coordinates are checked against the electrode coordinates
and the program stops if there is a mismatch to a certain precision
($10^{-4}\,\mathrm{Bohr}$). Note that the atomic coordinates are
compared relatively. Hence the \emph{input} atomic coordinates of the
electrode and the device need not be the same (see e.g. the tests in
the \shell{Tests}\index{Tests} directory.

To run an electrode calculation one should do:
\begin{shellexample}
  siesta --electrode RUN.fdf
\end{shellexample}
or define these options in the electrode fdf files:
\fdf{Save!HS} and \fdf{TS!DE.Save} to \fdf*{true} (the above
\code{--electrode} is a shorthand to forcefully define the two options).


\subsubsection{Matching coordinates}

Here are some rules required to successfully construct the appropriate
coordinates of the scattering region. Contrary to versions prior to
4.1, the order of atoms is largely irrelevant. One may define all
electrodes, then subsequently the device, or vice versa. Similarly,
buffer atoms are not restricted to be the first/last atoms.

However, atoms in any given electrode \emph{must} be consecutive in
the device file. I.e. if an electrode input option is given by:
\begin{fdfexample}
  %block TS.Elec.<>
    HS ../elec-<>/siesta.HSX
    bloch 1 3 1
    used-atoms 4
    electrode-position 10
    ...
  %endblock
\end{fdfexample}
then the atoms from $10$ to $10+4*3-1$ must coincide with the atoms of
the calculation performed in the \program{../elec-<>/}
subdirectory. The above options will be discussed in the following
section.

When using the Bloch expansion (highly recommended if your system
allows it) it is advised to follow the \emph{tiling} method. However
both of the below sequences are allowed.

\paragraph{Tile} \fdfindex{TS!Elec.<>!Bloch}%
Here the atoms are copied and displaced by the full
electrode. Generally this expansion should be preferred over the
\emph{repeat} expansion due to much faster execution.
\begin{fdfexample}
  iaD = 10 ! as per the above input option
  do iC = 0 , nC - 1
  do iB = 0 , nB - 1
  do iA = 0 , nA - 1
    do iaE = 1 , na_u
      xyz_device(:, iaD) = xyz_elec(:, iaE) + &
          cell_elec(:, 1) * iA + &
          cell_elec(:, 2) * iB + &
          cell_elec(:, 3) * iC
      iaD = iaD + 1
    end do
  end do
  end do
  end do
\end{fdfexample}

By using \sisl\cite{sisl} one can achieve the tiling scheme
by using the following command-line utility on an input
\program{ELEC.fdf} structure with the minimal electrode:
\begin{codeexample}
  sgeom -tx 1 -ty 3 -tz 1 ELEC.fdf DEVICE_ELEC.fdf
\end{codeexample}

\paragraph{Repeat} \fdfindex{TS!Elec.<>!Bloch}%
Here the atoms are copied individually. Generally
this expansion should \emph{not} be used since it is much slower than
tiling.
\begin{fdfexample}
  iaD = 10 ! as per the above input option
  do iaE = 1 , na_u
    do iC = 0 , nC - 1
    do iB = 0 , nB - 1
    do iA = 0 , nA - 1
      xyz_device(:, iaD) = xyz_elec(:, iaE) + &
          cell_elec(:, 1) * iA + &
          cell_elec(:, 2) * iB + &
          cell_elec(:, 3) * iC
      iaD = iaD + 1
    end do
    end do
    end do
  end do
\end{fdfexample}

By using \sisl\cite{sisl} one can achieve the repeating scheme by
using the following command-line utility on an input
\program{ELEC.fdf} structure with the minimal electrode:
\begin{codeexample}
  sgeom -rz 1 -ry 3 -rx 1 ELEC.fdf DEVICE_ELEC.fdf
\end{codeexample}



\subsubsection{Principal layer interactions} %
\index{transiesta!electrode!principal layer}%

It is \emph{extremely} important that the electrodes only interact
with one neighboring supercell due to the self-energy
calculation\cite{Sancho1985}. \tsiesta\ will print out a block as this
(\shell{<>} is the electrode name):
\begin{verbatim}
 <> principal cell is perfect!
\end{verbatim}
if the electrode is correctly setup and it only interacts with its
neighboring supercell.
%
In case the electrode is erroneously setup, something similar to the
following will be shown in the output file.
\begin{verbatim}
 <> principal cell is extending out with 96 elements:
    Atom 1 connects with atom 3
    Orbital 8 connects with orbital 26
    Hamiltonian value: |H(8,6587)|@R=-2 =  0.651E-13 eV
    Overlap          :  S(8,6587)|@R=-2 =   0.00
\end{verbatim}
It is imperative that you have a \emph{perfect} electrode as otherwise
nonphysical results will occur. This means that you need to add more
layers in your electrode calculation (and hence also in your
scattering region). An example is an ABC stacking electrode. If the
above error is shown one \emph{has} to create an electrode with ABCABC
stacking in order to retain periodicity.

By default \tsiesta\ will die if there are connections beyond the
principal cell. One may control whether this is allowed or not by
using \fdf{TS!Elecs!Neglect.Principal}.



\subsection{Convergence of electrodes and scattering regions}

For successful \tsiesta\ calculations it is imperative that the
electrodes and scattering regions are well-converged.
%
The basic principle is equivalent to the \siesta\ convergence, see
Sec.~\ref{sec:scf}.

The steps should be something along the line of (only done at
$0\, V$).
\begin{enumerate}

  \item%
  Converge electrodes and find optimal \fdf{Mesh!Cutoff},
  \fdf{kgrid!MonkhorstPack} etc.

  Electrode $k$ points should be very high along the semi-infinite
  direction. The default is $100$, but at least $>50$ should easily be
  reachable.


  \item%
  Use the parameters from the electrodes and also converge the
  same parameters for the scattering region SCF.

  This is an iterative process since the scattering region forces the
  electrodes to use equivalent $k$ points (see
  \fdf{TS!Elec.<>!check-kgrid}).

  Note that $k$ points should be limited in the \tsiesta\ run, see
  \fdf{TS!kgrid!MonkhorstPack}.

  One should always use the same parameters in both the electrode and
  scattering region calculations, except the number of $k$ points for
  the electrode calculations along their respective semi-infinite
  directions.


  \item%
  Once \tsiesta\ is completed one should also converge the
  number of $k$ points for \tbtrans. Note that $k$ point sampling in
  \tbtrans\ should generally be much denser but \emph{always} fulfill
  $N_k^{\tsiesta}\geq N_k^\tbtrans$

\end{enumerate}

The converged parameters obtained at $0\,\mathrm V$ should be used for
all subsequent bias calculations. Remember to copy the \sysfile{TSDE}
from the closest, previously, calculated bias for restart and much
faster convergence.


\tsiesta\ is also more difficult to converge during the SCF
steps. This may be due to several interrelated problems:
%
\begin{itemize}

  \item%
  A too short screening distance between the scattering atoms
  and the electrode layers.


  \item%
  In case buffer atoms (\fdf{TS!Atoms.Buffer}) are used with
  vacuum on the backside it may be that there are too few buffer atoms
  to accurately screen off the vacuum region for a sufficiently good
  initial guess. This effect is only true for $0\,\mathrm V$
  calculations.


  \item%
  The mixing parameters may need to be smaller than for \siesta,
  see Sec.~\ref{sec:scf:mix} and it is never guaranteed that it will
  converge. It is \emph{always} a trial and error method, there are
  \emph{no} omnipotent mixing parameters.


  \item%
  Very high bias' may be extremely difficult to
  converge. Generally one can force bias convergence by doing smaller
  steps of bias. E.g. if problems arise at $0.5\,\mathrm V$ with an
  initial DM from a $0.25\,\mathrm V$ calculation, one could try and
  $0.3\,\mathrm V$ first.


  \item%
  If a particular bias point is hard to converge, even by doing
  the previous step, it may be related to an eigenstate close to the
  chemical potentials of either electrode (e.g. a molecular eigenstate
  in the junction). In such cases one could try an even higher bias
  and see if this converges more smoothly.

\end{itemize}



\subsection{NEGF equations}
\label{sec:negf-equations}

The options available for \tsiesta\ will impact how the calculation
is performed. It is vital that the users carefully read this section
and the options that refer to these.

The NEGF equation are primarily concerning the Green function:
\begin{equation}
  \label{eq:negf:green}
  \G(E) = \big[ (E+i\eta)\SO - \Ham - \sum_\elec \SE_\elec(E).
\end{equation}
The electrode self-energy is calculated from the bulk electrode
calculation
\begin{equation}
  \SE_\elec(E) \leftarrow \big\{\Ham_\elec, \SO_\elec\big\}.
\end{equation}

\tsiesta\ has options to discern which Hamiltonian elements can be
used in which parts of the calculation.
Default is that the electrode matrices ($\Ham_\elec, \SO_\elec$) are
used whenever the electrode enters a matrix. Lets show a partitioning
of the Green function for a particular electrode ($z=E+i\eta$)
\begin{equation}
  \label{eq:negf:green-elec}
  \G(z) =
  \begin{bmatrix}
    \mathbf M_{\elec,\elec} & \mathbf M_{\elec, D} & \dots
    \\
    \mathbf M_{D,\elec} & \mathbf M_{D, D} &
    \\
    \vdots && \ddots
    \\
  \end{bmatrix}^{-1}=
  \begin{bmatrix}
    (z+\mu_elec)\SO_\elec - \Ham_\elec - \SE_\elec(E) & z\SO_{\elec,D} - \Ham_{\elec,D} & \dots
    \\
    z\SO_{D,\elec} - \Ham_{D,\elec} & z\SO_D - \Ham_D &
    \\
    \vdots & & \ddots
    \\
  \end{bmatrix}^{-1}.
\end{equation}
The following options alter the above equation slightly:
\begin{itemize}
  \item \fdf{TS!Elec.<>!Bulk}
  \item \fdf{TS!Elec.<>!Eta}
  \item \fdf{TS!Elec.<>!chemical-potential}
  \item \fdf{TS!Elec.<>!V-fraction} (experts only!)
  \item \fdf{TS!Elec.<>!delta-Ef} (experts only!)
\end{itemize}


\subsection{\texorpdfstring{\tsiesta\ }{TranSIESTA} Options}

The fdf options shown here are only to be used at the input file for
the scattering region. When using \tsiesta\ for electrode
calculations, only the usual \siesta\ options are relevant.
%
Note that since \tsiesta\ is a generic $\Nelec$ electrode NEGF code the
input options are heavily changed compared to versions prior to 4.1.

\subsubsection{Quick and dirty}

Since 4.1, \tsiesta\ has been fully re-implemented. And so have
\emph{every} input fdf-flag. To accommodate an easy transition between
previous input files and the new version format a small utility called
\program{ts2ts}. It may be compiled in \program{Util/TS/ts2ts}. It is
recommended that you use this tool if you are familiar with previous
\tsiesta\ versions.

%
One may input options as in the old \tsiesta\ version and then run
\begin{fdfexample}
  ts2ts OLD.fdf > NEW.fdf
\end{fdfexample}
which translates all keys to the new, equivalent, input format. If you
are familiar with the old-style flags this is highly recommendable
while becoming comfortable with the new input format. Please note that
some defaults have changed to more conservative values in the newer
release.

If one does not know the old flags and wish to get a basic example of
an input file, a script \program{Util/TS/tselecs.sh} exists that can
create the basic input for $\Nelec$ electrodes. One may call it like:
\begin{shellexample}
  tselecs.sh -2 > TWO_ELECTRODE.fdf
  tselecs.sh -3 > THREE_ELECTRODE.fdf
  tselecs.sh -4 > FOUR_ELECTRODE.fdf
  ...
\end{shellexample}
where the first call creates an input fdf for 2 electrode setups, the
second for a 3 electrode setup, and so on. See the help (\program{-h})
for the program for additional options.

Before endeavoring on large scale calculations you are advised to run
an analyzation of the system at hand, you may run your system as
\begin{shellexample}
  siesta -fdf TS.Analyze RUN.fdf > analyze.out
\end{shellexample}
which will analyze the sparsity pattern and print out several
different pivoting schemes. Please see \fdf{TS!Analyze} for more
information.


\subsubsection{General options}

One have to set \fdf{SolutionMethod} to \fdf*{transiesta} to enable
\tsiesta.

\begin{fdfentry}{TS!SolutionMethod}[string]<btd|mumps|full>

  Control the algorithm used for calculating the Green
  function. Generally the BTD method is the fastest and this option
  need not be changed.

  \begin{fdfoptions}
    \option[BTD]%
    \fdfindex*{TS!SolutionMethod:BTD}%
    Use the block-tri-diagonal algorithm for matrix inversion.

    This is generally the recommended method.

    \option[MUMPS]%
    \fdfindex*{TS!SolutionMethod:MUMPS}%
    Use sparse matrix inversion algorithm (MUMPS). This requires
    \tsiesta\ to be compiled with MUMPS.
    \index{MUMPS}%
    \index{External library!MUMPS}%

    \option[full]%
    \fdfindex*{TS!SolutionMethod:full}%
    Use full matrix inversion algorithm (LAPACK). Generally only
    usable for debugging purposes.

  \end{fdfoptions}

\end{fdfentry}

\begin{fdfentry}{TS!Voltage}[energy]<$0\,\mathrm{eV}$>

  Define the reference applied bias. For $\Nelec=2$ electrode calculations
  this refers to the actual potential drop between the electrodes,
  while for $\Nelec\neq2$ this is a reference bias. In the latter case it
  \emph{must} be equivalent to the maximum difference between the
  chemical potential of any two electrodes.

  \note Specifying \shell{-V}\fdfindex{Command line options:-V} on the
  command-line overwrites the value in the fdf file.

\end{fdfentry}

\begin{fdfentry}{TS!kgrid!MonkhorstPack}[block]<\fdfvalue{kgrid!MonkhorstPack}>

  $k$ points used for the \tsiesta\ calculation.

  For $\Nelec\neq2$ this should always be defined. Always take care to
  use only 1 $k$ point along non-periodic lattice vectors. An
  electrode semi-infinite region is considered non-periodic since it
  is integrated out through the self-energies.

  This defaults to \fdf{kgrid!MonkhorstPack}.

\end{fdfentry}

\begin{fdfentry}{TS!Atoms.Buffer}[block/list]
  \fdfindex{TS.BufferAtomsLeft|see TS!Atoms.Buffer}%
  \fdfindex{TS.BufferAtomsRight|see TS!Atoms.Buffer}%

  Specify atoms that will be removed in the \tsiesta\ SCF. They are
  not considered in the calculation and may be used to improve the
  initial guess for the Hamiltonian.


  An intended use for buffer atoms is to ensure a bulk behavior in the
  electrode regions when electrodes are different. As an example: a 2
  electrode calculation with left consisting of Au atoms and the right
  consisting of Pt atoms. In such calculations one cannot create a
  periodic geometry along the transport direction. One needs to add
  vacuum between the Au and Pt atoms that comprise the
  electrodes. However, this creates an artificial edge of the
  electrostatic environment for the electrodes since in \siesta\ there
  is vacuum, whereas in \tsiesta\ the effective Hamiltonian sees a
  bulk environment. To ensure that \siesta\ also exhibits a bulk
  environment on the electrodes we add \emph{buffer} atoms towards the
  vacuum region to screen off the electrode region. These
  \emph{buffer} atoms is thus a technicality that has no influence on
  the \tsiesta\ calculation but they are necessary to ensure the
  electrode bulk properties.

  The above discussion is even more important when doing $\Nelec$-electrode
  calculations.

  \note all lines are additive for the buffer atoms and the input
  method is similar to that of \fdf{Geometry!Constraints} for the
  \fdf*{atom} line(s).

  \begin{fdfexample}
    %block TS.Atoms.Buffer
       atom [ 1 -- 5 ]
    %endblock
    # Or equivalently as a list
    TS.Atoms.Buffer [1 -- 5]
  \end{fdfexample}
  will remove atoms [1--5] from the calculation.

\end{fdfentry}

\begin{fdfentry}{TS!ElectronicTemperature}[energy]<\fdfvalue{ElectronicTemperature}>

  Define the temperature used for the Fermi distributions for the
  chemical potentials.
  %
  See \fdf{TS!ChemPot.<>!ElectronicTemperature}.

\end{fdfentry}

\begin{fdfentry}{TS!SCF!DM.Tolerance}[real]<\fdfvalue{SCF.DM!Tolerance}>%
  \fdfdepend{SCF.DM!Tolerance,SCF.DM!Converge}

  The density matrix tolerance for the \tsiesta\ SCF cycle.

\end{fdfentry}

\begin{fdfentry}{TS!SCF!H.Tolerance}[energy]<\fdfvalue{SCF.H!Tolerance}>%
  \fdfdepend{SCF.H!Tolerance,SCF.H!Converge}

  The Hamiltonian tolerance for the \tsiesta\ SCF cycle.

\end{fdfentry}

\begin{fdflogicalT}{TS!SCF!dQ.Converge}

  Whether \tsiesta\ should check whether the total charge is within a
  provided tolerance, see \fdf{TS!SCF!dQ.Tolerance}.

\end{fdflogicalT}

\begin{fdfentry}{TS!SCF!dQ.Tolerance}[real]<$\mathrm{Q(device)}\cdot 10^{-3}$>%
  \fdfdepend{TS!SCF!dQ.Converge}

  The charge tolerance during the SCF.

  The charge is not stable in \tsiesta\ calculations and this flag
  ensures that one does not, by accident, do post-processing of files
  where the charge distribution is completely wrong.

  A too high tolerance may heavily influence the electrostatics of the
  simulation.

  \note Please see \fdf{TS!dQ} for ways to reduce charge loss in
  equilibrium calculations.

\end{fdfentry}

\begin{fdfentry}{TS!SCF.Initialize}[string]<diagon|transiesta>%

  Control which initial guess should be used for \tsiesta. The general
  way is the \fdf*{diagon} solution method (which is preferred),
  however, one can start a \tsiesta\ run immediately. If you start
  directly with \tsiesta\ please refer to these flags:
  \fdf{TS!Elecs!DM.Init} and \fdf{TS!Fermi.Initial}.

  \note Setting this to \fdf*{transiesta} is highly experimental and
  convergence may be extremely poor.

\end{fdfentry}

\begin{fdfentry}{TS!Fermi.Initial}[energy]<$\sum^{N_E}_iE_F^i/N_E$>

  Manually set the initial Fermi level to a predefined value.

  \note this may also be used to change the Fermi level for
  calculations where you restart calculations. Using this feature is
  highly experimental.

\end{fdfentry}

\begin{fdfentry}{TS!Weight.Method}[string]<orb-orb|[[un]correlated+][sum|tr]-atom-[atom|orb]|mean>

  Control how the NEGF weighting scheme is conducted. Generally one
  should only use the \fdf*{orb-orb} while the others are present for
  more advanced usage. They refer to how the weighting coefficients of
  the different non-equilibrium contours are performed. In the
  following the weight are denoted in a two-electrode setup while they
  are generalized for multiple electrodes.

  \def\mypropto{\,\oppropto^{||}\,} %
  \def\mn{{\mu\nu}} %
  Define the normalised geometric mean as $\mypropto$ via
  \begin{equation}
    w\mypropto \langle\cdot^L\rangle\equiv
    \frac{\langle\cdot^L\rangle}{\langle\cdot^L\rangle+\langle\cdot^R\rangle}.
  \end{equation}

  When applying a bias, \tsiesta\ will printout the following during
  the SCF cycle:
\begin{output}[fontsize=\footnotesize]
ts-err-D: ij(  447,   447), M =  1.8275, ew = -.257E-2, em = 0.258E-2. avg_em = 0.542E-06
ts-err-E: ij(  447,   447), M = -6.7845, ew = 0.438E-3, em = -.439E-3. avg_em = -.981E-07
ts-w-q:               qP1       qP2
ts-w-q:           219.150   216.997
ts-q:         D        E1        C1        E2        C2        dQ
ts-q:   436.147   392.146     3.871   392.146     3.871  7.996E-3
  \end{output}
  %
  The extra output corresponds to fine details in the integration
  scheme.
  \begin{description}[labelindent=3em, leftmargin=4.5em]
    \itemsep 10pt
    \parsep 0pt

    \item[\texttt{ts-err-*}] are estimated error outputs from the
    different integrals, for the density matrix (\texttt{D}) and the
    energy density matrix (\texttt{E}), see Eq.~(12) in
    \cite{Papior2017}. All values (except \texttt{avg\_em}) are for
    the given orbital site

    \begin{description}
      \itemsep 4pt
      \parsep 0pt

      \item[\texttt{ij(A,B)}] refers to the matrix element between orbital
      \texttt{A} and \texttt{B}

      \item[\texttt{M}] is the weighted matrix element value,
      $\sum_{\elec}w_\elec\DM^\elec$

      \item[\texttt{ew}] is the maximum difference between
      $\sum_{\elec}w_\elec\DM^\elec-\DM^\elec$ for all $\elec$.

      \item[\texttt{em}] is the maximum difference between
      $\DM^{\elec'}-\DM^\elec$ for all combinations of $\elec$ and
      $\elec'$.

      \item[\texttt{avg\_em}] is the averaged difference of \texttt{em} for all
      orbital sites.

    \end{description}

    \item[\texttt{ts-w-q}] is the Mulliken charge from the different
    integrals: $\Tr[w_\elec\DM^\elec\SO]$

  \end{description}

  \begin{fdfoptions}

    \option[orb-orb]%
    \fdfindex*{TS!Weight.Method:orb-orb}%
    Weight each orbital-density matrix element individually.

    \option[tr-atom-atom]%
    \fdfindex*{TS!Weight.Method:tr-atom-atom}%
    Weight according to the trace of the atomic density matrix sub-blocks
    \begin{equation}
      w_{ij}^{\Tr} \mypropto
      \sqrt{
          % First the i'th atom
          \sum_{\in\{i\}}(\Delta\rho_{\mu\mu}^L)^2
          \; % ensure a little space between them
          % second the j'th atom
          \sum_{\in\{j\}}(\Delta\rho_{\mu\mu}^L)^2
      }
    \end{equation}

    \option[tr-atom-orb]%
    \fdfindex*{TS!Weight.Method:tr-atom-orb}%

    Weight according to the trace of the atomic density matrix
    sub-block times the weight of the orbital weight
    \begin{equation}
      w_{ij,\mn}^{\Tr} \mypropto
      \sqrt{
          w_{ij}^{\Tr}
          w_{ij,\mn}
      }
    \end{equation}

    \option[sum-atom-atom]%
    \fdfindex*{TS!Weight.Method:sum-atom-atom}%

    Weight according to the total sum of the atomic density matrix
    sub-blocks
    \begin{equation}
      w_{ij,\mn}^{\Sigma} \mypropto
      \sqrt{
          % First the i'th atom
          \sum_{\in\{i\}}(\Delta\rho_{\mn}^L)^2
          \; % ensure a little space between them
          % second the j'th atom
          \sum_{\in\{j\}}(\Delta\rho_{\mn}^L)^2
      }
    \end{equation}

    \option[sum-atom-orb]%
    \fdfindex*{TS!Weight.Method:sum-atom-orb}%

    Weight according to the total sum of the atomic density matrix
    sub-block times the weight of the orbital weight
    \begin{equation}
      w_{ij,\mn}^{\Sigma} \mypropto
      \sqrt{
          w_{ij}^{\Sigma}
          w_{ij,\mn}
      }
    \end{equation}

    \option[mean]%
    \fdfindex*{TS!Weight.Method:mean}%

    A standard average.

  \end{fdfoptions}


  Each of the methods (except \fdf*{mean}) comes in a correlated and
  uncorrelated variant where $\sum$ is either outside or inside the
  square, respectively.

\end{fdfentry}

\begin{fdfentry}{TS!Weight.k.Method}[string]<correlated|uncorrelated>

  Control weighting \emph{per} $k$-point or the full sum. I.e. if
  \fdf*{uncorrelated} is used it will weight $n_k$ times if there are
  $n_k$ $k$-points in the Brillouin zone.

\end{fdfentry}

\begin{fdflogicalT}{TS!Forces}

  Control whether the forces are calculated. If \emph{not} \tsiesta\
  will use slightly less memory and the performance slightly
  increased, however the final forces shown are incorrect.

  If this is \fdftrue\ the file \sysfile{TSFA} (and possibly the
  \sysfile{TSFAC}) will be created. They contain forces for the atoms
  that are having updated density-matrix elements
  (\fdf{TS!Elec.<>!DM-update:all}).

  Generally one should not expect good forces close to the
  electrode/device interface since this typically has some
  electrostatic effects that are inherent to the \tsiesta\ method.
  Forces on atoms \emph{far} from the electrode can safely be
  analyzed.

\end{fdflogicalT}

\begin{fdfentry}{TS!dQ}[string]<none|buffer|fermi>
  \fdfindex*{TS!dQ:fermi}

  Any excess/deficiency of charge can be re-adjusted after each
  \tsiesta\ cycle to reduce charge fluctuations in the cell.

  \note recommended to \emph{only} use charge corrections for
  $0\,\mathrm{V}$ calculations.

  The non-neutral charge in \tsiesta\ cycles is an expression of one
  of the following things:
  \begin{enumerate}
    \item An incorrect screening towards the electrodes. To check
    this, simply add more electrode layers towards the device at each
    electrode and see how the charge evolves. It should tend to zero.

    The best way to check this is to follow these steps:
    \begin{enumerate}
      \item%
      Perform a \siesta-only calculation (the resulting DM
      should be used as the starting point for both following
      calculations)

      \item%
      Perform a \tsiesta\ calculation with the option
      \fdf{TS!Elecs!DM.Init:diagon} (please note that the electrode
      option has precedence, so remove any entry from the
      \fdf{TS!Elec.<>} block)

      \item%
      Perform a \tsiesta\ calculation with the option
      \fdf{TS!Elec.<>!DM-init:bulk} (please note that the electrode
      option has precedence, so remove any entry from the
      \fdf{TS!Elec.<>} block)

    \end{enumerate}

    Now compare the final output and the initial charge distribution,
    e.g.:
    \begin{output}
>>> TS.Elecs.DM.Init diagon
transiesta: Charge distribution, target =    396.00000
Total charge                  [Q]  :   396.00000

>>> TS.Elecs.DM.Init bulk
transiesta: Charge distribution, target =    396.00000
Total charge                  [Q]  :   395.9995
\end{output}

    The above shows that there is very little charge difference
    between the bulk electrode DM and the scattering region. This
    ensures that the charge distribution are similar and that your
    electrode is sufficiently screened.

    Additionally one may compare the final output such as total
    energies, calculated DOS and ADOS (see \tbtrans). If the two
    calculations show different properties, one should carefully
    examine the system setup.

    \item An incorrect reference energy level. In \tsiesta\ the Fermi
    level is calculated from the \siesta\ SCF. However, the \siesta\
    Fermi level corresponds to a periodic calculation and \emph{not}
    an open system calculation such as NEGF.

    If the first step shows a good screening towards the electrode it
    is usually the reference energy level, then use \fdf{TS!dQ:fermi}.

    \item A combination of the above, this is the typical case.
  \end{enumerate}

  \begin{fdfoptions}

    \option[none]%
    No charge corrections are introduced.

    \option[buffer]%
    Excess/missing electrons are placed in the buffer regions (buffer
    atoms are required to exist)

    \option[fermi] %
    Correct the charge filling by calculating a new reference energy
    level (referred to as the Fermi level). \\
    We approximate the contribution to be constant around the Fermi
    level and find
    \begin{equation}
      \label{eq:fermi-shift}
      \mathrm{d}E_F = \frac{Q'-Q}{Q|_{E_F}},
    \end{equation}
    where $Q'$ is the charge from a \tsiesta\ SCF step
    and $Q|_{E_F}$ is the equilibrium charge at the current Fermi
    level, $Q$ is the supposed charge to reside in the
    calculation. Fermi correction utilizes Eq.~\eqref{eq:fermi-shift} for
    the first correction and all subsequent corrections are based on a
    cubic spline interpolation to faster converge the
    ``correct'' Fermi level.

    This method will create a file called \file{TS\_FERMI}.

    \note correcting the reference energy level is a costly
    operation since the SCF cycle typically gets
    \emph{corrupted} resulting in many more SCF cycles.

  \end{fdfoptions}

\end{fdfentry}

\begin{fdfentry}{TS!dQ!Factor}[real]<0.8>

  Any positive value close to $1$. $0$ means no charge correction. $1$
  means total charge correction. This will reduce the fluctuations in
  the SCF and setting this to $1$ may result in difficulties in
  converging.

\end{fdfentry}

\begin{fdfentry}{TS!dQ!Fermi.Tolerance}[real]<0.01>

  The tolerance at which the charge correction will converge. Any
  excess/missing charge ($|Q'-Q|>\mathrm{Tol}$) will result in a
  correction for the Fermi level.

\end{fdfentry}

\begin{fdfentry}{TS!dQ!Fermi.Max}[energy]<$1.5\,\mathrm{eV}$>%

  The maximally allowed value that the Fermi level will change from a
  charge correction using the Fermi correction method. In case the
  Fermi level lies in between two bands a DOS of $0$ at the Fermi
  level will make the Fermi change equal to $\infty$. This is not
  physical and the user can thus truncate the correction.

  \note If you know the band-gab, setting this to $1/4$ (or smaller)
  of the band gab seems like a better value than the rather
  arbitrarily default one.

\end{fdfentry}

\begin{fdfentry}{TS!dQ!Fermi.Eta}[energy]<$1\,\mathrm{meV}$>%

  The $\eta$ value that we extrapolate the charge at the poles to.
  Usually a smaller $\eta$ value will mean larger changes in the
  Fermi level. If the charge convergence w.r.t. the Fermi level is
  fluctuating a lot one should increase this $\eta$ value.

\end{fdfentry}

\begin{fdflogicalT}{TS!HS.Save}
  \fdfindex*{TS!HS.Save:true}

  Must be \fdftrue\ for saving the Hamiltonian (\sysfile{TSHS}). Can only be set if
  \fdf{SolutionMethod} is not \fdf*{transiesta}.

  The default is \fdffalse\ for \fdf{SolutionMethod} different from
  \fdf*{transiesta} and if \code{--electrode} has not been passed as a
  command line argument.

  \note The \sysfile{TSHS} file format is deprecated and may be removed in
  future \siesta\ versions. The \sysfile{TS.HSX} file format (\fdf{Save!HS}) is
  feature complete, and should be used.

\end{fdflogicalT}

\begin{fdflogicalT}{TS!DE.Save}
  \fdfindex*{TS!DE.Save:true}

  Must be \fdftrue\ for saving the density and energy density matrix
  for continuation runs (\sysfile{TSDE}). Can only be set if
  \fdf{SolutionMethod} is not \fdf*{transiesta}.

  The default is \fdffalse\ for \fdf{SolutionMethod} different from
  \fdf*{transiesta} and if \code{--electrode} has not been passed as a
  command line argument.

\end{fdflogicalT}

\begin{fdflogicalF}{TS!S.Save}

  This is a flag mainly used for the Inelastica code to produce
  overlap matrices for Pulay corrections. This should only be used by
  advanced users.

\end{fdflogicalF}


\begin{fdflogicalF}{TS!SIESTA.Only}

  Stop \tsiesta\ right after the initial diagonalization run in
  \siesta. Upon exit it will also create the \sysfile{TSDE} file which
  may be used for initialization runs later.

  This may be used to start several calculations from the same initial
  density matrix, and it may also be used to rescale the Fermi level
  of electrodes. The rescaling is primarily used for semi-conductors
  where the Fermi levels of the device and electrodes may be
  misaligned.

\end{fdflogicalF}


\begin{fdflogicalF}{TS!Analyze}

  When using the BTD solution method (\fdf{TS!SolutionMethod}) this
  will analyze the Hamiltonian and printout an analysis of the
  sparsity pattern for optimal choice of the BTD partitioning
  algorithm.

  This yields information regarding the \fdf{TS!BTD!Pivot} flag.

  \note we advice users to \emph{always} run an analyzation step prior
  to actual calculation and select the \emph{best} BTD format. This
  analyzing step is very fast and may be performed on small
  work-station computers, even on systems of $\gg10,000$ orbitals.

  To run the analyzing step you may do:
  \begin{shellexample}
    siesta -fdf TS.Analyze RUN.fdf > analyze.out
  \end{shellexample}
  note that there is little gain on using MPI and it should complete
  within a few minutes, no matter the number of orbitals.

  Choosing the best one may be difficult. Generally one should choose
  the pivoting scheme that uses the least amount of memory. However,
  one should also choose the method with the largest block-size being
  as small as possible. As an example:
  \begin{output}[fontsize=\footnotesize]
TS.BTD.Pivot atom+GPS
...
    BTD partitions (7):
     [ 2984, 2776, 192, 192, 1639, 4050, 105 ]
    BTD matrix block size [max] / [average]: 4050 /   1705.429
    BTD matrix elements in % of full matrix:   47.88707 %

TS.BTD.Pivot atom+GGPS
...
    BTD partitions (6):
     [ 2880, 2916, 174, 174, 2884, 2910 ]
    BTD matrix block size [max] / [average]: 2916 /   1989.667
    BTD matrix elements in % of full matrix:   48.62867 %

  \end{output}
  Although the GPS method uses the least amount of memory, the GGPS
  will likely perform better as the largest block in GPS is $4050$
  vs. $2916$ for the GGPS method.

\end{fdflogicalF}

\begin{fdflogicalF}{TS!Analyze.Graphviz}
  \fdfdepend{TS!Analyze}

  If performing the analysis, also create the connectivity graph and
  store it as \file{GRAPHVIZ\_atom.gv} or \file{GRAPHVIZ\_orbital.gv}
  to be post-processed in Graphviz\footnote{\url{www.graphviz.org}}.

\end{fdflogicalF}


\subsection{\texorpdfstring{$k$}{k}-point sampling}


The options for $k$-point sampling are identical to the \siesta\
options, \fdf{kgrid!MonkhorstPack}, \fdf{kgrid!Cutoff} or
\fdf{kgrid!File}.

One may however use specific \tsiesta\ $k$-points by using these
options:

\begin{fdfentry}{TS.kgrid!MonkhorstPack}[block]<\fdfvalue{kgrid!MonkhorstPack}>%

  See \fdf{kgrid!MonkhorstPack} for details.

\end{fdfentry}

\begin{fdfentry}{TS.kgrid!Cutoff}[length]<$0.\,\mathrm{Bohr}$>

  See \fdf{kgrid!Cutoff} for details.

\end{fdfentry}

\begin{fdfentry}{TS.kgrid!File}[string]<none>

  See \fdf{kgrid!File} for details.

\end{fdfentry}


\subsubsection{Algorithm specific options}

These options adhere to the specific solution methods available for
\tsiesta. For instance the \fdf*{TS.BTD.*} options adhere only when
using \fdf{TS!SolutionMethod:BTD}, similarly for options with
\fdf*{MUMPS}.

\begin{fdfentry}{TS!BTD!Pivot}[string]<\nonvalue{first electrode}>

  Decide on the partitioning for the BTD matrix. One may denote either
  \fdf*{atom+} or \fdf*{orb+} as a prefix which does the analysis on
  the atomic sparsity pattern or the full orbital sparsity pattern,
  respectively. If neither are used it will default to \fdf*{atom+}.

  Please see \fdf{TS!Analyze}.

  \begin{fdfoptions}

    \option[<elec-name>|CG-<elec-name>]%
    The partitioning will be a connectivity graph starting from the
    electrode denoted by the name. This name \emph{must} be found in
    the \fdf{TS!Elecs} block. One can append more than one electrode
    to simultaneously start from more than 1 electrode. This may be
    necessary for multi-terminal calculations.

    \option[rev-CM] %
    Use the reverse Cuthill-McKee for pivoting the matrix elements to
    reduce bandwidth. One may omit \fdf*{rev-} to use the standard
    Cuthill-McKee algorithm (not recommended).

    This pivoting scheme depends on the initial starting
    electrodes, append \fdf*{+<elec-name>} to start the Cuthill-McKee
    algorithm from the specified electrode(s).

    \option[GPS] %
    Use the Gibbs-Poole-Stockmeyer algorithm for reducing the
    bandwidth.

    \option[GGPS] %
    Use the generalized Gibbs-Poole-Stockmeyer algorithm for reducing
    the bandwidth.

    \note this algorithm does not work on dis-connected graphs.

    \option[PCG] %
    Use the perphiral connectivity graph algorithm for reducing the
    bandwidth.

    This pivoting scheme \emph{may} depend on the initial starting
    electrode(s), append \fdf*{+<elec-name>} to initialize the PCG
    algorithm from the specified electrode(s).

  \end{fdfoptions}

  Examples are
  \begin{fdfexample}
    TS.BTD.Pivot atom+GGPS
    TS.BTD.Pivot GGPS
    TS.BTD.Pivot orb+GGPS
    TS.BTD.Pivot orb+PCG+Left
  \end{fdfexample}
  where the first two are equivalent. The 3rd and 4th are more heavy
  on analysis and will typically not improve the bandwidth reduction.

\end{fdfentry}

\begin{fdfentry}{TS!BTD!Optimize}[string]<speed|memory>

  When selecting the smallest blocks for the BTD matrix there are
  certain criteria that may change the size of each block. For very
  memory consuming jobs one may choose the \fdf*{memory}.

  \note often both methods provide \emph{exactly} the same BTD matrix
  due to constraints on the matrix.

\end{fdfentry}

\begin{fdfentry}{TS!BTD!Guess1.Min}[int]<\nonvalue{empirically determined}>
  \fdfdepend{TS!BTD!Guess1.Max}

  Constructing the blocks for the BTD starts by \emph{guessing} the
  first block size. One could guess on all different block sizes, but
  to speed up the process one can define a smaller range of guesses by
  defining \fdf{TS!BTD!Guess1.Min} and \fdf{TS!BTD!Guess1.Max}.

  The initial guessed block size will be between the two values.

  By default this is $1/4$ of the minimum bandwidth for a selected
  first set of orbitals.

  \note setting this to 1 may sometimes improve the final BTD matrix
  blocks.

\end{fdfentry}

\begin{fdfentry}{TS!BTD!Guess1.Max}[int]<\nonvalue{empirically determined}>
  \fdfdepend{TS!BTD!Guess1.Min}

  See \fdf{TS!BTD!Guess1.Min}.

  \note for improved initialization performance setting Min/Max flags
  to the first block size for a given pivoting scheme will drastically
  reduce the search space and make initialization much
  faster.

\end{fdfentry}

\begin{fdfentry}{TS!BTD!Spectral}[string]<propagation|column>

  How to compute the spectral function ($G\Gamma G^\dagger$).

  For $\Nelec<4$ this defaults to \fdf*{propagation} which should be the
  fastest.

  For $\Nelec\ge4$ this defaults to \fdf*{column}.

  Check which has the best performance for your system if you endeavor
  on huge amounts of calculations for the same system.

\end{fdfentry}


\begin{fdfentry}{TS!MUMPS!Ordering}[string]<\nonvalue{read MUMPS
      manual}>

  One may select from a number of different matrix orderings which are
  all described in the MUMPS manual.

  The following list of orderings are available (without detailing
  their differences): %
  \fdf*{auto}, \fdf*{AMD}, \fdf*{AMF}, \fdf*{SCOTCH}, \fdf*{PORD},
  \fdf*{METIS}, \fdf*{QAMD}.

\end{fdfentry}

\begin{fdfentry}{TS!MUMPS!Memory}[integer]<20>

  Specify a factor for the memory consumption in MUMPS. See the
  \fdf*{INFOG(9)} entry in the MUMPS manual. Generally if \tsiesta\
  dies and \fdf*{INFOG(9)=-9} one should increase this number.

\end{fdfentry}

\begin{fdfentry}{TS!MUMPS!BlockingFactor}[integer]<112>

  Specify the number of internal block sizes. Larger numbers increases
  performance at the cost of memory.

  \note this option may heavily influence performance.

\end{fdfentry}

\subsubsection{Poisson solution for fixed boundary conditions}

\tsiesta\ requires fixed boundary conditions and forcing this is an
intricate and important detail.

It is important that these options are exactly the same if one reuses
the \sysfile{TSDE} files.

\begin{fdfentry}{TS!Poisson}[string]<ramp|elec-box|\nonvalue{file}>

  Define how the correction of the Poisson equation is
  superimposed. The default is to apply the linear correction across
  the entire cell (if there are two semi-infinite aligned
  electrodes). Otherwise this defaults to the \emph{box} solution
  which will introduce spurious effects at the electrode
  boundaries. In this case you are encouraged to supply a \fdf*{file}.

  If the input is a \fdf*{file}, it should be a NetCDF file containing
  the grid information which acts as the boundary conditions for the
  SCF cycle.
  The grid information should conform to the grid size of the
  unit-cell in the simulation.
  %
  \note the file option is only applicable if compiled with CDF4
  compliance.

  \begin{fdfoptions}
    \option[ramp]%
    \fdfindex*{TS!Poisson:ramp}%

    Apply the ramp for the full cell. This is the default for 2
    electrodes.

    \option[<file>]%
    \fdfindex*{TS!Poisson:<file>}%

    Specify an external file used as the boundary conditions for the
    applied bias. This is encouraged to use for $\Nelec>2$ electrode
    calculations but may also be used when an \emph{a priori}
    potential profile is know.

    The file should contain something similar to this output
    (\code{ncdump -h}):
    \begin{output}[fontsize=\footnotesize]
netcdf <file> {
dimensions:
	one = 1 ;
	a = 43 ;
	b = 451 ;
	c = 350 ;
variables:
	double Vmin(one) ;
		Vmin:unit = "Ry" ;
	double Vmax(one) ;
		Vmax:unit = "Ry" ;
	double V(c, b, a) ;
		V:unit = "Ry" ;
}
    \end{output}
    Note that the units should be in Ry. \code{Vmax}/\code{Vmin}
    should contain the maximum/minimum fixed boundary conditions in
    the Poisson solution. This is used internally by \tsiesta\ to
    scale the potential to arbitrary $V$. This enables the Poisson
    solution to only be solved \emph{once} independent on subsequent
    calculations. For chemical potential configurations where the
    Poisson solution is not linearly dependent one have to create
    separate files for each applied bias.


    \option[elec-box]%
    \fdfindex*{TS!Poisson:elec-box}%

    The default potential profile for $\Nelec>2$, or when the electrodes
    does are not aligned (in terms of their transport direction).

    \note usage of this Poisson solution is \emph{highly}
    discouraged. Please see \fdf{TS!Poisson:<file>}.

  \end{fdfoptions}

\end{fdfentry}

\begin{fdfentry}{TS!Hartree.Fix}[string]<[-+][ABC]>

  Specify which plane to fix the Hartree potential at. For regular (2
  electrode calculations with a single transport direction) this
  should not be set.
  %
  For $\Nelec\neq2$ electrode systems one \emph{have} to specify a
  plane to fix. One can specify one or several planes to fix. Users
  are encouraged to fix the plane where the entire plane has the
  highest/lowest potential.

\end{fdfentry}

\begin{fdfentry}{TS!Hartree.Fix!Frac}[real]<$1.$>

  Fraction of the correction that is applied.

  \note this is an experimental feature!

\end{fdfentry}

\begin{fdfentry}{TS!Hartree.Offset}[energy]<$0\,\mathrm{eV}$>

  An offset in the Hartree potential to match the electrode potential.

  This value may be useful in certain cases where the Hartree
  potentials are very different between the electrode and device
  region calculations.

  This should not be changed between different bias calculations. It
  directly relates to the reference energy level ($E_F$).

\end{fdfentry}

\subsubsection{Electrode description options}

As \tsiesta\ supports $\Nelec$ electrodes one needs to specify all
electrodes in a generic input format.

Note that electrodes should be \emph{metallic} so that the Fermi-level
is well defined. Please see Sec.~\ref{sec:transiesta:description} for
more details.

\begin{fdfentry}{TS!Elecs}[block]

  Each line denote an electrode which is queried in \fdf{TS!Elec.<>}
  for its setup.

\end{fdfentry}

\begin{fdfentry}{TS!Elec.<>}[block]

  Each line represents a setting for electrode \fdf*{<>}.
  There are a few lines that \emph{must} be present, \fdf*{HS},
  \fdf*{semi-inf-dir}, \fdf*{electrode-pos}, \fdf*{chem-pot}. The
  remaining options are optional.

  \note Options prefixed with \fdf*{tbt} are neglected in \tsiesta\
  calculations. In \tbtrans\ calculations these flags has precedence
  over the other options and \emph{must} be placed at the end of the
  block.

  \begin{fdfoptions}

    \option[HS]%
    \fdfindex*{TS!Elec.<>!HS}%
    The Hamiltonian information from the initial electrode
    calculation. This file retains the geometrical information as well
    as the Hamiltonian, overlap matrix and the Fermi-level of the
    electrode.
    %
    This is a file-path and the Hamiltonian file need not be in the same
    directory, i.e. the path can be relative.

    \note \sysfile{HSX}, \sysfile{TSHS} and \sysfile{nc} files are supported.
    See \fdf{Save!HS}, \fdf{TS.HS!Save} or \fdf{CDF!Save} are the relevant flags
    for this.

    \note Please note that \tsiesta\ expects a metallic electrode.
    Results can not be trusted for semi-conductors.

    \option[semi-inf-direction|semi-inf-dir|semi-inf]%
    \fdfindex*{TS!Elec.<>!semi-inf-direction}%
    The semi-infinite direction of the electrode with respect to the
    electrode unit-cell.

    It may be one of \fdf*{[-+][abc]}, \fdf*{[-+]A[123]}, \fdf*{ab},
    \fdf*{ac}, \fdf*{bc} or \fdf*{abc}. The latter four all refer to a
    real-space self-energy as described in \cite{Papior2019}.

    \note this direction is \emph{not} with respect to the scattering
    region unit cell. It is with respect to the electrode unit
    cell. \tsiesta\ will figure out the alignment of the electrode
    unit cell and the scattering region unit-cell.

    \option[chemical-potential|chem-pot|mu]%
    \fdfindex*{TS!Elec.<>!chemical-potential}%
    The chemical potential that is associated with this
    electrode. This is a string that should be present in the
    \fdf{TS!ChemPots} block.

    \option[electrode-position|elec-pos]%
    \fdfindex*{TS!Elec.<>!electrode-position}%
    The index of the electrode in the scattering region.
    This may be given by either \fdf*{elec-pos <idx>}, which refers to
    the first atomic index of the electrode residing at index
    \fdf*{<idx>}. Else the electrode position may be given via
    \fdf*{elec-pos end <idx>} where the last index of the electrode
    will be located at \fdf*{<idx>}.

    \option[used-atoms]%
    \fdfindex*{TS!Elec.<>!used-atoms}%
    \fdfdepend{TS!Elec.<>!semi-inf-direction}%
    Number of atoms from the electrode calculation that is used in the
    scattering region as electrode. This may be useful when the
    periodicity of the electrodes forces extensive electrodes in the
    semi-infinite direction.

    If the semi-infinite direction is \emph{positive}, the first atoms will
    be retained.
    Contrary, if the semi-infinite direction is \emph{negative}, the last
    atoms will be retained.

    \note do not set this if you use all atoms in the electrode.

    \option[Bulk]%
    \fdfindex*{TS!Elec.<>!Bulk}%
    \fdfindex*{TS!Elec.<>!Bulk:true}%
    \fdfindex*{TS!Elec.<>!Bulk:false}%
    Control whether the Hamiltonian of the electrode region in the
    scattering region is enforced \emph{bulk} or whether the
    Hamiltonian is taken from the scattering region elements.

    This defaults to \fdftrue. If there are buffer atoms \emph{behind}
    the electrode it may be advantageous to set this to false to
    extend the electrode region, otherwise it is recommended to keep
    the default.

    This option changes how $\mathbf M_{\elec,\elec}$, see Eq.~\eqref{eq:negf:green-elec}, is setup.

    For \fdftrue\ $\big\{\Ham_\elec, \SO_\elec\big\}$ are taken from the
    electrode file (\fdf{TS!Elec.<>!HS}).

    For \fdffalse\ $\big\{\Ham_\elec, \SO_\elec\big\}$ are substituted by
    the device calculations electrode region.
    I.e. it is the self-consistent Hamiltonian.

    \option[DM-update]%
    \fdfindex*{TS!Elec.<>!DM-update}%
    \fdfindex*{TS!Elec.<>!DM-update:none}%
    \fdfindex*{TS!Elec.<>!DM-update:all}%
    \fdfdepend{TS!Elec.<>!Bulk}%
    String of values \fdf*{none}, \fdf*{cross-terms} or \fdf*{all}
    which controls which part of the electrode density matrix elements
    that are updated.
    %
    The density matrices that comprises an electrode and
    device-electrode region can be written as (omitting the central
    device region)
    \begin{equation}
      \label{eq:ts-dm-update}
      \DM =
      \begin{bmatrix}
        \DM_{\mathfrak e} & \DM_{\mathfrak eD} & 0
        \\
        \DM_{D\mathfrak e} & \ddots & \ddots
        \\
        0  & \ddots
      \end{bmatrix}
    \end{equation}
    This flag determines whether $\DM_{\mathfrak e}$ (\fdf*{all}) or
    $\DM_{\mathfrak eD}$ (\fdf*{cross-terms} and \fdf*{all}) or
    neither (\fdf*{none}) are updated in the SCF. The density matrices
    contains the charges and thus affects the Hamiltonian and Poisson
    solutions. Generally the default value will suffice and is
    recommended.

    If \fdf{TS!Elec.<>!Bulk:false} this is forced to \fdf*{all} and
    cannot be changed.

    If \fdf{TS!Elec.<>!Bulk:true} this defaults to \fdf*{cross-terms},
    but may be changed.

    \note if this is \fdf*{none} the forces on the atoms coupled to
    the electrode regions are \emph{not} to be trusted. The value
    \fdf*{none} should be avoided, if possible.

    \option[DM-init]%
    \fdfindex*{TS!Elec.<>!DM-init}%
    \fdfindex*{TS!Elec.<>!DM-init:diagon}%
    \fdfindex*{TS!Elec.<>!DM-init:bulk}%
    \fdfdepend{TS!Elecs!DM.Init,TS!Elec.<>!Bulk,TS!Voltage}%
    String of values \fdf*{bulk}, \fdf*{diagon} (default) or
    \fdf*{force-bulk} which controls whether the DM is initially
    overwritten by the DM from the bulk electrode calculation. This
    requires the DM file for the electrode to be present. Only
    \fdf*{force-bulk} will have effect if $V\neq0$. Otherwise this
    option only affects $V=0$ calculations.

    The density matrix elements in the electrodes of the scattering
    region may be forcefully set to the bulk values by reading in the
    DM of the corresponding electrode. If one uses
    \fdf{TS!Elec.<>!Bulk:false} it may be dis-advantageous to set this
    to \fdf*{bulk}.
    If the system is well setup (good screening towards electrodes),
    setting this to \fdf*{bulk} may be advantageous.

    This option may be used to check how good the electrodes are
    screened, see \fdf{TS!dQ:fermi}.

    \option[out-of-core]%
    \fdfindex*{TS!Elec.<>!Out-of-core}%
    \fdfdepend{TS!Elec.<>!Gf}%
    If \fdftrue\ (default) the GF files are created which contain
    the surface Green function.
    If \fdffalse\ the surface Green function will be calculated when
    needed.
    Setting this to \fdffalse\ will heavily degrade performance and
    it is highly discouraged!

    \option[Gf]%
    \fdfindex*{TS!Elec.<>!Gf}%
    String with filename of the surface Green function data
    (\sysfile{TSGF*}). This may be used to place a common surface
    Green function file in a top directory which may then be used in
    all calculations using the same electrode and the same contour.
    %
    If doing many calculations with the same electrode and $\mathbf k$, $E$ grids,
    then this can greatly improve throughput. It has a minor cost of disk-space.
    Note that the energy-grids are dependent on the applied bias.

    \option[Gf-Reuse]%
    \fdfindex*{TS!Elec.<>!Gf-Reuse}%
    \fdfdepend{TS!Elec.<>!Out-of-core,TS!Elec.<>!Gf}%
    Logical deciding whether the surface Green function file should be
    re-used or deleted.
    %
    If this is \fdffalse\ the surface Green function file is deleted
    and re-created upon start.

    \option[pre-expand]%
    \fdfindex*{TS!Elec.<>!pre-expand}%
    \fdfdepend{TS!Elec.<>!out-of-core}%
    String denoting how the expansion of the surface Green function
    file will be performed. This only affects the Green function file
    if \fdf*{Bloch} is larger than 1. By default the Green function
    file will contain the fully expanded surface Green function, but
    not Hamiltonian and overlap matrices (\fdf*{Green}). One may
    reduce the file size by setting this to \fdf*{Green} which only
    expands the surface Green function. Finally \fdf*{none} may be
    passed to reduce the file size to the bare minimum.
    %
    For performance reasons \fdf*{all} is preferred.

    If disk-space is a limited resource and the \sysfile{TSGF*} files
    are really big, try \fdf*{none}.

    \option[Eta]%
    \fdfindex*{TS!Elec.<>!Eta}%
    \fdfdepend{TS!Elecs!Eta}%
    Control the imaginary energy ($\eta$) of the surface Green
    function for this electrode.

    The imaginary part is \emph{only} used in the non-equilibrium
    contours since the equilibrium are already lifted into the complex
    plane. Thus this $\eta$ reflects the imaginary part in the
    $G\Gamma G^\dagger$ calculations. Ensure that all imaginary values
    are larger than $0$ as otherwise \tsiesta\ may seg-fault.

    \note if this energy is negative the complex value associated with
    the non-equilibrium contour is used. This is particularly useful
    when providing a user-defined contour along the real axis.

    See Sec.~\ref{sec:negf-equations} for details.
    This options changes the $\eta$ value in the calculated self-energy
    ($\SE(E+i\eta)$), while it does not change the $\eta$ value used
    in the device region.

    \option[DE]%
    \fdfindex*{TS!Elec.<>!DE}%
    \fdfdepend{TS!Elec.<>!DM-init}%
    Density and energy density matrix file for the electrode. This may
    be used to initialize the density matrix elements in the electrode
    region by the bulk values. See \fdf{TS!Elec.<>!DM-init:bulk}.

    \note this should only be performed on one \tsiesta\ calculation
    as then the scattering region \sysfile{TSDE} contains the
    electrode density matrix.

    \option[Bloch]%
    \fdfindex*{TS!Elec.<>!Bloch}%
    $3$ integers should be present on this line which each denote the
    number of times bigger the scattering region electrode is compared
    to the electrode, in each lattice direction. Remark that these
    expansion coefficients are with regard to the electrode unit-cell.
    This is denoted ``Bloch'' because it is an expansion based on
    Bloch waves.

    \note Using symmetries such as periodicity will greatly increase
    performance.

    \option[Bloch-A/a1|B/a2|C/a3]%
    \fdfindex{TS!Elec.<>!Bloch}%
    Specific Bloch expansions in each of the electrode unit-cell
    direction. See \fdf*{Bloch} for details.

    \option[Accuracy]%
    \fdfindex*{TS!Elec.<>!Accuracy}%
    \fdfdepend{TS!Elecs!Accuracy}%
    Control the convergence accuracy required for the self-energy
    calculation when using the Lopez-Sancho, Lopez-Sancho iterative
    scheme.

    \note advanced use \emph{only}.

    \option[delta-Ef]%
    \fdfindex*{TS!Elec.<>!delta-Ef}%
    Specify an offset for the Fermi-level of the electrode. This will
    directly be added to the Fermi-level found in the electrode file.

    Effectively this will transform the used chemical potential to
    \begin{equation}
      \mu'_{\mathrm{used}} = \mu_{\mathrm{used}} + \delta E_F.
    \end{equation}

    \note this option only makes sense for semi-conducting electrodes
    since it shifts the entire electronic structure. This is because
    the Fermi-level may be arbitrarily placed anywhere in the band
    gap. It is the users responsibility to define a value which does
    not introduce a potential drop between the electrode and device
    region. Please do not use unless you really know what you are
    doing.

    \option[V-fraction]%
    \fdfindex*{TS!Elec.<>!V-fraction}%

    Specify the fraction of the chemical potential shift in the
    electrode-device coupling region. This corresponds to altering Eq.~\eqref{eq:negf:green-elec} by:
    \begin{equation}
      \Ham_{\elec,D} \leftarrow \Ham_{\elec,D} +
      \mu_{\elec} \mathrm{V-fraction} \SO_{\elec,D}
    \end{equation}
    in the coupling region. Consequently the value \emph{must} be
    between $0$ and $1$.

    \note this option \emph{only} makes sense for
    \fdf{TS!Elec.<>!DM-update:none} since otherwise the electrostatic
    potential will be incorporated in the Hamiltonian.

    Only expert users should play with this number.

    \option[check-kgrid]%
    \fdfindex*{TS!Elec.<>!check-kgrid}%
    For $\Nelec$ electrode calculations the $\mathbf k$ mesh will sometimes
    not be equivalent for the electrodes and the device region
    calculations. However, \tsiesta\ requires that the device and
    electrode $\mathbf k$ samplings are commensurate. This flag
    controls whether this check is enforced for a given electrode.

    \note only use if fully aware of the implications!

  \end{fdfoptions}

\end{fdfentry}

There are several flags which are globally controlling the variables
for the electrodes (with \fdf{TS!Elec.<>} taking precedence).

\begin{fdflogicalT}{TS!Elecs!Bulk}

  This globally controls how the Hamiltonian is treated in all
  electrodes.
  %
  See \fdf{TS!Elec.<>!Bulk}.

\end{fdflogicalT}

\begin{fdfentry}{TS!Elecs!Eta}[energy]<$1\,\mathrm{meV}$>

  Globally control the imaginary energy ($\eta$) used for the surface
  Green function calculation on the non-equilibrium contour.
  %
  See \fdf{TS!Elec.<>!Eta} for extended details on the usage of this
  flag.

\end{fdfentry}

\begin{fdfentry}{TS!Elecs!Accuracy}[energy]<$10^{-13}\,\mathrm{eV}$>

  Globally control the accuracy required for convergence of the self-energy.
  %
  See \fdf{TS!Elec.<>!Accuracy}.

\end{fdfentry}

\begin{fdflogicalF}{TS!Elecs!Neglect.Principal}

  If this is \fdffalse\ \tsiesta\ dies if there are connections beyond
  the principal cell.

  \note set this to \fdftrue\ with care, non-physical results may
  arise. Use at your own risk!

\end{fdflogicalF}

\begin{fdflogicalT}{TS!Elecs!Gf.Reuse}

  Globally control whether the surface Green function files should
  be re-used (\fdftrue) or re-created (\fdffalse).

  See \fdf{TS!Elec.<>!Gf-Reuse}.

\end{fdflogicalT}

\begin{fdflogicalT}{TS!Elecs!Out-of-core}

  Whether the electrodes will calculate the self energy at each SCF
  step. Using this will not require the surface Green function files
  but at the cost of heavily degraded performance.

  See \fdf{TS!Elec.<>!Out-of-core}.

\end{fdflogicalT}

\begin{fdfentry}{TS!Elecs!DM.Update}[string]<cross-terms|all|none>

  Globally controls which parts of the electrode density matrix
  gets updated.

  See \fdf{TS!Elec.<>!DM-update}.

\end{fdfentry}

\begin{fdfentry}{TS!Elecs!DM.Init}[string]<diagon|bulk|force-bulk>
  \fdfindex*{TS!Elecs!DM.Init:bulk}%
  \fdfindex*{TS!Elecs!DM.Init:diagon}%

  Specify how the density matrix elements in the electrode regions of
  the scattering region will be initialized when starting \tsiesta.

  See \fdf{TS!Elec.<>!DM-init}.

\end{fdfentry}

\begin{fdfentry}{TS!Elecs!Coord.EPS}[length]<$0.001\,\mathrm{Ang}$>

  When using Bloch expansion of the self-energies one may experience
  difficulties in obtaining perfectly aligned electrode coordinates.

  This parameter controls how strict the criteria for equivalent
  atomic coordinates is. If \tsiesta\ crashes due to mismatch between
  the electrode atomic coordinates and the scattering region
  calculation, one may increase this criteria. This should only be
  done if one is sure that the atomic coordinates are almost similar
  and that the difference in electronic structures of the two may be
  negligible.

\end{fdfentry}


\subsubsection{Chemical potentials}
\label{sec:ts:chem-pot}

For $\Nelec$ electrodes there will also be $N_\mu$ chemical
potentials. They are defined via blocks similar to \fdf{TS!Elecs}.

\begin{fdfentry}{TS!ChemPots}[block]

  Each line denotes a new chemical potential which is defined in the
  \fdf{TS!ChemPot.<>} block.

\end{fdfentry}

\begin{fdfentry}{TS!ChemPot.<>}[block]

  Each line defines a setting for the chemical potential named
  \fdf*{<>}.

  \begin{fdfoptions}

    \option[chemical-shift|mu]%
    \fdfindex*{TS!ChemPot.<>!chemical-shift}%
    \fdfindex*{TS!ChemPot.<>!mu}%

    Define the chemical shift (an energy) for this chemical
    potential. One may specify the shift in terms of the applied bias
    using \fdf*{V/<integer>} instead of explicitly typing the energy.

    \option[contour.eq]%
    \fdfindex*{TS!ChemPot.<>!contour.eq}%
    A subblock which defines the integration curves for the
    equilibrium contour for this equilibrium chemical potential. One
    may supply as many different contours to create whatever shape of
    the contour

    Its format is
    \begin{fdfexample}
      contour.eq
       begin
        <contour-name-1>
        <contour-name-2>
        ...
       end
    \end{fdfexample}

    \note If you do \emph{not} specify \fdf*{contour.eq} in the block
    one will automatically use the continued fraction method and you
    are encouraged to use $50$ or more poles\cite{Ozaki2010}.

    \option[ElectronicTemperature|Temp|kT]%
    \fdfindex*{TS!ChemPot.<>!ElectronicTemperature}%
    \fdfindex*{TS!ChemPot.<>!Temp}%
    \fdfindex*{TS!ChemPot.<>!kT}%

    Specify the electronic temperature (as an energy or in
    Kelvin). This defaults to \fdf{TS!ElectronicTemperature}.

    One may specify this in units of \fdf{TS!ElectronicTemperature} by
    using the unit \fdf*{kT}.

    \option[contour.eq.pole]%
    \fdfindex*{TS!ChemPot.<>!contour.eq.pole}%

    Define the number of poles used via an energy
    specification. \tsiesta\ will automatically convert the energy to
    the closest number of poles (rounding up).

    \note this has precedence over
    \fdf{TS!ChemPot.<>!contour.eq.pole.N} if it is specified
    \emph{and} a positive energy. Set this to a negative energy to
    directly control the number of poles.

    \option[contour.eq.pole.N]%
    \fdfindex*{TS!ChemPot.<>!contour.eq.pole.N}%

    Define the number of poles via an integer.

    \note this will only take effect if
    \fdf{TS!ChemPot.<>!contour.eq.pole} is a negative energy.

  \end{fdfoptions}

  \note It is important to realize that the parametrization in 4.1 of
  the voltage into the chemical potentials enables one to have a
  \emph{single} input file which is never required to be changed, even
  when changing the applied bias (if using the command line options
  for specifying the applied bias).
  %
  This is different from 4.0 and prior versions since one had to
  manually change the \fdf*{TS.biasContour.NumPoints} for each applied
  bias.

\end{fdfentry}

These options complicate the input sequence for regular $2$ electrode
which is unfortunate.

Using \program{tselecs.sh -only-mu} yields this output:
\begin{fdfexample}
  %block TS.ChemPots
    Left
    Right
  %endblock
  %block TS.ChemPot.Left
    mu V/2
    contour.eq
      begin
        C-Left
        T-Left
      end
  %endblock
  %block TS.ChemPot.Right
    mu -V/2
    contour.eq
      begin
        C-Right
        T-Right
      end
  %endblock
\end{fdfexample}

Note that the default is a $2$ electrode setup with chemical
potentials associated directly with the electrode names
``Left''/``Right''. Each chemical potential has two parts of the
equilibrium contour named according to their name.



\subsubsection{Complex contour integration options}

Specifying the contour for $\Nelec$ electrode systems is a bit
extensive due to the possibility of more than 2 chemical
potentials. Please use the \program{Util/TS/tselecs.sh} as a means to
create default input blocks.

The contours are split in two segments. One, being the equilibrium
contour of each of the different chemical potentials. The second for
the non-equilibrium contour. The equilibrium contours are shifted
according to their chemical potentials with respect to a reference
energy. Note that for \tsiesta\ the reference energy is named the
Fermi-level, which is rather unfortunate (for non-equilibrium but not
equilibrium). Fortunately the non-equilibrium contours are defined
from different chemical potentials Fermi functions, and as such this
contour is defined in the window of the minimum and maximum chemical
potentials. Because the reference energy is the periodic Fermi level
it is advised to retain the average chemical potentials equal to
$0$. Otherwise applying different bias will shift transmission curves
calculated via \tbtrans\ relative to the average chemical potential.

In this section the equilibrium contours are defined, and in the next
section the non-equilibrium contours are defined.

\begin{fdfentry}{TS!Contours!Eq.Pole}[energy]<$1.5\,\mathrm{eV}$>

  The imaginary part of the line integral crossing the chemical
  potential. Note that the actual number of poles may differ between
  different calculations where the electronic temperatures are
  different.

  \note if the energy specified is negative,
  \fdf{TS!Contours!Eq.Pole.N} takes effect.

\end{fdfentry}

\begin{fdfentry}{TS!Contours!Eq.Pole.N}[integer]<8>

  Manually select the number poles for the equilibrium contour.

  \note this flag will only take effect if \fdf{TS!Contours!Eq.Pole}
  is a negative energy.

\end{fdfentry}

\begin{fdfentry}{TS!Contour.<>}[block]

  Specify a contour named \fdf*{<>} with options within the block.

  The names \fdf*{<>} are taken from the
  \fdf{TS!ChemPot.<>!contour.eq} block in the chemical potentials.

  The format of this block is made up of at least $4$ lines, in the
  following order of appearance.

  \begin{fdfoptions}

    \option[part]%
    \fdfindex*{TS!Contour.<>!part}%

    Specify which part of the equilibrium contour this is:
    \begin{fdfoptions}

      \option[circle]%
      The initial circular part of the contour

      \option[square]%
      The initial square part of the contour

      \option[line]%
      The straight line of the contour

      \option[tail]%
      The final part of the contour \emph{must} be a tail which
      denotes the Fermi function tail.

    \end{fdfoptions}

    \option[from \emph{a} to \emph{b}]%
    \fdfindex*{TS!Contour.<>!from}%

    Define the integration range on the energy axis.
    Thus \emph{a} and \emph{b} are energies.

    The parameters may also be given values \fdf*{prev}/\fdf*{next}
    which is the equivalent of specifying the same energy as the
    previous contour it is connected to.

    \note that \emph{b} may be supplied as \fdf*{inf} for \fdf*{tail}
    parts.

    \option[points/delta]%
    \fdfindex*{TS!Contour.<>!points}%
    \fdfindex*{TS!Contour.<>!delta}%

    Define the number of integration points/energy separation.
    If specifying the number of points an integer should be supplied.

    If specifying the separation between consecutive points an energy
    should be supplied.

    \option[method]%
    \fdfindex*{TS!Contour.<>!method}%

    Specify the numerical method used to conduct the integration. Here
    a number of different numerical integration schemes are accessible

    \begin{fdfoptions}
      \option[mid|mid-rule]%
      Use the mid-rule for integration.

      \option[simpson|simpson-mix]%
      Use the composite Simpson $3/8$ rule (three point Newton-Cotes).

      \option[boole|boole-mix]%
      Use the composite Booles rule (five point Newton-Cotes).

      \option[G-legendre]%
      Gauss-Legendre quadrature.

      \note has \fdf*{opt left}

      \note has \fdf*{opt right}

      \option[tanh-sinh]%
      Tanh-Sinh quadrature.

      \note has \fdf*{opt precision <>}

      \note has \fdf*{opt left}

      \note has \fdf*{opt right}

      \option[G-Fermi]%
      Gauss-Fermi quadrature (only on tails).

    \end{fdfoptions}

    \option[opt]%
    \fdfindex*{TS!Contour.<>!opt}%

    Specify additional options for the \fdf*{method}. Only a selected
    subset of the methods have additional options.

  \end{fdfoptions}

\end{fdfentry}

These options complicate the input sequence for regular $2$ electrode
which is unfortunate. However, it allows highly customizable contours.

Using \program{tselecs.sh -only-c} yields this output:
\begin{fdfexample}
  TS.Contours.Eq.Pole 2.5 eV
  %block TS.Contour.C-Left
    part circle
     from -40. eV + V/2 to -10 kT + V/2
       points 25
        method g-legendre
         opt right
  %endblock
  %block TS.Contour.T-Left
    part tail
     from prev to inf
       points 10
        method g-fermi
  %endblock
  %block TS.Contour.C-Right
    part circle
     from -40. eV -V/2 to -10 kT -V/2
       points 25
        method g-legendre
         opt right
  %endblock
  %block TS.Contour.T-Right
    part tail
     from prev to inf
       points 10
        method g-fermi
  %endblock
\end{fdfexample}
These contour options refer to input options for the chemical
potentials as shown in Sec.~\ref{sec:ts:chem-pot}
(p.~\pageref{sec:ts:chem-pot}). Importantly one should note the shift
of the contours corresponding to the chemical potential (the shift
corresponds to difference from the reference energy used in \tsiesta).


\subsubsection{Bias contour integration options}

The bias contour is similarly defined as the equilibrium
contours. Please use the \program{Util/TS/tselecs.sh} as a means to
create default input blocks.

\begin{fdfentry}{TS!Contours.nEq!Eta}[energy]<$\operatorname{min}[\eta_{\mathfrak e}]/10$>
  \fdfdepend{TS!Elecs!Eta}%

  The imaginary part ($\eta$) of the device states. While this may be
  set to $0$ for most systems it defaults to the minimum $\eta$ value
  for the electrodes ($\operatorname{min}[\eta_{\mathfrak
      e}]/10$). This ensures that the device broadening is always
  smaller than the electrodes while allowing broadening of localized
  states.

\end{fdfentry}

\begin{fdfentry}{TS!Contours.nEq!Fermi.Cutoff}[energy]<$5\,k_BT$>

  The bias contour is limited by the Fermi function tails. Numerically
  it does not make sense to integrate to infinity.
  %
  This energy defines where the bias integration window is turned into
  zero. Thus above $-|V|/2-E$ or below $|V|/2+E$ the DOS is defined as
  exactly zero.

\end{fdfentry}

\begin{fdfentry}{TS!Contours.nEq}[block]

  Each line defines a new contour on the non-equilibrium bias
  window. The contours defined \emph{must} be defined in
  \fdf{TS!Contour.nEq.<>}.

  These contours must all be \fdf*{part line} or \fdf*{part tail}.

\end{fdfentry}

\begin{fdfentry}{TS!Contour.nEq.<>}[block]

  This block is \emph{exactly} equivalently defined as the
  \fdf{TS!Contour.<>}. See page \pageref{TS!Contour.<>}.

\end{fdfentry}

The default options related to the non-equilibrium bias contour are
defined as this:
\begin{fdfexample}
  %block TS.Contours.nEq
    neq
  %endblock TS.Contours.nEq
  %block TS.Contour.nEq.neq
    part line
     from -|V|/2 - 5 kT to |V|/2 + 5 kT
       delta 0.01 eV
        method mid-rule
  %endblock TS.Contour.nEq.neq
\end{fdfexample}
If one chooses a different reference energy than $0$, then the limits
should change accordingly. Note that here \fdf*{kT} refers to
\fdf{TS!ElectronicTemperature}.


\subsection{Output}

\subsection{Standard output} \index{output!main output file}

\siesta\ writes a log of its workings to standard output (unit 6),
which is usually redirected to an ``output file''.

A brief description follows. See the example cases in the
siesta/Tests directory for illustration.

The program starts writing the version of the code which is
used. Then, the input \fdflib\ file is dumped into the output file as is
(except for empty lines). The program does part of the reading and
digesting of the data at the beginning within the \program{redata}
subroutine. It prints some of the information it digests. It is
important to note that it is only part of it, some other information
being accessed by the different subroutines when they need it during
the run (in the spirit of \fdflib\ input).  A complete list of the input
used by the code can be found at the end in the file \file{fdf.log},
including defaults used by the code in the run.

After that, the program reads the pseudopotentials, factorizes them
into Kleinman-Bylander form, and generates (or reads) the atomic basis
set to be used in the simulation. These stages are documented in the
output file.

The simulation begins after that, the output showing information of
the MD (or CG) steps and the SCF cycles within.  Basic descriptions of
the process and results are presented. The user has the option to
customize it, however,\index{output!customization} by defining
different options that control the printing of informations like
coordinates, forces, $\vec k$ points, etc.  The options are discussed
in the appropriate sections, but take into account the behavior of the
legacy \fdf{LongOutput} option, as in the current implementation might
silently activate output to the main .out file at the expense of
auxiliary files.

\begin{fdflogicalF}{LongOutput}
  \index{output!long}

  \siesta\ can write to standard output different data sets depending
  on the values for output options described below.  By default
  \siesta\ will not write most of them. They can be large for large
  systems (coordinates, eigenvalues, forces, etc.)  and, if written to
  standard output, they accumulate for all the steps of the
  dynamics. \siesta\ writes the information in other files (see Output
  Files) in addition to the standard output, and these can be
  cumulative or not.

  Setting \fdf{LongOutput} to \fdftrue\ changes the default of some
  options, obtaining more information in the output (verbose).  In
  particular, it redefines the defaults for the following:

  \begin{itemize}

    \item \fdf{WriteKpoints}%
    \index{output!grid $\vec k$ points}

    \item \fdf{WriteKbands}%
    \index{output!band $\vec k$ points}

    \item \fdf{WriteCoorStep}%
    \index{output!atomic coordinates!in a dynamics step}

    \item \fdf{WriteForces}%
    \index{output!forces}

    \item \fdf{WriteEigenvalues}%
    \index{output!eigenvalues}

    \item \fdf{WriteWaveFunctions}%
    \index{output!wave functions}

    \item \fdf{WriteMullikenPop}%
    \index{output!Mulliken analysis}%
    \index{Mulliken population analysis}%
    (it sets it to 1)

  \end{itemize}

  The specific changing of any of these options has precedence.

\end{fdflogicalF}


\subsection{Output to dedicated files}%
\index{output!dedicated files}

\siesta\ can produce a wealth of information in dedicated files,
with specific formats, that can be used for further analysis. See the
appropriate sections, and the appendix on file formats.
Please take into account the behavior of
\fdf{LongOutput}, as in the current implementation might silently
activate output to the main .out file at the expense of auxiliary
files.




\section{QM/MM with SIESTA}
\label{sec:qmmm}

 \siesta\ is ready to work as a QM-engine for hybrid Quantum Mechanics/
 Molecular Mechanics (QM/MM) simulations, using an external molecular
 mechanics software as a driver.
 For this purpose, you should run \siesta\ independently, since communication
 is done via either pipes or sockets; which is to say, you must have both your
 MM code and \siesta\ running simultaneously. You will need all of the
 regular components of a \siesta\ calculation, but atomic positions for the
 QM region (and cell dimensions) will be overwritten with data coming from
 the MM driver.

 \textbf{NOTE:} Please refer to the developer documentation on how to
 implement the communication from the MM side, and check the qmmm\_extern
 test for a reference implementation.

 It is also possible to run a stand-alone calculation using only \siesta\,
 providing a set of point charges and their coordinates in the fdf files.
 After calculation, \siesta\ will output the forces on the point charges in
 a .FAPC file. The following options refer to both the stand-alone
 calculations and when \siesta\ isused as a QM engine for an external driver
 program.

 \begin{fdflogicalF}{QMMM.Enabled}
  Enables the application of an external potential created by point charges
  near the QM system; i.e., enables QM/MM calculations.

 \end{fdflogicalF}

 \begin{fdfentry}{QMMM.CoulombType}[integer]<1>
  When set to 1 (default), uses Ewald summations for the electrostatic
  interaction between QM and MM regions.
  Setting this variable to 2 switches this scheme to just real space
  distance cut-off. This is not recommended, and the option is kept only
  for testing purposes.
 \end{fdfentry}

 \begin{fdfentry}{QMMM.CutOff}[length]<$20\,\mathrm{Bohr}$>
  Cut-off distance for QM-MM interactions. This magnitude is only relevant
  in the real part of the Ewald summation, or in the entire cutoff scheme.
  It is highly advisable that this magnitude does not go above one half
  of the simulation box's shortest side.

 \end{fdfentry}

 \begin{fdfentry}{QMMM.DensityCut}[length]<$1.10^{-6}\,\mathrm{Bohr}$>
  Tolerance when deciding if a given grid point is too close to a
  classical point charge. This is usually not worth changing.

 \end{fdfentry}

 \begin{fdfentry}{QMMM.Ewald.Rcut}[length]<$20\,\mathrm{Bohr}$>
  In Ewald summations, this value is used as a cut-off distance for
  how long the direct summation goes. Same considerations as with
  QMMM.CutOff apply here, and it is usually a good idea to keep of
  them in similar magnitudes if not the same.

  This radius is also used to set the value of the Ewald alpha coefficient,
  and the reciprocal space summation cut-off.

 \end{fdfentry}

 \begin{fdfentry}{PartialCharges}[block]

  This block contains the positions (in Angstrom) and charge values (in e-)
  for classical point charges. Each line corresponds to a given charge,
  indicating first x, y and z coordinates, and then the charge value.

  \begin{shellexample}
       from iq = 1 to ncharges
            read: x(iq), y(iq), z(iq), q(iq)
  \end{shellexample}

  It is important to note that this block will be ignored if \siesta\ is
  used by an external MM driver.
 \end{fdfentry}

 The following options are experimental and enable some form of
 smoothing potential along the Z-axis when dealing with electrodes.
 Use these only if you know exactly what you're doing.

 \begin{fdflogicalF}{QMMM.SmoothElectrode}
  Enables potential smoothing.

 \end{fdflogicalF}

 \begin{fdfentry}{QMMM.SmoothElectrode.Electrode}[length]<$0\,\mathrm{Bohr}$>
  Length of the electrodes.

 \end{fdfentry}

 \begin{fdfentry}{QMMM.SmoothElectrode.Smooth}[length]<$0\,\mathrm{Bohr}$>
  Length of the smoothing section.

 \end{fdfentry}

 \subsection{Using external MM driver}
 The following options are only used if \siesta\ is intended to run
 with an external MM driver. Most of these require knowing at least
 a bit of how the communication is handled by the MM program.

 \begin{fdfentry}{QMMM.Driver}[string]<$unknown$>
  Indicates to \siesta\ the name of the MM driver used. This
  option has currently no effect on any part of the run, but
  may be used in the future.

 \end{fdfentry}

 \begin{fdfentry}{QMMM.Driver.Type}[string]<$socket$>
  The type of communication used by the MM program, can be
  either "socket" or "pipe".

 \end{fdfentry}

 If the chosen communication type is "pipe", the following
 three options are also relevant:

 \begin{fdfentry}{QMMM.Driver.QMRegionFile}[string]<$siesta.qmatoms$>
  The name of the pipe/file containing the incoming QM coordinates
  and cell dimensions.

 \end{fdfentry}

 \begin{fdfentry}{QMMM.Driver.MMChargeFile}[string]<$siesta.mmatoms$>
  The name of the pipe/file containing the incoming MM coordinates
  and their charge values.

 \end{fdfentry}

 \begin{fdfentry}{QMMM.Driver.ForceOutFile}[string]<$siesta.allforces$>
  The name of the pipe/file containing the outgoing energy, forces and
  stress so that they are read by the MM program.

 \end{fdfentry}

 Otherwise, if the chosen communication type is "socket", the following
 three options are relevant instead:

 \begin{fdfentry}{QMMM.Driver.Address}[string]<$localhost$>
  The address of the socket used for communication. See also SocketType
  below in case the address is an IP address.

 \end{fdfentry}

 \begin{fdfentry}{QMMM.Driver.Port}[integer]<$10002$>
  For sockets that go over IP, the number of the port used for communications.

 \end{fdfentry}

 \begin{fdfentry}{QMMM.Driver.SocketType}[string]<$inet$>
  Can be either "inet" (for TCP/IP sockets) or "unix" for local sockets.

 \end{fdfentry}


 \subsection{The in-house QM/MM driver}
 \subsubsection{Running the driver}
 \siesta\ comes with a simple Molecular Mechanics program to use as the
 main driver for QM/MM simulations. As an additional input for this program,
 one needs to provide an extra "amber.parm" parameter file which contains
 all of the AMBER forcefield parameters needed for the classical part. There
 are a few examples provided in /Util/QMMM-driver/Tests.

 This driver can be used in two ways to communicate with siesta:
\begin{itemize}
  \item Launching both the driver and \siesta\ as different processes:

  \begin{shellexample}
    siesta_qmmm < driver_input.fdf > driver_out.out &
    siesta_qmmm < siesta_input.fdf > siesta_out.out
  \end{shellexample}

  Note that two separate input files are needed, one for the driver and
  one for \siesta\ .

  \item Launching \siesta\ as a subprocess from the driver, using
  the \fdf{LaunchSiesta} option. In this case, a separate \siesta\ input
  file is not needed, and the driver will pass along the options present
  in its own fdf file.
\end{itemize}

  Communications are done via pipes (see previous section), and if \siesta\
  is launched independently (first option above), the following block is
  necessary on the \siesta\ input file:

  \begin{shellexample}
    QMMM.Driver.QMRegionFile {label}.siesta.coords
    QMMM.Driver.MMChargeFile {label}.siesta.pc
    QMMM.Driver.ForceOutFile {label}.siesta.forces
  \end{shellexample}

  Replace \{label\} for the system label you are using in the QMMM driver.


  \subsubsection{The MM parameter file}
  By default, the MM parameter file is taken as an "amber.parm" file residing in
  the working directory. There are two different ways to specify a different
  path or filename:

  \begin{itemize}
    \item Via the SIESTA\_MM\_PARM\_FILE environment variable. It should point
          to the specific file. ( for example:
          SIESTA\_MM\_PARM\_FILE=/home/myparm.parm )

    \item Via the \fdf{MM.ParmFile} option within the driver's fdf file.
  \end{itemize}

  \subsubsection{QM/MM driver-specific options}
  The driver inherits from \siesta\ most of the options related to system
  description and molecular dynamics, thus variables such as \fdf{NumberOfAtoms}
  will remain the same.

  However, there are a few driver-specific options.

  \begin{fdfentry}{QM.AtomTypes}[block]
    This block contains the MM identities for atoms in the QM region. This
    is only relevant for Lennard-Jones interactions. Atoms must be in the
    same order as in the \fdf{AtomicCoordinatesAndAtomicSpecies} block. For
    example, for a single water molecule in the QM region:

    \begin{shellexample}
      %block QM.AtomTypes
        OW
        HW
        HW
      %endblock QM.AtomTypes
    \end{shellexample}

    The atom types (OW, HW) must coincide with the information available in the
    amber.parm file.

  \end{fdfentry}

  \begin{fdfentry}{MM.ParmFile}[string]<$'amber.parm'$>
    File containing the classical (MM) forcefield parameters.
  \end{fdfentry}

  \begin{fdfentry}{NumberOfMMAtoms}[integer]<$0$>
    The number of classical MM atoms in the system (not counting link atoms).
  \end{fdfentry}

  \begin{fdfentry}{MM.Atoms}[block]
    This block contains names and coordinates for MM atoms. It follows a
    structure similar to that of a PDB file, thus making it easier to copy
    from an available PDB:

    \begin{shellexample}
      %block MM.Atoms
        ATOM      1  O   HOH     1     11.610  15.591  10.441
        ATOM      2  H1  HOH     1     10.681  15.570  10.811
        ATOM      3  H2  HOH     1     11.621  15.201   9.521
        ATOM      4  O   HOH     2     11.560  12.061   0.651
        ATOM      5  H1  HOH     2     11.911  11.890   1.571
        ATOM      6  H2  HOH     2     10.681  11.610   0.530
      %endblock MM.Atoms
    \end{shellexample}

    It begins with a single word, followed by the atom index, then the
    atom name, then the residue name and the residue index, and finally the X,
    Y, and Z coordinates. Both indices start from 1. \textbf{Residue and atom
    names must coincide with the information present in the amber.parm file}.

  \end{fdfentry}

  \begin{fdfentry}{CoulombType}[string]<$'ewald'$>
    Sets the type of description for coulomb interactions in MM-MM and QM-MM
    interactions. Can be either "ewald" or "cut-off".
  \end{fdfentry}

  \begin{fdfentry}{MM.Cutoff}[length]<$0.8,\mathrm{Ang}$>
    For MM-MM interactions. When Ewald summations are used, this will also be
    the value for \fdf{QMMM.Ewald.Rcut}.
  \end{fdfentry}

  \begin{fdfentry}{Block.Cutoff}[length]<$100.0,\mathrm{Ang}$>
    If given a value of 99.0 Ang or below, MM atoms that are further than a
    \fdf{Block.Cutoff} distance from the QM region will be frozen and prevented from
    moving during the molecular dynamics.
  \end{fdfentry}

  \begin{fdflogicalF}{CenterQMSystem}
    At every MD step, re-center the entire system on the geometrical center for
    the QM region.
  \end{fdflogicalF}

  \begin{fdflogicalF}{LaunchSiesta}
    Launch \siesta\ as a subprocess for the QMMM driver. This avoids the need
    to run separate processes. By default, the driver will look for a \siesta\
    excecutable in the working directory, but a different location can be
    provided using the SIESTA environment variable. For example:

    \begin{shellexample}
      SIESTA=path/to/siesta/bin  \
        siesta_qmmm < driver_input.fdf > driver_out.out
    \end{shellexample}

    SIESTA\_PARALLEL\_COMMAND can be combined with the SIESTA environment variable,
    see \fdf{MPICommand}.
  \end{fdflogicalF}

  \begin{fdfentry}{NumberMPInodes}[integer]<$1$>
    When using \fdf{LaunchSiesta}, this option can be used to choose the amount
    of CPUs used for \siesta\ itself (the driver always runs in serial). When
    set to 1, \fdf{MPICommand} will be ignored.

    This option can be overridden with the SIESTA\_PARALLEL\_COMMAND environment
    variable (see \fdf{MPICommand}).
  \end{fdfentry}

  \begin{fdfentry}{MPICommand}[string]<$'mpirun'$>
    When using \fdf{LaunchSiesta}, this option can be used to choose a different
    program for parallelization, for example 'srun'.

    This can be overridden with the SIESTA\_PARALLEL\_COMMAND environment
    variable. In that case, SIESTA\_PARALLEL\_COMMAND must contain all of the
    information needed for the MPI launcher (e.g., the number of CPUs). This
    can be used as follows:

    \begin{shellexample}
      SIESTA_PARALLEL_COMMAND="srun -n 8" \
        siesta_qmmm < driver_input.fdf > driver_out.out
    \end{shellexample}

    SIESTA\_PARALLEL\_COMMAND can be combined with the SIESTA environment variable,
    see \fdf{LaunchSiesta}.
  \end{fdfentry}

\section{ANALYSIS TOOLS}

There are a number of analysis tools and programs in the \texttt{Util}
directory. Some of them have been directly or indirectly mentioned in
this manual. Their documentation is the appropriate sub-directory of
\texttt{Util}. See \texttt{Util/README}.

In addition to the shipped utilities \siesta\ is also officially
supported by \sisl\cite{sisl} which is a Python library enabling many
of the most commonly encountered things.


\subsection{2D Lindhard function calculator}
\label{sec:lindhard}

The Lindhard Function calculator computes the 2D Lindhard response
~\cite{doi:10.1088/1361-648X/ab8522} from \siesta\ outputs. The outputs
required are the EIG file (with the eigenvalues for each k-point) and the KP
file (with the coordinates for each k-point). The original \siesta\ calculation
must be done with the \fdf{TimeReversalSymmetryForKpoints} option set to
\fdffalse. Other than that, the Lindhard-function specific options should be
added to the same fdf file as the \siesta\ calculation. Then, the utility can be
used as:

\begin{verbatim}
  lindhard < <fdf file> > <output file>
\end{verbatim}

The lindhard utility accepts the following options:

\begin{fdfentry}{Lindhard!Temperature}[temperature]<$300\,\mathrm{K}$>

  The temperature for the Fermi distribution function used to calculate the
  Lindhard response.
\end{fdfentry}

\begin{fdfentry}{Lindhard!FirstBand}[integer]<$0$>
    Index of the first band to be included in the calculation; i.e., the lowest
    energy eigenvalue.
\end{fdfentry}

\begin{fdfentry}{Lindhard!LastBand}[integer]<$0$>
    Index of the last band to be included in the calculation; i.e., the highest
    energy eigenvalue.
\end{fdfentry}

\begin{fdfentry}{Lindhard!ngridx}[integer]<$0$>
    Number of k-grid points in the direction of the first reciprocal lattice
    vector. Must be an even number.
\end{fdfentry}

\begin{fdfentry}{Lindhard!ngridy}[integer]<$0$>
    Number of k-grid points in the direction of the second reciprocal lattice
    vector. Must be an even number.
\end{fdfentry}

\begin{fdfentry}{Lindhard!ngridz}[integer]<$0$>
    Number of k-grid points in the direction of the third reciprocal lattice
    vector. Must be an even number.
\end{fdfentry}

\begin{fdfentry}{Lindhard!nq1}[integer]<$1$>
    Step size in the first direction. In the direction of the first reciprocal
    lattice vector, calculate only every nq1 points. This only affects the
    granularity of the output, not the calculation itself.
\end{fdfentry}

\begin{fdfentry}{Lindhard!nq2}[integer]<$1$>
    Step size in the second direction. In the direction of the second reciprocal
    lattice vector, calculate only every nq2 points. This only affects the
    granularity of the output, not the calculation itself.
\end{fdfentry}

\section{SCRIPTING}

In the \texttt{Util/Scripting} directory we provide an experimental
python scripting framework built on top of the ``Atomic Simulation
Environment'' (see \url{https://wiki.fysik.dtu.dk/ase}) by the CAMD
group at DTU, Denmark.

(NOTE: ``ASE version 2'', not the new version 3, is needed)

There are objects implementing the ``Siesta as server/subroutine'' feature, and
also hooks for file-oriented-communication usage. This interface is
different from the \siesta-specific functionality already
contained in the ASE framework.

Users can create their own scripts to customize the ``outer geometry loop''
in \siesta, or to perform various repetitive calculations in compact form.

Note that the interfaces in this framework are still evolving and are
subject to change.

Suggestions for improvements can be sent to Alberto Garcia
(\href{mailto:albertog@icmab.es}{albertog@icmab.es})

\section{PROBLEM HANDLING}

\subsection{Error and warning messages}

\begin{description}
\itemsep 10pt
\parsep 0pt

\item[\texttt{chkdim: ERROR: In \textit{routine} dimension \textit{parameter} =
\textit{value}. It must be  ...}]

And other similar messages.

\textit{Description:} Some array dimensions which change infrequently,
and do not lead to much memory use, are fixed to oversized
values. This message means that one of this parameters is too small
and neads to be increased.  However, if this occurs and your system is
not very large, or unusual in some sense, you should suspect first of
a mistake in the data file (incorrect atomic positions or cell
dimensions, too large cutoff radii, etc).

\textit{Fix:} Check again the data file.  Look for previous warnings or
suspicious values in the output.  If you find nothing unusual, edit
the specified routine and change the corresponding parameter.

\end{description}

\section{REPORTING BUGS}
\index{bug reports}

Your assistance is essential to help improve the program. If you find
any problem, or would like to offer a suggestion for improvement,
please follow the instructions in the file
\texttt{Docs/REPORTING\_BUGS}.

Since \siesta\ has moved to
\url{https://gitlab.com/siesta-project/siesta} you are encouraged to
follow the instructions by pressing ``New Issue'' and selecting
``Bug'' in the Description drop-down. Also please follow the debug
build options, see Sec.~\ref{sec:build:debug}

\section{ACKNOWLEDGMENTS}

We want to acknowledge the use of a small number of routines,
written by other authors, in developing the siesta code.
In most cases, these routines were acquired by now-forgotten
routes, and the reported authorships are based on their headings.
If you detect any incorrect or incomplete attribution, or suspect
that other routines may be due to different authors, please
let us know.

\begin{itemize}
  \item%
  The main nonpublic contribution, that we thank thoroughly, are
  modified versions of a number of routines, originally written by
  \textbf{A. R.\ Williams} around 1985, for the solution of the radial
  Schr\"odinger and Poisson equations in the APW code of Soler and
  Williams (PRB \textbf{42}, 9728 (1990)).  Within \siesta, they are
  kept in files arw.f and periodic\_table.f, and they are used for the
  generation of the basis orbitals and the screened pseudopotentials.

  \item%
  The exchange-correlation routines contained in SiestaXC were written
  by J.M.Soler in 1996 and 1997, in collaboration with
  \textbf{C.\ Balb\'as} and \textbf{J. L.\ Martins}.  Routine pzxc,
  which implements the Perdew-Zunger LDA parametrization of xc, is
  based on routine velect, written by \textbf{S.\ Froyen}.

  \item%
  The serial version of the multivariate fast fourier transform used
  to solve Poisson's equation was written by \textbf{Clive Temperton}.

  \item%
  Subroutine iomd.f for writing MD history in files was originally
  written by \textbf{J. Kohanoff}.

\end{itemize}

We want to thank very specially \textbf{O. F.\ Sankey}, \textbf{D. J.\
    Niklewski} and \textbf{D. A.\ Drabold} for making the FIREBALL
code available to P.\ Ordej\'on.  Although we no longer use the
routines in that code, it was essential in the initial development of
the \siesta\ project, which still uses many of the algorithms
developed by them.

We thank \textbf{V. Heine} for his support and encouraging us in this
project.

The \siesta\ project is supported by the Spanish DGES through
several contracts. We also acknowledge past support by the Fundaci\'on
Ram\'on Areces.

\section{APPENDIX: Physical unit names recognized by FDF}
\label{sec:fdf-units}

Since \siesta\ 5.0 the units follow the CODATA 2018 values. This
affects comparisons with prior versions of \siesta\ due to small
numeric differences.

To compare numerical values between \siesta\ 5.0 and prior versions one
have to recompile 5.0 with
\begin{shellexample}
  cmake [.....] -DWITH_UNIT_CONVENTION=legacy
\end{shellexample}
\index{compile!pre-processor!-DSIESTA\_\_UNITS\_ORIGINAL}. Please
\emph{only} use this for comparisons and not for production runs.

\program{fdf} accepts nearly all conventional units used in physics
and chemistry. If a unit is not accepted a list of accepted units for
the requested dimension will be printed to standard out.

\newpage
\section{APPENDIX: XML Output}
\index{XML}
\index{CML}

From version 2.0, \siesta\ includes an option to write its output to an
XML file. The XML it produces is in accordance with the CMLComp subset of
version 2.2 of the Chemical Markup Language. Further information
and resources can be found at \url{http://cmlcomp.org/} and tools for working
with the XML file can be found in the \texttt{Util/CMLComp} directory.

The main motivation for standarised XML (CML) output is as a step
towards standarising formats for uses like the following.

\begin{itemize}

\item To have \siesta\ communicating with other software, either
for postprocessing or as part of a larger workflow scheme. In such a
scenario, the XML output of one \siesta\ simulation may be easily parsed
in order to direct further simulations. Detailed discussion of this is
out of the scope of this manual.

\item To generate webpages showing \siesta\ output in a more accessible,
graphically rich, fashion. This section will explain how to do this.

\end{itemize}

\subsection{Controlling XML output}

\begin{fdflogicalF}{XML!Write}

  Determine if the main XML file should be created for this run.

\end{fdflogicalF}

\subsection{Converting XML to XHTML}

The translation of the \siesta\ XML output to a HTML-based webpage is
done using XSLT technology. The stylesheets conform to XSLT-1.0 plus
EXSLT extensions; an xslt processor capable of dealing with this is
necessary. However, in order to make the system easy to use, a script
called ccViz is provided in \texttt{Util/CMLComp} that works on most Unix or
Mac OS X systems. It is run like so:

\texttt{./ccViz SystemLabel.xml}

A new file will be produced. Point your web-browser at \texttt{SystemLabel.xhtml}
to view the output.

The generated webpages include support for viewing three-dimensional
interactive images of the system. If you want to do this, you will
either need jMol (\url{http://jmol.sourceforge.net}) installed or access
to the internet. As this
is a Java applet, you will also need a working Java Runtime
Environment and browser plugin - installation instructions for these
are outside the scope of this manual, though. However, the webpages
are still useful and may be viewed without this plugin.

An online version of this tool is avalable from
\url{http://cmlcomp.org/ccViz/}, as are updated versions of
the ccViz script.

\newpage
\section{APPENDIX: Selection of precision for storage}
\index{Precision selection}

Some of the real arrays used in \siesta\ are by default
single-precision, to save memory. This applies to the array that holds
the values of the basis orbitals on the real-space grid, to the
historical data sets in Broyden mixing, and to the arrays used in the
O(N) routines. Note that the grid functions (charge densities,
potentials, etc) are (since mid January 2010) in double
precision by default.

The following options and pre-processing symbols control the
precision selection.

\begin{itemize}

  \item Add \texttt{-DWITH\_GRID\_SP} to the CMake invocation to use
    single-precision for all the grid magnitudes, including the
    orbitals array and charge densities and potentials.  This will
    cause some numerical differences and will have a negligible effect
    on memory consumption, since the orbitals array is the main user
    of memory on the grid, and it is single-precision by default. This
    setting will recover the default behavior of versions prior to
    4.0.\index{Grid precision}


  \item Use \texttt{-DFortran\_FLAGS="-DGRID\_DP"} to use
    double-precision for all the grid magnitudes, including the
    orbitals array. This will significantly increase the memory used
    for large problems, with negligible differences in accuracy.


  \item Use \texttt{-DFortran\_FLAGS="-DBROYDEN\_DP"} to use
    double-precision arrays for the data sets in the Broyden mixing
    for SCF convergence acceleration.\index{Broyden mixing}

  \item Use \texttt{-DFortran\_FLAGS="-DON\_DP"} to use
    double-precision for all the arrays in the O(N) routines.

\end{itemize}

\newpage
\section{APPENDIX: Data structures and reference counting}
\index{Reference counting}
\index{Data Structures}

To implement some of the new features (e.g. charge mixing
and DM extrapolation), \siesta\ uses new flexible data structures. These are defined and
handled through a combination and extension of ideas already in the
Fortran community:
\begin{itemize}
\item Simple templating using the ``include file'' mechanism, as for example in
  the FLIBS project led by Arjen Markus
  (\url{http://flibs.sourceforge.net}).
\item The classic reference-counting mechanism to avoid memory leaks, as
  implemented in the PyF95++ project
  (\url{http://blockit.sourceforge.net}).
\end{itemize}

Reference counting makes it much simpler to store data in container
objects. For example, a circular stack is used in the charge-mixing
module. A number of future enhancements depend on this paradigm.


\clearpage
\addcontentsline{toc}{section}{Bibliography}
\bibliographystyle{plainnat}
\bibliography{siesta}

% Indices
\clearpage
\addcontentsline{toc}{section}{Index}
\printindex

\printindex[sfiles]
\printindex[sfdf]


\end{document}




%%% Local Variables:
%%% mode: latex
%%% ispell-local-dictionary: "american"
%%% fill-column: 70
%%% TeX-master: t
%%% End:
